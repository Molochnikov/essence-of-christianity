\documentclass[12pt,oneside]{book}
\usepackage[utf8]{inputenc}
\usepackage[a6paper]{geometry}
\usepackage[LGR,T2A]{fontenc}
\usepackage[russian]{babel}
\usepackage{indentfirst}
\usepackage{tocloft}
\usepackage{fancyhdr}
\usepackage{calc}
\usepackage{xcolor}
\usepackage{tikz}
\usepackage{titling}
\usepackage{hyperref}
\usepackage[explicit]{titlesec}

\titleformat{\chapter}[display]
  {\large\bfseries}{\filcenter\chaptertitlename\ \thechapter}
  {20pt}{\filcenter\large#1}
\titlespacing*{\chapter}
  {0pt}{0pt}{20pt} 

\setlength{\headheight}{35pt}

\counterwithout*{footnote}{chapter}

\newcommand{\textgreek}[1]{
\begingroup\fontencoding{LGR}\selectfont#1\endgroup
}

\newcommand{\rom}[1]{\uppercase\expandafter{\romannumeral #1\relax}}

\pagestyle{fancy}
\fancyhead[LO]{\fontsize{10}{12}\selectfont\nouppercase\leftmark} 
\fancyhead[RO]{}

\title{Сущность\\\\христианства}
\author{Людвиг Андреас\\фон Фейербах}
\date{1841 г.}

\tolerance=10000
\hbadness=10000
\vbadness=10000
\clubpenalty=10000
\widowpenalty=10000
\displaywidowpenalty=10000
\interfootnotelinepenalty=10000

\begin{document}

\begin{titlepage}

\color{white}

\tikz[remember picture,overlay] \node[opacity=1.0,inner sep=0pt] at (current page.center){\includegraphics[width=\paperwidth,height=\paperheight]{DSA09695.jpeg}};

\noindent{\Huge\textbf{\thetitle}}

\vspace{1.5cm}
\noindent\theauthor

\vspace{1.5cm}
\noindent\thedate

\color{black}

\end{titlepage}

\tableofcontents

\phantomsection
\addcontentsline{toc}{chapter}{Предисловие к первому изданию}
\chapter*{Предисловие к первому изданию}

В предлагаемой книге сосредоточены рассеянные в различных трудах большею частью только случайные, полемические и афористические мысли автора по поводу религии и христианства, теологии и спекулятивной философии, причем они развиты, обработаны, обоснованы, сохранены или видоизменены, сжаты или расширены, смягчены или заострены в той мере, в какой это было целесообразно и, следовательно, необходимо; но ни в какой мере этот труд не является исчерпывающим уже по одному тому, что автор, не любящий общих мест, старался, как и всегда, не уклоняться от совершенно определенной темы.

Предлагаемое сочинение заключает в себе элементы, всего лишь элементы, для критики философии положительной религии или откровения, но, как и следовало ожидать, философии религии не в ребячески-фантастическом смысле нашей христианской мифологии, готовой принять за факт каждую нелепую историческую сказку, и не в педантическом смысле нашей спекулятивной теологии, которая, подобно схоластике, прямо возводит в логически-метафизическую истину каждый символ веры.

Спекулятивная философия религии приносит религию в жертву философии, христианская мифология приносит философию в жертву религии. Одна делает религию игрушкой умозрительного произвола, другая превращает разум в игрушку фантастического религиозного материализма. Одна заставляет религию говорить только то, что она сама желает, другая принуждает ее говорить вместо разума. Одна не способна выйти из себя и поэтому превращает образы религии в свои собственные мысли; другая не может войти в себя и делает образы вещами.

Само собой разумеется, что философия или религия вообще, то есть независимо от их специфического различия, тождественны. Другими словами, если одно и то же существо одновременно мыслит и верит, то и образы религии выражают одновременно мысли и вещи. Ведь каждая определенная религия, каждое верование есть в то же время известный образ мышления; ведь ни один человек не может верить в то, что противоречит его образу мысли и представлениям. Для человека, верующего в чудеса, чудо не есть нечто противоречащее разуму; оно, наоборот, кажется ему вполне естественным следствием божественного всемогущества, которое тоже является в его глазах совершенно естественным представлением. Человек верующий считает воскресение мертвых таким же понятным и естественным явлением, как восход солнца после его заката, пробуждение весны после зимы и появление растений из брошенных в землю семян. Вера и религия особенно противоречат разуму только там, где нарушается гармония между мыслями и чувствами человека и его верой и где, следовательно, вера перестает быть непреложной для человека истиной. Во всяком случае вера, как таковая, тоже считает свои предметы непостижимыми, противоречащими разуму, но она делает различие между христианским и языческим, просвещенным и естественным разумом. Это различие имеет следующее значение. Предметы веры могут казаться противоречащими разуму только людям неверующим, но всякий, кто верит в них, тот убежден в их истинности и признает их самих высшим разумом.

Но и при условии этой гармонии между христианской, или религиозной, верой и христианским, или религиозным, разумом все-таки всегда остается существенное противоречие между верой и разумом, так как и вера не может отрешиться от естественного разума. Естественный разум есть не что иное, как разум \textgreek{ηατ' έξοχην}\footnote{по преимуществу (др.-гр.).}, как всеобщий разум, --- разум, которому свойственны общие истины и законы. Напротив, христианская вера, или, что то же, христианский разум, есть средоточие особых истин, особых привилегий и изъятий, следовательно особый разум. Говоря короче и определеннее: разум есть правило, вера есть исключение из правила. Поэтому столкновение между ними неизбежно даже при условии гармонии, ибо специфичность веры и универсальность разума не покрывают друг друга, не насыщают друг друга, благодаря чему во всяком случае в известные моменты обнаруживается противоречие между остаткам самостоятельного, свободного разума и разумом, лежащим в основе религии. Таким образом, различие между верой и разумом становится психологическим фактом.

Сущность веры заключается не в сходстве с всеобщим разумом, а в отличии от него. Корень веры есть своеобразие, поэтому ее содержание даже внешним образом связано с особым периодом истории, с особым местом и именем. Отождествлять веру с разумом --- значит ослаблять веру, уничтожать это различие. Если, например, вера в первородный грех выражает только убеждение, что человек рождается не таким, каким он должен быть, --- значит это обыкновенная рационалистическая истина, истина, известная каждому, не исключая даже грубого дикаря, прикрывающего свой стыд звериной шкурой: ведь прикрывая себя таким образом, дикарь находит, что человек создан не таким, каким он должен быть. Разумеется, и в основе первородного греха лежит эта общая мысль, но предметом веры и религиозной истиной первородный грех становится только благодаря тем своим особенностям, которые противоречат всеобщему разуму.

Конечно, мышление всегда и неизбежно освещает и просветляет объекты религии, а с точки зрения религии или во всяком случае теологии разум только затемняет и разрушает религию; поэтому цель предлагаемого сочинения --- доказать, что в глубине сверхъестественных тайн религии кроются совершенно простые, естественные истины. При этом необходимо иметь в виду существенное различие между философией и религией, если мы хотим вскрыть не самих себя, а сущность религии. Существенное отличие религии от философии составляет символ, образ. Религия драматична по существу. Сам бог есть существо драматическое, то есть индивидуальное. Отняв у религии образ, мы отнимем у нее предмет, и у нас в руках останется только caput mortuum\footnote{череп, останки; букв. мертвая голова (лат.).}. Образ, как образ, есть вещь.

В предлагаемом сочинении образы религии рассматриваются только как образы, а не как мысли (во всяком случае, не с точки зрения спекулятивной философии религии) и не как вещи, то есть теология рассматривается не как мистическая прагматология, вразрез с христианской мифологией, и не как онтология, вразрез с умозрительной философией религии, а как психопатология.

Автор избрал наиболее объективный метод: метод аналитической химии. Поэтому он везде, где необходимо и возможно, ссылается на документы, помещенные частью внизу текста, частью в отдельном приложении. Цель этих источников --- узаконить заключения, выработанные путем анализа, то есть доказать их объективность. Поэтому если результаты его метода покажутся кому-либо странными и незаконными, то вина за это лежит не на авторе, а на предмете.

Автор недаром воспользовался свидетельствами давно минувших веков. Христианство тоже пережило некогда свой классический период, а ведь только истинное, великое, классическое достойно быть предметом мышления; все же, что не является классическим, заслуживает суда комедии или сатиры. Поэтому, чтобы представить христианство в качестве объекта, достойного мышления, автор должен был отрешиться от трусливого, бесхарактерного, комфортабельного, беллетристического, кокетливого, эпикурейского христианства наших дней и перенестись в те времена, когда христова невеста была еще целомудренной, чистой девой, когда она еще не вплетала в терновый венец своего небесного жениха розы и мирты языческой Венеры, чтобы не упасть в обморок при виде страждущего бога, когда она еще не имела сокровищ на земле, но зато в избытке блаженно наслаждалась тайнами сверхъестественной любви.

Современное христианство могло бы представить только один документ: testimonia paupertatis\footnote{свидетельство бедности, несостоятельности (лат.).}. Всем, чем обладает еще современное христианство, оно обязано не себе, --- оно живет подаянием минувших веков. Если бы современное христианство было предметом, достойным философской критики, то автору не пришлось бы тратить столько труда на изучение прошлого. То, что в этом сочинении доказывается, так сказать, a priori\footnote{из предшесвующего (лат.), знание, предшествующее опыту.}, а именно, что тайна теологии есть антропология, давно уже доказано и установлено a posteriori\footnote{из последующего (лат.), знание, исходящее из опыта.} историей теологии. <<История догмата>>, говоря проще, история теологии вообще, есть <<критика догмата>>, критика теологии в целом. Теология давно сделалась антропологией. История реализовала, сделала предметом сознания то, что было сущностью теологии самой в себе. Здесь метод Гегеля оказался совершенно верным и исторически обоснованным.

Хотя <<безграничная свобода и индивидуальность>> современного мира отразились на христианской религии и теологии настолько, что давно исчезло различие между творящим святым духом божественного откровения и все перерабатывающим человеческим духом, а сверхъестественное и сверхчеловеческое содержание христианства давно уже приняло естественный, антропоморфический характер. И всё же в силу нерешительности и неопределенности современной теологии в ней живет, подобно призраку, сверхчеловеческая, сверхъестественная сущность старого христианства. Но автор не намерен доказывать, что этот современный призрак есть только иллюзия, самообман человека; он считает такое доказательство неинтересным с философской точки зрения. Призраки --- это тени прошлого; они неизбежно наводят на вопрос: что представлял собой некогда призрак, когда он был облечен в плоть и кровь?

Но автор просит благосклонного и, главным образом, неблагосклонного читателя не забывать о том, что автор, черпающий материал из прошлого, --- сын не старого, а нового времени и пишет для современников; что, рассматривая первоначальную сущность призрака, он не теряет из виду его современного облика; что этот труд, носящий патологический или физиологический характер, преследует терапевтическую, или практическую, цель.

Эта цель заключается в рекомендации духовных ванн, в наставлении относительно употребления и пользы холодной воды естественного разума, в попытке воскресить древнюю простую ионическую гидрологию в области умозрительной философии, в частности в умозрительной философии религии. Древнее ионическое учение, в особенности Фалеса, в своем первоначальном виде таково: вода есть начало всех вещей и существ, следовательно также и богов. Ведь дух, или бог, который, по Цицерону, присутствует в воде при возникновении всех вещей как особое существо, есть, очевидно, позднейшее прибавление языческого теизма.

Сократовское \textgreek{γνῶθσεαυτόν}\footnote{познай самого себя (др.-гр.).}, служащее истинным эпиграфом и темой предлагаемого труда, вовсе не противоречит естественной простоте ионической мудрости, когда она понимается в должном смысле. Вода есть не только физическое средство рождения и питания, каким ее считала старая, ограниченная гидрология, но полезное психическое и оптическое лекарство. Холодная вода проясняет зрение. Один вид прозрачной воды доставляет неизъяснимое наслаждение. Эта оптическая ванна освежает душу и проясняет ум. Вода с магической силой влечет нас к себе, вглубь природы; в ней же отражается наш собственный образ. Вода есть подобие самосознания, подобие человеческого глаза, естественное зеркало человека. В воде смело освобождается человек от всяких мистических покровов; он погружается в воду в своем истинном, в своем обнаженном виде; в воде исчезают все сверхъестественные иллюзии. Так, некогда в воде ионической натурфилософии угас светильник языческой астрогеологии.

В этом заключается целительная сила воды, благотворность и необходимость духовного водолечения, особенно для такого робкого, самообольщающегося и изнеженного поколения, как наше.

Мы вовсе не намерены строить иллюзий относительно воды, светлой, чистой воды естественного разума, и снова связывать сверхъестественные представления с противоядием супранатурализма. Разумеется, \textgreek{ἂριστον ὕδωρ}\footnote{вода есть лучшее (др.-гр.).}, однако и 
\textgreek{ἂριστον μέτρον}\footnote{мера есть лучшее (др.-гр.).}. Сила воды есть сила, в себе самой ограниченная, определенная мерой и целью. Существуют болезни, которые невозможно излечить водой. К ним относится венерическая, сладострастная болезнь современных ханжей, рифмоплетов и эстетов, которые судят о ценности вещей только по их поэтическим достоинствам и, признавая иллюзию иллюзией, не стыдятся защищать ее ради ее красоты, которые так пусты и лживы, что даже не чувствуют, что иллюзия прекрасна лишь до тех пор, пока она считается истиной. Но этих безнадежно пустых, сладострастных субъектов вовсе и не имеет в виду духовный врач-гидропат. Только тот, кто ставит скромный дух истины выше поверхностной, обманчивой красоты, кто считает истину прекрасной, а ложь отвратительной, --- только тот достоин и способен принять святое крещение водой.

\bigskip
\hfill\emph{Людвиг Фейербах.}

\hfill\emph{1841 г.}

\phantomsection
\addcontentsline{toc}{chapter}{Предисловие ко второму изданию}
\chapter*{Предисловие ко второму изданию}

Нелепые и недобросовестные отзывы, высказанные по поводу моей книги со времени появления первого издания, вовсе не удивили меня, потому что других я не ждал и, рассуждая разумно и справедливо, не мог ждать. Эта книга поссорила меня с богом и миром. Я имел <<преступную дерзость>> уже в предисловии указать на то, что <<христианство тоже пережило некогда свой классический период, а ведь только истинное, великое, классическое достойно быть предметом мышления; все же фальшивое, ничтожное, не являющееся классическим, заслуживает суда комедии или сатиры. Поэтому, чтобы представить христианство в качестве объекта, достойного мышления, я должен был отрешиться от трусливого, бесхарактерного, комфортабельного, беллетристического, кокетливого, эпикурейского христианства наших дней и перенестись в те времена, когда Христова невеста была еще целомудренной, чистой девой, когда она еще не вплетала в терновый венец своего небесного жениха розы и мирты языческой Венеры, когда она еще не имела сокровищ на земле, но зато в избытке блаженно наслаждалась тайнами сверхъестественной любви>>.

Итак, я имел преступную дерзость вызвать на свет из мрака прошлого истинное христианство, от которого отреклись современные мнимые христиане, и сделал это не с похвальной и разумной целью представить современное христианство как non plus ultra\footnote{и не далее, дальше нельзя (лат.); в смысле предела, крайней степени чего-либо.} человеческого ума и сердца, а с противоположной, <<безрассудной>> и <<дьявольской>> целью: свести христианство к более высокому и общему началу. Благодаря этой преступной моей дерзости меня, как и следовало ожидать, предали проклятию современные христиане и особенно богословы. Я поразил умозрительную философию в ее самое чувствительное место, затронул ее самый point d'honneur\footnote{вопрос чести (фр.).}. Я безжалостно разоблачил ее мнимое согласие с религией, доказав, что ради этого согласия она лишила религию ее истинного, существенного содержания. С другой стороны, я показал в чрезвычайно неблагоприятном свете так называемую положительную философию, доказав, что оригинал ее идола есть человек, что личность нераздельна с плотью и кровью. Одним словом, моя экстраординарная книга явилась неожиданным ударом для ординарных философов-профессионалов. Далее, к сожалению, мои чрезвычайно бестактные, хотя интеллектуально и морально необходимые, разъяснения темной сущности религии навлекли на меня немилость политиков, как тех, которые пользуются религией как наиболее политическим средством унижения и эксплуатации человека, так и тех, которые находят, что религия не имеет ничего общего с политикой, и поэтому любят свободу и свет в области промышленности и политики и ненавидят их в области религии. Наконец, моя неосторожная попытка назвать вещи своими именами оказалась ужасным, непростительным нарушением современного этикета.


В <<хорошем обществе>> господствует нейтральный, бесстрастный тон условных иллюзий и лжи. Поэтому не только политические, что понятно само собой, но даже религиозные и научные вопросы, id est\footnote{то есть (лат.).}, составляющие <<злобу дня>>, обсуждаются в этом тоне. Сущностью нашего века является притворство, притворство во всем, начиная с политики и науки и кончая религией и нравственностью. Всякий, кто говорит теперь правду, считается дерзким, и <<невоспитанным>>, а <<невоспитанность>> равносильна безнравственности. Правда в наш век стала безнравственностью. Нравственным и достойным уважения считается лицемерное отрицание христианства, которое придает себе видимость его утверждения; безнравственным и преступным --- искреннее, нравственное отрицание христианства– отрицание, откровенно называющее себя отрицанием. Нравственными считаются произвольные сделки с христианством, позволяющие фактически игнорировать основной член христианского вероучения и сохранять видимость другого; ведь <<одно сомнение, --- согласно Лютеру*\let\svthefootnote\thefootnote
\let\thefootnote\relax\footnotetext{*в другом месте Лютер еще так выражает эту мысль: <<Одно из двух: или верить во все начисто и без всякого исключения, или же ни во что не верить. Святой дух нельзя делить на части так, чтобы одну часть считать истинной, а другую --- ложной\dots Колокол, давший трещину, уже не звучит и весь никуда не годится>>. Как это правильно! Как оскорбляет музыкальный слух колокольный звон современной веры! Но и то сказать: колокол это весь в трещинах.}
\let\thefootnote\svthefootnote, --- влечет за собой все другие>>, по крайней мере в принципе. Безнравственно серьезное отречение от христианства в силу внутренней необходимости; нравственна бестактная половинчатость; безнравственна уверенная в себе, убежденная целостность; нравственно легкомысленное противоречие; безнравственна строгая последовательность; нравственна посредственность, ничего не доводящая до конца, не умеющая проникать в глубину вещей; безнравственен гений, изучающий предмет до основания. Короче: нравственна одна только ложь, скрывающая зло истины или --- что теперь одно и то же --- истину зла.

Истина считается в наши дни не только безнравственной, но и ненаучной. Истина --- граница науки. Свобода плавания немецких кораблей по Рейну простирается только jusques \`a la mer\footnote{только до моря (фр.).}; свобода немецкой науки --- ju\. sques \` a la v\' erit\' e\footnote{только до истины (фр.).}. Когда наука достигает истины, она перестает быть наукой и делается объектом полиции: полиция есть граница между истиной и наукой. Истиной является человек, а не разум in abstracto\footnote{в абстракции (лат.).}, жизнь, а не мысль, остающаяся на бумаге, к этому стремящаяся и находящая на бумаге свое оправдание. Поэтому мысли, переходящие непосредственно из пера в кровь, из разума в человека, перестают быть только научными истинами. Наука в сущности является теперь бесполезной, но зато и безвредной игрушкой ленивого разума; она стала заниматься вещами безразличными для жизни и человека; но даже и в противном случае она все же настолько скучна и бесполезна, что никто о ней не думает. Поэтому необходимым качеством настоящего, почтенного, признанного ученого является пустота ума, холодность сердца, отсутствие настроения --- одним словом, бесхарактерность. Эти качества особенно необходимы для ученого, который в силу своей науки должен касаться щекотливых вопросов современности. Но ученый, отличающийся неподкупной любовью к истине, решительностью и твердостью, поражающий зло в корне, доводящий всякое дело до кризиса, до окончательного разрешения, --- такой ученый уже не есть ученый. Боже сохрани! --- он <<Герострат>>. Его необходимо как можно скорей вздернуть на виселицу или по крайней мере пригвоздить к позорному столбу. Последнее предпочтительнее; ведь смерть на виселице, по началам современного <<христианского государственного права>>, считается неполитической и <<нехристианской>>, так как она есть смерть очевидная и бесспорная; а пригвождение к позорному столбу, как смерть гражданская, отвечает лицемерному духу политики и христианства, так как не кажется смертью. А ведь отличительной чертой нашего времени является притворство, чистейшее притворство во всех сколько-нибудь щекотливых вопросах.

Немудрено поэтому, что в век мнимого, номинального, хвастливого христианства <<Сущность христианства>> была встречена с таким скандалом. Христианство настолько отстало от жизни и практики, что даже официальные и ученые представители христианства, богословы, не знают уже или, по крайней мере, не хотят знать, что такое христианство. Чтобы убедиться в этом собственными глазами, стоит только сравнить упреки, сделанные мне богословами, например, по поводу веры, чуда, провидения, ничтожества мира, с историческими свидетельствами, которые я в этом втором издании снабдил особенно многочисленными примечаниями. Рассмотрев эти упреки, можно увидеть, что они относятся не ко мне, а к самому христианству, что <<негодование>> богословов вызвано не моей книгой, а истинным, но ставшим совершенно чуждым им по духу содержанием христианской религии. Немудрено, что люди, которые, вероятно от скуки, отнеслись с аффектированной страстностью к давно отжившему и --- увы! --- столь ничтожному вопросу, как противоречие между католицизмом и протестантизмом, и не постыдились поднять серьезный спор о смешанном браке, --- немудрено, что эти люди сочли возмутительным анахронизмом книгу, доказывающую на основании исторических документов, что не только смешанный брак, брак между верующими и неверующими, но и брак вообще противоречит истинному христианству, ибо истинный христианин --– а разве <<христианские правительства>>, христианские пастыри, христианские учителя не обязаны заботиться, чтоб все мы были истинными христианами? --- не знает другого зачатия, кроме зачатия от святого духа, не ведает иного населения, кроме населения небес, а не Земли.

Поэтому шум, вызванный моей книгой и против нее направленный, не заставил меня изменить свой взгляд. Я еще раз подверг свою книгу спокойной и самой строгой исторической и философской критике, исправил по возможности ее формальные недостатки, развернул ее содержание, по-новому осветил его, дополнил новыми убедительными и неопровержимыми историческими свидетельствами. Может быть, теперь, когда я шаг за шагом пополнил свой анализ историческими справками, всякий, имеющий глаза, сознается, хотя бы и помимо своей воли, что моя книга есть точный, верный перевод христианской религии с образного языка восточной фантазии на простой, всем понятный немецкий язык. А цель моей книги --- именно быть точным переводом или, выражаясь без метафоры, эмпирико- или историко-философским анализом, решением загадки христианской религии. Общие положения, высказанные мной во введении, являются не априорными, вымышленными суждениями, не плодами умозрения; они вытекают из анализа религии и, подобно всем основным мыслям моего труда, суть облеченные в форму мыслей, то есть общих выражений, фактические проявления человеческой сущности, именно религиозной сущности и религиозного сознания человека. Мысли, высказанные в моем труде, вытекают из предпосылок, каковыми являются не отвлеченные мысли, а объективные, живые или исторические факты --- факты, которых в их громоздком изложении в фолиантах моя голова не могла вовсе вместить. Я вообще безусловно отвергаю абсолютное, нематериальное, самодовольное умозрение, черпающее материал из самого себя. Я не имею ничего общего с теми философами, которые закрывают глаза, чтобы легче было думать. Я мыслю при помощи чувств, главным образом зрения, основываю свои суждения на материалах познаваемых нами посредством внешних чувств произвожу не предмет от мысли, а мысль от предмета; предмет же есть только то, что существует вне моей головы. Я --- идеалист только в области практической философии, где я не считаю границ настоящего и прошедшего границами человечества, границами будущего, где я непоколебимо верю, что многое, что кажется современным недальновидным и малодушным практикам фантазией, неосуществимой мечтой, призраком, станет совершившимся фактом завтра, то есть в следующем столетии; ведь то, что является столетием по отношению к отдельным личностям, можно считать днем в жизни человечества. Короче: я смотрю на идею, как на веру в историческую будущность, в торжество истины и добродетели, и поэтому идея имеет для меня только политическое и нравственное значение. Зато в области собственно теоретической философии я, в прямую противоположность философии Гегеля, где дело обстоит как раз наоборот, считаюсь только с реализмом и материализмом в указанном смысле. Основной принцип господствовавшей доселе умозрительной философии: все, что мое, я ношу в себе самом --- древнее onania nia mecum porto --- к сожалению, ко мне неприменим. Вне меня существует множество вещей, которые я не могу носить с собою ни в кармане, ни в голове, но которые я в то же время как философ, а не просто как человек, о чем здесь не может быть речи, считаю своими. Я --- только духовный естествоиспытатель, а естествоиспытатель должен прибегать к инструментам, к материальным средствам. В качестве духовного естествоиспытателя я и написал предлагаемую книгу, содержащую в себе практически, то есть in concreto\footnote{конкретно (лат.).}, на особом предмете, но предмете всеобщего значения, а именно религии, доказанный, развитый и разработанный принцип новой философии, такой философии, которая существенно отличается от прежней философии и вполне отвечает истинной, действительной, целостной сущности человека и которая именно поэтому противоречит воззрениям людей, искалеченных и испорченных сверхчеловеческой, то есть противочеловеческой, религией и умозрением. Эта философия, как я уже говорил в другом месте, не признает единственным достойным органом откровения гусиное перо, а имеет глаза, уши, руки и ноги; она не отождествляет мысли о предмете с самим предметом с целью сделать при помощи пера действительное бытие бытием, существующим только на бумаге, а отделяет одно от другого и только в силу этого размежевания доходит до самой вещи; она представляет себе вещь не как объект отвлеченного разума, а как объект действительного, цельного человека, то есть как цельную, действительную вещь. Эта философия опирается не на разум в себе, не на абсолютный, безымянный, неизвестно кому принадлежащий, а на ум человека, далекого от умозрения и христианства. Она говорит человеческим, а не безымянным, неопределенным языком; она и на словах и на деле усматривает сущность философии в отрицании философии, то есть объявляет истинной философией только такую, которая облечена in succum et sanguinem\footnote{в плоть и кровь (лат.).}, в человеческую философию, и торжествует, поскольку пустые невежды, принимающие за сущность философии ее призрак, не желают вовсе признавать ее философией.

Эта философия отрешилась от субстанции Спинозы, от <<Я>> Канта и Фихте, от абсолютного тождества Шеллинга, от абсолютного духа Гегеля и тому подобных отвлеченных, только мыслимых или воображаемых вещей и сделала своим принципом действительную, вернее, самую действительную сущность, истинную ens realissimum\footnote{реальнейшая сущность (лат.).} человека, то есть самое положительное, реальное начало. Она производит каждую мысль из ее противоположности, из материи, из сущности, из чувств, и относится к предмету чувственно, то есть страдательно и рецептивно, прежде чем определить его мысленно. Поэтому моя книга, являясь, с одной стороны, истинным, облеченным в плоть и кровь результатом прежней философии, с другой --- имеет столь малое отношение к области умозрения, что представляет скорее ее прямую противоположность, пожалуй, конец умозрения. Умозрение заставляет религию говорить только то, что само оно измыслило и выразило гораздо лучше, чем религия; она определяет религию, но сама ею не определяется; она не выходит из пределов себя самой. Я даю религии возможность высказаться самой; я играю роль слушателя, переводчика, а не суфлера. Моя единственная цель --- не изобретать, а <<вскрыть существование>>; мое единственное стремление --- правильно видеть. Не я, а религия поклоняется человеку, хотя она, или, вернее теология, отрицает это. Не только я, но и сама религия говорит: бог есть человек, человек есть бог. Не я, а сама религия отвергает и отрицает такого бога, который не есть человек, а только ens rationis\footnote{рациональная (разумная) сущность (лат.).}; она заставляет бога сделаться человеком, и только этот ставший человеком, по-человечески чувствующий и мыслящий бог становится предметом ее поклонения и почитания. Я только разоблачил тайну христианской религии, сорвал с нее противоречивый и ложный покров теологии и --- как оказывается --- совершил настоящее святотатство. Если даже моя книга носит отрицательный, безбожный, атеистический характер, то ведь не надо забывать, что атеизм --- по крайней мере в смысле моей книги --- есть тайна самой религии; ибо религия не только внешним образом, но и по существу, не только в мыслях и воображении, но всем сердцем своим верит исключительно в истинность и божественность человеческого существа. Или пусть мне докажут ошибочность и неправильность моих исторических и рациональных аргументов. Пусть попытаются, я прошу об этом, опровергнуть их, но не путем юридических оскорблений, не путем богословских иеремиад, избитых спекулятивных фраз или анонимных недостойных приемов, а путем доводов, и притом таких доводов, которых я сам основательно не опроверг в этой книге.

Разумеется, моя книга носит характер отрицательный, разрушительный, но это относится не к человеческой, а к сверхчеловеческой сущности религии. Поэтому она распадается на две части, из которых одна по отношению к главному предмету является утверждением, другая со включением приложения --- отрицанием, пусть неполным, но в большей его части. Но в обеих частях доказывается одно и то же --- только различным или, вернее, противоположным образом. Первая часть ищет разгадки религии в ее сущности, в ее истине; вторая --- в ее противоречиях; в первой дано подробное изложение, вторая --- полемическая; первая носит характер спокойный, а вторая --- более живой. Изложение подвигается вперед медленно, борьба --– быстро; ведь изложение находит удовлетворение на каждой стадии, а борьба --- только в конечной цели. Изложение отличается озабоченностью, борьба --- решительностью. Изложение требует света, борьба --- огня. Поэтому обе части различаются даже в формальном отношении. В первой части я доказываю, что истинный смысл теологии есть антропология, что между определениями божественной и человеческой сущности, следовательно, между божественным и человеческим субъектом или существом нет различия, что они тождественны; ведь если повсюду, как в теологии, предикаты выражают не случайные качества, акциденции, а сущность субъекта, то между предикатом и субъектом нет разницы и предикат может быть поставлен на место субъекта; в этом отношении я отсылаю к <<Аналитике>> Аристотеля\footnote{имеется в виду <<Первая аналитика>>, одно из логических сочинений Аристотеля.} или хотя бы к <<Введению>> Порфирия\footnote{имеется в виду сочинение философа-неоплатоника Порфирия <<Введение в ,,Категории`` Аристотеля>>.}. Во второй части я показываю, что различие, которое делают или, вернее, считают нужным делать, между теологическими и антропологическими предикатами, ничтожно и бессмысленно.

Наглядный пример. В первой части я доказываю, что сын божий в религии есть сын действительный, сын бога в том же смысле, в каком человек есть сын человека, и вижу истину, сущность религии в том, что она понимает и утверждает глубоко человеческое отношение, как отношение божественное. Во второй части я утверждаю, что если не в самой религии, то в рефлексии ее сын божий считается сыном не в естественном, человеческом, а в совершенно ином, противоречащем природе и разуму, следовательно, в нелепом и непонятном смысле, и нахожу в этом отрицании человеческого смысла и ума противоречие истине, отрицательный момент религии.

Согласно этому, первая часть является прямым, вторая --– косвенным доказательством, что теология есть антропология. Вторая часть тесно связана с первой и не имеет самостоятельного значения. Ее цель --- доказать только, что смысл, в котором там понимается религия, должен быть истинным, потому что противоположный смысл есть бессмыслица. Короче: в первой части я имею дело главным образом с религией, а во второй --- с теологией; я говорю: главным образом, потому что и в первой части мне иногда приходилось касаться теологии, а во второй --- религии. Но многие ошибаются, предполагая, что я имел в виду только обыкновенную теологию, от которой я, напротив, старался держаться как можно дальше, ограничиваясь, самым существенным, самым строгим, самым необходимым определением предмета. Так, например, рассматривая таинства, я остановился только на двух, потому что в строгом смысле, согласно толкованию Лютера, есть только два таинства\footnote{здесь и далее Фейербах цитирует Лютера по Собранию его сочинений: Martin Luther. S\" amtliche deutsche Schriften und Wercke. \rom{22} Theile. Leipzig, 1729--1734 г.}. Следовательно, я ограничивался только тем определением, которое сообщает предмету общий интерес, возвышает его над ограниченной теологической сферой, и, таким образом, я имел в виду умозрительную теологию, или философию. Я говорю теологию, а не теологов, так как я останавливаюсь всюду только на том, что составляет prima casua\footnote{первопричина (лат.).}, оригинал, а не копию, принципы, а не личности, род, а не индивидов, объекты истории, а не объекты chronique scandaleuse\footnote{скандальная хроника (фр.).}.

Если бы моя книга заключала в себе только вторую часть, то можно было бы упрекнуть меня в исключительно отрицательной тенденции и положение <<религия есть ничто, нелепость>> принять за существенное содержание ее. Но я не говорю: бог есть ничто, троица --- ничто, слово божие --- ничто и так далее (поступить так было бы весьма легко). Я показываю только, что они не то, чем представляют их нам теологические иллюзии, что они не иноземные, а родные нам мистерии, мистерии человеческого рода. Я показываю, что религия принимает мнимую, поверхностную сущность природы и человечества за их истинную, внутреннюю сущность, а их истинную, эзотерическую сущность представляет себе в качестве другого, особого существа, благодаря чему все религиозные определения бога, например, определение слова божия --- по крайней мере не отрицательные в вышеуказанном смысле --- определяют или объективируют только истинную сущность человеческого слова. Упрек в том, что по смыслу моей книги религия является бессмыслицей, ничем, чистой иллюзией, имел бы основание только в том случае, если бы для меня бессмыслицей, ничто, чистой иллюзией было то, к чему я свожу религию и что я считаю ее подлинным предметом и содержанием, то есть человека, антропологию. Но я далек от того, чтобы придавать антропологии ничтожное или хотя бы второстепенное значение --- такое значение свойственно ей лишь постольку, поскольку ей противопоставляется теология как нечто высшее. Низводя теологию к антропологии, я возвышаю антропологию до теологии, подобно христианству, которое, унизив бога до человека, сделало человека богом --- хотя и далеким от человека, трансцендентным, фантастическим богом. Поэтому и самое слово антропология понимается мною не в смысле гегелевской или иной старой философии вообще, а в бесконечно более высоком и всеобщем смысле.
Религия есть сон человеческого духа; но и во сне мы находимся не на небе, а на земле --- в царстве действительности; только мы видим действительные предметы не в реальном свете необходимости, а в чарующем, произвольном блеске воображения и прихоти. Я только открываю религии и спекулятивной философии или теологии глаза или, вернее, обращенный внутрь взгляд направляю на внешний мир, то есть превращаю предмет воображаемый в предмет действительный.

Но, разумеется, в наше время, предпочитающее образ самой вещи, копию --- оригиналу, представление --- действительности, видимость --- сущности, такое превращение является отказом и, следовательно, абсолютным отрицанием или, по крайней мере, дерзкой профанацией, ибо священна только видимость, истина же нечестива. В представлении современников святость возрастает по мере того, как уменьшается истина и растет видимость, так что высшая ступень видимости в то же время составляет для них высшую ступень святости. Религия исчезла, и ее место заступила даже у протестантов иллюзия религии --- церковь, имеющая целью внушить невежественной и слепой толпе веру в то, что христианская вера еще существует, потому что теперь, как и тысячу лет тому назад, существуют христианские церкви и соблюдаются все внешние обряды веры. Вера современного мира есть вера мнимая, не верящая в то, во что она якобы верит, нерешительное, недостаточно сознательное неверие, как это неоднократно доказывалось мной и другими. Все отжившее в смысле веры должно жить во мнении людей; все, что перестало быть священным само по себе, должно по крайней мере казаться священным. Этим объясняется мнимо религиозное негодование нашего века, века лжи и призраков, по поводу моего анализа таинств. От писателя, который заботится не о благосклонности современников, а только о неприкрашенной, голой истине, нельзя требовать, чтобы он лицемерил и относился с уважением к простой видимости, тем более если предмет этой видимости сам по себе является кульминационным пунктом религии, то есть тем пунктом, где религиозность превращается в безбожие. Это я говорю для обоснования, а не для оправдания моего анализа таинств.

Что касается истинного смысла этого анализа, заключающегося главным образом в конце книги, то здесь я подтверждаю наглядным примером существенное содержание моей книги, ее прямую тему, в особенности по отношению к ее практическому значению; здесь я призываю чувства в свидетели правдивости моего анализа и мыслей, показываю 
ad oculos\footnote{наглядно, воочию (лат.).}, даже ad tactum, ad gustum\footnote{для осязания, на вкус (лат.).}, вкусовой очевидности то, что я разрабатываю в целой книге ad captum\footnote{букв. для головы (лат.), т.е. в теоретическом смысле.}. Вода крещения, вино и хлеб причастия в их естественной силе и значении гораздо более важны и действенны, чем в смысле сверхъестественном, иллюзорном. И вообще предмет религии, понимаемый в том смысле, какой я вкладываю в свою книгу, то есть в смысле антропологическом, является гораздо более плодотворным и более реальным предметом теории и практики, чем в богословском смысле. Ведь свойства, которые приписываются или должны приписываться воде, вину и хлебу в качестве чего-то отличного от этих естественных веществ, существуют только в представлении, в воображении, а никак не в действительности, не доподлинно. То же можно сказать и о предмете религии вообще --- о божественной сущности в отличие от сущности природы и человечества. Определения этой сущности, как разум, любовь и т. п., долженствующие иметь иное значение, чем те же определения, относящиеся к сущности человека и природы, даны не в действительности, а только в представлении и воображении. Мораль этой басни такова: мы не должны, подобно теологии и умозрительной философии, определения и подлинные силы действительности, вообще действительные существа и предметы делать произвольными знаками, орудиями, символами или предикатами отличного от них, трансцендентного, абсолютного, то есть абстрактного, существа; мы должны понимать и усваивать их в том значении, какое они имеют сами по себе, какое отвечает их качеству и определенности, делающей их тем, что они есть. Только таким образом мы найдем ключ к реальной теории и практике. Действительно, я выдвигаю на место бесплодной воды крещения благодеяние действительной воды. Как это <<водянисто>>, как тривиально! Разумеется, очень тривиально. Но очень тривиальной истиной был в свое время и брак, который Лютер во имя естественного человеческого чувства противопоставил мнимо священной иллюзии безбрачия. Поэтому вода является для меня во всяком случае вещью и в то же время также орудием, образом, примером, символом <<нечестивого>> духа моей книги, подобно тому как предмет моего анализа, вода крещения, является одновременно настоящей и образной или символической водой. То же относится к вину и хлебу. Людская злоба сделала из этого нелепый вывод, будто омовение, еда и питье составляют summa summarum\footnote{конечный итог (лат.).}, положительный результат моей книги. На это я могу возразить только следующее: если всё содержание религии заключается в таинствах, если нет других религиозных актов или действий, кроме тех, которые совершаются во время крещения и причащения, тогда, конечно, все содержание и положительный результат моей книги сводятся к омовению, пище и питью, что лишний раз доказывает, что моя книга есть точный, верный своему предмету историко-философский анализ, разоблачение, самосознание религии.

Я говорю: историко-философский анализ в отличие от исключительно исторических анализов христианства. Историк доказывает, например, подобно Даумеру, что причастие есть ритуал, относящийся к древнему культу человеческих жертв, что некогда вместо вина и хлеба употреблялись в пищу настоящее тело и кровь человека. Я же делаю предметом своего анализа и объяснения только христианское, христианством освященное значение причастия и согласно своему основному принципу признаю, что источником известного догмата или института (независимо от существования его в других религиях) можно считать только то значение, какое этот догмат имел в христианстве, но, разумеется, не в современном, а в древнем, истинном христианстве. Или другой историк, как, например, Люцелъбергер, показывает, что легенды о чудесах Христа полны противоречий и несообразностей, что они являются позднейшими вымыслами, что, следовательно, Христос не был чудотворцем, вообще не был таким, каким его изображает Библия. Я же, напротив, не рассматриваю вопроса, чем был действительный, естественный Христос в отличие от вымышленного или супранатуралистического Христа. Я принимаю данного религиозного Христа, но показываю, что это сверхъестественное существо есть лишь продукт и объект сверхъестественного человеческого чувства. Я не спрашиваю, возможно ли то или другое чудо или чудо вообще; я только показываю, чт\'{о} такое чудо, и притом не отвлеченно, а на примерах чудес, изображенных в Библии как действительные события; и тем самым я отвечаю на вопрос о возможности или действительности или даже необходимости чуда; этим путем уничтожается самая возможность всех этих вопросов. Таково различие между мной и антихристианскими историками. Что же касается моего отношения к Штраусу и Бруно Бауэру, в связи с которыми постоянно называют и мое имя, то я тут обращу лишь внимание на различие трактуемых нами вопросов: это видно уже из различия предметов, указанных в заглавиях наших сочинений. Бауэр избрал предметом своей критики евангельскую историю, то есть библейское христианство, или, вернее, библейскую теологию; Штраус --- христианское вероучение и жизнь Иисуса, следовательно, догматическое христианство или, лучше, догматическое богословие. Я же избрал своей темой христианство вообще, то есть христианскую религию, и уже как следствие ее --- христианскую философию или теологию. Поэтому я цитирую главным образом только таких авторов, для которых христианство было не только теоретическим или догматическим предметом, не только теологией, но и религией. Мой главный предмет --– христианство, религия как непосредственный объект, непосредственная сущность человека. Эрудиция и философия служат для меня только средствами обнаружить скрытые в человеке сокровища.

Затем я должен напомнить и о том, что моя книга совершенно вопреки моему намерению и ожиданию проникла в широкую публику. Правда, я всегда считал мерилом наилучшего метода обучения и стиля не ученого, не абстрактного, факультетского философа-специалиста, а универсального человека; я считал критерием истины вообще человека, а не того или другого философа. Я всегда полагал высшую добродетель философа в самоотречении философа, в том, чтобы он не выносил своей философии напоказ, как человек и писатель, чтобы он был философом по существу, а не по форме, --– философом скромным, а не крикливым и вульгарным. В этом, как и во всех моих сочинениях, я поставил себе правилом высшую ясность, простоту и определенность, благодаря чему эта книга может быть доступна всякому образованному и мыслящему человеку. Но, несмотря на это, должной оценки и полного понимания я могу ожидать, разумеется, только от правдивого, компетентного ученого, от такого, который стоит выше убеждений и предрассудков образованной и необразованной черни; ведь несмотря на всю свою самостоятельность, мой труд есть в то же время и необходимый вывод из истории. Иногда я ссылаюсь на то или другое историческое явление, не называя его по имени, что я считал излишним: такие указания могут быть понятны только ученому. Так, например, в первой главе, где я развиваю необходимые следствия философии чувства, я имею в виду философов Якоби и Шлейермахера; во второй главе очень часто упоминаю о кантианстве, скептицизме, деизме, материализме, пантеизме; в главе <<Основная точка зрения религии>>, разбирая противоречие между религиозным, или теологическим, и физическим, или натурфилософским, взглядом на природу, я имею в виду философию эпохи догматизма, преимущественно философию Декарта и Лейбница, где это противоречие обнаруживается особенно ясно. Поэтому всякий, кто незнаком с историческими предпосылками и материалами моего сочинения, не будет в состоянии усвоить исходные пункты моих аргументов и идей; и неудивительно, если мои утверждения покажутся взятыми с воздуха, на какую бы твердую почву они ни опирались. Предмет моего сочинения заключает в себе общечеловеческий интерес, и не подлежит сомнению, что его основные мысли --- хотя и не в том виде, в каком они выражены здесь и могут быть выражены при существующих отношениях --- сделаются некогда достоянием человечества, так как в наше время им можно только противопоставить нелепые, бессильные, противоречащие истинной сущности человека иллюзии и предрассудки. Но я отнесся к своему предмету, как к научному вопросу, как к объекту философии, и не мог отнестись к нему иначе. Исправляя заблуждение религии, теологии и умозрения, я должен был употреблять их выражения и даже пускаться в метафизику, тогда как я, собственно, отрицаю умозрение и свожу теологию к антропологии. Моя книга заключает в себе, как я уже сказал, конкретно развитое начало новой философии, не школьной, а человеческой. Но так как она извлекает этот новый принцип из недр религии, то новая философия отличается от старой католической и современной протестантской схоластики тем, что ее согласие с религией не обусловливается согласием с христианской догматикой. Будучи порождена самой сущностью религии, она носит эту истинную сущность в себе и сама по себе, как философия, является религией. Но именно эта чисто формальная особенность делает недоступным для широкой публики мой генетический труд, построенный в силу этого на объяснениях и доказательствах.

Если некоторые мои утверждения, высказанные в этой книге, покажутся читателю недостаточно обоснованными, то я советую ему обратиться к моим более ранним сочинениям, в особенности к книге <<Пьер. Бейль. К истории философии человечества>>\footnote{эта работа была опубликована в 1838 г. и составила третий том историко-философских сочинений Фейербаха.}, а также к <<Философии и христианству>>\footnote{имеетеся в виду работа <<О философии и христианстве в связи с упреком и нехристианском характере, сделанном философии Гегеля>>, опубликованная в 1839 г.}, в них я немногими, но резкими чертами обрисовал историческое разложение христианства и показал, что христианство давно уже перестало отвечать требованиям разума и человеческой жизни и есть не что иное, как id\' ee fixe\footnote{навязчивая идея (фр.).}, резко противоречащая нашим страховым обществам, железным дорогам и пароходам, нашим картинным галлереям и музеям слепков, военным и промышленным школам, нашим театрам и физическим кабинетам. Навязчивая идея. 

\bigskip
\hfill\emph{Людвиг Фейербах.}

\hfill\emph{Брукберг, 14 февраля 1843 г.}

\bigskip

\emph{Р.S.} Когда я писал это предисловие, я не знал, что новошеллингианская философия провозглашена в газетах <<опорой государства>>. Эта философия нечистой совести давно уже копошится во тьме, потому что хорошо знает, что день ее появления на свет будет днем ее гибели. Это --- философия смешного тщеславия, опирающаяся не на аргументы, а на имена и титулы и притом на какие имена и титулы! Это --- теософическая шутка философствующего Калиостро \rom{19} столетия*\let\svthefootnote\thefootnote
\let\thefootnote\relax\footnotetext{*документальные доказательства правельности этой характеристики можно найти в исчерпывающем виде в безапелляционной книге Каппа, посвященной Шеллингу.}\let\thefootnote\svthefootnote. Сознаюсь, что если бы эта шутка появилась раньше, я бы иначе написал свое предисловие.

\bigskip

\hfill\emph{31 марта.}

\bigskip

Бедная Германия! Тебя нередко надували в области философии, и чаще всего тебя обманывал только что упомянутый Калиостро, который постоянно тебя морочил, никогда не выполнял того, что обещал, никогда не доказывал того, что утверждал. Прежде он по крайней мере номинально опирался на разум, на природу, то есть на названия вещей, теперь он хочет одурачить тебя окончательно именами лиц, вроде Савиньи, Твестена и Неандера! Бедная Германия! У тебя хотят отнять даже твою научную славу. Подписи стали играть роль научных доказательств и доводов разума. Но ты не позволишь одурачить себя. Ты не забыла еще истории с августинским монахом. Ты знаешь, что истина является в мир не в блеске декорации, не в сиянии тронов, не под звуки труб и литавров, а в тишине и неизвестности, среди слез и стона. Ты знаешь, что волны всемирной истории влекут за собой низы, а не лиц <<высокопоставленных>>, так как они поставлены слишком высоко.

\bigskip

\hfill\emph{1 апреля.}

\bigskip

\phantomsection
\addcontentsline{toc}{chapter}{Предисловие к третьему изданию}
\chapter*{Предисловие к третьему изданию}
Я убежден, что можно без конца говорить и писать, но я привык молчать там, где говорят дела. Поэтому я и теперь, при выходе нового издания, отказываюсь говорить читателю a priori то, в чем он может убедиться собственными глазами a posteriori. Я хочу только заранее обратить внимание на то, что в этом издании я по возможности избегал иностранных слов и перевел все наиболее длинные латинские и греческие цитаты, чтобы сделать их доступными более широким кругам. В этих переводах я строго придерживался смысла, а не буквы оригинала.

\bigskip

\hfill\emph{Людвиг Фейербах.}

\hfill\emph{1848 г.}




\part{Введение}





\chapter{Общая сущность человека}


Религия коренится в существенном отличии человека от животного: у животных нет религии. Хотя старшие, не владевшие критическим методом зоографы и приписывали слону религиозность наряду с другими похвальными качествами, тем не менее религия слона --- это басни. Кювье, один из величайших знатоков животного мира, на основании личных наблюдений не ставит слона выше собаки.

В чем же заключается это существенное отличие человека от животного? Самый простой, самый общий и вместе с тем самый обычный ответ на этот вопрос: в сознании в строгом смысле этого слова; ибо сознание в смысле самоощущения, в смысле способности чувственного различения в смысле восприятия и даже распознавания внешних вещей по определенным явным признакам свойственно и животным. Сознание в самом строгом смысле имеется лишь там, где субъект способен понять свой род, свою сущность. Животное сознает себя как индивид, почему оно и обладает самоощущением, --- а не как род, так как ему недостает сознания, происходящего от слова <<знание>>. Сознание нераздельно со способностью к науке. Наука --- это сознание рода. В жизни мы имеем дело с индивидами, в науке --- с родом. Только то существо, предметом познания которого является его род, его сущность, может познавать сущность и природу других предметов и существ.

Поэтому животное живет единой, простой, а человек двоякой жизнью. Внутренняя жизнь животного совпадает с внешней, а человек живет внешней и особой внутренней жизнью. Внутренняя жизнь человека тесно связана с его родом, с его сущностью, человек мыслит, то есть беседует, говорит с самим собой. Животное не может отправлять функций рода без другого индивида, а человек отправляет функции мышления и слова --- ибо мышление и слово суть настоящие функции рода, без помощи другого. Человек одновременно и <<Я>> и <<ты>>; он может стать на место другого именно потому, что объектом его сознания служит не только его индивидуальность, но и его род, его сущность.

Сущность человека в отличие от животного составляет не только основу, но и предмет религии. Но религия есть сознание бесконечного, и поэтому человек познает в ней свою не конечную и ограниченную, а бесконечную сущность. Доподлинно конечное существо не может иметь о бесконечном существе ни малейшего представления, не говоря уже о сознании, потому что предел существа является одновременно пределом сознания. Сознание гусеницы, жизнь и сущность которой ограничивается известным растением, не выходит за пределы этой ограниченной сферы; она отличает это растение от других растений, и только. Такое ограниченное и именно, вследствие этой ограниченности, непогрешимое, безошибочное сознание мы называем не сознанием, а инстинктом. Сознание в строгом или собственном смысле слова и сознание бесконечного совпадают; ограниченное сознание не есть сознание; сознание по существу всеобъемлюще, бесконечно. Сознание бесконечного есть не что иное, как сознание бесконечности сознания. Иначе говоря, в сознании бесконечного сознание обращено на бесконечность собственного существа.

Но в чем же заключается сущность человека, сознаваемая им? Каковы отличительные признаки истинно человеческого в человеке? Разум, воля и сердце*\let\svthefootnote\thefootnote
\let\thefootnote\relax\footnotetext{*бездушный материалист говорит: <<Человек отличается от животного только сознанием; он --- животное, но такое, которое обладает сознанием>>. Он не принимает, таким образом, во внимание, что в существе, в котором пробудилось сознание, происходит качественное изменение всей его сущности. Впрочем, этим нисколько не умаляется достоинство животных. Здесь не место глубже исследовать этот вопрос.}
\let\thefootnote\svthefootnote. Совершенный человек обладает силой мышления, силой воли и силой чувства. Сила мышления есть свет познания, сила воли --- энергия характера, сила чувства --- любовь. Разум, любовь и сила воли --- это совершенства. В воле, мышлении и чувстве заключается высшая, абсолютная сущность человека, как такового, и цель его существования. Человек существует, чтобы познавать, любить и хотеть. Но какова цель разума? Разум. Любви? Любовь. Воли? Свобода воли. Мы познаем, чтобы познавать, любим, чтобы любить, хотим, чтобы хотеть, то есть быть свободными. Подлинное существо есть существо мыслящее, любящее, наделённое волей. Истинно совершенно, божественно только то, что существует ради себя самого. Таковы любовь, разум и воля. Божественная <<троица>> проявляется в человеке и даже над индивидуальным человеком в виде единства разума, любви и воли. Нельзя сказать, чтобы разум (воображение, фантазия, представление, мнение), воля и любовь были силами, принадлежащими человеку, так как он без них --- ничто, и то, что он есть, он есть только благодаря им. Они составляют коренные элементы, обосновывающие его сущность, не являющуюся ни его непосредственным достоянием, ни продуктом. Это силы, оживотворяющие, определяющие, господствующие, это божественные, абсолютные силы, которым человек не может противостоять\dag\let\svthefootnote\thefootnote
\let\thefootnote\relax\footnotetext{\dag<<Каждое убеждение достаточно сильно, чтобы заставить себя отстаивать ценой жизни>> (Монтень).}\let\thefootnote\svthefootnote\footnote{Фейербах цитирует сочинение <<Опыты>> Монтеня (кн. \rom{1}, гл. \rom{14}). См.: Монтень М. Опыты, кн. \rom{1}, М., 1960 г., с. 67.}.

Как бы мог чувствующий человек противиться чувству, любящий --- любви, разумный --- разуму? Кто не испытал чарующей силы звуков? А что такое сила звуков, как не сила чувства? Музыка --- язык чувства; звук --- это громко выраженное чувство, которое сообщается другим. Кто не испытывал силы любви или по крайней мере не слыхал о ней? Кто сильнее: любовь или индивидуальный человек? Человек ли владеет любовью или, напротив, любовь человеком? Когда любовь побуждает человека даже с радостью идти на смерть ради любимого существа, то что это --- его собственная индивидуальная сила или скорее сила любви? Кто из мыслящих людей не испытал на себе силы мышления, тихой, бесшумной силы мышления? Когда ты погружаешься в глубокое раздумье, забывая о себе самом, об окружающем, ты ли владеешь тогда разумом или разум владеет тобой и поглощает тебя? Разве научное вдохновение не есть величайшая победа разума над человеком? Разве жажда знания не есть безусловно непреодолимая, всепобеждающая сила? А когда ты подавляешь свою страсть, отказываешься от своих привычек, одним словом, одерживаешь победу над самим собой, --- что это --– всепобеждающая сила, твоя личная, самодовлеющая сила или скорее сила воли, моральная сила, которая овладевает тобой помимо твоего желания и наполняет тебя негодованием против тебя самого и твоих личных слабостей?\ddag\let\svthefootnote\thefootnote
\let\thefootnote\relax\footnotetext{\ddagдля темы этой книги не имеет значения вопрос о том, обосновывается ли природой вещей различение между индивидуумом --- слово, которое, подобно всем отвлеченным словам, крайне неопределенно, двусмысленно и сбивчиво --- и любовью, разумом, волей. Религия абстрагирует от человека его силы, свойства, существенные определения и обожествляет их как самостоятельные существа, причем безразлично, остаются ли они раздельными --- в политеизме или сливаются воедино --- в монотеизме; таким образом, при объяснении характера этих существ и при их сведении к человеку необходимо не упускать из виду этого различия. Впрочем, оно обосновано не только самим предметом, но и практикой языка и --- что то же --- логикой, ибо человек отличает себя от своего духа, от своей головы, от своего сердца, как будто без них он представляет собою нечто.}\let\thefootnote\svthefootnote


Человек --- ничто без объекта. Великие, выдающиеся люди, раскрывающие пред нами сущность человека, подтверждали это своею жизнью. Они знали только одну преобладающую страсть: желание достигнуть цели, которая была главным объектом их деятельности. Но тот предмет, с которым субъект связан по существу, в силу необходимости, есть не что иное, как собственная, но объективная сущность этого субъекта. Общий предмет нескольких одинаковых по роду, но различных по виду индивидов является их собственной объективной сущностью, во всяком случае в той мере, в какой он служит этим индивидам объектом сообразно их особенностям.

Так, солнце есть общий объект всех планет, но оно не одинаково для Меркурия, Сатурна, Венеры, Урана и Земли. У каждой планеты свое собственное солнце. Солнце, поскольку оно освещает и согревает Уран, имеет для земли не физическое, а лишь астрономическое, научное значение. На Уране солнце не только кажется, но и действительно является иным, чем на Земле. В отношении земли к солнцу обнаруживается её отношение к самой себе и к своей собственной сущности, ведь размер, величина и сила света солнца, в качестве объекта земли, обусловливаются величиной расстояния земли от солнца, от него зависят особенности нашей планеты. Каждая планета поэтому имеет в своем солнце отражение своей сущности.

Человек самого себя познает из объекта: сознание объекта есть самосознание человека. По объекту мы можем узнать человека и его сущность. В объекте обнаруживается сущность человека, его истинное объективное <<Я>>. Это относится не только к умственным, но и к чувственным объектам. Наиболее отдаленные от человека объекты являются откровениями его человеческой сущности, поскольку и потому, что они являются его объектами. Луна, солнце и звезды взывают к человеку: \textgreek{γνῶθσεαυτόν}, познай самого себя. То, что он их видит, и видит так, а не иначе, свидетельствует о его собственной сущности. На животное производят впечатление только непосредственно для жизни необходимые лучи солнца, на человека --- равнодушное сияние отдаленнейших звезд. Только человеку доступны чистые, интеллектуальные, бескорыстные радости и аффекты; только человеческие глаза знают духовные пиршества. Взор, обращённый к звездному небу, созерцает бесполезные и безвредные светила и видит в сиянии их свою собственную сущность, свое собственное происхождение. Природа глаза небесна. Поэтому человек возвышается над землей только благодаря зрению, поэтому теория начинается там, где взор обращается к небу. Первые философы были астрономами. Небо напоминает человеку о его назначении, о том, что он создан не только для деятельности, но и для созерцания.

Собственная сущность человека есть его абсолютная сущность, его бог; поэтому мощь объекта есть мощь его собственной сущности. Так, сила чувственного объекта есть сила чувства, сила объекта разума --- сила самого разума, и, наконец, сила объекта воли --- сила воли. Человек, сущность которого определяется звуком, находится во власти чувства, во всяком случае того чувства, которое в звуке находит соответствующий элемент. Но чувством овладевает не звук, как таковой, а только звук, полный содержания, смысла и чувства. Чувство определяется только полнотой чувства, то есть самим собой, своей собственной сущностью. То же можно сказать и о воле и о разуме. Какой бы объект мы ни познавали, мы познаем в нем нашу собственную сущность; что бы мы ни осуществляли, мы в этом проявляем самих себя. Воля, чувство, мышление есть нечто совершенное, поэтому нам невозможно чувствовать или воспринимать разумом --- разум, чувством --- чувство и волей --- волю, как ограниченную, конечную, то есть ничтожную, силу. Ведь конечность и ничтожество --- понятия тождественные; конечность есть только эвфемизм для ничтожества. Конечность есть метафизическое, теоретическое выражение; ничтожество --- выражение патологическое, практическое. Что конечно для разума, то ничтожно для сердца. Но мы не можем считать волю, разум и сердце конечными силами, потому что всякое совершенство, всякая сила и сущность непосредственно доказывают и утверждают самих себя. Нельзя любить, хотеть и мыслить, не считая этих факторов совершенствами, нельзя сознавать себя любящим, желающим и мыслящим существом, не испытывая при этом бесконечной радости. Сознавать для существа значит быть предметом самого себя; поэтому сознание не есть нечто отличное от познающего себя существа, иначе как бы могло оно сознавать себя? Поэтому нельзя совершенному существу сознавать себя несовершенством, нельзя чувство ощущать ограниченным и мышлению ставить пределы.

Сознание --- это самоосуществление, самоутверждение, любовь к себе самому, наслаждение собственным совершенством. Сознание есть отличительный признак совершенного существа. Оно может быть только в полнокровном, совершенном существе. Доказательством этого служит даже человеческое тщеславие. Человек смотрится в зеркало и испытывает удовольствие, рассматривая свой облик. Это удовольствие является необходимым, непроизвольным следствием совершенства и красоты человека. Красивая форма довлеет себе и естественно радуется этому, она отражается в себе самой. Тщеславен только тот, кто восхищается своей личной красотой, а не человеческой красотой вообще. Человеческой наружностью следует восхищаться, в мире нет ничего более прекрасного и величественного*\let\svthefootnote\thefootnote\let\thefootnote\relax\footnotetext{*<<Человеку кажется, будто нет ничего красивее человека>> (Цицерон, О природе богов, кн. \rom{1}). И это не есть признак ограниченности, так как он и другие существа помимо себя находит прекрасными; его радует красота животных форм, красота растительных форм, красота природы вообще. Но лишь абсолютное, совершенное сознание может без чувства зависти любоваться обликом других существ.}\let\thefootnote\svthefootnote\footnote{см.: Цицерон. Философские трактаты. М., 1985 г., с. 84.}. Во всяком случае каждое существо любит себя, свое бытие и должно его любить. Бытие есть благо. <<Всё, --- говорит Бэкон, --- что достойно бытия, достойно и знания>>\footnote{Фейербах цитирует сочинение Ф. Бэкона <<Новый органон>> (кн. \rom{1}).}. Все существующее ценно, представляет нечто выдающееся и поэтому утверждает и отстаивает себя. Сознание --- это высшая форма самоутверждения, та форма, которая сама есть отличие, это --- совершенство, счастье, благо.



Всякое ограничение разума и вообще человеческой сущности вытекает из обмана, из заблуждения. Разумеется, человеческий индивид может и даже должен считать себя существом ограниченным --- этим он отличается от животного; но он может сознавать свою конечность, свою ограниченность только в том случае, если его объектом является совершенство, бесконечность рода, независимо от того, будет ли то объект чувства, совести или мыслящего сознания. Если, однако, человек приписывает свою ограниченность целому роду, то он заблуждается, отождествляя себя с родом, --- это заблуждение тесно связано с любовью к покою, леностью, тщеславием и эгоизмом. Ограниченность, которую я приписываю исключительно себе, унижает, смущает и беспокоит меня. Чтобы освободиться от чувства стыда и беспокойства, я приписываю свою личную ограниченность человеческому существу вообще. Что непонятно для меня, непонятно и для других, чего же мне смущаться? это не моя вина, это зависит не от моего рассудка, а свойственно рассудку рода, но это --- смешное и преступное заблуждение. На существо человеческой природы, на сущность рода, то есть на абсолютную сущность индивида, нельзя смотреть как на нечто конечное, ограниченное. Каждое существо довлеет себе. Ни одно существо не может отрицать себя, то есть свою сущность, ни одно существо не есть по себе существо ограниченное. Напротив, каждое существо по себе бесконечно и заключает в себе своего бога, свою высшую сущность. Всякая ограниченность того или иного существа заметна только для существа другого, высшего рода. Жизнь насекомых несравненно короче жизни более долговечных животных, но эта кратковременная жизнь кажется им самим не менее длинной, чем долголетняя жизнь --- другим. Листик, на котором живет гусеница, представляется ей целым миром, бесконечным пространством.

Человек становится тем, что он есть, благодаря своему таланту, богатству, украшениям. Как же можно считать свое бытие небытием, свое богатство --- нищетой, свой талант --- неспособностью? Если бы растения обладали зрением, вкусом и способностью суждения, то каждое из них считало бы свой цветок наиболее прекрасным, ибо рассудок цветка, его вкус не простирался бы дальше его производительной способности. Высший продукт этой способности казался бы растению величайшим произведением в мире. Рассудок, вкус, сила суждения не могут отрицать того, что утверждается самим существом, иначе этот рассудок принадлежал бы не данному, а какому-нибудь иному существу. Рассудок измеряет вещи меркой существа. Если существо ограниченно, то чувство и рассудок также ограниченны. Но ограниченный рассудок не есть граница в глазах ограниченного существа; такое существо вполне счастливо и довольно своим рассудком, оно сознает его, прославляет, считает его дивной, божественной силой; ограниченный рассудок, в свою очередь, ценит и восхваляет то ограниченное существо, которому он принадлежит. Оба как нельзя более подходят друг к другу; таким образом они могли бы распасться? Рассудок --– это кругозор данного существа, наше существо не простирается за пределы нашего зрения, и наоборот. Зрение животного не простирается дальше его потребностей, а его существо --- дальше этих же потребностей. И сколь обширна твоя сущность, столь же неограниченно твое самоощущение, настолько ты --- бог. Разлад между рассудком и существом, между силой мыслительной и силой производительной в человеческом сознании является, с одной стороны, разладом личным, не имеющим общего значения, с другой --- только кажущимся. Тот, кто признает, что его плохие стихи --- плохи, менее ограничен в своем познании и, следовательно, в своем существе, чем тот, кто считает хорошими свои плохие стихи.

Следовательно, мысля о бесконечном, ты мыслишь и утверждаешь бесконечность мыслительной способности; чувствуя бесконечное, ты чувствуешь и утверждаешь бесконечность чувствующей способности. Объектом разума является объективированный разум, объектом чувства --- объективированное чувство. Если ты не понимаешь и не чувствуешь музыки, то самая лучшая музыкальная пьеса произведет на тебя такое же впечатление, как шум ветра, дующего над твоим ухом, или журчание ручья под ногами. Почему же звуки музыки действуют на тебя? Что ты о них слышишь? Разве не слышишь ты в них голоса твоего сердца? Чувство обращается непосредственно к чувству и понятно только чувству, то есть самому себе, --- ведь объектом чувства является только чувство. Музыка --- монолог чувства, но и диалог философии есть в сущности не что иное, как монолог разума; мысль говорит лишь к мысли. Блеск кристаллов пленяет наши чувства, мысль говорит лишь к мысли. Блеск кристаллов пленяет наши чувства, но наш разум интересуется только кристаллономией. Объектом разума может быть только разумное\dag\let\svthefootnote\thefootnote\let\thefootnote\relax\footnotetext{\dag<<Ум восприимчив только к уму и к тому, что от него исходит>> (Реймарус).}\let\thefootnote\svthefootnote\footnote{Фейербах цитирует сочинение Г.С. Реймаруса <<О возвышенных истинах естественной религии>> (1754 г.). Reimarus, Wahrheit der natuerlichen Religion, \rom{4} Abt., par. 8.}.





Поэтому все то, что в смысле трансцедентного умозрения и религии имеет лишь значение производного, субъективного или человеческого, значение средства, органа в смысле истины, имеет значение первоначального, существенного, объективного. Так, например, если чувство --- существенный орган религии, то, следовательно, сущность бога есть не что иное, как сущность чувства. Истинный, но скрытый смысл слов: <<чувство есть орган божественный>> --- заключается в том, что чувство есть самое благородное и возвышенное, то есть божественное в человеке. Ты бы не мог постигать божественное чувством, если бы чувство не было божественного происхождения. Божественное познается только через божественное, <<бог только через себя самого познается>>. Божественная сущность, постигаемая чувством, есть не что иное, как очарованная и восхищенная собой сущность чувства, --- восторженное, блаженное в себе чувство.

Это явствует хотя бы из того, что там, где чувство становится органом бесконечного, субъективной сущностью религии, объект последней теряет свою объективную ценность. С тех пор как чувство сделалось главной основой религии, люди стали равнодушны к внутреннему содержанию христианства. Если чувство и приписывает предмету некоторую ценность, то это делается только ради самого чувства, которое связывается с ним, быть может, только по случайным основаниям; если бы другой предмет возбуждал те же чувства, он был бы столь же желательным. Предмет чувства становится безразличным, потому что чувство, признаваемое субъективной сущностью религии, действительно становится также её объективной сущностью, хотя это и не признается непосредственно. Непосредственно, говорю я, потому, что косвенное признание этого факта заключается в том, что чувство, как таковое, признается религиозным, чем уничтожается всякое различие между специфически религиозными и иррелигиозными или во всяком случае нерелигиозными чувствами --- необходимое следствие взгляда на чувство, как на единственный орган божественного. Ты считаешь чувство органом бесконечного, божественного существа только в силу его сущности, его природы. Но свойства чувства вообще присущи каждому отдельному чувству независимо от его объекта. Что же делает это чувство религиозным? Определенный объект? --- нисколько, потому что каждый объект религиозен только в том случае, если он является объектом не холодного рассудка или памяти, а чувства. Значит, что же? --- самая природа чувства, присущая каждому отдельному чувству, независимо от объекта последнего. Следовательно, чувство признается священным только потому, что оно чувство; причина его религиозности заключается в природе самого чувства, лежит в нем самом. Значит чувство признается абсолютным, божественным? Если чувство хорошо и религиозно, то есть священно, божественно по себе, то разве оно не заключает своего бога в себе самом?

Тем не менее, если ты хочешь определить объект чувства и в то же время правильно истолковать свое чувство, не привнося помощью рассудка ничего постороннего, ты должен провести грань между своими личными чувствами и общей сущностью чувства, отделить эту сущность от посторонних, оскверняющих влиянии, с которыми у тебя, условного индивида, связано чувство. Таким образом, единственное, что ты можешь объективировать, представить бесконечным, определить как его сущность, это природа чувства. Следовательно, бога можно определить только так: бог есть чистое, неограниченное, свободное чувство. Всякий иной бог, собою предполагаемый навязан твоему чувству извне. Чувство атеистично в смысле ортодоксальной веры, которая связывает религию с внешним объектом; чувство отрицает предметного бога оно есть бог само для себя. Отрицание чувства равносильно, с точки зрения чувства, отрицанию бога. Ты только слишком робок и ограничен, чтобы открыто признаться в том, что втайне утверждается твоим чувством. Связанный внешними условностями, неспособный постигнуть величие чувства, ты боишься религиозного атеизма твоего сердца и поэтому допускаешь раздвоение чувства, измышляешь отдельное от чувства объективное существо и неизбежно возвращаешься к старым вопросам и сомнениям: существует ли бог, или нет? Вопросы и сомнения эти немыслимы там, где сущностью религии признается чувство. Чувство твоя сокровеннейшая и вместе с тем отдельная, не зависящая от себя сила, действующая в тебе, выше тебя; это твоя подлинная суть, однако воздействующая на тебя, как другое существо, короче --- это твой бог. Зачем же ты воздаешь помимо этой сущности ещё новое предметное существо, вне твоего чувства?

Впрочем, я взял чувство только в виде примера. Те же доводы можно привести по отношению ко всякой другой силе, способности, потенции, реальности и деятельности, --- дело, разумеется, не в названии, --- одним словом, ко всякому существенному органу того или другого объекта. Все, что имеет значение сущности в смысле субъективном, имеет значение сущности и в смысле объективном или предметном. Человек никогда не может освободиться от своей подлинной сущности. Он может представить себе при помощи фантазии существо другого, высшего рода но не может абстрагировать себя от своего рода, от своей сущности; определения сущности, которыми он наделяет этих других индивидов, почерпаются им из своей собственной сущности, и в его определениях отражается и объективизируется он сам. Вероятно, на других планетах нашей солнечной системы есть мыслящие существа помимо человека, но, предполагая это, мы не изменяем своей точки зрения, обогащаем её лишь количественно, а не качественно. Ведь если там действуют те же законы движения, что у нас, то так же обстоит дело с законами чувствования и мышления. Мы не допускаем, чтобы другие планеты были населены иными существами, чем мы; мы полагаем; что там живут ещё другие существа, более или менее подобные нам\ddag\let\svthefootnote\thefootnote\let\thefootnote\relax\footnotetext{\ddagтак, например, Христиан Гюйгенс говорит в своем <<Cosmotheoros>> (lib. \rom{1}): <<Есть основание полагать, что удовольствие от музыки и математики свойственно не только людям, а распространяется на много других существ>>. Это, другими словами, значит: качество тождественно, тот же вкус к музыке, к науке; но число тех, которые их воспринимают, должно быть неограниченным.}\let\thefootnote\svthefootnote.






\chapter{Общая сущность религии}


Все сказанное выше по поводу чувственных объектов, об отношении человека к объекту вообще можно повторить, в частности, и об отношении человека к объекту религиозному.

По отношению к чувственным объектам сознание объекта, конечно, отличается от самосознания, а по отношению к объекту религиозному сознание и самосознание непосредственно совпадают. Чувственный объект находится вне человека, религиозный --- в нём, внутри него. Поэтому религиозный объект, подобно его самосознанию и совести, есть нечто интимное, интимнейшее, наиболее близкое человеку. <<Бог, --- говорит Августин, --- ближе и понятнее нам, чем чувственные, телесные предметы, и потому мы легче познаем его>>*\let\svthefootnote\thefootnote\let\thefootnote\relax\footnotetext{*<<De genesi ad litteram>>, lib. \rom{5}, c. 16.}\let\thefootnote\svthefootnote\footnote{Фейербах цитирует комментарии Августина <<О книге бытия\dots>>, кн. 5, гл. 16 (лат.).}. Чувственный объект сам по себе безразличен, он не зависит ни от настроения, ни от способности суждения, тогда как религиозный объект есть объект избранный, существо исключительное, первое, высшее. Он, естественно, предполагает критическое суждение, разграничение между божественным и небожественным, между достойным и недостойным поклонения\dag\let\svthefootnote\thefootnote\let\thefootnote\relax\footnotetext{\dag<<Вы не понимаете, --- говорит Минуций Феликс в своем ,,Октавиане`` (гл. 24), обращаясь к язычникам, --- что скорее нужно познавать бога, чем поклоняться ему>>.}\let\thefootnote\svthefootnote. Здесь поэтому без всяких ограничений имеет силу следующее положение: объект человека есть не что иное, как его же объективная сущность. Бог человека таков, таковы его мысли и намерения. Ценность бога не превышает ценности человека. Сознание бога есть самосознание человека, познание бога --- самопознание человека. О человеке можно судить по богу и о боге --- по человеку. Они тождественны, божество человека заключается в его духе и сердце, а дух, душа и сердце человека обнаруживаются в его боге. Бог есть откровение внутренней сути человека, выражение его <<Я>>; религия есть торжественное раскрытие тайных сокровищ человека, признание его сокровенных помыслов, открытое исповедание его тайн любви.



Если религия, сознание бога, и характеризуется как самосознание человека, то это ещё не значит, будто религиозный человек непосредственно сознает, что его сознание бога влечет за собой сознание своей сущности, --- ведь отсутствие этого сознания является отличительным признаком религии. Во избежание недоразумения мы лучше скажем: религия есть первое и к тому же косвенное самосознание человека. Поэтому религия всегда предшествует философии не только в истории человечества, но и в истории личности. Прежде чем искать свою сущность в себе, человек полагает её вне себя. Свою собственную сущность он объективирует в качестве другой сущности. Религия --- младенческая сущность человечества; но ребенок свою сущность, человека, рассматривает как нечто постороннее: человек, поскольку он ребенок, объективирует себя в качестве другого человека. Поэтому исторический прогресс религии заключается в том, что всё, казавшееся более ранней религии объективным, теперь кажется субъективным; то, что раньше считалось и почиталось божественным, ныне считается человеческим. Свою собственную сущность он объективирует в качестве другой сущности. Религия --- младенческая сущность человечества; но ребенок свою сущность, человека, рассматривает как нечто постороннее: человек, поскольку он ребенок, объективирует себя в качестве другого человека. Поэтому исторический прогресс религии заключается в том, что все, казавшееся более ранней религии объективным, теперь кажется субъективным; то что раньше считалось и почитается божественным, ныне считается человеческим. Всякая предшествующая религия кажется последующей идолопоклонством; человек раньше поклонялся собственной сущности. Человек себя объективировал, но не усматривал в объекте своей сущности; последующая религия проникается этим сознанием, и потому всякий прогресс в религии есть проявление более глубокого самопознания. Но всякая определенная религия считает своих старших сестер идолопоклонницами, не предполагая, что в будущем её ждет такая же участь; это убеждение неизбежно --- иначе она не была бы религией; она только приписывает другим религиям вину, свойственную религии вообще, если только это вина. Она имеет другой объект, другое содержание, которое выше содержания предшествующей религии, и поэтому мнит себя сильнее неизменных, вечных законов, определяющих сущность религии; она воображает, что её объект, её содержание есть нечто сверхчеловеческое. Но мыслитель прозревает сущность религии, от нее самой скрытую, потому что относится к ней объективно, чего не может сделать сама религия. Поэтому мы должны доказать, что противоположность между божественным и человеческим --- только иллюзия, что она объясняется противоположностью человеческой сущности человеческому индивиду и что, следовательно, объект и содержание христианской религии есть нечто вполне человеческое.

Во всяком случае в христианской религии выражается отношение человека к самому себе, или, вернее, к своей сущности, которую он рассматривает как нечто постороннее. Божественная сущность --- не что иное, как человеческая сущность, очищенная, освобождённая от индивидуальных границ, то есть от действительного, телесного человека, объективированная, то есть рассматриваемая и почитаемая в качестве посторонней, отдельной сущности. Поэтому все определения божественной сущности относятся и к сущности человеческой\ddag\let\svthefootnote\thefootnote\let\thefootnote\relax\footnotetext{\ddag<<Совершенства бога не что иное, как совершенства наших душ, но он обладает ими неограниченно\dots У нас есть известные способности, известное познание, известная доброта, но у бога всё это совершенно>> (Лейбниц, Теодицея, предисловие). <<Всё, что характеризует человеческую душу, свойственно и божественному существу. Всё, что устранено из бога, не составляет также существенных определений души>> (св. Григорий Нисский, De aima. Lips, 1837, p.42). <<Поэтому из всех наук самая ценная и важная --- самопознание, ибо кто познал самого себя, познал и бога>> (Климент Александрийский, Paedag, lib. \rom{3}, c. 1).}\let\thefootnote\svthefootnote\footnote{см.: Лейбниц Г.В. Соч. в 4-х т. Т. 4 М., 1989 г., с. 51.} \footnote{Фейербах цитирует сочинение Григория Нисского <<О душе>> и Климента Александрийского <<Педагог>>, кн. \rom{3}, гл. \rom{1} (лат.).}.



Всё это нисколько не оспаривается в применении к предикатам, то есть к свойствам бога, но никоим образом не считается возможным в применении к субъекту, то есть к основной сущности этих предикатов. Безбожием, атеизмом считается отрицание субъекта, но не его свойств. Но всё, что лишено определений, не производит на меня действия; а что не действует, то и не существует для меня, отрицание определений равносильно отрицанию сущности. Существо, лишенное свойств, не имеет объективного бытия, следовательно, не существует. Если отнять у бога его свойства, то он окажется лишь существом отрицательным, то есть не существующим. Отсутствие определений, или, что то же, непостижимость бога, есть продукт нового времени, плод современного неверия.

Разум может быть признан и признается как нечто ограниченное только там, где человеку представляется безусловным, истинным чувственное наслаждение, религиозное чувство, эстетическое созерцание или нравственные убеждения. Точно как же непостижимость или неопределенность бога выставляется и признается догматом лишь в том случае, если этот предмет перестает возбуждать познавательный интерес и человека начинает занимать только действительность, приобретающая для него значение существенного, абсолютного, божественного объекта и вместе с тем вопреки этим чисто мирским тенденциям у него ещё сохраняются остатки религиозности. И эти пережитки религиозной совести заставляют человека извинять свое безбожие, свою суетность непостижимостью бога; он отрицает бога практически своими поступками, потому что мирская суета поглощает все его чувства и мысли, но не отрицает его теоретически, --- он не посягает на существование бога и признает его бытие. Такое существование не смущает и не стесняет его, это --- только отрицательное существование, бытие без бытия, противоречащее самому себе, это --- бытие, которое по своему, действию ничем не отличается от небытия. Отрицание определенных положительных свойств божественного существа есть не что иное, как отрицание религии, сохраняющее вид религии и поэтому не считающееся атеизмом, но в действительности --- утонченный, лукавый атеизм. Якобы религиозная боязнь приписать богу определенные свойства и чем сделать его конечным объясняется атеистическим желанием забыть о боге, выкинуть его из головы. Кто боится быть конечным, тот боится существовать. Всё реально существующее, то есть всякое подлинное бытие, определяется качественно. Кто серьезно, действительно, истинно верит в существование бога, того не коробят даже его грубо чувственные свойства. Кто не хочет быть грубым, не хочет, чтобы его существование кого-нибудь оскорбляло, тот должен отказаться от существования. Бог, полагающий, что определенность оскорбляет его, не имеет достаточно мужества и силы, чтобы существовать. Качество --- это огонь, кислород, соль бытия. Бытие вообще, без определенного качества, есть безвкусица, нелепость. В боге не больше содержания, чем в религии; религия и самое существование бога становятся безвкусными, если человек теряет вкус к религии.

Помимо этого прямого способа отрицания божественных свойств есть другой, более утонченный. Признается, что свойства бога конечны, в частности, что они имеют человеческий характер, но отрицается их отрицание; их даже защищают, потому что человеку необходимо иметь определенное представление о боге, и в то же время он как человек не может иметь о боге иного представления, кроме человеческого. По отношению к богу, говорят такие люди, эти определения, разумеется, не имеют значения, но все же мне, поскольку я заинтересован в боге, нельзя представить себе его иначе, как в качестве человеческого или, по крайней мере, человекоподобного существа.

Но это различение между богом самим по себе и богом для меня разрушает мирное существование религии, к тому же такое различение неосновательно и шатко. Откуда мне знать, есть ли бог по себе или для себя нечто иное, чем для меня; в том виде, каков он для меня, он для меня --- все. По-моему, те свойства, которые я ему приписываю, и составляют его сущность; он для меня такой, каким он только и может быть для меня. Религиозный человек вполне довольствуется тем представлением, какое он имеет о боге, и не допускает другого представления, потому что бог кажется ему тем, чем он вообще может казаться человеку. Делая различение, о котором упоминалось выше, человек отрешается от самого себя, от своей сущности, от своего абсолютного мерила, но это отрешение не больше, как иллюзия. Различие между объектом действительным и объектом, существующим для меня, возможно только тогда, когда объект может мне действительно казаться иным, чем он кажется, а не тогда, когда мое представление о нем отвечает моему абсолютному мерилу и не может быть другим. Мое представление может быть субъективным, то есть не иметь ничего общего с моим родом. Но если мое представление соответствует мерилу рода и, следовательно, абсолютно, то различие между бытием в себе и бытием для меня отпадает. Мерило рода есть абсолютное мерило, закон и критерий человека. Религия убеждена, что каждый человек, желающий иметь истинное представление о боге, должен и обязан разделять её взгляд на свойства божии, что они --- необходимые представления человеческой природы, представления объективные, соответствующие богу. Каждая религия считает чужих богов лишь представлениями о боге и только свое собственное представление о боге --- настоящим, истинным, богом, каков он сам в себе. Религия удовлетворяется лишь цельным, безусловным богом; ей нужны не просто проявления божества, а сам бог, личный бог. Отрицая сущность бога, религия отрицает самое себя; она перестает быть истиной, как только отрекается от обладания истинным богом. Скептицизм --- заклятый враг религии. А различение между предметом и представлением, между богом самим по себе и богом для меня есть различение скептическое, следовательно безбожное.

Бог есть существо, которое кажется человеку самодовлеющим, высшим. Человек не может представить себе ничего выше бога и, следовательно, не может задаваться вопросом что такое бог сам по себе. Если бы бог был объектом для птицы, он казался бы ей существом пернатым; для птицы нет большего счастья, чем обладать крыльями. Смешно было бы, если бы птица рассуждала так: <<Я представляю себе бога в виде птицы, но не знаю, чем он является в действительности>>. Птица кажется ей высшим существом. Если ты у нее отнимешь представление о сущности птицы, ты отнимешь у нее представление о высшем существе. Итак птица не сомневалась бы в том что бог есть существо пернатое. Вопрос, отвечает ли бог моему представлению о нем, равносилен вопросу, действительно ли бог есть бог. Предлагать такие вопросы может только тот, кто хочет быть выше своего бога и восстает против него.

Когда человек приходит к сознанию, что божественные свойства суть только антропоморфизмы, то есть человеческие представления, у него закрадывается сомнение, неверие. Если это сознание не влечет за собой полного отрицания свойств, а затем и субъекта, которому их приписывают, то это объясняется лишь малодушием и слабостью мышления. Сомневаясь в объективной истинности этих свойств, ты не можешь не сомневаться в объективной истинности субъекта, которого ими наделяют. Если эти свойства антропоморфизмы, то и субъект их также антропоморфизм. Если любовь, доброта, индивидуальность суть определения человеческие, то и основное существо, обладающее ими, самое существование бога и вера в его бытие– всё это антропоморфизмы, исключительно человеческие предположения. Не есть ли вообще вера в бога предел человеческого представления? Может быть, высшие существа, в которых ты веришь, так счастливы сами по себе, как согласны с собой, что между ними и высшим существом нет разлада? Познать бога и не быть богом, знать о блаженстве и не наслаждаться им --- это разлад с самим собой, это несчастье*\let\svthefootnote\thefootnote\let\thefootnote\relax\footnotetext{*поэтому в потустороннем мире уничтожается этот разлад между богом и человеком. В потустороннем мире человек перестает быть человеком, оставаясь им разве лишь в своем представлении, --- он теряет свою собственную волю, отличную от божественной, а следовательно, --- ибо чем является существо без воли? --- и свою собственную сущность; он сливается с богом. Таким образом, в потустороннем мире исчезают различие и противоположность между богом и человеком. Но там, где есть только бог, нет никакого бога. Где нет противоположности величий, нет и самого величия.}\let\thefootnote\svthefootnote. Высшие существа незнакомы с этим несчастьем, они не представляют себе того, чем они сами не являются.



Ты приписываешь богу любовь, потому что любишь сам, ты находишь бога мудрым и благим, потому что считаешь доброту и рассудок своими высшими качествами, ты веришь в то, что бог существует, что он субъект или существо, потому что сам ты существуешь, сам ты --- существо, --- ведь все, что существует, есть существо, независимо от того, рассматривается ли оно как субстанция, личность или как-нибудь иначе. Ты считаешь любовь, доброту и мудрость высшим человеческим состоянием, а бытие --- высшим счастьем; сознание всякой действительности, всякого счастья связано в тебе с сознанием сущностного бытия, существования. Для тебя бог существует, он --- существо, по тому же самому, почему ты считаешь его мудрым, блаженным, добрым. Различие между божественными свойствами и божественным существом заключается в том, что существо, существование не кажутся тебе антропоморфизмом, потому что в том, что ты --- существо, заключена необходимость бога, как существующего, как существа, а свойства кажутся тебе антропоморфизмами, потому что они (бог мудр, благ, справедлив и так далее) являются не непосредственной, тождественной с человеческим бытием необходимостью, а лишь необходимостью, обусловленной самосознанием, деятельностью мышления. Я --- субъект, я --- существо, я существую независимо от того, умен я или глуп, хорош или дурён. Существование для человека важнее всего, субъект имеет в его представлении большее значение, чем предикаты. Поэтому отрицание предикатов он допускает, а существование бога кажется ему непреложной, неприкосновенной, абсолютно верной, объективной истиной. И тем не менее это различие --- только кажущееся. Необходимость субъекта заключается в необходимости его предикатов. Ты являешься существом только в качестве определенного человеческого существа, подлинность и реальность твоего существования заключаются в подлинности и реальности твоих человеческих качеств. Сущность субъекта выражается в его предикатах, предикаты --- это истинность субъекта; разница между субъектом и предикатом та же, что между существованием и существом. Отрицание предикатов есть поэтому отрицание субъекта. Что останется от человеческого существа, если отнять у него человеческие свойства? Даже на обыденном языке название божественного существа часто заменяется названиями его свойств: провидение, мудрость, всемогущество.

Уверенность в бытии бога, которое, по свидетельству многих, кажется человеку более достоверным, чем его собственное существование, не есть нечто непосредственное: она обусловливается уверенностью в качествах бога. Христианин уверен в существовании только христианского, язычник --- в существовании только языческого бога. Язычник не сомневался в существовании Юпитера, потому что существо Юпитера его не смущало, он не мог представить себе бога обладающим иными качествами и считал свойства Юпитера божественно– реальными. Реальность предиката есть единственный залог существования.

Человек считает действительным то, что кажется ему истинным, потому что сначала ему кажется истинным (в противоположность вымышленному, грезам, воображаемому) только то, что действительно. Понятие бытия, существования, есть первое, первичное понятие истины. Иначе говоря, сначала человек думает, что истина обусловливается существованием, затем, что существование обусловливается истиной. Бог есть сущность человека, рассматриваемая в качестве высшей истины --- истины человеческой. Но бог или, что то же, религия настолько же различны, насколько различны представления человека о своей сущности, как о высшем существе. Поэтому человеческие определения бога --- для него истина, а в связи с этим --- высшее бытие или, вернее, вообще бытие, потому что только высшее бытие есть собственно бытие и достойно этого названия. Следовательно, бог на том же основании есть сущее, действительное существо, по которому он оказывается данным определенным существом, потому что свойство и определение бога есть существенное свойство человека. Но каждый человек представляет собой нечто, себе довлеющее; в его качествах заключается его существование, его реальность. Грека нельзя лишить его свойств грека, не лишая его существования. Поэтому уверенность в бытии бога есть непосредственная уверенность каждой религии; насколько непроизвольно и неизбежно грек был греком, настолько же неизбежно его боги должны были быть греческими существами и действительно существовать. Религия есть тождественное с сущностью человека воззрение на сущность мира и человека. Но не человек возвышается над своим воззрением, а оно возвышается над ним, одухотворяет и определяет его, господствует над ним. Необходимость доказательства, связи сущности, или качества, с существованием, возможность сомнения отпадают сами собой. Я могу сомневаться только в том, что я отделяю от своей сущности. А бог --- моя собственная сущность, и поэтому я не могу сомневаться в нём. Сомневаться в существовании своего бога --- значит сомневаться в себе самом. Только тогда, когда бог рассматривается как нечто отвлечённое и его предикаты подвергаются философской абстракции, возникает различие или разграничение между субъектом и его предикатом, возникает предположение, будто субъект есть нечто отличное от предиката, нечто непосредственное, несомненное, в противоположность сомнительному предикату. Но это только кажущееся различие. Богу, обладающему отвлеченными предикатами, свойственно и отвлеченное существование. Существование, бытие, так же различно, как и качество.

Тождественность субъекта и предиката обнаруживается особенно ясно по ходу развития религии, которое идет рука об руку с развитием человеческой культуры. Поскольку самому человеку свойственно первобытное состояние, постольку его божество также носит характер первобытный. Как только человек начинает строить жилища, он сооружает храм для своего бога. Сооружение храма свидетельствует о том, что человек ценит красивые здания. Храм в честь бога является в сущности храмом в честь архитектуры. По мере того как человек выходит из первобытного, дикого состояния и становится культурным, он начинает различать, что подобает и не подобает человеку и что приличествует и не приличествует богу. Бог есть олицетворение величия, высшего достоинства; религиозное чувство --- высшее чувство благопристойности. Только позднейшие образованные художники Греции начали воплощать в статуях богов понятия достоинства, величия души, невозмутимого спокойствия и бодрости. Но почему они считали эти свойства атрибутами, предикатами бога? Потому что каждое из этих свойств казалось им божеством само по себе. Почему они не увековечивали позорных, низких страстей? Потому что они смотрели на них, как на нечто неприличное, недостойное, нечеловеческое и, следовательно, небожественное. Боги Гомера едят и пьют; это значит, что еда и питье --- наслаждение богов. Боги Гомера обладают физической силой: --- Зевс сильнейший из богов. Почему? Потому что физическая сила сама по себе есть нечто прекрасное, божественное. Древние германцы считали высшей добродетелью добродетель воина, поэтому их главным богом был бог войны; Один --- значит война, --- значит <<основной или древнейший закон>>. Первой, истинной, божественной сущностью является не свойство божества, а богоподобие или божественность свойства. Поэтому то, что теологией и философией признавалось богом, абсолютом, сущностью, не есть бог. Бог --- это именно то, что они не считали богом, то есть свойство, качество, определенность, действительность вообще. Подлинным атеистом, то есть атеистом в обычном смысле, надо считать не того, для кого божественный субъект --- ничто, а того, кто отрицает божественные предикаты, как то: любовь, мудрость, справедливость. Отрицание субъекта не есть отрицание предикатов самих по себе. Они имеют собственное, самостоятельное значение: человек необходимо должен признавать их в силу их содержания, их подлинность заключается непосредственно в них самих; они действенно обнаруживают себя. Призрачность или подлинность доброты, справедливости, мудрости не зависит от призрачности или подлинности существования божия. Понятие бога обусловливается понятием справедливости, благости, мудрости; бог не благой, не справедливый, не мудрый не есть бог, а не наоборот. Качество божественно не потому, что оно свойственно богу, а, напротив, оно свойственно богу, потому что божественно само по себе, потому что без него бог был бы существом несовершенным.

Справедливость, мудрость и вообще всякое свойство, которое приписывается богу, определяется самим собой, а бог характеризуется лишь определением, качеством. Я определяю бога самим богом только в том случае, если я отождествляю бога и справедливость, представляю себе бога в качестве реального воплощения идеи справедливости или другого какого-либо качества. Но если бог как субъект является определяемым, а свойство, предикат --- определяющим, то на самом деле подобает не субъект, а предикат называть первичным существом.

Лишь в том случае, если несколько противоречивых свойств соединяются в одном существе и это существо понимается как личность, и тем самым личность выдвигается на первое место, только тогда мы можем забыть о происхождении религии и о том, что предикат, который мы рассматриваем теперь как нечто отличное, отдельное от субъекта был первоначально подлинным субъектом. Так, римляне и греки обожествляли акциденции в качестве субстанций, добродетели, душевные волнения и страсти в качестве самостоятельных существ. Человек, особенно религиозный человек, есть мера всех вещей, всякой реальности. Он возводит на степень божества все, что ему импонирует, все, что производит особенное впечатление на его душу, будь то даже странный, необъяснимый звук. Религия охватывает все предметы в мире; все существующее было предметом религиозного почитания. Сущность и сознание религии исчерпывается тем, что заключается в сущности человека, его сознании и самосознании. У религии нет собственного, особого содержания. Римляне сооружали храмы даже в честь таких эффектов, как страх и ужас. Христиане тоже превращали душевные явления в существа, чувства --- в качества вещей, и свои аффекты в страсти, господствующие над миром, одним словом, превращали свойства своей собственной, известной или неизвестной, сущности в самостоятельные существа. Черти, домовые, ведьмы, привидения, ангелы были священной истиной до тех пор, пока религиозное чувство целиком, нераздельно владело человечеством.

Не желая признавать тождественность божественных и человеческих предикатов и вместе с тем тождественность божественной и человеческой сущности, люди представляют себе, что бог в качестве абсолютно реального существа обладает бесконечным множеством различных предикатов, причем некоторые из них, аналогичные или подобные нам, мы познаем в настоящей, а остальные, совершенно отличающие бога от человека, --- только в будущей жизни, то есть на том свете. Но бесконечная полнота или множество предикатов, которые действительно отличаются друг от друга настолько, что по одному из них нельзя непосредственно судить о других, такое бесконечное множество предикатов реализуется и проявляется только в бесконечной полноте и множестве различных существ или индивидов. Поэтому человеческая сущность, бесконечно богатая различными предикатами, в связи с этим изобилует и разнообразными индивидами. Каждый новый человек есть новый предикат, новый талант человечества. Количество сил и свойств человечества равняется количеству людей. Каждый отдельный человек обладает силой, присущей всем, но она определяется и складывается в нём таким образом, что кажется особенной, новой силой. Тайна неисчерпаемого источника божественных свойств есть не что иное, как тайна человеческой, бесконечно разнообразной, бесконечно определяемой и именно в силу этого чувственной сущности. Только в чувственности, только в пространстве и времени может получить место бесконечная, действительно бесконечная и изобилующая определениями сущность. Различие свойств обусловливается различием времени. Положим, что какой-нибудь человек --- прекрасный музыкант, прекрасный писатель, прекрасный врач, но он не может одновременно заниматься музыкой, писательством и леченьем. Не гегелевская диалектика, а время является средством объединения в одном и том же существе нескольких противоречий. Но представление о боге, как о существе, обладающем бесконечным множеством предикатов, отличных и отмежёванных от сущности человека, является просто-напросто вымыслом, чувственным представлением, лишенным необходимых условий, лишенным истинности чувственности, и прямо противоречащим божеству как духовной, то есть отвлеченной, простой, единственной, сущности; отличительная черта свойств божьих заключается именно в том, что каждое свойство дает мне возможность судить о других, потому что между ними нет никакого реального различия. Поэтому если настоящие предикаты отличаются от будущих и настоящий бог отличается от будущего, то, следовательно, настоящий и будущий бог не одно и то же, а два различных существа \dag\let\svthefootnote\thefootnote\let\thefootnote\relax\footnotetext{\dagдля религиозной веры между настоящим и будущим богом существует лишь то различие, что первый является объектом веры, представления, фантазии, а второй --- объектом непосредственного, то есть личного, чувственного восприятия. И здесь и там он один и тот же, но здесь он неотчетлив, а там ясен.}\let\thefootnote\svthefootnote. Но это различие противоречит тому, что бог --- единственное, единое и простое существо. Почему это свойство есть свойство божие? Потому что оно божественной природы, то есть выражает нечто неограниченное, совершенное. Почему те или другие свойства божественны? Потому что они, несмотря на кажущееся различие, имеют одну общую черту: все они выражают совершенство, неограниченность. Таким образом, я могу представить себе бесконечное множество божественных свойств, потому что все они совпадают в отвлеченном представлении о боге и общей их чертой является то, что делает божественным каждое отдельное свойство. То же находит и Спиноза. Он говорит о бесконечном множестве атрибутов божественной субстанции, но не называет ни одного, кроме мышления и протяженности. Почему? Потому что нам неважно знать эти свойства, так как они безразличны и излишни сами по себе и не могут ничего прибавить к тому, что уже выражено двумя понятиями: мышление и протяжение. Почему мышление --- атрибут субстанции? Потому что оно, согласно Спинозе, постигается само собой и выражает нечто неделимое, совершенное, бесконечное. Почему субстанции свойственно протяжение, материя? В силу тех же самых соображений. Следовательно, субстанция может иметь неопределенное количество предикатов, потому что эти свойства становятся атрибутами субстанции не в силу определенности, различия, а в силу неразличимости, равенства. Вернее: субстанция обладает бесчисленными свойствами только потому, что она, как это ни странно, не имеет в действительности ни одного реального, определенного свойства. Неопределенное единое мысли дополняется неопределенной множественностью фантазии. Если предикат не multum\footnote{множество, большое количество (лат.).}, то он multa\footnote{очень много, самое большое количество (лат.).}. В действительности есть только два положительных предиката: мышление и протяжение. Эти два понятия выражают собой гораздо больше, чем бесчисленное множество безымянных предикатов; они представляют собой нечто определенное, они дают мне понятие о чем-то. Но субстанция слишком безразлична и бесстрастна, чтобы она могла воодушевиться чем-нибудь. Не желая быть чем-нибудь, она предпочитает быть ничем.



Мы доказали, что субъект, или сущность, сводится к его определениям, то есть что предикат есть истинный субъект; поэтому если божественные предикаты суть человеческие свойства, то и субъект, обладающий ими, также человеческого происхождения. Божественные предикаты разделяются на общие и личные. К общим свойствам относятся метафизические, которые служат только внешней связью религии и не сообщают ей определенного характера. Только личные свойства составляют сущность религии и характеризуют существо божие как объект религии. К таким предикатам относятся, например, следующие: бог есть личность, нравственный законодатель, отец людей, святой, справедливый, благой, милосердный. Но из этих и других свойств очевидно, что они в качестве личных свойств являются чисто человеческими свойствами. Следовательно, в отношении религиозного человека к богу выражается его отношение к своей собственной сущности. Религия считает свойства божьи реальной истиной, а не представлением, не образом, который человек составляет о боге, независимо от того, что есть бог сам в себе. Религия не признает антропоморфизмов, для нее антропоморфизмы не антропоморфизмы. Сущность религии такова, что эти определения служат для нее выражением существа божья. Только рассудок, размышляющий над религией, её одновременно защищающий и отрицающий, приписывает свойствам божьим значение образов. Но в религии бог есть действительный отец, действительная любовь и милосердие, потому что она считает его действительным, живым, личным существом, а его свойства --- истинными, живыми, личными свойствами. Главными свойствами религия считает именно те, которые кажутся наиболее сомнительными разуму, отвергающему их путем размышления над религией. Религия субъективно есть аффект, поэтому для нее и объективно аффект --- божественного свойства. Даже гнев не кажется ей недостойным бога, если только в основе этого гнева не лежит ничего злого.

Здесь необходимо упомянуть об одном достопримечательном явлении, которое характеризует внутреннюю сущность религии. Чем бог человечнее по существу, тем большим кажется различие между богом и человеком, то есть тем настойчивее опровергается путем религиозной рефлексии и богословия единство божественной и человеческой сущности, тем больше унижается достоинство человеческих свойств, которые служат, как таковые, объектом человеческого сознания\ddag\let\svthefootnote\thefootnote\let\thefootnote\relax\footnotetext{\ddag<<Сколь бы большим мы ни мыслили сходство между творцом и тварью, несходство между ними мы должны мыслить ещё большим>>. Лютеранский собор, кан. 2 (Summa omn. Conc. Carranza, Antv. 1559, p. 526). Высшее различение между человеком и богом, между конечным существом и существом бесконечным, до которого способно подняться религиозное умозрение, это --- различение между нечто и ничто, между ens и non-ens; ибо лишь в ничто уничтожается всякая общность со всеми другими существами.}\let\thefootnote\svthefootnote\footnote{сущее, не-сущее (лат.).}. Это объясняется тем, что положительными свойствами в созерцании или определении божественной сущности являются только человеческие свойства, вследствие чего и взгляд на человека, как на объект сознания, может быть только отрицательным, человеконенавистническим. Чтобы обогатить бога, надо разорить человека; чтобы бог был всем, человек должен сделаться ничем. Но и он может быть ничем для себя, раз все, что он от себя отделяет, не теряется в боге, а сохраняется в нём. Сущность человека заключается в его боге, каким же образом он может иметь её в себе и для себя? Зачем нужно одно и то же дважды полагать и дважды иметь? Все, что человек отнимает у себя, чего он лишается, служит для него несравненно более высоким и обильным источником наслаждения в боге.



Монахи давали богу обет целомудрия, они подавляли в себе половую любовь, но зато признавали небесную, божественную любовь в образе девы Марии. Они могли обходиться без настоящей женщины, потому что предметом их истинной любви была идеальная, воображаемая дева. Чем больше значения они придавали отрицанию чувственности, тем больше значения приобретала в их глазах небесная дева: она заменяла им Христа и бога. Чем больше человек отрицает чувственность; тем чувственнее становится его бог, в жертву которому приносится эта чувственность. Тому, что мы жертвуем богу, придается особенная ценность, то и признается особенно угодным богу. Что имеет высокую цену в глазах человека, имеет такую же жену и в глазах бога; вообще что нравится человеку, нравится и богу. Евреи приносили в жертву Иегове не нечистых и скверных животных, а таких, которые казались им наиболее ценными. Пища людей служила пищей и богу*\let\svthefootnote\thefootnote\let\thefootnote\relax\footnotetext{*Cibus Dei. 3. Mose. 3. 11.}\let\thefootnote\svthefootnote\footnote{Пища бога (лат.). Фейербах ссылается на Третью книгу Моисееву (Левит), гл. 3, 11.}. Поэтому там, где человек из отрицания чувственности создает особую сущность, богоугодную жертву, там чувственности приписывается величайшая ценность, там торжество чувственности выражается в том, что сам бог заступает место чувственного предмета, который принесли в жертву. Монахиня обручается с богом, она обретает небесного жениха, подобно тому как монах --- небесную невесту. Но небесная дева служит только чувственным проявлением общей истины, касающейся сущности религии. Человек приписывает богу то, что он отрицает в себе\dag\let\svthefootnote\thefootnote\let\thefootnote\relax\footnotetext{\dag<<Кто презирает себя, --- говорит, например, Ансельм, --- тот в чести у бога; кто себе не мил, тот мил богу. Будь поэтому ничтожен в твоих собственных глазах, чтобы быть великим в глазах господа; тем больше будет тебя ценить бог, чем презреннее ты будешь во мнении людей>>. (Anselmi, Opp. Paris, 1721, p. 191).}\let\thefootnote\svthefootnote. Религия отвлекается от человека и от мира, но она может абстрагировать только от действительных или воображаемых, недостаточных или ограниченных, ничтожных явлений, а не от сущности и не от положительных свойств мира и человечества. Поэтому в её абстракции и отрицании снова проявляется то, от чего она абстрагирует или предполагает абстрагировать. Таким образом, религия снова бессознательно приписывает богу все то, что она сознательно отрицает, разумеется, в том случае, если она отрицает что-нибудь существенное, истинное, чего поэтому нельзя отрицать. В религии человек отрицает свой разум: из себя он ничего не знает о боге, его мысли носят светский, земной характер: он может только верить в божественное откровение. Но зато богу свойственны земные, человеческие помыслы; он строит планы, подобно человеку, приспособляется к обстоятельствам и умственным способностям людей, как учитель к своим ученикам, точно рассчитывает эффект своих благодеяний и откровений; наблюдает за всеми действиями и поступками человека и знает все, даже самое земное, самое пошлое, самое дурное. Одним словом, человек отрицает ради божества свое знание, свое мышление, но зато приписывает это знание, это мышление богу. Человек отрекается от своей личности и вместо этого считает личным существом всемогущего, неограниченного бога; он отказывается от человеческого достоинства, от человеческого я и в то же время бог кажется ему себялюбивым, эгоистичным существом, которое ищет во всем личного удовлетворения, личных почестей, личной выгоды. Отсюда вытекает самоудовлетворение бога, враждебное всему остальному, и его самонаслаждение эгоизмом\ddag\let\svthefootnote\thefootnote\let\thefootnote\relax\footnotetext{\ddag<<Бог может любить только себя, думать только о себе, работать только для себя. Создавая человека, бог ищет своей пользы, своей славы>> и т. д. См. P. Bayle, Ein Beitrag zur Geschichte der Phil. u. Menschheit, 1. Aufl., S. 104; 2. Aufl. (Werke, 6 Bd.), S. 131.}\let\thefootnote\svthefootnote. Религия отнимает, далее, у человека все хорошие качества: человек зол, испорчен, неспособен творить добро, но зато бог только благ; бог --- благое существо. Существенное требование религии заключается в том, чтобы объектом человека были хорошие качества в лице бога; но разве этим самым не признается добро в качестве главного свойства человека? Если я зол, грешен абсолютно, то есть по природе, по существу, то разве может быть моим объектом святость, доброта, независимо от того, дается ли мне этот объект извне или изнутри? Если у меня злое сердце, испорченный ум, как я могу воспринимать и считать святое святым и хорошее хорошим? Как я могу признавать достоинства хорошей картины, если душа моя эстетически извращена? Пусть я не художник и не обладаю способностью создавать прекрасные произведения, я все-таки могу воспринимать красоту извне, если у меня есть эстетический вкус и понимание. Или хорошие качества вовсе не существуют для человека, или они понятны ему, и в таком случае в них обнаруживаются для человека святость и достоинства человеческого существа. То, что абсолютно противоречит моей природе, с чем меня не связывают узы родства, я не могу ни представить себе, ни чувствовать. Святость является для меня объектом как противоположность моей личности, представляющая единство с моей сущностью. Святость --- это упрек моей греховности, благодаря ей я признаю себя грешником, порицаю себя, постигаю, чем я должен и могу быть сообразно своему назначению, --- ведь долженствовать и не мочь --- это смешная химера, не способная привлечь к себе мой дух. Но, признавая добро своим назначением, своим законом, я тем самым признаю его сознательно или бессознательно своей собственной сущностью. Посторонняя, чуждая мне сущность не трогает меня. Грех кажется мне грехом только в том случае, если я чувствую в нём противоречие между моей личностью и моей сущностью. Чувство греха, основанное на противоречии с божественной, как посторонней сущностью, необъяснимо и бессмысленно.







Различие между Августином и Пелагием заключается только в том, что первый высказывает языком религии то, что второй излагает рациональным способом. Оба говорили одно и то же, оба приписывали человеку добро: Пелагий --– прямо, рациональным, моральным путем, а Августин --- косвенно, мистическим, религиозным способом*\let\svthefootnote\thefootnote\let\thefootnote\relax\footnotetext{*пелагианство отрицает бога, религию, <<пелагиане приписывают воле такое могущество, что умаляют силу небесной молитвы>> (Августин, De nat. et grat. cont. Pelagium, c. 58); пелагианство опирается лишь на творца, то есть на природу, а не на спасителя, который только и является религиозным богом, --- одним словом, оно отрицает бога, но зато поднимает человека на степень божества, делая его не нуждающимся в боге, самодовлеющим и независимым существом (см. об этом Лютер, Против Эразма и Августина, loc. cit., c.33).}\let\thefootnote\svthefootnote\dag\let\svthefootnote\thefootnote\let\thefootnote\relax\footnotetext{\dagавгустинианство отрицает человека, но зато оно снижает бога на степень человека вплоть до позора крестной смерти ради человека, пелагианство ставит человека на место бога, августинианство --- бога на место человека. Оба приходят к тем же результатам; различие между ними лишь мнимое, оно не больше, чем благочестивая иллюзия. Августинианство есть пелагианство навыворот; то, что первое полагает как субъект, второе полагает как объект.}\let\thefootnote\svthefootnote. Все, что приписывается богу человеком, приписывается в действительности самому человеку; все, что человек говорит о боге, он на самом деле говорит о самом себе. Августинианство только в том случае было бы истиной, и притом истиной, противоположной пелагианству, если бы человек сознательно считал своим богом дьявола и поклонялся ему, почитая его в качестве высшего существа, притом сознавая, что он дьявол. Но пока человек поклоняется в лице бога существу достойному, в боге будет отражаться его собственная благая сущность.

Учение об основной испорченности человека тождественно учению о том, что человек не может достичь ничего хорошего самостоятельно, одними собственными силами. Отрицание человеческой способности и деятельности было бы истинным только в том случае, если бы человек отрицал и в боге моральную деятельность и говорил, подобно восточному нигилисту или пантеисту, что бог есть существо абсолютно безвольное, бездеятельное и равнодушное, не знающее различия между добром и злом. Но кто представляет себе бога в качестве деятельного, и притом нравственно деятельного, морально критического существа, которое любит, вознаграждает, творит добро и которое наказывает, отвергает, проклинает зло, кто как определяет бога, тот отрицает человеческую деятельность только кажущимся образом, а на самом деле возводит её на степень высшей, реальнейшей деятельности. Кто приписывает богу человеческий образ действий, тот считает человеческую деятельность божественной; тот говорит: бог, недеятельный в смысле моральном или человеческом, не есть бог, и, таким образом, связывает понятие бога с понятием деятельности, и именно человеческой, потому что высшей деятельности он не знает.

Человек --- и в этом заключается тайна религии, --- объективирует свою сущность\ddag\let\svthefootnote\thefootnote\let\thefootnote\relax\footnotetext{\ddagрелигиозное, первичное самообъективирование человека нужно, впрочем, --- это ясно высказано в этой книге --- отличать от самообъективирования рефлексии и умозрения. Последнее произвольно, тогда как первое непроизвольно, необходимо, --- необходимо в такой мере, как искусство, как язык. Правда, с течением времени богословие и религия будут всегда совпадать.}\let\thefootnote\svthefootnote и сделает себя предметом этой объективированной сущности, превратившейся в субъект, и личность, он относится к себе как к объекту, но как к объекту другого объекта, другого существа. Так и здесь. Человек есть объект бога. Богу не безразлично, хорош ли человек или дурен. Религиозное, первичное самообъективирование человека нужно, впрочем, --- это ясно высказано в этой книге, --- отличать от самообъективирования рефлексии и умозрения. Последнее произвольно, тогда как первое непроизвольно, необходимо, --- необходимо в такой же мере, как искусство, как язык. Правда, с течением времени богословие и религия будут всегда совпадать. Нет, он живо, искренне заботится о том, чтобы человек был добрым; он хочет, чтобы человек был добрым и блаженным существом, потому что без добра нет блаженства. Религиозный человек отменяет ничтожество человеческой деятельности тем, что делает свои мысли и поступки объектом бога, человека --- целью бога объект в смысле духа есть цель в смысле практическом) и божественную деятельность --- средством человеческого спасения. Деятельность бога направлена к тому, чтобы человек был добр и блаженствовал. Таким образом, кажущееся тяжкое унижение человека является в сущности его величайшим превознесением. Человек в боге и чрез него ищет только самого себя. Разумеется, человек прежде всего стремится к богу, но сам бог стремится исключительно к нравственному и вечному спасению человека, и, таким образом, целью человека является он сам. Божественная деятельность не отличается от деятельности человеческой.



Я не могу быть объектом божественной деятельности, если она существенно отличается от человеческой, я не могу себе ставить человеческие цели, цели исправления и счастья человеческого, если божественная деятельность не есть деятельность человеческая. Разве цель не определяет поступков? Если человек стремится к нравственному усовершенствованию, то он питает божественные намерения и планы, если же бог стремится к человеческому спасению, значит он задается человеческой целью и добивается этой цели посредством соответствующей человеческой деятельности. Таким образом, объектом человека в лице бога является только его собственная деятельность. Но так как он рассматривает эту деятельность объективно, как нечто от себя отличное, и относится к добру, как к объекту, то все его импульсы и побуждения исходят не от него самого, а от этого объекта. Он рассматривает свою сущность в качестве добра, как нечто внешнее; поэтому само собой понятно, --- так как это лишь тавтология, --- что импульс к добру исходит для него лишь оттуда, куда перенес он добро.

Бог есть обособленная, выделенная, субъективная, своеобразная сущность человека, поэтому все добрые поступки человека исходят не от него, а от бога. Чем субъективнее, чем человечнее бог, тем более отказывается человек от своей субъективности, от своей человечности, потому что бог --- это отделенное от человека его Я, которое он затем снова присваивает себе. Как деятельность артерий гонит кровь к отдаленнейшим конечностям, как деятельность вен гонит её обратно, как жизнь состоит в беспрерывной смене систолы и диастолы, так и в религии. В религиозной систоле человек выделяет из себя свою собственную сущность, отгоняя и отрицая самого себя, в религиозной диастоле он снова принимает эту отторгнутую сущность в свое сердце. Бог есть существо, действующее и деятельное по собственному побуждению --- вот акт религиозной отталкивающей силы; бог --- существо, действующее во мне, со мной, через меня, на меня, для меня, он принцип моего спасения, моих нравственных помыслов и действий и, следовательно, мой собственный нравственный принцип и сущность --- вот акт религиозной силы притяжения.

Следовательно, намеченный выше общий путь развития религии заключается, собственно, в том, что человек все более отнимает у бога и приписывает себе. Сначала человек отделяет от себя все без различия. Это обнаруживается особенно ясно в вере в откровение. То, что последующие века, что культурный народ приписывает природе или разуму, в прежние времена, первобытным народом приписывалось исключительно богу. Израильтяне облекали в форму определенной божественной заповеди самые естественные стремления человека, даже, например, стремление к чистоте. При этом из данного примера опять-таки видно, что достоинство бога умаляется по мере того, как человек отрекается от себя. Смирение, самоотречение человека достигает крайних пределов, если он перестает считать себя способным исполнять по собственному побуждению требования простого приличия*\let\svthefootnote\thefootnote\let\thefootnote\relax\footnotetext{*\rom{5} книга Моисея, 23, 12, 13.}\let\thefootnote\svthefootnote\footnote{имеется в виду книга Второзаконие.}. Христианская религия, напротив, различала побуждения и аффекты людей по их свойствам и содержанию. Она приписывала божественному откровению и влиянию только добрые аффекты, нравственные мысли и чувства, которые она считала чувствами, аффектами и мыслями бога. Ибо божественное откровение определяется том, что предназначено самим богом; от избытка сердца уста глаголют; действие соответствует причине, откровение --- существу, которое открывает себя. Если бог обнаруживается только в нравственных побуждениях, то, его существенным качеством является нравственное достоинство. Христианская религия отличает внутреннюю чистоту от внешней, физической; иудейская религия отождествляла их\dag\let\svthefootnote\thefootnote\let\thefootnote\relax\footnotetext{\dagСм., например, \rom{1} кн. Моисея, 35, 2; \rom{3} кн. Моисея, 11, 44; 20, 25.}\let\thefootnote\svthefootnote\footnote{имеется в виду книга Бытие и Левит.}. Христианская религия в противоположность иудейской есть религия критики и свободы. Израильтянин делал только то что предписывал бог; он был безволен даже в чисто внешних вопросах, авторитет религии простирался даже на его пищу. Христианская религия, напротив, в вопросах внешних предоставляла человека самому себе, то есть вменяла человеку то, что израильтянин отделял от человека и вменял богу. Израиль превосходно характеризует позитивизм. В сравнении с израильтянином христианин есть свободомыслящий. Такова изменчивость вещей. То, что вчера было религией, сегодня перестает быть ею; то, что сегодня кажется атеизмом, завтра станет религией. 





\part{Истинная, то есть антропологическая сущность религии}





\chapter{Бог как сущность рассудка}


В религии человек раздваивается в самом себе: он противопоставляет себе бога, как нечто противоположное ему. Бог есть не то, что человек, а человек не то, что бог. Бог --- бесконечное, человек --- конечное существо; бог совершенен, человек несовершенен; бог вечен, человек смертен; бог всемогущ, человек бессилен; бог свят, человек греховен. Бог и человек составляют крайности. Бог понятие положительное, совокупность всех реальностей, человек --- понятие отрицательное, совокупность всего ничтожного.

Но человек воплощает в религии свою собственную сокровенную сущность. Следовательно, нужно доказать, что та противоположность, тот разлад между богом и человеком, на котором основана религия, есть разлад человека с его собственной сущностью.

Внутренняя необходимость этого доказательства очевидна уже по одному тому, что если бы божественная сущность, как объект религии, действительно отличалась от сущности человеческой, то между ними не было бы раздвоения, не было бы разлада. Если бог действительно иное существо, то что мне за дело до его совершенств? Разлад возможен только между такими двумя сущностями, которые раздвоились, но могут и должны составлять и, следовательно, по существу, доподлинно составляют единство. В силу этого общего основания та сущность, с которой раздвоился человек, должна быть ему врожденной, но в то же время отличной от той сущности или силы, которая внушает человеку чувство примирения и единства с богом или, что то же, с самим собой.

Эта сущность не что иное, как ум (Intelligenz) --- разум или рассудок. Бог, как противоположность человеку, как нечеловеческая, то есть не индивидуально человеческая, сущность, является объективированной сущностью рассудка. Чистая, совершенная, безущербная божественная сущность есть самосознание рассудка, сознание рассудком своего собственного совершенства. Рассудок не знает страданий сердца; ему чужды желания, страсти, потребности и, следовательно, недостатки и слабости, свойственные сердцу. Рассудочные люди, люди, воплощающие в себе и олицетворяющие собой несколько односторонним, но тем более характерным образом сущность рассудка, выше душевных мук, страстей и волнений, свойственных людям чувства; они не привязываются со всей страстью к конечному, то есть определенному, предмету, они не закабаляют себя, они свободны. <<Не иметь потребностей и благодаря этому уподобляться бессмертным богам>>; <<не подчиняться обстоятельствам, а подчинять их себе>>; <<всё суета>> --- эти и подобные изречения были девизом людей отвлеченного рассудка. Рассудок есть нейтральная, безразличная, неподкупная, неомрачённая в нас сущность, чистый, свободный от эффектов свет познания. Рассудок есть категорическое, беспощадное сознание вещи как таковой, потому что он объективен по природе, --- сознание всего, что свободно от противоречий, потому что он сам есть свободное от противоречий единство, источник логического тождества, --- сознание закона необходимости, правила, меры, потому он сам есть действие закона, необходимость природы вещей, как самодеятельность, правило правил, абсолютное мерило, мерило мерил. Только благодаря рассудку человек думает и поступает вопреки своим самым заветным человеческим, то есть личным, чувствам, если того потребует бог рассудка, закон, необходимость, право. Отец, приговаривающий в качестве судьи своего сына к смерти за совершенное им преступление, делает это, как человек рассудка, а не чувства. Благодаря рассудку мы замечаем недостатки и слабости людей любимых и близких --- даже наши собственные. Он поэтому часто приводит нас к мучительным столкновениям с самим собой, с влечениями сердца. Мы не хотим признать прав рассудка, не хотим из чувства сожаления и снисхождения произнести справедливый, но жесткий безоговорочный приговор рассудка. Рассудок --- достояние рода. Сердце является носителем частных, индивидуальных побуждений, рассудок --- побуждений всеобщих. Рассудок сверхчеловеческая, то есть сверхличная, безличная сила или сущность в человеке. Только посредством рассудка и в рассудке человек получает силу абстрагировать от самого себя, то есть от своей субъективной, личной сущности, возвышаться до всеобщих понятий и отношений, отличать объект от производимого им впечатления, рассматривать его по существу, помимо его отношения к человеку. Философия, математика, астрономия, физика, одним словом, всякая наука является продуктом и, следовательно, фактическим доказательством этой поистине бесконечной и божественной деятельности. Поэтому религиозные антропоморфизмы противоречат рассудку, и он оспаривает их у бога, отрицает их. Но этот свободный от антропоморфизмов, беспощадный, бесстрастный бог есть не что иное, как собственная объективная сущность рассудка.

Бог, как таковой, то есть как не-конечная, не-человеческая, нематериальная, не-чувственная сущность, есть только объект мышления. Это --- отвлеченное, аморфное, непостижимое, без\'{о}бразное, абстрактное, отрицательное существо; оно познается, то есть становится объектом, только при помощи абстракции и отрицания (via negationis). Почему? Потому, что оно есть не что иное, как объективная сущность мыслительной способности, то есть вообще той силы или деятельности, --- называйте её, как хотите, --- благодаря которой человек сознает разум, дух, ум. Духовной сущностью является исключительно мышление, познание, рассудок, а всякая другая духовная сущность есть призрак, порождение фантазии. Человек не может допускать, подозревать, представлять, мыслить другую духовную сущность, другой разум, кроме того, который светится и проявляется в нем. Он может только вынести разум в пределы своей индивидуальности. <<Бесконечный дух>> в отличие от конечного есть не что иное, как разум, вынесенный за пределы индивидуальности и телесности (потому что индивидуальность и телесность неотделимы одна от другой), разум, утверждаемый или мыслимый сам в себе. Схоластики, отцы церкви и ещё раньше их языческие философы говорили, что бог --- существо бестелесное, дух, чистый рассудок. Образ бога, как такового, нельзя себе представить, но ведь ты не можешь представить себе и образ рассудка, разума. Какой у него вид? Разве можно постичь и представить себе его деятельность? Бог непостижим, но постигаешь ли ты сущность разума? Исследовал ли ты таинственные функции мышления, сокровенную сущность самосознания? Разве самосознание не есть величайшая из тайн? Даже древние мистики, схоластики и отцы церкви сравнивали непостижимость существа божия с непостижимостью человеческого духа. Разве это не свидетельствует о том, что они отождествляли сущность божественную с сущностью человеческой?*\let\svthefootnote\thefootnote\let\thefootnote\relax\footnotetext{*в своем сочинении <<Contra Academicos>>, которое Августин написал тогда, когда до известной степени был ещё язычником, он говорит (кн. \rom{3}, гл. 12), что высшее благо человека заключается в духе, или в разуме. Напротив, в своем <<Libr. retractationum>>, где он является уже изощренным христианским богословом, Августин следующим образом (кн. \rom{1}, гл. 1) исправляет высказанную им прежде мысль: <<Правильнее было бы сказать: в боге, ибо дух, чтобы достигнуть блаженства, наслаждается богом как своим высшим благом>>. Но в чем же тут разница? Разве моя сущность не там, где мое высшее благо?}\let\thefootnote\svthefootnote Бог, как таковой, то есть как существо, служащее лишь объектом разума, то есть объективированный разум. Только бог объясняет тебе, что такое рассудок или разум. Все должно выражаться, обнаруживаться, осуществляться, утверждаться. Бог есть разум, выражающий и утверждающий себя в качестве высшей сущности. Для воображения разум является откровением бога, а для разума бог является откровением разума. Сущность разума, его могущество воплощается в боге. Бог есть потребность мышления, необходимая мысль, высшая ступень мыслительной способности. <<Разум не удовлетворяется чувственными предметами и существами>>; он успокаивается только тогда, когда возвращается к высшей, главной, необходимой сущности, служащей только объектом для разума. Почему? Потому, что только в этой сущности разум сосредоточен в себе самом, потому что только в мысли о высшей сущности заключена высшая сущность разума, высшая ступень мыслительной и абстрагирующей способности. Мы не удовлетворяемся и, следовательно, чувствуем себя несчастными до тех пор, пока наши силы не разовьются до крайних пределов, до того, quo nihil majus cogitari potest\footnote{выше чего нельзя ничего помыслить (лат.).}, пока прирожденная способность к тому или другому искусству, к той или другой науке не достигнет высшей степени совершенства. Только высшая ступень искусства есть искусство, только высшая ступень мышления --- мышление, разум. Ты мыслишь в строгом смысле этого слова только тогда, когда мыслишь о боге, потому что только бог реализует, олицетворяет, исчерпывает собой мыслительную способность. Ты сознаешь действительный разум только в том случае, если ты познаешь бога, несмотря на то, что ты при помощи воображения представляешь себе эту сущность как нечто отличное от разума, потому что ты, как чувственное существо, привык отличать объект от созерцания, подлинный объект --- от представления и теперь посредством воображения переносишь эту привычку на сущность разума и, таким образом, ошибочно приписываешь бытию разума, бытию умозрительному, чувственное бытие, от которого бы абстрагировался.




Бог, как сущность метафизическая, --- себе довлеющий разум, или, вернее, наоборот: самодовлеющий, сознающий себя абсолютной сущностью разум есть бог в качестве метафизической сущности. Поэтому все метафизические определения бога --- действительные определения лишь постольку, поскольку они мысленные определения, поскольку они применяются к разуму, к рассудку.

Разум есть <<исходная, первичная>> сущность. Разум выводит все вещи из бога как первоисточника. Мир, исключающий разумную причину, кажется разуму игрушкой бессмысленного и бесцельного случая, то есть он видит основу и цель мира только в себе самом, в своей сущности; существование мира кажется ему ясным и понятным только тогда, когда оно вытекает из ясных и определенных понятий, то есть из самого разума. Ясной, определенной, обоснованной, истинной сущностью рассудок\dag\let\svthefootnote\thefootnote\let\thefootnote\relax\footnotetext{\dagто есть само собой разумеется, рассудок, как он здесь рассматривается, очищенный от чувственности, оторванный от природы, --- теистический рассудок.}\let\thefootnote\svthefootnote считает только такую сущность, которая действует согласно разуму и руководится намерением и целью. Поэтому все бесцельное находит оправдание своего существования в цели какой-нибудь другой разумной сущности. Таким образом, рассудок сознает себя причиной, первой, предвечной сущностью, то есть, будучи первой по достоинству и последней по времени сущностью, он считает себя также и первой по времени сущностью.



Рассудок --- критерий всего существующего, всей действительности. Всё бессмысленное, противоречащее себе, ничтожно; всё, что противоречит разуму, противоречит богу. Так, например, разуму противоречит соединение понятия высшей сущности с определенными границами места и времени, и поэтому он отрицает эти границы, как противоречащие божественной сущности. Разум может верить только в такого бога, который соответствует его сущности, в бога, который не ниже его по достоинству, а, напротив, воплощает в себе его собственную сущность. Разум верит только в себя, в реальность, в истину своей собственной сущности. Не разум зависит от бога, а бог от разума. Даже в те времена, когда господствовала вера в чудеса, разум считался, по крайней мере формально, критерием божества. Бог всесилен, он может делать все благодаря своему безграничному всемогуществу, но в то же время он ничто, он совершенно бессилен в отношении всего, что противоречит разуму. И всемогущество не может учинить неразумного. Следовательно, сила всемогущества уступает высшей силе рассудка и над сущностью бога стоит сущность рассудка как критерий всего, что утверждается и отвергается богом, всего положительного и отрицательного. Можешь ли ты считать богом существо неразумное и злое? Никогда. А почему? Потому, что обоготворение злого и неразумного существа противоречит твоему рассудку. Что ты утверждаешь, что ты воплощаешь в боге? Твой собственный рассудок. Бог --- твое высшее понятие и рассудок, твоя высшая мыслительная способность. Бог есть <<совокупность всех реальностей>>, то есть совокупность всех истин рассудка. Я воплощаю в боге те свойства рассудка, которые мне кажутся существенными; бог есть то, что рассудок мыслит как наивысшее. В том, что я считаю существенным, обнаруживается сущность моего рассудка, проявляется сила моей мыслительной способности.

Таким образом, рассудок есть наиреальнейшая сущность древней онтотеологии. <<В сущности, --- говорит онтотеология, --- мы не можем представить себе бога иначе как приписывая ему без всякого ограничения все наши реальные свойства>>\ddag\let\svthefootnote\thefootnote\let\thefootnote\relax\footnotetext{\ddagКант И., Лекция о философском учении о религии. Лейпциг, 1817 г., стр. 39.}\let\thefootnote\svthefootnote. Наши положительные существенные качества, наши реальности являются реальностями бога но в нас они ограничены, а в боге не ограничены. Но что освобождает реальности от границ? Рассудок. Следовательно, сущность, свободная от границ, есть не что иное как сущность освобождающего от границ рассудка. Ты сознаешь себя таким, каким ты сознаешь бога; мерило твоего бога есть мерило твоего разума. Если ты представляешь себе бога ограниченным, --- значит, твой рассудок также ограничен. Если ты представляешь себе бога существом телесным, --- то материя есть граница, предел твоего рассудка и ты не можешь представить себе ничего бестелесного; если же, напротив, ты считаешь бога бесплотным то этим ты доказываешь и утверждаешь свободу своего рассудка от материальных границ. В неограниченной сущности ты воплощаешь свой собственный неограниченный рассудок. И объявляя, эту самую неограниченную сущность высшей сущностью, ты тем самым поистине признаешь, что рассудок есть \^etre supr\^eme, верховная сущность.



Далее, рассудок есть сущность самостоятельная и независимая. Зависимо и несамостоятельно только то, что лишено рассудка. Человек без рассудка не обладает и волей. Он позволяет обольщать себя, поддается ослеплению, становится средством в руках окружающих. Человек несамостоятельный в смысле разума не может быть самостоятельным и в смысле воли. Только мыслящее существо свободно и независимо. Только при помощи рассудка человек низводит окружающие его и подчиненные ему существа на степень простого средства для своего существования. Самостоятельно и независимо только то, что заключает свою цель, свой объект в самом себе. То, что является целью и объектом для самого себя, не может быть средством и объектом для другого существа, поскольку оно объект для самого себя. Словом, отсутствие разума есть бытие для других, объект; рассудок --- бытие для себя, субъект. Все, существующее не для других, а для себя, отвергает всякую зависимость от другого существа. Разумеется, мы, как существа физические, зависим от существ вне нас даже в момент мышления; но поскольку мы мыслим, поскольку вопрос касается деятельности рассудка, как таковой, мы не зависим ни от какого другого существа*\let\svthefootnote\thefootnote\let\thefootnote\relax\footnotetext{*это относится даже к мыслительному акту, как акту физиологическому, так как деятельность мозга, хотя она и предполагает дыхательный и другие процессы, является особой, самостоятельной деятельностью.}\let\thefootnote\svthefootnote. Деятельность мышления есть самодеятельность. Когда я мыслю, --- говорит Кант в только что названном сочинении, --- я сознаю, что во мне мыслит мое <<Я>>, а не что-либо постороннее. Из этого я заключаю, что мое мышление зависит не от какой-либо внешней вещи, а от меня самого и, следовательно, я --- субстанция, то есть существую для себя, а не в качестве предиката другой вещи. Мы нуждаемся в воздухе, но тем не менее мы в качестве естествоиспытателей превращаем воздух из предмета потребности в объект независимой деятельности мышления, то есть низводим его на степень вещи для нас. Когда я дышу, я --- объект воздуха, а воздух --- субъект; когда я делаю воздух объектом мышления, исследования, анализа, тогда, напротив, я становлюсь субъектом, а воздух --- объектом для меня. Зависимо только то, что служит объектом для другой сущности. Так, растение зависит от воздуха и света, то есть служит объектом для воздуха и света, а не для самого себя. В то же время воздух и свет являются объектами для растения. Физическая жизнь есть не что иное, как постоянная смена субъективного и объективного бытия, цели и средства. Мы поглощаем воздух и поглощаемся им; наслаждаемся сами и доставляем наслаждение другим. Только рассудок есть сущность, наслаждающаяся всем и не доставляющая наслаждение ничему, сущность самодовлеющая, наслаждающаяся только собой, --- абсолютный субъект, сущность, которую нельзя низвести на степень объекта другой сущности, потому что она, не будучи вещью сама и будучи свободной от всех вещей, делает все вещи объектами и предикатами для себя самого.


Единство рассудка есть единство бога. Рассудку присуще сознание своего единства и универсальности, он сознает себя в качестве абсолютного единства; именно: все что рассудок находит разумным, кажется ему неизменным, абсолютным законом; он не допускает, чтобы что-нибудь противоречащее себе, ложное, бессмысленное могло быть где-нибудь истинным, и, наоборот, чтобы истинное, разумное могло быть где-нибудь ложным, бессмысленным. <<Я допускаю, --- говорит Мальбранш, --- что некоторые разумные люди не похожи на меня, но не могу допустить, что есть такие разумные существа, которые признают другие законы и истины, чем я, потому что каждый разумный человек непременно должен понимать, что дважды два --- четыре и что нельзя предпочитать собаку своему другу>>\dag\let\svthefootnote\thefootnote\let\thefootnote\relax\footnotetext{\dagточно так же говорит астроном Христиан Гюйгенс в своем уже раньше цитированном <<Cosmotheoros>>: <<Возможно ли, чтобы где-нибудь существовал разум, отличный от нашего? Возможно ли чтобы на Юпитере и на Марсе считалось несправедливым и отвратительным то, что у нас считается справедливым и достойным похвалы.  Поистине, это невероятно и даже невозможно>>.}\let\thefootnote\svthefootnote. Я не могу допустить, представить себе такой рассудок, который существенно отличался бы от человеческого. Скорее, этот мнимо другой рассудок есть только утверждение моего собственного рассудка, то есть моя идея, мое представление, которое зарождается в моем уме и, следовательно, выражает мой рассудок. Что я мыслю, то я сам созидаю, разумеется, в вопросах чисто интеллектуальных: я связываю то, что мне кажется связанным, различаю то, что мне кажется различным, и отрицаю то, что мне кажется невозможным. Так, например, если я представляю себе такой рассудок, в котором созерцание или действительность объекта непосредственно связаны с мыслью объекта, то я их действительно связываю; мой рассудок или сила моего воображения обладают способностью связывать эти различия или противоречия. Но каким образом мог бы ты их связывать отчетливо или неясно, если бы они не соединялись в тебе самом? Однако, --- как бы мы ни определяли рассудок, который кажется определенному человеческому индивиду отличным от его собственного, --- этот другой рассудок есть не что иное, как рассудок, свойственный людям вообще, но только вынесенный за пределы этого определенного, существующего во времени индивида. Понятию рассудка присуще единство. Рассудок не может представить себе две высшие сущности, две бесконечные субстанции, двух богов, потому что он не может противоречить самому себе, отрицать свою собственную сущность, сознавать себя разделенным и размноженным.



Рассудок есть сущность бесконечная. Бесконечность нераздельна с единством, ограниченность --- с множеством. Ограниченность в смысле метафизическом основана на различии между бытием и сущностью, индивидом и родом; бесконечность --- на единстве бытия и сущности. Поэтому конечно всё то, что можно сравнивать с другими индивидами того же рода, а бесконечно всё то, что тождественно лишь самому себе, что не имеет себе подобных и, следовательно, не является индивидом известного рода, а соединяет в едином нераздельно род и индивид, сущность и бытие. Таков рассудок; его сущность заключается в нем самом, и поэтому наряду с ним нет ничего, что бы существовало подле него или вне его. Он вне всяких сравнений, потому что сам является источником всех сравнений; он неизмерим, потону что он мерило всех мерил, и мы измеряем все только рассудком; его нельзя подчинить какой-нибудь высшей сущности или роду, потому что он как наивысшее начало всех иерархий сам подчиняет себе все вещи и существа. Умозрительные философы и теологи считают бога существом, в котором сливаются воедино бытие и сущность, чем обусловливается тождественность субъекта и предиката; все эти определения являются понятиями, отвлеченными от сущности разума.

Наконец рассудок, или разум, есть сущность необходимая. Разум существует только потому, что его существование разумно: если бы не было разума, то не было бы и сознания, все было бы ничем, бытие уподоблялось бы небытию. Различие между бытием и небытием основывается только на сознании. Лишь в сознании открывается ценность бытия, ценность природы. Почему существует вообще нечто, почему существует мир? Только потому, что если бы не существовало этого нечто, то существовало бы ничто, если бы не было разума, то было бы неразумие. Следовательно, мир существует потому, что было бы нелепостью, если бы он не существовал. В нелепости его небытия заключается истинный смысл его бытия, в неосновательности предположения, будто он мог бы не существовать, --- основа его существования. Ничто, небытие бесцельно, бессмысленно, непонятно. Только бытие имеет цель, основание и смысл. Бытие существует, потому что только бытие разумно и истинно. Бытие есть абсолютная потребность, абсолютная необходимость. В чем заключается основание самоощущающего бытия, жизни? В потребности жить. Чему нужна эта потребность? Тому, что не живет. Глаза созданы не зрячим существом; если оно уже видит, зачем ему глаза? Нет, глаза нужны только тому, кто не видит. Все мы родились на свет помимо нашего знания и воли, но именно затем, чтобы существовали знание и воля. Из чего, следовательно, вытекает существование мира? Из нужды, потребности, необходимости, но не из той необходимости, которая заключается в другой, отличной от мира сущности --- это было бы явным противоречием, --- а из собственной внутренней необходимости, из необходимости необходимости, потому что без мира не может быть необходимости, а без необходимости не может быть разума, не может быть рассудка. Ничто, из которого произошел мир, есть ничто без мира. Итак, в самом деле причиной мира является ничто, по выражению умозрительных философов, но ничто, упраздняющее само себя, ничто, которое per imossibile\footnote{через невозможное (лат.), во что бы то ни стало.} существовало бы, если бы не было мира. Так или иначе, мир вытекает из недостатка, но этот недостаток нельзя рассматривать как онтологическую сущность; этот недостаток заключается в предполагаемом небытии мира. Следовательно, необходимость мира вытекает из самого мира и через него. Но необходимость мира есть необходимость разума. Разум, как понятие всего существующего, --- потому что все достоинства мира ничто без света, а внешний свет ничто без внутреннего, --- разум есть самая необходимейшая сущность, глубочайшая и самая существенная потребность. Только разум является самосознанием бытия, самосознающим бытием; только в разуме открывается цель и смысл бытия. Разум есть бытие, объектом которого служит его собственная цель --- конечная цель вещей. Все, что служит объектом самого себя, есть высшая, последняя сущность; всё, что владеет собой, всесильно.

\chapter{Бог как моральная сущность или закон}

Бог как бог --- бесконечная, всеобщая, свободная от антропоморфизмов сущность разума --- имеет для религии такое же значение, какое имеет для отдельной науки общее начало, на котором она основывается; это только высшая, последняя, основная, связующая, как бы математическая точка религии. Связанное с сознанием этой сущности сознание человеческой ограниченности и ничтожества ни в коем случае не является сознанием религиозным, оно скорее характеризует скептика, материалиста, натуралиста, пантеиста. Вера в бога --- по крайней мере в бога религии --- утрачивается только там, где, как в скептицизме, пантеизме и материализме, утрачивается вера в человека, во всяком случае, в такого, с каким считается религия. Религия не утверждает и не может серьезно утверждать, что человек ничтожен*\let\svthefootnote\thefootnote\let\thefootnote\relax\footnotetext{*представлением или выражением ничтожества человека перед лицом бога является в пределах религии гнев божий, ибо как любовь бога есть утверждение человека, так гнев божий есть его отрицание. Но дело в том, что гнев этот принимается не очень-то всерьез. <<Бог\dots в действительности не гневен. Бог не может по-настоящему гневаться, нам только кажется, что он гневается и карает>> (Лютер, Полное собрание сочинений. Лейпциг, 1729, ч. \rom{8}, стр. 208). К этому изданию относятся и дальнейшие цитаты в этой книге, причем обозначается только соответствующая часть.}\let\thefootnote\svthefootnote, и серьезно признавать ту отвлеченную сущность, с которой связано сознание этого ничтожества. Религия в действительности признает только такие определения, которые для человека объективируют человека. Отрицание человека равносильно отрицанию религии.



Религии нужен такой объект, который отличался бы от человека и в то же время обладал бы человеческими качествами. Отличие от человека касается лишь формы существования, сходство с ним составляет внутреннюю сущность его. Если бы это существо значительно отличалось от человека, то человеку не было бы никакого дела до бытия или небытия этого существа. Человек не стал бы интересоваться существованием бога, если бы не был причастен ему.

Вот пример: <<Если допустить, --- говорится в ,,Конкорданциях``, --- что за меня страдало только человеческое естество, то Христос является в моих глазах плохим спасителем, так как ему самому нужен спаситель>>. Таким образом мы выходим за пределы человека и требуем для спасения другое существо, отличное от человека, но, предположив такое существо, человек, чувствующий влечение к самому себе, к своей сущности, тотчас наделяет его человеческими качествами. <<Здесь бог, а не человек, и не соделавшийся человеком\dots Я не хочу такого бога\dots Христос, не обладающий, в качестве отвлеченного бога человеческими качествами, не удовлетворяет меня. Нет, друг, дай мне такого бога, который был бы подобен человеку>>.

Человек ищет успокоения в религии, религия --- его высшее достояние. Но каким образом он мог бы обрести в боге утешение и мир, если бы бог отличался от него по существу? Я не могу разделить блаженства того кто существенно отличается от меня. Если его сущность отличается от моей сущности, то и его покой существенно иной, для меня это --- не покой. Я не могу быть причастным его покою, если я не причастен его сущности, а как я могу быть причастным его сущности, если я существенно отличаюсь от него? Все живущее обретает покой только в своей стихии, в своей собственной сущности. Следовательно, если человек обретает успокоение в боге, то это значит, что бог --- его истинная сущность, а все, в чем он искал успокоения и что считал своею сущностью до сих пор было посторонней, чуждой ему сущностью. Поэтому если человек хочет и должен обрести в боге успокоение, он должен сначала обрести в нем самого себя. <<Кто хочет насладиться божеством, тот должен искать его в человечности Христа; если же ты таким путем не обретешь божества и в нем, значит, тебе не суждено обрести покой>>\dag\let\svthefootnote\thefootnote\let\thefootnote\relax\footnotetext{\dagЛютер, ч. \rom{3}, стр. 589.}\let\thefootnote\svthefootnote. <<Каждая вещь обретает покой в своей родной стихии. Я происхожу от божества. Бог --- мое отечество. Бог --- не отец ли мне? Да, бог не только мой отец, но и мое подлинное ,,Я``. Прежде чем явиться на свет, я уже существовал в боге>>\ddag\let\svthefootnote\thefootnote\let\thefootnote\relax\footnotetext{\ddag<<Predigten etzlicher Lehrer vor und zu Tauleri Zeiten>>. Hamburg, 1621, S. 81.}\let\thefootnote\svthefootnote.




Бог, выражающий собою только сущность разума, не удовлетворяет религии и не есть бог в смысле религиозном. Рассудок интересуется не только человеком, но и тем, что вне его, --- природой. Рассудочный человек ради природы забывает о самом себе. Христиане смеялись над языческими философами, потому что те думали не о себе, не о своем спасении, а исключительно о вещах, им посторонних. Христианин думает только о себе. Рассудок относится с одинаковым энтузиазмом к блохе, ко всему созданному по подобию божию человеку. Рассудок есть абсолютная нейтральность и тождество всех вещей и сущностей. Существованием ботаники, минералогии, зоологии, физики, астрономии мы обязаны не христианству, не религиозному воодушевлению, а энтузиазму рассудка. Одним словом, рассудок есть универсальная, пантеистическая сущность, любовь ко вселенной; а наиболее специфическим определением религии, в особенности христианской, будет столкование её как сущности антропотеистической, это --- исключительная любовь человека к самому себе, исключительное самоутверждение человеческой, и притом субъективно человеческой, сущности. Ведь рассудок тоже утверждает сущность человека, но только его объективную сущность, относящуюся к предмету ради самого предмета, который изображается в науке. Если человек хочет и должен найти удовлетворение в религии, то его религиозным объектом должно быть, помимо сущности рассудка, ещё нечто другое, что и является истинным ядром религии.

Каждая религия, и в том числе христианская, приписывает богу, раньше всех других рассудочных определений, нравственное совершенство. Бог, как существо нравственно совершенное, есть не что иное, как осуществлённая идея, олицетворённый закон нравственности*\let\svthefootnote\thefootnote\let\thefootnote\relax\footnotetext{*Даже Кант говорит в уже неоднократно цитированных и читанных ещё при Фридрихе \rom{2} <<Лекциях о философском учении о религии>> (стр. 135): <<Бог есть как бы сам нравственный закон, который, мыслится олицетворённым>>.}\let\thefootnote\svthefootnote, обращённая в абсолютную сущность собственная моральная сущность человека. Ибо моральный бог требует от человека, чтобы тот был таким же, как он. <<Бог, свят, и вы должны быть святы, как бог>>. Бог есть собственная совесть человека, потому что иначе он не мог бы бояться бога, каяться перед ним в своих грехах, считать его судьей своих самых сокровенных помыслов и побуждений.



Но сознание нравственно совершенного существа, как сознание отвлеченной, очищенной от всех человеческих страстей сущности, не согревает нас, не заполняет нашей внутренней пустоты, потому что мы чувствуем разрыв между нами и этим существом. Это --- бессердечное сознание, потому что оно обусловливается сознанием нашего личного ничтожества и притом самого чувствительного, нравственного ничтожества. Сознание всемогущества и предвечности божьей, в противоположность моей ограниченности в пространстве и времени, не причиняет мне боли, потому что всемогущество не обязывает меня быть всемогущим, а предвечность --- быть вечным. Но нравственное совершенство я могу сознавать только как закон для себя. Нравственное совершенство, по крайней мере для нравственного сознания, зависит не от природы, а только от воли, оно есть совершенство воли, совершенная воля. Я не могу мыслить совершенной воли, которая тождественна с законом и сама является законом, не мысля её как объект воли, то есть долга. Короче, представление о нравственно совершенной сущности есть не только теоретическое, спокойное, но вместе с тем и практическое представление, требующее действия, подражания и служащее источником моего разлада с самим собой, потому что оно предписывает мне, чем я должен быть, и в то же время без всякого лицеприятия указывает мне, что я не таков\dag\let\svthefootnote\thefootnote\let\thefootnote\relax\footnotetext{\dag<<\dotsТо, что уменьшает наше самомнение в нашем собственном суждении, смиряет. Следовательно, моральный закон неизбежно смиряет каждого человека, сопостовляющего с этим законом чувственные влечения своей природы>> (Кант И., Критика практического разума, изд. 4, стр. 132).}\let\thefootnote\svthefootnote\footnote{перевод дан по: Кант И. Соч. в 6 т. Т. 4, ч. \rom{1}. М., 1965 г., с. 399.}. Этот разлад тем мучительнее, тем ужаснее, чем больше религия противопоставляет человеку его собственную истинную сущность, как другое и притом личное существо, которое лишает грешников своей милости, источника всякого спасения и блаженства, которое ненавидит и проклинает их.

Что же избавляет человека от разлада между ним и совершенным существом, от тяжелого чувства греховности, от мучительного сознания своего ничтожества? Чем притупляется смертоносное жало греха? Только тем, что человек сознает, что сердце, любовь есть высшая, абсолютная сила и истина, и видит в боге не только закон, моральную сущность и сущность разума, но главным образом любящее, сердечное, даже субъективно-человеческое существо.

Рассудок судит только по строгости закона, сердце приспособляется; оно судит справедливо, снисходительно с оглядкой, \textgreek{````νατ''''άνθρωπον}\footnote{применительно к человеку (др.-гр.).}. Закон, требующий от нас нравственного совершенства, недоволен ни одним из нас, но поэтому-то человек, его сердце также недовольны законом. Закон обрекает на гибель; сердце проникается жалостью к грешнику. Закон утверждает меня как абстрактное, сердце --- как действительное существо. Сердце вселяет в меня сознание, что я человек; закон --- только сознание, что я грешник, ничтожество\ddag\let\svthefootnote\thefootnote\let\thefootnote\relax\footnotetext{\ddag<<Мы все обременены грехами\dots Вместе с законом появились отцеубийцы>> (Сенека). <<Закон нас губит>>. (Лютер, ч. \rom{16}, стр. 320).}\let\thefootnote\svthefootnote\footnote{Фейербах цитирует сочинение Сенеки <<О милосердии>> (\rom{1}, 23).}. Закон подчиняет себе человека, любовь его освобождает.



Любовь есть связь, посредствующее начало между совершенным и несовершенным, греховным и безгрешным, всеобщим и индивидуальным, законом и сердцем, божеским и человеческим. Любовь есть сам бог, и вне любви нет бога. Любовь делает человека богом и бога --- человеком. Любовь укрепляет слабое и ослабляет сильное, унижает высокое и возвышает низкое, идеализирует материю и материализирует дух. Любовь есть подлинное единство бога и человека, духа и природы. Любовь претворяет обыденную природу в дух и возвышенный дух --- в природу. Любить --- значит исходя из духа, отрицать дух и, исходя из материи, отрицать материю. Любовь есть материализм; нематериальная любовь есть нелепость. Отвлеченный идеалист, приписывающий любви стремление к отдаленному предмету, только утверждает этим, помимо своей воли, истину чувственности. Но в то же время любовь есть идеализм природы; любовь есть дух. Только любовь научает соловья искусству пения, только любовь украшает половые органы растений венком цветка. Какие чудеса творит любовь даже в нашей будничной мещанской жизни! Любовь объединяет то, что разделяют вера, исповедание, предрассудок. Даже нашу аристократию любовь с немалым юмором отождествляет с городской чернью. Древние мистики говорили, что бог есть высшее и в то же время самое обыкновенное существо. Это поистине относится к любви, но не к вымышленной, воображаемой, нет! А к настоящей любви, любви, облеченной в плоть и кровь.

Да, это относится только к любви, облеченной в плоть и кровь, потому что только такая любовь способна прощать грехи, содеянные плотью и кровью. Исключительно нравственное существо не может прощать того, что противоречит закону нравственности. То, что отрицает закон, в свою очередь само отрицается им. Моральный судья, который не может влить человеческую кровь в свои приговоры, судит грешника беспощадно и неумолимо. Итак, если бог отпускает грехи, то, следовательно, он, хотя, не безнравственное, но больше чем моральное, словом, человечное существо. Отрицание греха равносильно отрицанию отвлеченной нравственной справедливости --- утверждению любви, милосердия, чувственности. Только чувственные, а не отвлеченные существа бывают милосердны. Милосердие есть правосознание чувственности. Поэтому бог отпускает грехи не как отвлеченный, рассудочный бог, а как человек, как воплощенный, чувственный бог. Бог, в качестве человека, разумеется, не грешит, но он знает, он принимает на себя страдания, потребности, нужды чувственности. Кровь Христа очищает нас в глазах бога от наших грехов; только его человеческая кровь смягчает бога, укрощает его гнев. Это значит, что наши грехи отпускаются нам потому, что мы существа не абстрактные, а облеченные в плоть и кровь*\let\svthefootnote\thefootnote\let\thefootnote\relax\footnotetext{*<<Этот мой господь и бог принял мою природу: плоть и кровь и испытал и выстрадал все то, что и я, но безгрешно; поэтому он может иметь состраданье к моей слабости>> (Лютер, ч. \rom{17}, стр. 533). <<Чем глубже в нашу плоть вберем мы Христа, тем лучше>> (там же, стр. 565). <<Сам бог, если обращаться с ним помимо юриста, есть страшный бог, у которого нельзя найти утешение, а только гнев и немилость>> (ч. \rom{15}, стр. 298--334).}\let\thefootnote\svthefootnote.






\chapter{Тайна воплощения, или бог как сущность сердца}

Благодаря сознанию любви человек примиряется с богом, или вернее, с собой, со своей собственной сущностью, которую он противопоставляет себе в законе как отдельную сущность. Тайна воплощения или вочеловечения бога заключается в сознании божественной любви, или, что то же, во взгляде на бога, как на человеческое существо. Воплощение есть не что иное, как фактическое, чувственное проявление человеческого естества в боге. Бог сделался человеком не ради себя; причиной воплощения была нужда, потребность человека, которая поныне остается потребностью религиозной души. Бог сделался человеком из милосердия, следовательно, прежде чем сделаться истинным человеком, он уже был человеческим богом в себе самом, иначе он не принял бы так близко к сердцу человеческую нужду и человеческое горе. Воплощение было слезой божественного сострадания, то есть проявлением человечески чувствующего, поэтому по преимуществу человеческого существа.

Если мы будем усматривать в воплощении только вочеловечившегося бога, то это вочеловечение, конечно, покажется нам поразительным, необъяснимым, чудесным. Но вочеловечившийся бог есть только сделавшийся богом человек, потому что нисхождению бога до человека должно непременно предшествовать возвышение человека до бога. Прежде чем бог сделался человеком, то есть явился в образе человека, человек уже был в боге, был сам уже богом*\let\svthefootnote\thefootnote\let\thefootnote\relax\footnotetext{*<<Такие места в Писании, в которых говорят о боге как о человеке и присваивают ему человеческие черты, весьма приятны и утешительны; утешительно, что он так ласково говорит с нами и о таких вещах, о которых обыкновенно говорят люди, что он радуется, печалится и страдает, как человек, ради тайны грядущего вочеловечения Христа>> (Лютер, ч. \rom{2}, стр. 334).}\let\thefootnote\svthefootnote. Иначе бог не мог бы сделаться человеком. Старое правило <<из ничего ничто не может произойти>> действует и здесь. Царь, который не принимает близко к сердцу блага подданных, который не проникает мысленно в их жилища, сидя на троне, и который не является по своему настроению, как говорит народ, <<обыкновенным человеком>>, --- такой царь никогда не потрудится сойти со своего трона и осчастливить свой народ своим личным присутствием. Следовательно, если подданный не возвысится до царя, то и царь не снизойдет до него. Если личное посещение царя способно осчастливить и польстить подданному, то это чувство надо отнести не к появлению царя, как к таковому, а к обнаружившемуся в нем человеколюбивому настроению, составляющему основу этого его появления. Но что в истине религии является основанием, приобретает в сознании религии значение следствия. Так и здесь религия считает возвышение человека до бога следствием уничижения, снисхождения бога до человека. Бог, по смыслу религии, вочеловечился для того, чтобы человек сделался богом\dag\let\svthefootnote\thefootnote\let\thefootnote\relax\footnotetext{\dag<<Бог сделался человеком, чтобы человек стал богом>>. (Августин, Sermones ad populum). Но у Лютера и некоторых отцов церкви мы находим места, указывающие на истинное соотношение. Так, например, Лютер (ч. \rom{1}, стр. 334), говорит, что Моисеи, называя человека <<образом божиим, богоравным>>, намекал на то, что <<бог должен стать человеком>>. Здесь воплощение бога довольно ясно характеризуется как следствие божественности человека.}\let\thefootnote\svthefootnote.


Выражение <<бог стал человеком>>, кажется нам глубоким и непостижимым, то есть противоречивым, только потому, что мы смешиваем понятие или определения всеобщей, неограниченной, метафизической сущности с понятием или определениями религиозного бога, то есть определения рассудка с определениями сердца. Это смешение служит величайшим препятствием для правильного понимания религии. В действительности речь идет только о человеческом образе такого бога, который уже по существу, в глубине своей души, есть милосердный, то есть человечный бог.

Согласно учению церкви, воплощается не первое, а второе лицо божества, представляющее человека в боге и перед богом. В действительности же, как мы увидим ниже, второе лицо является истинным, целостным, первым лицом религии. И только без этого посредствующего понятия, служащего исходным пунктом воплощения, последнее, кажется таинственным, непостижимым, <<умозрительным>>; в связи же с ним оно является необходимым, само собой понятным следствием. Поэтому утверждение, будто воплощение есть факт чисто эмпирический, или исторический, обнаруживающийся только в откровении божественном, есть проявление самого нелепого религиозного материализма. Воплощение --- это вывод, основанный на вполне понятной предпосылке. Но нельзя также объяснять воплощение чисто умозрительными, то есть метафизическими, отвлеченными причинами, потому что метафизика относится только к первому лицу, которое не воплощается и не является лицом драматическим, такая дедукция оправдывалась бы только в том случае, если бы мы сознательно выводили из метафизики отрицание метафизики.

Из этого примера видно, насколько антропология отличается от умозрительной философии. Антропология не усматривает в воплощении особой, необычайной тайны, подобно ослепленной мистическим призраком умозрительной философии; она, напротив, разрушает иллюзию, будто в воплощении заключается особая, сверхъестественная тайна: она критикует догмат и сводит его к его естественным, прирожденным человеку элементам, к его внутреннему началу и средоточию --- к любви.

Догмат говорит о двояком: о боге и о любви. Бог есть любовь. Но что это значит? разве бог есть нечто отдельное от любви, отличная от нее сущность? Разве это определение бога не похоже на то, как мы определяем какого-нибудь человека, восклицая о нем в возбуждении: <<это --- сама любовь>>. Разумеется, иначе нужно было бы упразднить имя <<бог>>, которое обозначает особое, индивидуальное существо, субъект в отличие от предиката: Итак, любовь рассматривается как нечто особое: бог по своей любви послал своего единородного сына. Таким образом, любовь отодвигается и обесценивается тусклым фоном? Богом. Любовь становится индивидуальным, хотя и определяющим сущность качеством; и благодаря этому она приобретает для ума и сердца (в объективном и субъективном смысле) значение только предиката, а не субъекта, не сущности. Она становится в моих глазах второстепенной вещью, акциденцией; то она кажется мне чем-то существенным, то она снова исчезает. Бог представляется мне не только в образе любви, но и в образе всемогущества, в образе темной силы, --- силы, не связанной любовью, силы, свойственной демонам и чертям, --- правда, в меньшей степени.

Пока любовь не возвысится до субстанции, сущности, до тех пор за нею будет открываться субъект, представляющий собой нечто и без любви, какое-то безжалостное чудовище, отличаемое и действительно отличающееся от любви, демоническое существо, фантом религиозного фанатизма, которому доставляет наслаждение кровь еретиков и неверующих. Тем не менее, сущность воплощения заключается в любви, хотя и затемнённой мраком религиозного сознания. Любовь побудила бога к обнаружению своего божества\ddag\let\svthefootnote\thefootnote\let\thefootnote\relax\footnotetext{\ddagТак, в таком именно смысле прославляла воплощение старая, безусловная и восторженная вера. <<Любовь побеждает бога>> (<<Amor triumphat de deo>>), --- говорит, например, святой Бернард. И только на действительном самоотчуждении, самоотрицании божества покоятся реальность, сила и значение воплощения, хотя это самоотрицание есть само по себе лишь продукт фантазии, ибо на самом деле воплощением бог не отрицает самого себя, а лишь проявляет себя тем, что он есть, --- человеческим существом. Возражения позднейшего рационалистически-ортодоксального и библейско-пиэтистскирационалистичсского богословия против экстатических представлений и выражений старой веры, относящиеся к воплощению, не заслуживают не только опровержения, но даже упоминания.}\let\thefootnote\svthefootnote. Самоотречение бога вытекает не из его божественности, как таковой, в силу которой он является субъектом в положении: <<бог есть любовь>>, а из любви, из предиката; следовательно, любовь могущественнее и истиннее божества. Любовь побеждает бога. Любви пожертвовал бог своим божеским величием. Но какова была эта любовь? Чем она отличалась от нашей любви, которой мы жертвуем своим состоянием и жизнью. Была ли это любовь к себе, как к богу? Нет, это была любовь к человеку. А разве любовь к человеку не есть любовь человеческая? Могу ли я любить человека не человеческой любовью, не той любовью, какой он любит сам, когда любит действительно? Не была ли то дьявольская любовь? Ведь дьявол тоже любит человека, только не ради человека, а ради себя, то есть из эгоизма, с целью возвеличить себя, расширить свою власть. Но бог, любя человека, любит человека ради человека, то есть он желает сделать его совершенным, доставить ему счастье и блаженство. Следовательно, он любит человека так, как истинный человек любит себе подобных. Может ли вообще любовь определяться множественным числом? Не всюду ли она оказывается себе равной? Таким образом, истинная, настоящая причина воплощения заключается в любви без всяких дополнений, без различия любви божественной и человеческой. Ведь хотя среди людей и встречается иногда любовь корыстная, тем не менее истинная, достойная этого имени человеческая любовь всегда приносит свои интересы в жертву другому. Кто же наш спаситель и примиритель? Бог или любовь? Любовь, потому что мы спасены не богом, как таковым, а любовью, которая выше различия между божественной и человеческой личностью. Бог отрёкся от себя ради любви, так же мы из любви должны отречься от него, и если мы не принесем бога в жертву любви, то принесем любовь в жертву богу и найдем в нем, несмотря на предикат любви, злого идола религиозного фанатизма.



Извлекая эту мысль из воплощения, мы показали несостоятельность догмата и низвели мнимо сверхъестественную и сверхразумную тайну на степень простой, свойственной человеку истины --- истины, присущей, по крайней мере в зачатке, не только христианской, но и всякой другой религии, как таковой. Всякая религия, претендующая на это название, предполагает, что бог не равнодушен к существам, его почитающим, что человеческое ему не чуждо, что, как предмет человеческого почитания, он в то же время сам является человеческим богом. В каждой молитве заключена тайна воплощения, каждая молитва фактически есть воплощение бога. В молитве я склоняю бога к человеческому горю, делаю его причастным к моим страданиям и потребностям. Бог не остается глухим к моим мольбам; он жалеет меня; он таким образом отрекается от своего божеского величия ради всего конечного и человеческого; он с человеком становится человеком; ибо если он слышит и жалеет меня, значит мои страдания его трогают. Бог любит человека, то есть бог страдает от человека. Любовь немыслима без сочувствия, сочувствие --- без сострадания. Могу ли я сочувствовать бесчувственному существу? --- Нет! Я могу почувствовать только тому, кто чувствует, кого я сознаю однородным со мной по существу, в ком я чувствую самого себя, чьи страдания я сам разделяю, сострадание предполагает однородность существа. Эта однородность бога и человека выражается в воплощении, в промысле божьем, в молитве*\let\svthefootnote\thefootnote\let\thefootnote\relax\footnotetext{*<<Мы знаем, что бог сострадает нам и не только видит наши слезы, но и ведет счет всем нашим слезинкам>>, --- как сказано в 56-м псалме. <<Сын божий испытывает истинное сочувствие к нашим страданиям>> (Melanchtonis et aliorum Declam. Argentor., T. \rom{3}, p. 286, 450). <<Ни одна слезинка, --- говорит Лютер по поводу девятого стиха цитированного выше 56-го псалма, --- не проливается даром; огромными выразительными буквами отмечается она на небе>>. Но ведь существо, которое считает и <<собирает>> человеческие слезинки, несомненно, --- в высшей степени сентиментальное существо.}\let\thefootnote\svthefootnote\footnote{в русском Синодальном переводе это место читается так: <<У тебя исчислены мои скитания; положи слёзы мои в сосуд у тебя, --- не в книгели они твоей?>> (Псалтирь, 55, 9).} \footnote{Фейербах цитирует сочинение Ф. Меланхтона, сподвижника М. Лютера.}.



Богословие, разрабатывающее и придерживающееся метафизических определений вечности, непостижимости, неизменяемости и других таких же отвлеченных определений, выражающих на самом деле сущность разума, отрицает способность бога страдать и тем самым отрицает истинность религии\dag\let\svthefootnote\thefootnote\let\thefootnote\relax\footnotetext{\dagСв. Бернард выходит из затруднения с помощью тонко софистической игры слов: Impassibilis est deus, sed non incompassibilis, cui proprium est misereri semper et parcere (Sup. Cantica Sermo 26). Как будто состраданье не есть страданье, --- правда, страданье любви, страданье сердца! Но что страдает, как не сочувствующее сердце? Без любви нет страданья. Источником страданья является не что иное, как всеобщее сердце, объединяющее все существа.}\let\thefootnote\svthefootnote\footnote{<<Бог, всегда милосердный и опекающий, свободен от страдания, но не от сострадания>> (лат.).}. Ибо религия, религиозный человек, обращающийся к богу с благоговейной молитвой, верит в то, что бог принимает действительное участие в его страданиях и нуждах, верит в волю божию, обусловленную искренностью молитвы, то есть силой сердца, верит в то, что бог действительно и немедленно слышит его в самый момент молитвы. Истинно религиозный человек, не задумываясь, выкладывает перед богом свою душу, бог кажется ему сердцем, восприимчивым ко всему человеческому. Сердце может обращаться только к сердцу, оно находит утешение только в себе, в своей собственной сущности.



Утверждение, что исполнение молитвы было предопределено уже от вечности, что оно входило изначала в план мироздания, является пустым, нелепым вымыслом механического мышления, абсолютно противоречащим сущности религии. <<Нам нужен не стесненный в своем произволе бог>>, --- говорит совершенно справедливо в духе религии Лафатер. Впрочем и в вышеупомянутом вымысле и в уверенности, что бог слышит человека в самый момент молитвы, обнаруживается взгляд на бога, как на существо, определяемое человеком, с той только разницей, что в первом случае противоречивые свойства неизменяемости и непостижимости отодвигаются в туманную даль прошедшего или вечности. В сущности, совершенно безразлично, решается ли бог исполнить мою молитву именно теперь, или это было решено им уже раньше.

В высшей степени непоследовательно отвергать представление о боге, определяемом молитвой, то есть силой души, как представление недостойное, антропоморфическое, если мы верим в существо, служащее предметом почитания, предметом молитвы, предметом души, в существо, всеведущее, пекущееся о нас, немыслимое без любви, верим в провидение, в существо любящее, руководящееся в своих поступках исключительно любовью, то, следовательно, мы верим в существо, обладающее человеческим сердцем, если не в анатомическом, то в психологическом смысле, религиозный человек, как мы уже сказали, выкладывает перед богом свою душу, за исключением того, что кажется ему постыдным. Христиане не приписали богу страстей, противоречащих их нравственным понятиям, но зато, не задумываясь, они приписали ему чувства любви и милосердия, что было вполне естественно. Любовь, предполагаемая религиозным человеком в боге, есть не кажущаяся и воображаемая, а настоящая, подлинная любовь. Люди любят бога, и бог любит людей; в божественной любви только объективируется, утверждается любовь человеческая. Углубляясь, любовь находит в боге самое себя, свою истинность.

Развиваемый здесь взгляд на смысл воплощения может встретить возражение, что христианское воплощение (и это до некоторой степени справедливо) носит совершенно иной характер, чем вочеловечение богов языческих, например греческих и индусских. Последние были просто продуктом человека или обоготворёнными людьми, а в христианстве дана идея истинного бога, вследствие чего соединение божеского естества с человеческим получает здесь впервые особое значение и носит <<умозрительный>> характер. Юпитер превращался также и в быка; языческие превращения богов являются простым вымыслом. Сущность языческого бога не превосходила образа, в котором он являлся на землю, тогда как христианский бог принял на себя образ человека, не переставая в то же время быть богом, существом сверхчеловеческим. Но это возражение опровергается тем, что предпосылка христианского воплощения, как уже замечено выше, содержит в себе человеческую сущность. Бог любит человека; кроме того, он в себе имеет сына; бог есть отец; ему не чужды человеческие отношения: человеческое ему близко, ему знакомо. Следовательно, и здесь сущность бога не превосходит его проявления. В воплощении религия только признает то, чего она не хочет допустить как теология, рефлектирующая над самой собой, --- а именно, что бог есть существо насквозь человеческое. Следовательно, воплощение, тайна <<богочеловека>> не есть мистическое сочетание противоположностей, не есть синтетический факт, каким его считает умозрительная философия религии, питающая особую склонность ко всякого рода противоречиям, а факт аналитический --- человеческое слово с человеческим смыслом. Если бы здесь действительно заключалось противоречие, то оно имело бы место до и вне воплощения; оно вытекало бы из соединения провидения, любви с божеством. Ибо если это любовь истинная, то она ничем существенно не отличается от нашей любви --- стоит только устранить границы. Следовательно, воплощение есть только самое сильное, самое искреннее, самое чувственное, самое сокровенное выражение этого провидения, этой любви. Высшее счастье для объекта любви заключается в том, что любовь радует его своим личным присутствием, что она позволяет себя созерцать. Любовь питает пламенное желание видеть незримого благодетеля лицом к лицу. Созерцание есть божественный акт. Простой взгляд на любимое существо доставляет нам блаженство. Взгляд есть залог любви. И воплощение есть не что иное, как несомненный залог любви бога к человеку. Любовь вечна, воплощение же однажды свершилось на земле. Явление бога на землю было ограничено пространством и временем, и свидетелями его были лишь немногие, но сущность этого явления носит вечный и всеобщий характер. Мы должны верить в воплощение, но не ради самого акта, а ради его сущности, ибо нам только осталось созерцание любви.

Человек является в религии божественным объектом, божественной целью, и, следовательно, в религии выражается его отношение к самому себе, к своей собственной сущности. Самым очевидным и неопровержимым доказательством этого служит любовь бога к человеку --- основа и средоточие религии. Ради человека бог отказывается от своей божественности. В этом и заключается возвышающее значение воплощения: высшее, самодовлеющее существо смиряется, унижается ради меня. В боге поэтому мое собственное существо доходит до моего созерцания; для бога я имею ценность; божественное значение моего существа открывается для меня. Ведь высокое значение человека нагляднее всего выражается в том, что бог становится человеком ради человека, что человек служит конечной целью, предметом божественной любви. Любовь бога к человеку есть существенное определение божественного существа. Бог есть существо, любящее меня и людей вообще. На этом покоится сила религии, её основной пафос. Любовь бога побуждает меня к любви; любовь бога к человеку есть основание любви человека к богу; божественная любовь вызывает, пробуждает любовь человеческую. <<Будем любить его, ибо он прежде возлюбил нас>>\ddag\let\svthefootnote\thefootnote\let\thefootnote\relax\footnotetext{\ddag1 посл. Иоанна, 4, 19.}\let\thefootnote\svthefootnote. Итак, что же люблю я в боге? Любовь, именно любовь к человеку. Но если люблю я любовью, какою бог любит человека, то, следовательно, я люблю человека, и моя любовь к богу является косвенной любовью к человеку. Итак, если бог любит человека, значит, человек есть содержание бога. Я люблю только то, с чем я тесно связан. Есть ли у меня сердце, раз я неспособен любить? Нет! Сердце человека проявляется только в любви. А любовь немыслима без предмета любви. Предмет моей любви есть мое сердце, мое внутреннее содержание, моя сущность. Почему человек, потерявший любимое существо, сокрушается и теряет охоту жить? Потому что в лице любимого существа он потерял свое сердце, источник своей жизни. Если, таким образом, бог любит человека, значит, человек есть сердце божие, благо человека --- его главный помысел. Следовательно, человек, будучи объектом бога, является в то же время своим собственным объектом в боге. Сущность бога есть человеческая сущность, так как бог есть любовь, а существенное содержание этой любви есть человек. Любовь бога к человеку, составляющая основу и средоточие религии, есть любовь человека к самому себе, объективированная и созерцаемая как высшая истина, как высшая сущность человека. Поэтому догмат <<бог любит человека>> есть ориентализм, а религия по существу --- восточного происхождения; в переводе этот догмат гласит: <<наивысшее есть человеческая любовь>>.


Истина, к которой сводится посредством анализа тайна воплощения, проникла даже в религиозное сознание. Так, например, Лютер говорит: <<Кто хочет следовать примеру воплотившегося бога, тот должен ради плоти и крови, находящейся одесную отца, любить всякую плоть и кровь здесь на земле и никогда не враждовать с людьми. Человеческая кротость Христа, нашего бога, должна наполнять сердца людей радостью и изгонять оттуда все злые, недоброжелательные помыслы. Ради этой нашей плоти и крови человеку следует носить своих ближних на руках>>. <<Этот акт должен наполнять нас радостью и блаженной гордостью, потому, что он ставит нас превыше всякой твари, даже превыше ангелов, чем мы, поистине, можем гордиться: моя собственная плоть и кровь восседает одесную бога и правит всем. Такая честь не выпадает на долю никакой твари, даже ангела. Это должно быть для нас такой хлебопекарной печью, которая сплавила бы наши сердца в одно общее сердце и воспламенила бы горячей любовью к ближним>>*\let\svthefootnote\thefootnote\let\thefootnote\relax\footnotetext{*Лютер, ч. \rom{15}, стр. 44.}\let\thefootnote\svthefootnote. Но что в религиозной истине составляет сущность мифа, самое ядро, то в религиозном сознании является лишь моралью мифа, только чем-то второстепенным.





\chapter{Тайна страдающего бога}


Существенным определением вочеловечившегося или, что то же, человеческого бога, то есть Христа, являются страсти господни. Любовь обнаруживает себя в страдании. Все мысли и чувства, ближайшим образом связанные с Христом, сосредоточиваются в понятии страдания. Бог, как бог, олицетворяет собой все человеческие совершенства; бог как Христос воплощает в себе все человеческие страдания. Языческие философы считали высшей, божественной деятельностью деятельность вообще и, в особенности, самодеятельность разума; христиане считали страдание священным, приписали его самому богу. Бог, как actus purus, как чистая деятельность, есть бог отвлеченной философии; а христианский бог, Христос, passio pura, чистое страдание, есть высшая метафизическая идея, \^{e}tre supr\'eme\footnote{высшая сущность (фр.).} сердца. Ничто не производит такого сильного впечатления на сердце, как страдание, а тем более страдание существа, свободного от страданий, возвышающегося над всяким страданием, страдание безгрешного, невинного, который превыше всех страданий, страдание ради блага других, страдание любви, самопожертвования. История страданий

Христа сильнейшим образом трогает всякое человеческое сердце, или вообще сердце (ведь смешна попытка представить себе другое сердце, кроме человеческого), а это служит неопровержимым доказательством того, что здесь выражается, объективируется сущность сердца, что эти страдания являются изобретением не человеческого разума или фантазии, а человеческого сердца. Но сердце созидает иначе, чем свободное воображение или разум; оно страдает, воспринимая; все, исходящее от него, кажется ему данным извне, насильственным, действующим с неотразимой необходимостью. Сердце покоряет, овладевает человеком; всякий, кто охвачен им, чувствует его демоническую, божественную силу. Сердце не знает другого бога, другой высшей сущности, кроме самого себя, кроме того бога, который хотя может отличаться от него по имени, но сущность, субстанцию которого составляет собственная сущность сердца. Высшая, истинная, очищенная от теологических элементов и противоречий сущность христианства вытекает именно из сердца, из внутренней потребности к добру, к страданию и смерти ради людей, из божественного побуждения к благодеяниям, желающего осчастливить всех, не исключая даже самого отверженного и презренного, из нравственного долга благотворения в высшем смысле, когда оно стало внутренней необходимостью, то есть сердцем, из человеческой сущности, которая раскрывается как сердце и через сердце.

То, что в религии является предикатом, приобретает в силу вышесказанного значение субъекта, и наоборот. Поэтому мы можем перевернуть религиозные изречения, представить их себе как contr\'everit\'es\footnote{противоположное истине (фр.).}, и таким образом обретем истину. Бог страдает, --- страдание есть предикат, --- но страдает он за людей, за других, а не за себя. Что это по нашему значит? Только то, что страдание ради других божественно; кто страдает за других, полагает за них свою душу, поступает по-божески, является богом для людей*\let\svthefootnote\thefootnote\let\thefootnote\relax\footnotetext{*религия говорит примерами. Пример есть закон в религии. Что делал Христос, есть закон. Христос страдал за других, следовательно, и мы должны делать то же. <<Лишь для того господь так унижал, так умалял себя, чтобы и вы делали то же>> (Бернард, Nat. Domini). <<Мы должны ревностно следовать примеру Христа\dots Его пример должен побуждать нас от всего сердца служить и помогать другим людям, хотя бы это и было трудно и нам пришлось бы пострадать за это>> (Лютер, ч. \rom{15}, стр. 40).}\let\thefootnote\svthefootnote.


Но страдание Христа символизирует не только нравственное, самочинное страдание любви, силу, жертвующую собой за благо других; оно символизирует ещё страдание, как таковое, поскольку оно служит выражением способности страдания вообще. Христианская религия отнюдь не сверхчеловечна; она освящает даже человеческие слабости. Языческий философ, узнав о смерти своего ребенка, восклицает: <<Я знал, что произвел на свет смертного>>\footnote{речь идет о Ксенофонте Афинском, ученике Сократа.}; а Христос --- по крайней мере библейский, ибо об ином, добиблейском или небиблейском Христе мы ничего не знаем, --- оплакивает мнимую кончину Лазаря. Сократ с невозмутимым спокойствием выпивает чашу, наполненную ядом, а Христос, напротив того, молится: <<Если возможно, да минует меня чаша сия>>\dag\let\svthefootnote\thefootnote\let\thefootnote\relax\footnotetext{\dag<<Многих, --- говорит св. Амвросий, --- смущают эти слова. Я же изумляюсь здесь, как нигде, смирению и величину Христа, ибо он был бы менее для меня полезен, если б не принял на себя мой аффект>> (Expos. in Lucae Ev., lib. \rom{10}, c. 22). <<Разве мы дерзнули бы приблизиться к богу, если бы он был не способен к страданию (in sua impassibilitate)>> (Бернард, Tract. de \rom{12} grad. humil. et superb.). <<Хотя, --- говорит христианский врач И. Милихий, друг Меланхтона, --- смешно с точки зрения стоиков приписывать богу чувства и душевные движения (affectus), однако родителям, оплакивающим несчастье своих детей, следовало бы помнить, что и бог чувствовал подобную же любовь к своему сыну и к вам\dots любовь истинную, а не холодную или притворную>> (declam. Melanchtн, ч. \rom{2}, стр. 147).}\let\thefootnote\svthefootnote\footnote{см.: Матф., 26, 39.}. В этом отношении Христос есть самопризнание человеческой чувствительности. Христианин, в противоположность языческому, в частности стоическому, началу с его суровой силой воли и самостоятельностью внес в сознание бога сознание собственной возбудимости и чувствительности. Христианский бог не отрицает, не проклинает человеческих слабостей, если только они не греховны.

Страдание есть высшая заповедь христианства --- сама история христианства представляет историю страданий человечества. У язычников к культу богов примешивалось чувственное ликование; у христиан, разумеется у древних христиан, к богослужению присоединялись слезы и сердечные воздыхания. Но как чувственному богу, богу жизни оказываются почести там, где чувственный крик радости входит в состав культа, где это чувственное ликование служит чувственным определением сущности богов, вызывающих это ликование, так и сердечные вздохи христиан исходят из глубины сердца, из внутренней сущности их бога. Истинного человеческого бога надо искать не в боге софистической теологии, а в боге культа, у христиан культа внутреннего. А христиане, разумеется древние христиане, считали высшей формой религиозного почитания слезы, слезы раскаяния и тоски. Таким образом, слезы --- чувственный апогей христианской религиозной души, в которых отражается сущность её бога. Бог, находящий удовольствие в слезах, выражает собой сущность сердца и особенно чувства. Правда, христианская религия учит, что Христос сделал все для нас: спас нас, примирил нас с богом. Из этого можно было бы сделать вывод: радуйтесь; нечего больше заботиться о примирении с богом, ведь примирение уже состоялось. Но продолжительные страдания производят более сильное и прочное впечатление, чем однократный акт спасения. Спасение есть только результат страдания; страдание --- основа спасения. Поэтому страдание глубже укрепляется в душе; страдание, а не спасение, делается предметом подражания. Если сам бог страдал за меня, то я не могу радоваться и наслаждаться, по крайней мере в этом испорченном мире, служившем ареной его страданий\ddag\let\svthefootnote\thefootnote\let\thefootnote\relax\footnotetext{\ddag<<Мой бог распят на кресте, могу ли я предаваться сладострастию?>> (Form. нon. vitae. В числе апокрифических сочинении Св. Бернарда). <<Мысль о Распятом пусть и в тебе распинает твою плоть>> (Иоган Бергард, Medit. sacrae. Med. 37).}\let\thefootnote\svthefootnote. Неужели я лучше бога? Могу ли я не разделять его страданий? Разве поступки моего бога и господа не должны служить для меня примером? Имею ли я право пользоваться только прибылью, не неся расходов? Разве мне только известно, что он меня искупил? Разве история страданий Христа не стала для меня также объектом? Неужели она является для меня только объектом холодного воспоминания или даже предметом радости, ибо эти страдания доставили мне блаженство? Но кто может допустить такую мысль, кто захочет исключить себя из страданий своего бога?



Христианская религия есть религия страдания*\let\svthefootnote\thefootnote\let\thefootnote\relax\footnotetext{*<<Претерпевать зло несравненно лучше, чем делать добро>> (Лютер, ч. \rom{4}, стр. 15).}\let\thefootnote\svthefootnote. Образ распятия, который мы и поныне встречаем во всех храмах, представляет нам не спасителя, а распятого страдальца. Даже распространенное среди христиан самораспинание является психологически глубоко обоснованным следствием их религиозного миросозерцания. Тот, кто постоянно носит в душе образ распятого, естественно чувствует потребность распинать себя или других. По крайней мере, мы имеем такое же право сделать это заключение, какое имели Августин и другие отцы церкви, укорявшие языческую религию за то, что непристойные религиозные изображения язычников будто побуждают их на непристойности.



Бог страждет --- значит, в сущности, что бог есть сердце. Сердце есть источник, средоточие всех страданий. Кто не страдает, у того нет сердца. Поэтому тайна страдающего бога есть тайна чувства. Страдающий бог есть бог чувствующий, чувствительный\dag\let\svthefootnote\thefootnote\let\thefootnote\relax\footnotetext{\dag<<Страдать пожелал он, чтоб научиться состраданию, стать жалким, чтоб научиться жалости>> (Бернард, De grad). <<Сжалься над нами, ибо Ты сам познал слабость плоти чрез собственное страдание>> (Климент Александрийский, Paedag, lib. \rom{1}, c. 8).}\let\thefootnote\svthefootnote. Но положение: бог есть существо чувствующее, есть только религиозное выражение мысли: чувство --- божественного происхождения.



Человек сознает в себе не только источник деятельности, но и источник страданий. Я ощущаю и ощущаю чувство, как однородное с моей сущностью, не только волю и мышление, которое очень часто противоречит мне и моим ощущениям; я сознаю также, что хотя оно есть источник страданий, слабости и горя, но в то же время я ощущаю его как величественную божественную силу и совершенство. Что такое человек, лишенный чувства? Чувство есть музыкальная сила в человеке. Что такое был бы человек без звука? Человек, чувствующий влечение к музыке и испытывающий потребность выражать свои чувства в звуках и песне, неизбежно изливает в религиозных воздыханиях и слезах сущность чувства как объективную, божественную сущность.

Религия есть рефлекс, отражение человеческой сущности в себе самой. Все существующее, естественно, должно нравиться себе, радоваться себе, любить себя и любить по праву. Порицание за любовь к себе равносильно порицанию за существование. Существовать --- значит утверждать, отстаивать себя, любить себя; тот, кому надоело жить, лишает себя жизни. Поэтому там, где чувство не отодвигается на задний план, не подавляется по примеру стоиков, где ему предоставляется возможность существовать, там ему присваивается религиозная сила и значение, там оно возвышается до той ступени, где оно отображается и рефлектирует в самом себе, заглядывает в бога, в свое собственное зеркало. Бог есть зеркало человека.

Человек считает богом только то, что имеет для него существенное значение, кажется ему совершенным, превосходным и доставляет ему истинное наслаждение. Если чувство кажется тебе превосходным, то оно и есть для тебя божественное качество. Поэтому чувствующий, впечатлительный человек верит только в чувствующего, отзывчивого бога, то есть он верит только в истинность своего собственного бытия и сущности, ибо он может верить только в то, что он есть сам в своей сущности. Его вера является сознанием того, что для него священно, а священно для человека только то, что составляет его внутреннее, собственное, последнее основание, саму суть его личности. Бесчувственный бог кажется исполненному чувств человеку бессодержательным, абстрактным, отрицательным богом, то есть он для него ничто, ибо в нем нет того, что для человека ценно и священно. Бог служит для человека летописью его возвышенных побуждений и помыслов, родословной книгой, где записаны имена самых дорогих и священных для него существ.

Отличительным признаком хозяйственности, домовитости, женственности является стремление собирать и сохранять все ценное, вместо того, чтобы доверить его волнам забвения, случайности воспоминания, вообще предоставить самому себе. Свободный мыслитель подвергается опасностям расточительной, рассеянной, разнузданной жизни, а религиозный человек, связывающий все воедино, не теряется в чувственной жизни, но зато подвергается опасности реакции, духовного эгоизма и корыстолюбия. Поэтому религиозный человек считает безбожного или, по крайней мере, нерелигиозного человека субъективным, своевольным, высокомерным, дерзким, но не потому, что для одного не священно по себе то, что священно для другого, а лишь потому, что нерелигиозный человек лишь запечатлевает у себя в уме то, что религиозный ставит как предмет вне себя и над собой, благодаря чему останавливает для себя отношение формальной подчиненности. Одним словом, религиозный человек имеет свою летопись, средоточие, цель и, следовательно, твердую почву под ногами. Не воля, как таковая, не просто знание, а лишь целесообразная деятельность, объединяющая теоретическую и практическую деятельность, дает человеку нравственную основу и выдержку, то есть характер. Каждый человек должен иметь бога, то есть преследовать какую-нибудь цель. Конечная цель есть сознательное, добровольное, существенное жизненное стремление, взор гения, светоч самопознания --- единство природы и духа в человеке. Кто имеет конечную цель, тот повинуется закону, ибо он не только руководит сам собой, но и подчиняется руководству. У кого нет конечной цели, у того нет родины, нет святыни. Отсутствие цели есть величайшее несчастье. Человек, преследующий даже самые обыденные цели, счастливее того, кто, быть может, и лучше его, но не имеет цели. Цель ограничивает, но ограничение есть наставник добродетели. Человек, имеющий цель, такую цель, которая истинна и существенна сама по себе, имеет тем самым религию, если и не в ограниченном смысле богословской черни, то во всяком случае (что и требуется) в смысле разума, в смысле истины.




\chapter{Тайна триединства и матери божией}


Если человек, как существо чувствующее и страдающее, не удовлетворяется богом бесчувственным, не способным к страданию, то он не удовлетворяется также и существом, только чувствующим, существом, лишенным разума и воли. Полноценный человек может удовлетвориться только таким существом, которое носит в себе всего человека. Сознание человеком себя в своей цельности есть сознание троичности, троица соединяет в единство определения или силы, которые дотоле рассматривались как разъединенные, и тем низводит всеобщую сущность разума, то есть бога, как бога, на степень особой сущности, особой способности.

То, что богословие определяет как отображение, образ, подобие троицы, мы должны рассматривать исключительно как саму вещь, как сущность, первообраз, оригинал, и тогда загадка будет решена. Мнимые образы, в которых олицетворялась троица, чтобы стать понятной, суть: дух, рассудок, память, воля, любовь (mens, intellectus, memoria, voluntas, amor или caritas).

Бог мыслит и любит, но мыслит он и любит себя; все мыслимое, познаваемое, любимое есть сам бог. Первое, что мы встречаем в троице, есть объективирование самосознания. Самосознание неизбежно, невольно действует на человека как нечто абсолютное. Бытие для него равносильно самосознанию. Бытие с сознанием есть для него просто бытие. Не существовать или существовать не зная того, что существуешь, --- это одно и то же. Самосознание имеет абсолютное значение для человека и само по себе. Бог, не знающий себя, бог без сознания, не есть бог. Человек одинаково не может мыслить как себя, так и бога без сознания. Божественное самосознание есть не что иное, как сознание сознания, как абсолютной или божественной сущности.

Впрочем, троичность никоим образом не исчерпывается этим определением. Мы поступили бы вполне произвольно, если бы свели только к нему и ограничили только им тайну троичности. Сознание, рассудок, воля и любовь в значении абстрактных сущностей или определений относятся к области только абстрактной философии. Но религия есть сознание человеком себя в своей живой цельности, в которой единство самосознания существует только как завершившееся единство <<Я>> и <<ты>>.

Религия, по крайней мере христианская, абстрагирует от мира; внутренняя жизнь составляет её существо. Религиозный человек ведет жизнь уединенную, сосредоточенную в боге, тихую, лишенную радостей мира. Но он отчуждается от мира только потому, что сам бог есть существо, отчужденное от мира, вне и сверхмировое, --- выражаясь строго и абстрактно философски, --- есть небытие мира. Однако бог как внемировое существо есть не что иное, как сущность человека, свободная от всяких связей и соотношений с действительностью, возвысившаяся над миром, реализованная и рассматриваемая как объективная сущность. Такой бог есть сознание способности абстрагировать себя от всего другого, довольствоваться собой и зависеть только от себя, подобно тому как в религии эта способность становится объектом для человека в качестве отличной от человека, особого существа*\let\svthefootnote\thefootnote\let\thefootnote\relax\footnotetext{*<<Существо божие стоит вне всех тварей, подобно тому как бог от вечности был в себе самом; поэтому отврати любовь свою от тварей>>. (Иог. Бергард. Гергард, Medit. sacrae. Med. 31). <<Если ты хочешь обресть творца тварей, ты должен отречься от тварей\dots Чем меньше тварей, тем больше бога. Поэтому гони от себя всех тварей со всей их утехой>> (И. Таулер, Postilla. 1621, стр. 312). <<Если человек не может в сердце своем правдиво сказать: бог и я только и существуем в мире, и ничего иного не существует, --- то нет ещё покоя в нем>> (Г. Арнольд, Von Verscнmaeнung der Welt. Waнre Abbild der ersten Cнristen, кн. 4, гл. 2, \S 7).}\let\thefootnote\svthefootnote. Бог как бог, как существо простое есть существо одинокое --- абсолютное одиночество и самостоятельность, ибо одиноким может быть только то, что самостоятельно. Способность к одиночеству есть признак характера и мыслительной способности. Одиночество есть потребность мыслителя, общение --- потребность сердца. Мыслить можно одному, любить надо непременно другого. В любви мы зависимы, так как она есть потребность в другом существе; мы самостоятельны только в одиночном акте мышления. Одиночество есть независимость, самодовольство.

Но одинокий бог исключает существенную потребность всякой двойственности, любви, общения, действительного, полного самосознания --- у него нет другого <<Я>>. Религия удовлетворяет эту потребность тем, что соединяет одинокое божественное существо с другим, вторым, отличающимся от него как личность, но однородным с ним по существу --- с богом-сыном, отличным от бога-отца. Бог-отец --- это <<Я>>, бог-сын --- <<ты>>. <<Я>> --- это рассудок, <<ты>> --- любовь. Любовь в связи с рассудком и рассудок в связи с любовью образуют впервые дух, а дух есть цельный человек.

Только общественная жизнь есть истинная, себе довлеющая, божественная жизнь --- эта простая мысль, эта естественная для человека, врожденная истина и составляет сверхъестественную тайну троичности. Но эта истина, как и всякая другая, признается религией только косвенным, то есть превратным, образом; религия рассматривает эту всеобщую истину как частную и придает подлинному субъекту значение только предиката. Она говорит: бог есть общественная жизнь, жизнь и сущность любви и дружбы. Третье лицо в троице служит лишь выражением обоюдной любви двух первых лиц божества, есть единство отца и сына, понятие общения, которое довольно нелепо воплощается в свою очередь в особом, личном существе.

Святой дух обязан своим личным существованием только одному имени, одному слову. Даже древние отцы церкви, как известно, отождествляли св. дух с сыном. Позднейшие догматические олицетворения его тоже недостаточно обоснованы. Он есть любовь бога к самому себе и людям и в то же время любовь человека к богу и к человеку. Следовательно, он есть тождество бога и человека, поскольку оно объективируется в религии как особенная сущность для человека. Но для нас это единство заключается уже в отце и ещё в большей степени в сыне. Поэтому нам не придется делать св. дух особым предметом нашего анализа. Ещё одно замечание. Поскольку св. дух представляет субъективную сторону, постольку он является представителем религиозной души перед ней самой, представителем религиозного эффекта, религиозного воодушевления, или олицетворением, объективированием религии в религии. Поэтому святой дух есть воздыхающее творение, томление твари по богу.

В действительности есть только два лица, ибо третье, как уже сказано, представляет собой только любовь. Это объясняется тем, что любовь в строгом смысле довольствуется двумя. Два --- это принцип и вместе с тем замен а множества. Предположение многих лиц уменьшило бы силу любви; она раздробилась бы. Но любовь и сердце тождественны, сердце не есть особая способность --- сердце есть человек, поскольку он любит. Поэтому второе лицо есть самоутверждение человеческого сердца как принципа двойственности, общественной жизни; сын есть теплота, отец есть свет, хотя свет является главным образом предикатом сына, ибо в сыне божество впервые становится ясным, отчетливым и понятным человеку. Однако вне зависимости от этого, так как отец представляет собой бога как бога, представляет собой холодную сущность разума, то мы можем приписать отцу свет как неземную сущность, а сыну --- теплоту как земную сущность. Бог-сын впервые согревает человека; здесь бог из предмета созерцания, безразличного чувства света становится предметом чувства, аффекта, воодушевления, восторга; по только потому, что сам сын есть не что иное, как горячая любовь, воодушевление. Бог, как сын, есть первоначальное воплощение, первоначальное самоотрицание бога, отрицание бога в боге, ибо сын есть конечное существо, так как он ab alio\footnote{от другого (лат.).} имеет причину, а отец беспричинен и существует a se\footnote{сам собой (лат.).}. Таким образом, второе лицо исключает существенное определение божества как бытия от себя самого. Но бог-отец сам рождает сына, следовательно, он добровольно отрекается от своей суровой, исключительной божественности; он умаляет, унижает себя, полагает себе начало конечности, бытия, обусловленного причиной; он становится в сыне человеком и прежде всего не по образу, а по существу. Через это впервые бог, только в качестве сына делается объектом человека, объектом чувства, сердца.

Сердце трогает только то, что вытекает из сердца. Из свойств субъективных впечатлений можно безошибочно заключить о свойстве объекта. Чистый, свободный рассудок отрицает сына, а рассудок, озаряемый сердцем, определяемый чувством, не отрицает его; напротив, находит в сыне глубину божества, ибо находит в нем чувство, чувство темное само по себе и потому являющееся человеку некоторой тайной. Сын действует на сердце, потому что истинным отцом сына божия является человеческое сердце\dag\let\svthefootnote\thefootnote\let\thefootnote\relax\footnotetext{\dagПодобно тому как женственность католицизма, --- в отличие от протестантизма, принципом которого является мужественный бог, мужественное чувство и сердце, --- выражается в матери божией.}\let\thefootnote\svthefootnote; и сам сын есть не что иное, как божественное сердце, то есть объективированное в качестве божественной сущности человеческое сердце.

Конечное, чувственное существо не может поклоняться такому богу, которому чужды сущность ограниченности, начало чувственности, чувство зависимости. Религиозный человек не может любить бога, не заключающего в себе сущности любви; и человек, вообще конечное существо, не может быть объектом бога, не заключающего в себе начала ограничения. Такому богу недостает смысла, разумения, участия к конечному. Как может бог быть отцом людей и любить их, как любят другие, подчиненные ему существа, если он не вмещает в себе самом подчиненного ему существа --- сына и не знает по собственному опыту, по отношению к самому себе, что такое любовь. Так, одинокий человек принимает гораздо меньше участия в семейных горестях другого лица, чем тот, кто сам живет семейной жизнью. Поэтому бог-отец любит людей в сыне и ради сына. Любовь его к людям есть производная любовь от любви его к сыну.

Два первых лица троицы являются сыном и отцом не в переносном, а в собственном смысле. Отец есть действительный отец по отношению к сыну; сын есть действительный сын по отношению к отцу, или к богу как отцу. Их существенное индивидуальное отличие заключается только в том, что один рождает, другой рождается. Если отнять у них это естественное, чувственное определение, то уничтожится их личное существование и реальность. Христиане, разумеется, древние христиане, которые едва ли признали бы в современных суетных, тщеславных, языческих христианах своих братьев во Христе, заменяли естественную, врожденную человеку любовь и единство единой религиозной любовью и единством; они отрицали действительную семейную жизнь, внутренние узы естественной любви, как нечто не божественное, не небесное, то есть на самом деле ничтожное. Но зато, взамен их, они имели в боге единого отца и единого сына, соединенных между собой глубочайшей любовью, той любовью, которую порождает только естественное родство. Поэтому тайна троичности была для древних христиан предметом высокого изумления, воодушевления и восторга, так как в предмете созерцания, в боге, они находили удовлетворение глубочайших человеческих потребностей, которые отрицались ими в действительной жизни\ddag\let\svthefootnote\thefootnote\let\thefootnote\relax\footnotetext{\ddag<<Назидательно рассматривать свойства и общение отца и сына, но самое назидательное содержится в их взаимной любви>> (Ансельм в <<Rixners Gescн. d. Pнil>>, \rom{2}. B., Anн., S. 18).}\let\thefootnote\svthefootnote.



Поэтому вполне в порядке вещей было и то, что в божественную семью, в союз любви между отцом и сыном включено было на небе ещё третье --- женское лицо; ибо лицо св. духа слишком неопределенно; скорее оно служит только поэтическим олицетворением взаимной любви отца и сына и поэтому не могло бы заменить этого третьего, восполняющего существа. Правда, Мария была поставлена между отцом и сыном не в том смысле, что отец с ней породил сына, так как христиане считали связь мужчины с женщиной чем-то порочным, греховным; но довольно и того, что наряду с отцом и сыном было выдвинуто материнское существо.

Нельзя не удивляться, почему мать считается чем-то греховным, то есть недостойным бога, в то время как сам бог является отцом и сыном. Если даже отец не есть отец в смысле естественного рождения, и рождение божественное должно отличаться от естественного, человеческого, то и в таком случае бог все-таки является по отношению к сыну отцом, отцом действительным, а не номинальным или аллегорическим. Кажущееся нам теперь столь странное понятие <<божией матери>> в сущности не более странно или парадоксально, чем понятие <<сын божий>>; оно не более противоречит всеобщим, отвлеченным определениям божества, чем представление отца и сына. Мария входит логическим звеном в состав троицы: она без мужа зачинает сына, которого отец рождает без жены*\let\svthefootnote\thefootnote\let\thefootnote\relax\footnotetext{*<<От отца он рождается всегда, а от матери родился однажды, он зачат от отца без пола, а от матери --- без супружества. У отца не было лона воспринимающей, а у матери --- объятий производящего>> (Августин, Serm. ad pop. стр. 272, гл. \rom{1}, ed. Bened., Antv. 1701).}\let\thefootnote\svthefootnote. Таким образом, Мария является внутренне необходимой антитезой к отцу в составе троицы. Женское начало мы имеем уже в сыне, если не в определенном лице и конкретно, то мысленно, в зачаточной форме. Сын божий кроткое, нежное, незлобивое, примиряющее существо, женственная душа бога. Бог, как отец, есть только производитель, принцип мужской самодеятельности; но сын рождается, не рождая сам, Deus genitus\footnote{Бог рожденный (лат.).}, страдающее, воспринимающее существо: сын получает от отца свое бытие. Сын зависит от отца, разумеется, не как бог, а как сын; он подчинен отческому авторитету. Таким образом, сын представляет женственное чувство зависимости в боге; сын невольно пробуждает в нас потребность в действительном женском существе\dag\let\svthefootnote\thefootnote\let\thefootnote\relax\footnotetext{\dagв иудейской мистике бог есть мужская, а св. дух --- женская первоначальная сущность, от полового сочетания обоих возник сын и вместе с ним мир (Gfroerer, Jaнrн. d. Н., \rom{1}. Abt., стр. 332--334). Также и гернгутеры называли святой дух матерью Спасителя.}\let\thefootnote\svthefootnote.





Сын --- я разумею естественного человеческого сына --- является сам по себе и для себя связующим звеном между мужественностью отца и женственностью матери; он как бы наполовину мужчина, наполовину женщина: у него ещё нет полного, строгого сознания самостоятельности, характеризующего мужчину, он чувствует больше влечения к матери, чем к отцу. Любовь сына к матери есть первая любовь мужского существа к женскому. Любовь мужа к жене, юноши к деве приобретает свою религиозную, свою единственно подлинно религиозную, окраску в любви сына к матери. Любовь сына к матери есть первое страстное желание, первое проявление смирения мужчины перед женщиной.

Поэтому мысль о сыне божием необходимо связана с мыслью о матери божией, --- то же самое сердце, которое нуждается в сыне божием, нуждается и в матери божией. Где есть сын, там должна быть и мать. Сын единороден отцу, мать единородна сыну. Не отец заменяет сыну мать, а сын заменяет её отцу. Мать необходима сыну; сердце сына есть сердце матери. Почему бог-сын вочеловечился только через посредство женщины? Разве всемогущий бог не мог явиться среди людей другим путем, не непосредственно как человек? Почему сын вселился в утробу женщины?\ddag\let\svthefootnote\thefootnote\let\thefootnote\relax\footnotetext{\ddag<<Ибо богу было бы не трудно и не невозможно послать и без матери сына своего в мир, но он пожелал воспользоваться для этого женщиной>> (Лютер, ч. \rom{2}, стр. 348).}\let\thefootnote\svthefootnote Потому что сын есть стремление к матери, потому что его женственное, любвеобильное сердце нашло соответствующее выражение только в женском тело. Хотя сын, как естественный человек, находится под сердцем женщины только девять месяцев, но он получает здесь неизгладимые впечатления; мать навсегда остается в уме и сердце сына. Поэтому, если почитание сына божия не есть идолопоклонство, то и почитание матери божией не есть идолопоклонство. Если любовь бога к нам познается из того, что он, ради нашего спасения, предал на смерть своего единородного сына, то есть самое дорогое и возлюбленное, что было у него, то эта любовь познается нами ещё в большей степени, если припишем богу материнское сердце. Высшая и глубочайшая любовь есть любовь материнская. Отец утешается после потери сына; в нем есть стоическое начало. Мать, напротив, безутешна, мать есть страдалица; но безутешность есть истинность любви.



Там, где падает вера в матерь божию, падает и вера в бога-сына и богаотца. Отец только там есть истина, где и мать есть истина. Любовь сама по себе женственна по происхождению и по существу. Вера в любовь божию есть вера в женское начало как божественную сущность*\let\svthefootnote\thefootnote\let\thefootnote\relax\footnotetext{*в самом деле любовь к женщине есть основание всеобщей любви. Кто не любит женщины, не любит человека.}\let\thefootnote\svthefootnote. Любовь без естества есть бессмыслица, фантом. В любви познается священная необходимость и глубина природы!



Протестантизм отодвинул матерь божию на задний план\dag\let\svthefootnote\thefootnote\let\thefootnote\relax\footnotetext{\dagв Книге Согласия (Erkl\"ar., Art. 8) и в Апол. Аугсбург. Исповед. Мария называется <<преславной девой, истинной матерью божией и приснодевой, достойной всяческой хвалы>>.}\let\thefootnote\svthefootnote, но отверженная женщина жестоко отомстила ему за себя. Оружие, употреблённое им против матери божией, обратилось против него самого, против сына божия, против всей троицы. Кто однажды принес в жертву рассудку матерь божию, тот может принести ему в жертву и тайну сына божия, как антропоморфизм. Исключение женского существа скрывает антропоморфизм, но только скрывает, а не уничтожает его. Правда, протестантизм не нуждался в небесной женщине, так как он с распростёртыми объятиями принял в свое сердце земную женщину. Но он должен быть последователен и мужествен до конца и отвергнуть вместе с матерью также отца и сына. Небесные родители нужны только тому, у кого нет земных. Триединый бог есть бог католицизма, он имеет внутреннее, необходимое, истинно религиозное значение только как противовес отрицанию всех естественных связей, как противовес институту отшельников, монахов и монахинь\ddag\let\svthefootnote\thefootnote\let\thefootnote\relax\footnotetext{\ddag<<Монах пусть будет, как Мелхиседек, без отца, без матери без предков, и пусть никого не называет он на земле своим отцом пусть думает он о себе так, как будто существуют только он и бог>>. Specul. Monacн. (Pseudobernard). <<Священник пусть будет по примеру Мелхиседека без отца и без матери>> (Амвросий, где-то).}\let\thefootnote\svthefootnote. Триединый бог полон содержания, поэтому он нужен тому, кто абстрагирует себя от содержания действительной жизни. Чем бессодержательнее жизнь, тем полнее, тем конкретнее бог. Бог преисполняется по мере того, как опустошается действительный мир. Только бедный человек имеет богатого бога. Бог возникает из чувства недостатка: бог есть то, чего определенно, то есть сознательно или бессознательно, недостает человеку. Так, безотрадное чувство пустоты и одиночества нуждается в боге, который для него есть общество, союз искренне любящих друг друга существ.


В этом заключается истинное объяснение, почему троица в новейшее время утратила начала свое практическое, а в конце концов и свое теоретическое значение.




\chapter{Тайна логоса и подобия божия}



Существенное значение троичности для религии всегда сосредоточивается в существе второго лица. Христиане живо интересовались троичностью главным образом из-за интереса к сыну божьему*\let\svthefootnote\thefootnote\let\thefootnote\relax\footnotetext{*<<Христианин получил свое имя от Христа. Поэтому что не признает Христа своим господом и богом, не может быть безоговорочно христианином>> (Фульгенций, AdDonatum lib. unus). На этом же основании латинская церковь в противность греческой так твердо держится догмата, что св. дух исходит не только от отца, но вместе с тем и от сына. (См. об этом у И. Г. Валхия, hist. Contr. Gr. et Lat. de proc. Spir. S., Jenae 1751).}\let\thefootnote\svthefootnote. Горячий спор о homousios\footnote{единосущный (др.-гр. в латин. транскрипции).} и homoiusios\footnote{подобносущный (др.-гр. в латин. транскрипции).} не был пустым спором, несмотря на то, что вся разница заключалась в одной только букве. Здесь шла речь о равенстве с богом, о божественном достоинстве второго лица, о чести самой христианской религии: ведь специфической особенностью последней является именно второе лицо, а специфический объект всякой религии --- её истинный существенный бог. Истинный, подлинный бог религии есть вообще так называемый посредник, ибо он и служит непосредственным объектом религии. Всякий, кто вместо бога обращается к святому, обращается к нему в предположении, что бог повинуется этому святому, находится в его руках и с готовностью исполняет его просьбы, то есть его желания и волю. Просьба есть средство проявлять под видом смирения и покорности свое господство и превосходство над другим существом. Существо, к которому я прежде всего обращаюсь, является для меня поистине первым существом. Я обращаюсь к святому не потому что он свят от бога, а потому что бог зависит от него и определяется, обусловливается его просьбами, то есть волею или сердцем святого. Различие, которое католические богословы делали между latria, dulia и hyperdulia\footnote{почитание, подобающее богу; почитание, подобающее святым; почитание, подобающее Богоматери (др.-гр. в латин. транскрипции).}, есть не что иное, как нелепый, необоснованный софизм. Словом, бог, заслонённый посредником, есть отвлеченное, пустое представление, представление или идея божества вообще; и роль посредника заключается не в примирении с этой идеей, а в её уничтожении, отрицании, ибо она не есть предмет для религии\dag\let\svthefootnote\thefootnote\let\thefootnote\relax\footnotetext{\dagэто особенно отчетливо выражено в вочеловечении. Бог отвергает, отрицает свое величие, могущество и бесконечность, чтобы стать человеком, то есть человек отрицает бога, который сам не есть человек, и утверждает только бога, утверждающего человека. <<Он отрёкся, --- говорит св. Бернард, --- от своего величия и могущества, но не от благости и милосердия>>. Таким образом, то, от чего он не отрекся, есть божественная благость и милосердие, то есть самоутверждение человеческого сердца.}\let\thefootnote\svthefootnote. Короче, бог, стоящий над посредником, есть не что иное, как холодный рассудок над сердцем, подобно судьбе, распоряжавшейся олимпийскими богами.







Только образ может владеть человеком и воодушевлять его, как существо чувствительное и чувственное. Образный, чувствительный, чувственный разум есть фантазия. Второе существо в боге, в действительности первое существо религии, есть объективированная сущность фантазии. Определения второго лица по преимуществу образны. И эти образы не вытекают из неспособности человека мыслить предмет иначе, как образно: это --- ложное толкование; образность объясняется тем, что сам предмет есть образ. Поэтому сын выразительно называется подобием божиим; его сущность в том, что он есть образ --- фантазия бога, видимая слава невидимого бога. Сын есть удовлетворенная потребность образного созерцания, объективированная сущность фантазии как абсолютной, божественной деятельности. Человек создает образ бога, то есть превращает отвлеченную сущность разума, сущность мыслительной способности в чувственный, объект или сущность фантазии\ddag\let\svthefootnote\thefootnote\let\thefootnote\relax\footnotetext{\ddagсамо собой разумеется, что подобие божие имеется и другое значение, именно, что сам личный, видимый человеки и есть бог. Но здесь образ рассматривается только как образ.}\let\thefootnote\svthefootnote. Но он переносит этот образ на самого бога, потому что, естественно, его потребность осталась бы неудовлетворенной, если бы этот образ казался ему не объективной истиной, а лишь субъективным образом, отличным от бога и созданным человеком. Да и на самом деле этот образ не есть созданный, произвольный образ: ведь в нем выражается необходимость фантазии, необходимость утверждать фантазию как божественную силу. Сын есть отблеск фантазии, любимый образ сердца, и поэтому он в противоположность богу как олицетворённой сущности абстракции, является только предметом фантазии, только объективированной сущностью фантазии*\let\svthefootnote\thefootnote\let\thefootnote\relax\footnotetext{*<<Вечный отец, --- говорит Меланхтон в своей книге ,,De Anima``, --- создает, созерцая себя, свой образ. Что мышлением создаются образы, это мы знаем и по себе самим. А так как бог пользуется нашими словами, то он и хотел показать, что сын создается мышлением>>. <<Бог захотел, --- говорит он далее, --- чтобы наши мысли были образами предметов, так как он желал, чтобы в нас были образы его самого. Ведь отец создает, созерцая себя, путем мышления, сына, который является образом вечного отца>>. Итак, что же другое объективируется в сыне божием, как не сила воображения, или фантазия?}\let\thefootnote\svthefootnote.





Из этого видно, как заблуждается догматическое умозрение, когда оно, упуская из виду внутренний генезис сына божия, как образа божия, объясняет сына как метафизическое ens\footnote{сущность (лат.).}, как мыслительную сущность, ибо сын есть удаление, отпадение от метафизической идеи божества, --- отпадение, которое религия естественно переносит на самого бога, чтобы оправдать это отпадение и не чувствовать его отпадением. Сын есть высший и последний принцип иконопочитания, ибо он есть образ божий; но образ неизбежно заступает место самого предмета. Почитание святого в образе есть почитание образа, как святого. Образ есть сущность религии там, где образ является существенным выражением, органом религии.

Никейский собор наряду с другими доказательствами правильности употребления Никон сослался, между прочим, на авторитет Григория Нисского, который говорит, что образ, представляющий жертвоприношение Исаака, всегда исторгал у него ручьи слез --- так живо рисовался ему этот священный рассказ. Но действие изображенного предмета не есть действие предмета, как такового, а только действие образа. Священный предмет есть только ореол, в котором образ скрывает свою таинственную силу. Религиозный предмет служит лишь предлогом для искусства или фантазии, чтобы беспрепятственно проявить свою власть над человеком. В религиозном сознании святость образа естественно и необходимо связывается со святостью предмета, но религиозное сознание не есть мерило истины. Церковь делала различие между образом и предметом, его вызывающим, и отрицала поклонение образу, но вместе с тем она, опять-таки невольно, по крайней мере косвенным образом, признавала истину и сама удостоверяла святость образа\dag\let\svthefootnote\thefootnote\let\thefootnote\relax\footnotetext{\dag<<Мы постановляем, чтобы святому образу господа нашего Иисуса Христа, как и св. Евангелию, воздавались честь поклонения и т. д.>> (Gener. Const. Concil. \rom{8}, Act. 10, can. 3).}\let\thefootnote\svthefootnote.



Но последним, высшим основанием иконопочитания является почитание образа божьего в боге. <<Отблеск бога>> есть восхищающий блеск фантазии, проявляющейся в видимых образах. Образ подобия божия есть образ образов не только во внутреннем, но и во внешнем смысле. Образа святых суть только оптическое умножение одного и того же образа. Умозрительная дедукция образа божия есть, поэтому только бессознательная дедукция и обоснование иконопочитания, ибо санкция принципа естественно является санкцией его неизбежных последствий. Но санкция первообраза есть санкция отображения. Если бог имеет свой образ, то почему мне не иметь образа божия? Если бог любит свой образ, как самого себя, почему я не могу любить образ бога, как самого бога? Если образ бога есть сам бог, почему образ святого не есть сам святой? Если это не ость суеверие, что образ божий не есть образ, не есть представление, а есть сущность, есть личность, то почему же будет суеверием, что образ святого есть чувствующая сущность самого святого? Образ божий льет слезы и кровоточит, почему же образ святого не может лить слезы ли истекать кровью? Неужели это различие в том, что образ святого есть дело рук? Но ведь этот образ создается не руками, а духом, одушевившим эти руки, фантазией; и если бог сделал себе образ самого себя, то этот образ есть также только продукт силы воображения. Или, может быть, разница в том, что образ божий есть произведение самого бога, а образ святого создается Другим существом? Но ведь образ святого есть также самоосуществление святого, потому что святой является художнику, а художник изображает его таким, каким он сам представился ему.

Другое определение второго лица, находящееся в связи с сущностью образа, заключается в том, что сын есть слово божие\ddag\let\svthefootnote\thefootnote\let\thefootnote\relax\footnotetext{\ddagо значении слова logos в Новом завете много писалось. Мы придерживаемся здесь значения слова божьего как священного в христианстве. О логосе у Филона, см. Gfr\"oer. Филон употребляет вместо logos также \textgreek{ῤῆµα θεοῦ}, см. также у Тертуллиана adv. Praxeam c. 5, где он доказывает, что все равно, переводить ли logos словом sermo или ratio. Что слово верно передает смысл логоса, видно уже из того, что творение в Ветхом завете обусловлено повелением и в этом творческом слове уже давно усматривали logos. Конечно, logos имеет также смысл virtus, spiritus, силы, разума и т. д., ибо что такое слово без смысла, то есть без силы?}\let\thefootnote\svthefootnote\footnote{слово, речь; разум (лат.).}.



Слово есть отвлеченный образ, мнимый предмет или, поскольку всякий предмет непременно является предметом мысли, --- воображённая мысль. Поэтому люди, знающие слово, название предмета, соображают, что они также знают и самый предмет. Слово есть дело силы соображения; спящие говорят во сне, говорят также больные в бреду. То, что возбуждает фантазию, делает нас разговорчивыми, воодушевление сообщает нам красноречие. Дар слова есть поэтический дар; животные не говорят, потому что у них нет поэзии. Мысль выражается только в образах; способность выражения мысли обусловливается силою воображения; но сила воображения проявляется внешним образом в языке. Говорящий пленяет, очаровывает того, кому говорит; но сила слова есть мощь силы воображения. Поэтому древние народы, обладавшие неразвитым воображением, считали слово существом таинственным, магически действующим. Христиане, не только простые, но и ученые, отцы церкви связывали с простым именем Христа таинственную, спасительную силу*\let\svthefootnote\thefootnote\let\thefootnote\relax\footnotetext{*<<Сила имени Иисуса так велика против демонов, что оно действует даже и тогда, когда произносится нечестивыми>> (Origenes adv. Celsum, lib. \rom{1}. См. также lib. \rom{3}).}\let\thefootnote\svthefootnote. Простонародье и до сих пор верит в возможность заворожить человека одними словами. Чем же объясняется эта вера в воображаемую силу слова? Тем, что само слово есть только сущность силы воображения, почему оно и действует на человека подобно наркозу, отдавая его во власть фантазии. Слова обладают революционной силой; слова владеют человечеством. Предание считается священным, а дело разума и истины не пользуется доброй славой.



Поэтому утверждение или объективирование сущности фантазии связывается с утверждением или объективированном сущности языка, слова. У человека есть побуждение, потребность не только думать, чувствовать и воображать, но и говорить, выражать свои мысли, сообщать их другим. Эта потребность так же божественна, как божественна сила слова. Слово есть образная, откровенная, сияющая, блестящая, освещающая мысль. Слово --- свет мира. Слово вводит нас в полноту истины, открывает все тайны, являет невидимое, воспроизводит отдаленное прошлое, ограничивает бесконечное, увековечивает преходящее. Люди умирают, слово бессмертно; слово есть жизнь и истина. Слову дано всемогущество: оно исцеляет слепых, хромых и больных, воскрешает мертвых; слово творит чудеса и притом лишь чудеса разумные. Слово есть евангелие, параклит, утешитель человечества. Чтобы убедиться в божественной сущности языка, представь себе, что ты, одинокий и покинутый, впервые слышишь человеческую речь; разве она не покажется тебе ангельским пением, голосом самого бога, небесной музыкой? На самом деле слово нисколько не беднее, не бездушнее музыкального звука, хотя нам и кажется, что звук бесконечно выразительнее, глубже и богаче слова, так как его окружает этот призрак, эта иллюзия.

Слово обладает спасающей, искупляющей, благодетельной, освобождающей силой. Грехи, в которых мы каемся, отпускаются нам благодаря божественной силе слова. Примирённым отходит умирающий после исповеди в своих долго таимых греках. Сознание греха влечет за собой прощение греха. Наши страдания облегчаются наполовину, если мы делимся ими с другом. Наши страсти утрачивают свою остроту благодаря тому, что мы говорим о них; наступает просветление; предмет гнева, злобы, огорчения проявляется в таком свете, что ясно проступает непристойность страсти. Стоит нам раскрыть рот, чтобы спросить друга о чем-нибудь сомнительном и непонятном, как все сомнения и колебания разлетаются, точно дым. Слово делает человека свободным. Кто не умеет высказаться, есть раб. Поэтому чрезмерная страсть, чрезмерная радость, чрезмерное горе лишают нас языка. Речь есть акт свободы; слово --- сама свобода. Поэтому обогащение языка справедливо считается основой просвещения, где культивируется слово, там культивируется человечество. Варварство средних веков исчезло с развитием языка.

Как не можем мы в качестве божественной сущности представить себе ничего, кроме разумного, благого и прекрасного, которое мы мыслим, любим и чувствуем, так но знаем мы другой, более высокой, духовной силы, чем сила слова\dag\let\svthefootnote\thefootnote\let\thefootnote\relax\footnotetext{\dag<<Бог открывает нам, что on обладает вечным несотворённым словом, которым он сотворил мир и всё сущее, что это было для него, легким долом, именно только актом слова, что это было ему не труднее, чем нам называть вещи>> (Лютер, ч. \rom{1}, стр. 302).}\let\thefootnote\svthefootnote. Бог есть средоточие всей действительности, то есть всего существенного и совершенного. Все, что человек чувствует и познает как действительность, он должен полагать в боге или как бога. Поэтому религия должна признавать силу слова божественной силой. Слово божие есть божественность слова, поскольку она объективируется для человека в религии, --- истинная сущность человеческого слова. Слово божие якобы том отличается от человеческого, что оно является не преходящим дуновением, а сообщённой другим сущностью. Но разве человеческое, по крайней мере истинное человеческое слово, не заключает в себе сущности человека, его сообщаемого другим <<Я>>? Таким образом религия принимает видимость человеческого слова за его сущность; поэтому неизбежно она представляет себе истинную сущность слова как особую, отличную от человеческого слова сущность.





\chapter{Тайна миротворческого начала в боге}


Второе лицо, как открывающий себя, проявляющий себя, изрекающий о себе бог (deus se dicit), есть миротворяющее начало в боге.

Мир не есть бог, он есть нечто другое, противоположное богу, или, по крайней мере, если это выражение слишком сильно, поскольку вещь названа своим именем, --- нечто отличное от бога. Но то, что отличается от бога, не может происходить непосредственно от бога, а вытекает из различия в самом боге. Второе лицо есть сознающий себя бог, отличающий себя от себя в себе, противопоставляющий себя самому себе и поэтому являющийся своим собственным объектом. Самоотличение бога от себя самого есть основа всего того, что от него отличается; следовательно, первопричина мира есть самосознание. Бог мыслит мир прежде всего потому, что мыслит себя; мыслить себя --- значит рождать себя; мыслить мир --- значит творить мир. Рождение предшествует творению. Продуктивная идея мира, как сущности, отличной от бога, обусловливается продуктивной идеей другого существа, подобного богу.

Но этот миротворческий процесс есть не что иное, как мистическая перифраза психологического процесса, не что иное, как объективирование единства сознания и самосознания. Бог мыслит себя --- значит, он сознает себя; бог есть самосознание, представляемое как объект, как сущность; но, сознавая себя, мысля себя, бог вместе с тем мыслит нечто другое, чем он сам есть; ибо сознавать себя --- значит отличать себя от другого возможного, только представляемого или действительного. Поэтому и мир, по крайней мере, возможность, идея мира, связывается с сознанием, или скорее, обусловливается им. Сын, мыслимый от себя, объективированный, первообразный, другой бог, есть начало мироздания. Истина, лежащая в основе этого, есть сущность человека: единство его самосознания с сознанием чего-то другого, одновременно с ним тождественного и не тождественного. И второе, подобное ему по существу, неизбежно является посредником между первым и третьим. Мысль о другом, вообще отличном по существу, вытекает впервые из мысли о другом, подобном мне по существу.

Сознание мира есть сознание моей собственной ограниченности (если бы я не знал о мире, я ничего не знал бы об ограниченности); но сознание моей ограниченности противоречит стремлению моей личности к неограниченности. Я не могу непосредственно перейти от абсолютной личности (бог есть абсолютное я) к её противоположности; я должен смягчить, подготовить, сгладить это противоречие сознанием такого существа, которое, правда, есть другое существо и в силу этого обнаруживает мою ограниченность, но в то же время утверждает, объективирует мою сущность. Сознание мира есть смиряющее сознание; творение было <<актом смирения>>, но первым камнем преткновения, о который разбивается гордость личности, является <<ты>>, другое <<Я>>. Прежде чем вынести взгляд, не отражающий в себе образа нашего <<Я>>, <<Я>> должно закалить свой взор созерцанием <<ты>>. Другой человек является связующим звеном между мною и миром. Я завишу от мира и сознаю эту зависимость, ибо сперва я чувствую свою зависимость от другого человека. Если бы я не нуждался в людях, то не нуждался бы и в мире. Я примиряюсь, я сближаюсь с миром только посредством другого человека. Без другого лица мир казался бы мне не только мертвым и пустым, но и бессмысленным и неразумным. Человек уясняет и сознает себя только благодаря другому; а для того, чтобы понимать мир, я должен понимать самого себя. Человек, существующий исключительно ради себя, бессознательно и безразлично потерялся бы в океане природы; он не понимал бы ни себя как человека, ни природы как таковой. Первый объект человека есть человек. Понимание природы, влекущее за собой сознание мира как мира, относится к более позднему времени, так как оно возникает впервые чрез акт отделения человека от себя самого. Греческим натурфилософам предшествовали так называемые семь мудрецов, мудрость которых относилась только к человеческой жизни.

Следовательно, посредником между <<Я>> и сознанием мира является сознание второго лица. Так человек становится богом человека. Своим существованием он обязан природе, а тем, что он человек, --- человеку. Без другого человека он ничего не может сделать не только в физическом, но и в духовном отношении. Четыре руки могут сделать больше, чем две; две пары глаз видят больше, чем одна. И эта объединённая сила отличается от единичной не только количественно, но и качественно. Единичная человеческая сила ограничена; соединенная с другими, она бесконечна. Знание одного человека ограничено, но не ограничен разум, не ограничена наука, поскольку она есть общий акт человечества, и не только потому, что над созданием науки совместно работает бесчисленное множество людей, но и потому, что научный гений известной эпохи соединяет в себе определенным, индивидуальным образом умственные силы всех предыдущих гениев, благодаря чему его сила не является единичной. Остроумие, проницательность, фантазия, чувство, как нечто отличное от ощущения, разум --- все эти так называемые душевные силы суть силы всего человечества, а не отдельного человека; это --- продукты культуры, продукты человеческого общества. Огонь остроумия зажигается только там, где человек сталкивается с человеком, поэтому его больше в городе, чем в деревне, и в больших городах больше, чем в маленьких; чувство и фантазия возникают только там, где человек освещается и согревается близостью человека; поэтому любовь, акт общественный, порождающий величайшие страдания при отсутствии взаимности, является первоисточником поэзии; разум возникает только там, где человек говорит с человеком; он зарождается только в речи, акте общественном. Первые акты мышления были вопросы и ответы. Сначала мышление обусловливалось двумя лицами. На более высокой ступени культуры человек удваивается, и теперь он может одновременно играть роль первого и второго лица. Поэтому древние народы, жившие чувственными восприятиями, отождествляли мышление и слово. Они мыслили только тогда, когда говорили; их мышление состояло только в разговоре.

Простые люди, то есть люди, не получившие отвлеченного образования, и до сих пор не понимают написанного если они не читают вслух, не произносят читаемого громко. В этом отношении Гоббс вполне справедливо производит человеческий рассудок от слуха.

Космогоническое начало в боге, сведенное на отвлеченные логические категории, выражает собой только тавтологию: различное может произойти только из начала различия, а не из простой сущности. Христианские философы и богословы, защищавшие учение о сотворении мира из ничего, не могли совершенно обойти старого правила: <<из ничего не возникает ничего>>, ведь оно выражает закон мышления. Правда, они не считали началом различных материальных вещей действительную материю, но зато они сделали началом действительной материи --- божественный рассудок, который есть средоточие всех вещей, духовная материя, сын не есть мудрость, наука, рассудок отца. Различие между языческой вечностью материи и христианским творением заключается только в том, что язычники приписывали миру действительную, объективную, а христиане беспредметную вечность. Вещи существовали до начала века, но не как объект чувства, а как объект духа. Христиане, принципом которых был принцип абсолютной субъективности*\let\svthefootnote\thefootnote\let\thefootnote\relax\footnotetext{*уже во втором издании я стремился изгнать из своего труда иностранные слова, в том числе ненавистное мне слово, <<субъективность>>. Соответствующие ей слова: <<своеобразие>>, <<индивидуальность>>, <<Я>>; далее: <<душа>>, <<приятность>> или <<человечность>>, пожалуй, духовность, <<нечувственность>>. Для слова <<субъективность>> или <<субъективный>> совершенно тождественного выражения нет --- в смысле противоположности предметному, во всяком случае нет общеупотребительного выражения. То же самое приложимо и ко многим другим словам.}\let\thefootnote\svthefootnote, мыслили все только опосредствованным образом, пользуясь этим принципом. Поэтому материя субъективная, воображаемая, представляемая их субъективным мышлением, кажется им первоначальной материей, значительно превосходящей действительную, чувственную материю. Но тем не менее это различие есть только различие в способе бытия. Мир вечно пребывает в боге. Или, может быть, он возник в нем подобно внезапной затее, капризу? Конечно, можно думать и так, но в таком случае человек обожествляет только свое собственное неразумие. Если я сохраняю разум, то я могу выводить мир только из его сущности, из его идеи, то есть выводить один образ его существования из другого --- иначе говоря: я могу выводить мир только из него самого. Мир заключает свою причину в самом себе, как и все в мире, что претендует на название истинной сущности. Differentia specifica\footnote{различные признаки (лат.).}, своеобразная сущность, все, что сообщает существу определенный характер, есть нечто необъяснимое, непроизводное в обыкновенном смысле, что существует чрез себя и имеет свое основание в себе самом.


Различие между миром и богом как творцом мира --- только формальное, несущественное. Божественный рассудок, средоточие всех вещей, есть сама божественная сущность; поэтому бог, мысля и сознавая себя, одновременно мыслит и познает мир, полноту бытия. Существо божие есть не что иное, как отвлеченная, производная, мыслимая сущность мира; а сущность мира --- не что иное, как действительное, конкретное, чувственно созерцаемое существо божие; таким образом творение есть не что иное, как только формальный акт. Ведь все, что до начала творения было предметом мысли, рассудка, становится благодаря творению предметом чувства, но по содержанию пребывает неизменным, хотя и остается необъяснимым происхождение действительной, материальной вещи от предмета мысли\dag\let\svthefootnote\thefootnote\let\thefootnote\relax\footnotetext{\dagПоэтому объяснение мира бытием творца --- не более, чем самообман.}\let\thefootnote\svthefootnote.



То же можно сказать о множественности и различии, если мы возводим мир к этой отвлеченной категории в противоположность простоте и единству божественной сущности. Действительное различие может быть выведено только от существа, различного в самом себе. Но я отношу различие к первоначальному существу, так как для меня различие уже из начала есть истина и сущность. Если различие ничтожно само по себе, то оно немыслимо и в принципе. Я полагаю различие как сущность, как истину, там, где я вывожу его из первоначальной сущности, и наоборот: то и другое тождественно. Разумное выражение этого есть следующее: различие составляет такую же неотъемлемую принадлежность разума, как и единство.

Но если различие является существенным определением разума, то я могу выводить различие только из различия, объяснять его только им самим, потому что оно есть первоначальная, сама собой понятная, самое себя утверждающая вещь. Откуда произошел мир, отличный от бога? Из различения богом себя в самом боге. Бог мыслит себя, он служит объектом для себя, он отличает себя от себя --- так возникает это различие, мир --- из различия другого рода, внешнее из внутреннего, сущее из деятельного, из акта различения. Таким образом, я основываю различие только через самого себя: оно есть первоначальное понятие, граница моего мышления, закон, необходимость, истина. Последнее различие, которое я могу мыслить, есть различие существа от себя и в себе. Различие одного существа от другого понятно само собой, оно обусловливается их существованием, является очевидной истиной: их два. В мышлении же я обосновываю различие только тогда, если я приписываю его одному и тому же существу, если я связываю его с законом тождества. В этом заключается последнее истинное различие. Космогоническое начало в боге, сведенное к его последним основаниям, есть не что иное, как акт мышления, объективированный в своих простейших элементах. Если я отниму от бога различие, то у меня не будет материала для мышления; бог перестанет быть объектом мышления, ибо различение есть существенный принцип мышления. И поэтому, допуская в боге различие, я утверждаю, я объективирую истинность и необходимость этого принципа мышления.






\chapter{Тайна мистицизма или природы в боге}


Интересный материал для критики космогонических и теогонических измышлений представляет учение о вечной природе в боге, почёрпнутое из Якова Бёме и обновлённое Шеллингом.

Бог есть чистый дух, светлое самосознание, нравственная личность; природа, напротив, по крайней мере в некоторых частностях, беспорядочна, темна, мрачна, безнравственна или ненравственна. Но нечистое не может исходить из чистого, мрак из света. Как же произвести от бога эти очевидные противоречия в божественных свойствах? Для этого необходимо предположить эту нечистоту, этот мрак в боге, различать в самом боге начала света и мрака. Другими словами: мы только тем можем объяснить происхождение мрака, что вообще откажемся от представления о происхождении и допустим, что мрак существовал от века*\let\svthefootnote\thefootnote\let\thefootnote\relax\footnotetext{*в наши цели не входит критика этой грубой мистической теории. Мы только заметим, что тьма лишь тогда объяснима, когда она выводится из света, что объяснение тьмы в природе из света представляется невозможным только в том случае, если человек настолько слеп, что не видит света во тьме и не замечает, что в природе нет абсолютной тьмы, а есть только тьма умеренная, смягченная светом.}\let\thefootnote\svthefootnote.



Мрачная сторона природы заключается в иррациональном, материальном, в природе, как таковой, в отличие от разума. Простой смысл этого учения таков: природу, материю, нельзя объяснять разумом и выводить из разума; она скорее является основой разума, основой личности; сама же она не имеет основания; дух без природы есть только абстракция; сознание развивается только из природы. Но это материалистическое учение окутывается в мистический, хотя и спокойный мрак и выражается не ясными, простыми словами разума, а с помощью священного слова <<бог>>. Если свет в боге происходит из мрака, пребывающего в боге, то он происходит потому лишь, что свет обладает свойством освещать мрак и, следовательно, предполагает мрак, но не делает его. Если ты подчиняешь бога всеобщему закону (что необходимо, если ты не хочешь приписывать богу бессмысленного произвола), если самосознание в боге, как и сам осознание само по себе, обусловливается естественным началом, то почему ты не абстрагируешь его от бога? То, что является законом для сознания вообще, есть закон для сознания всякого личного существа, будь то человек, ангел, демон, бог или что угодно. К чему же сводятся по внимательном рассмотрении оба начала, существующие в боге? Одно сводится к природе, по крайней мере к природе в том виде, как она существует в твоем представлении, к природе, отвлеченной от действительности; другое --- к духу, сознанию, личности. Ты называешь бога богом не в силу его оборотной стороны, его изнанки, а в силу лишь его фасадной, лицевой стороны, куда он является тебе как дух, как сознание.

Следовательно, характерной сущностью его является то, что делает его богом, это то, что он дух, разум, сознание. Почему же ты сводишь то, что является истинным субъектом, в боге, то есть дух, к простому предикату, как будто бог есть бог помимо духа, помимо сознания? Потому, конечно, что ты мыслишь, как раб мистически религиозного воображения, и тебе приятно блуждать в обманчивом полусвете мистицизма.

Мистицизм есть толкование на основании намеков. Мистик размышляет о сущности природы или человека, но воображая при этом, что он созерцает другое, отличное от них обоих личное существо. Мистик исследует те же предметы, что и обыкновенный сознательный мыслитель, но действительный предмет представляется мистику воображаемым, а воображаемый --- действительным. Поэтому и здесь, в мистическом учении о двух началах в боге, действительный предмет --- патология, соображаемый --- теология; то есть патология обращается в теологию. Это не заслуживало бы возражения, если бы действительная патология сознательно и открыто признавалась теологией; наша задача в том и состоит, чтобы показать, что теология есть скрытая от самой себя, эзотерическая патоантропопсихология и что поэтому действительная патология, действительная антропология, действительная психология имеют больше права на название теологии, чем теология, которая на самом деле есть не что иное, как воображаемая психология и антропология. Но содержанием этого учения или созерцания должна быть не патология, --- и именно потому это учение есть мистическое и фантастическое, --- а теология, теология в старом или обыкновенном смысле слова; теология должна объяснять нам сущность другого, отличного от нас существа, но вместо этого она раскрывает нашу собственную сущность и снова её затемняет, утверждая, что она должна быть сущностью другого существа. Не разум человеческих индивидов (это было бы слишком тривиальной истиной), а разум бога должен сперва освободиться от страстей природы; не мы, а бог должен подняться из сферы смутных, неясных чувств и стремлений на высоту ясного познания; не в нашем способе представления, а в самом боге ужас ночи должен предшествовать радостному сознанию света; одним словом, в этом учении должна быть изображена не история развития болезни человечества, а история развития болезни бога, так как всякое развитие есть болезнь.

Миротворческий процесс как божественный процесс дифференциации обнаруживает перед нами свет силы различения как божественную сущность; ночь же или природа в боге представляют нам pens\'ees confuses\footnote{смутные мысли (фр.).} Лейбница, как божественные силы или потенции. Но смутные, темные представления и мысли, вернее, образы, выражают собой плоть, материю; чистому, отделенному от материи, разуму свойственны только ясные, свободные мысли; у него нет смутных, то есть плотских представлений, материальных, возбуждающих фантазию, волнующих кровь образов. Поэтому ночь в боге говорит только о том, что бог есть не только духовное, но и материальное, телесное, плотское существо. Но человек является и называется человеком не в силу своей плоти, а в силу своего духа; подобным образом и бог.

Однако мистическое учение выражает это лишь в смутных, неопределенных, двусмысленных образах. Вместо сильного и потому точного и острого выражения: плоть, оно употребляет неопределенные, отвлеченные слова: природа и основание. <<До бога не было ничего и вне бога нет ничего; следовательно, основание его бытия заключается в нем самом. Это говорят все философы; но они рассматривают это основание как простое понятие, не придавая ему ничего реального и действительного. Это основание бытия бога не есть сам бог, созерцаемый абсолютно, то есть бог сущий; ибо оно есть только основание его бытия. Оно есть природа в боге, неотделимая, но отличная от него сущность. Пояснить это отношение можно аналогией с отношением между силой тяжести и светом в природе>>\footnote{Фейербах цитирует сочинение Шеллинга <<Философские исследования о сущности человеческой свободы>> (1809 г.). Перевод дан по: Шеллинг Ф.В.Й. Соч. в 2-х т. Т. 2. М., 1989 г., с. 109.}. Но это основание есть неразумное начало в боге. <<То, что является началом разума (в нем самом), не может быть разумным>>. <<Из этого отдалённого от неразумного (Verstandlosen) порожден разум в собственном смысле. Без этого предварительного мрака не было бы реальности творения>>\footnote{см.: там же, с. 109.}. <<С помощью столь отвлеченных понятий о боге, таких, как actus purissimus\footnote{чистейший акт (лат.).}, вроде тех, которые выдвигала древняя философия, так же как и понятия новой философии, которая стремится удалить бога из природы, совершенно неудовлетворительны. Бог есть нечто более реальное, нежели простой моральный миропорядок, и несет в себе совсем иные и более движущие силы, чем те, которые приписывают ему абстрактные идеалисты с их скудоумной утонченностью\dots Идеализм, не основанный на живом реализме, приобретает значение такой же бессодержательной отвлеченной системы, как догматические системы Лейбница, Спинозы и любая другая\dots>>\footnote{см.: там же, с. 105--106.}. <<Пока бог современного теизма будет оставаться существом простым, но на самом деле лишенным всякой сущности, каким он является во всех новейших системах, пока мы не признаем в нем действительной двойственности и не противопоставим утверждающей, расширяющей силе силу ограничивающую, отрицающую, до тех пор отрицание личного бога будет выражением научной добросовестности>>\footnote{эта цитата и две последующие взяты из сочинения Шеллинга <<Памятка о ,,Божественно сущем`` г-на Ф.Г. Якоби>> (1812 г.). В русском переводе отсутствует.}. <<Всякое сознание есть сосредоточенность, напряжение, собранность в самом себе. Эта отрицающая, к себе самой возвращающаяся сила существа есть истинная сила его личности, сила его самоутверждения, его ,,Я``>>. <<Мы не боялись бы бога, если бы в нем не заключалась сила. Свойство божие, основанное исключительно на силе и крепости, не может казаться нам странным, если только мы не утверждаем, что бог есть лишь сила, а не другое что>>.

Но сила и крепость, являющаяся только силой и крепостью, есть не что иное, как только телесная сила и крепость? В отличие от силы добра и разума располагаешь ли ты другой силой, помимо силы мускулов? Если ты не можешь достичь своей цели при помощи доброты и разума, тебе остается прибегнуть к силе. А чего ты можешь <<достичь>>, без сильной руки и кулака? Помимо силы нравственного миропорядка тебе известна только одна <<более живая и деятельная сила>> --- сила уголовного суда. Природа, лишенная плоти, не есть ли <<бессодержательное, отвлеченное>> понятие, <<скудная утонченность>>. Тайна природы не есть ли тайна тела? Система <<живого реализма>> не есть ли система органического тела? Существует ли вообще другая, противопоставляемая разуму сила, кроме силы плоти и крови? Есть ли другая сила природы, кроме силы чувственного побуждения? Не есть ли половое стремление --- сильнейшее стремление природы? Кто не помнит древней пословицы: a mare et sapere vix Deo competit\footnote{даже бог не в состоянии, любя, сохранять разум (лат.).}? Если поэтому мы хотим воплотить в боге природу, сущность, противопоставляемую свету разума, то мы не можем представить себе более живой, более реальной противоположности, чем противоположность любви и мышления, духа и плоти, свободы и полового стремления. Ты пугаешься этих последствий и выводов, но они являются законными плодами священного брачного союза между богом и природой. Ты сам родил их под благодетельным покровом ночи; теперь я только показываю их тебе при свете дня.

Личность, <<эгоизм>>, сознание, лишенное природы, есть ничто или, что то же, есть нечто пустое, лишенное сущности, абстракция. Но природа, что доказано и само собой понятно, есть ничто без тела. Только тело является той отрицающей, ограничивающей, сосредоточивающей, суживающей силой, без которой немыслима личность. Отними у своей личности её тело --- и ты отнимешь у нее её спайку. Тело есть основание, субъект личности. Действительная личность отличается от воображаемой личности какого-нибудь привидения только телом. Мы были бы абстрактными, неопределенными, бессодержательными личностями, если бы нам не был присущ предикат непроницаемости, если бы на том же самом месте, под тем же самым обликом могли одновременно находиться другие. Реальность личности определяется ограничением в пространстве. Но тело ничто без плоти и крови. Плоть и кровь есть жизнь, а только жизнь есть действительность тела. Но плоть и кровь неразлучны с половым различием. Половое различие не есть различие поверхностное, ограниченное определенными частями тела; оно гораздо существеннее; оно пронизывает весь организм. Сущность мужчины есть мужественность; сущность женщины --- женственность. Как бы мужчина ни был одухотворен и свободен от чувственности, он все-таки останется мужчиной; точно так же и женщина. Поэтому личность есть ничто без полового различия. Мужская и женская личность существенно отличаются друг от друга. Там, где нет второго лица, не может быть и первого; но различие между первым и вторым лицом, основное условие всякой личности, всякого сознания, становится действительнее, живее, ярче в виде различия между мужчиной и женщиной. <<Ты>>, обращённое мужчиной к женщине, звучит совершенно иначе, чем монотонное <<ты>> между друзьями.

Природа в отличие от личности означает не что иное, как половое различие. Личное существо, лишенное естества, есть существо, лишенное пола, и наоборот. Природа свойственна богу <<в том смысле, в каком мы называем человека сильной, крепкой, здоровой натурой>>. Что может быть болезненнее, отвратительнее, противоестественнее личности, лишенной пола или отрицающей свой пол своим характером, нравами, чувствами? В чем заключается добродетель, достоинство человека как мужчины? --- В мужественности. Человека как женщины? --- В женственности. Но человек существует только как мужчина или как женщина. Поэтому сила, здоровье человека состоит только в том, чтобы он как женщина был женщиной и как мужчина был мужчиной. Ты отрицаешь <<отвращение ко всему реальному, основанное на предположении, будто духовное оскверняется от соприкосновения с ним>>. Но в таком случае ты должен прежде всего отречься от своего собственного презрения к половому различию. Если бог не оскверняется природой, то он не оскверняется и половым различием. Боязнь полового бога есть ложный стыд, ложный по двум основаниям: во-первых, потому, что мрак, вложенный тобою в бога, освобождает тебя от стыда: стыд обнаруживается только при свете; во-вторых, потому, что он идет вразрез с твоим собственным принципом. Нравственный бог вне природы не имеет под собой основы, а основа нравственности есть половое различие. Даже животное становится способным к самопожертвованию благодаря половому различию. Вся красота природы, все её могущество, вся её мудрость и глубина сосредоточиваются и индивидуализируются в различии полов. Почему же ты боишься назвать природу бога её настоящим именем? Очевидно, только потому, что ты вообще боишься истины и действительности и смотришь на все сквозь обманчивый туман мистицизма. Природа в боге есть только обманчивая, лишенная всякой сущности иллюзия, фантастический призрак природы, ведь она, как уже сказано, опирается не на плоть и кровь, не на реальное основание. Так как и это обоснование личного бога тоже несостоятельно, то я заключу свою мысль словами: <<Отрицание личного бога останется научно добросовестным положением>>, и прибавлю: научной истиной, до

тех пор, пока мы не выразим и не докажем ясными, определенными словами, сначала априори, исходя из умозрительных оснований, что облик, место, плоть, пол не противоречат понятию божества, и затем апостериори --- поскольку действительность личного существа опирается только на эмпирические основания --- каков образ бога, где он существует --- хотя бы на небе --- и, наконец, какого он пола: мужчина, женщина или даже гермафродит. Впрочем, уже в 1682 году один священник поставил смелый вопрос: <<Женат ли бог и сколько у него есть способов производить людей>>. Пусть же глубокомысленные умозрительные немецкие философы возьмут пример с этого честного, простодушного священника! Пусть они мужественно отряхнут с себя стеснительный остаток рационализма, который резко противоречит их существу, и реализуют наконец мистическую потенцию природы бога и действительно дееспособного производительного бога. Аминь.

Учение о природе в боге заимствовано у Якова Бёме, но в подлиннике оно имеет более глубокий и интересный смысл, чем во втором, сокращенном и модернизированном, издании. У Якова Бёме --- глубокая, глубоко-религиозная душа; религия является центром его жизни и мышления. Но в то же время в его религиозное сознание проникло то значение природы, которое она приобрела за последнее время благодаря изучению естественных наук, благодаря спинозизму, материализму и эмпиризму. Он чувствовал природу и пожелал проникнуть в её таинственную глубину. Но природа испугала его, и он не мог согласовать страх, внушаемый природой, со своими религиозными представлениями. <<Я созерцал необъятную глубину этого мира, наблюдал за солнцем, звездами, облаками, дождем и снегом, представлял себе мысленно все творение этого мира и находил добро и зло, любовь и гнев во всем --- не только в людях и животных, но и в неразумной твари: в дереве, камнях, земле и стихиях\dots Я убедился, что добро и зло присущи всему: не только тварям, но и стихиям, что безбожники так же блаженствуют, как и люди благочестивые, что варварские народы располагают лучшими землями и наслаждаются большим счастьем, чем народы благочестивые. Это повергло меня в печаль и сильно огорчило. Я не мог найти утешения даже в Писании, с которым я был хорошо знаком. Но надо мной не восторжествовал дьявол, внушавший мне часто языческие мысли, о которых я не хочу здесь упоминать>>\dag\let\svthefootnote\thefootnote\let\thefootnote\relax\footnotetext{\dagкраткое извлечение из Якова Бёме. Амстердам, 1718, стр. 58. Следующие места находятся на страницах 480, 338, 340, 323.}\let\thefootnote\svthefootnote. Мрачная сущность природы, не соответствующая религиозным представлениям Бёме о творце небесном, действовала на его душу удручающим образом, тем не менее он с восхищением отзывается о блестящей стороне природы. Яков Бёме понимал природу. Он понимал и даже ощущал радость минеролога, ботаника, химика, одним словом, радость <<безбожного натуралиста>>. Он восхищается блеском драгоценных камней, звуком металлов, запахом и цветом растений, грацией и кротостью многих животных. В другом месте он пишет: <<это (то есть откровение бога в лучезарном мире, процесс, сообщающий различные краски небу и определенный характер каждой твари) несравнимо ни с чем, кроме драгоценных камней: рубина, смарагда, дельфина, оникса, сапфира, бриллианта, яшмы, гиацинта, аметиста, берилла, сердолика, альмандина и др.>>. В другом месте он говорит: <<Драгоценные камни, превосходящие собой все остальные, как то: альмандин, рубин, смарагд, дельфин, оникс и так далее, зарождаются там, где загорается свет любви. А этот свет загорается в кротости и есть сердце в центре рудниковых духов, поэтому эти камни так кротки, прочны и красивы>>. Как видите, у Якова Бёме был хороший минералогический вкус. О его любви к цветам и, следовательно, ботаническом вкусе свидетельствуют следующие места: <<Небесные силы рождают небесные радостные плоды и цветы, всевозможные деревья и кусты, убранные прекрасными и изящными плодами жизни. Благодаря этим силам цветы получают чудную небесную окраску и запах; их вид разнообразен, но каждый из них является по-своему священным, божественным, радостным>>. <<Если ты хочешь видеть небесный божественный блеск и величие, если ты хочешь знать, каковы растения, веселие или радость на небе, то посмотри внимательно вокруг себя, постарайся разглядеть земные плоды и растения: деревья, кусты, траву, корни, цветы, масла, тина, злаки, одним словом --- все, что доступно твоему исследованию. Во всем этом отражается небесное великолепие>>.



Яков Бёме не мог удовольствоваться деспотическим <<да будет!>> в качестве объяснения природы; он слишком понимал и любил природу. Поэтому он пытался дать природе естественное объяснение и нашел его только в качествах самой природы, которые производили глубокое впечатление на его душу. Яков Бёме --- и в этом заключается его существенное значение --- мистический натурфилософ, теософический вулканист и нептунист, так как, по его мнению, все вещи первоначально возникли из огня и воды. Природа очаровала религиозную душу Якова Бёме --- недаром блеск оловянной посуды напоминал ему о мистическом свете. Но религиозная душа замыкается в себе самой; у нее нет ни силы, ни мужества проникнуть в действительность вещей; она созерцает все глазами религии, она созерцает все в боге, то есть в чарующем, ослепляющем душу блеске воображения, представляет себе все в образе и как образ. Природа вселяла в душу Бёме противоречие, и он должен был усмотреть это противоречие в самом боге, так как иначе предположение двух самостоятельно существующих, противоположных первопричин растерзало бы его религиозную душу: он должен был различать в самом боге кроткое, благодетельное и суровое, карающее существо. Все огненное, горькое, жесткое, ограничивающее, мрачное, холодное исходит из божественной жестокости, гнева, безразличия и суровости; все кроткое, блестящее, согревающее, мягкое, нежное, податливое исходит из кроткого, мягкого, светлого качества в боге. Одним словом, небо так же богато, как и земля; все, что мы видим на земле, есть и на небе\ddag\let\svthefootnote\thefootnote\let\thefootnote\relax\footnotetext{\ddagпо Сведенборгу, ангелы на небе имеют даже одежду и жилища. <<Их жилища во всем подобны земным жилищам, называемым домами, но они гораздо лучше; в них много горниц, комнат, спален, кругом дворы, сады, поля и луга>>. (E. v. Swedenborgs auserlesene Schrifen, \rom{1}. Frankfurt a. M. 1776, S. 190 и 96). Таким образом, для мистика земная жизнь тождественна с загробной, и наоборот.}\let\thefootnote\svthefootnote; все, что есть в природе, есть и в боге. Но там все это божественного, небесного свойства, у нас все носит земной, видимый, внешний, материальный характер, по существу же --- одно и то же. <<Когда я пишу о деревьях, кустах и плодах, ты не должен понимать моих слов в земном, обыкновенном смысле; я не держусь того мнения, будто на небе находятся мертвые, грубые деревья или земные камни. Нет, мое представление божественно и духовно, но в то же время истинно и правдоподобно. Я описываю вещи так, как я их себе представляю. Следовательно, на небе также есть деревья и цветы, но небесные деревья таковы, какими они кажутся моему воображению, они не производят на меня грубого, материального впечатления; земные же деревья таковы, какими они представляются моему чувственному, действительному созерцанию>>. Различие между теми и другими есть различие между соображением и созерцанием. <<Я не намереваюсь, --- говорит он сам*\let\svthefootnote\thefootnote\let\thefootnote\relax\footnotetext{*в цитированном произведении, стр. 339, 69.}\let\thefootnote\svthefootnote, --- описывать свечение, место и название звезд, их ежегодные конъюнкции, противостояния или квадраты, их ежегодное и ежечасное изменение. Я не изучал этого и предоставляю это ученым. Моя задача заключается в том, чтобы писать, сообразуясь с умом и чувством, а не с созерцанием>>.




Учение, что природа --- в боге, делает натурализм основой теизма, именно теизма, рассматривающего высшее существо, как существо личное. Личный теизм мыслит бога существом личным, отвлеченным от всего материального; он исключает из бога всякое развитие; развитие есть не что иное, как самоотделение какого-нибудь существа от состояний и качеств, не соответствующих его истинному понятию. Но к богу это не применимо: в нем нельзя различать ни начала, ни конца, ни средины; он именно таков, каким он изначала должен и может быть. Бог есть чистое единство бытия и сущности, реальности и идеи, деяния и воли. Deus suum esse est\footnote{Бог есть сущность самого себя (лат.).}. В этом смысле теизм соответствует сущности религии. Все даже наиболее положительные религии основаны на абстракции; они отличаются друг от друга только предметом абстракции. Даже боги Гомера --- абстрактные образы, несмотря на всю их жизненность и человекоподобие; они обладают телами, подобно людям, но их тела свободны от несовершенств и недомоганий человеческого тела. Первое определение бога заключается в том, что он есть отвлеченное, дистиллированное существо. Разумеется, эта абстракция непроизвольна, она обусловливается существенным аспектом человека. Каков он есть сам, как вообще он мыслит, так он и абстрагирует.

В абстракции заключается одновременно оправдательный и обвинительный приговор, хвала и порицание. Богом человек считает то, что он хвалит и превозносит\dag\let\svthefootnote\thefootnote\let\thefootnote\relax\footnotetext{\dag<<Что человек ставит выше всего, то и есть его бог>> (Origines, Explan. in Epist. Pauli ad Rom., C. 1).
}\let\thefootnote\svthefootnote\footnote{Ориген. <<Толкования послания к римлянам апостола Павла>>, гл. \rom{1} (лат.).}; небожественным --- то, что он отвергает и порицает. Религия есть приговор. Поэтому существенным определением в религии, в идее существа божия является отделение достойного от недостойного, совершенного от несовершенного, короче --- существенного от ничтожного. Самый культ состоит только в постоянном возобновлении происхождения религии --- в критическом, но торжественном отделении божественного от небожественного.

Сущность бога есть преображённая чрез смерть абстракции сущность человека --- отошедший в вечность дух человека. В религии человек освобождается от границ жизни; здесь он сбрасывает с себя все, что его тяготит, стесняет, раздражает; бог есть самоощущение человека, освобождённое от всего неприятного; человек чувствует себя свободным, счастливым, блаженным только в своей религии, потому что здесь он верит в свой гений, празднует свое торжество. Опосредствование, обоснование идеи бога лежит, по мнению религиозного человека, вне этой идеи; истинность же ее --- в суждении, согласно которому все, не свойственное богу, он считает небожественным, а все небожественное --- ничтожным. Если мы вложим обоснование этой идеи в самую идею, то она утратит свое существенное значение, свою истинную цену, свою чудодейственную силу. Процесс отделения, отграничения разумного от неразумного, личности от природы, совершенного от несовершенного необходимо переносится в человека, не в бога, а идея божества приурочивается не к началу, а к концу чувственности, мира и природы: <<бог начинается там, где кончается природа>> --- ведь бог есть граница абстракции. Бог --- это то, от чего я уже не могу абстрагировать, это последняя и, следовательно, высшая, доступная мне мысль. Id quo majus nihil cogitari potest, Deus est\footnote{Бог есть то, выше чего нельзя помыслить (лат.).}. Легко понять, что эта омега чувственности становится альфой, но главная суть заключается в том, что это омега. Альфа есть только следствие, ибо, поскольку альфа есть последнее, постольку она есть также и первое. Также предикат: <<первое существо>> имеет не творческое значение, а лишь значение высшего ранга. В религии Моисея творение имеет целью сделать Иегову предикатом высшего, первого, истинного, исключительного бога в противоположность богам языческим\ddag\let\svthefootnote\thefootnote\let\thefootnote\relax\footnotetext{\ddag<<Я --- господь, творящий все>>. <<Я господь, я нет иного; пет бога кроме меня>>. <<Я первый, и я последний, и кроме меня нет бога>> (Исайя, гл. 41--47). Отсюда выясняется значение творения, о чем подробнее будет ниже.}\let\thefootnote\svthefootnote. 

В основе стремления к обоснованию личности бога через природу заключается нечестивое, беззаконное смешение философии с религией, полнейшее отсутствие сознательного и критического отношения к генезису личного бога. Если личность является существенным определением бога, если установлено, что безличный бог не есть бог, значит, личность сама по себе есть нечто высшее и реальнейшее. Здесь в основе лежит следующее утверждение: все безличное мертво и ничтожно, и только личное бытие есть жизнь и истина; природа же безлична, то есть ничтожна. Истинность личности опирается на ложность природы. Приписывание богу свойства личности равносильно признанию личности абсолютной сущностью. Но личность понимается только как нечто отличное, отвлеченное от природы. Разумеется, только личный бог есть абстрактный бог, но он и должен быть таким; это свойство заключается в самом понятии бога; бог есть не что иное, как отрешившаяся от всякой связи с миром, освободившаяся от всякой зависимости от природы личная сущность человека. В личности бога человек чтит сверхъестественность, бессмертие, независимость, неограниченность своей собственной личности.

Потребность в личном боге, вообще говоря, объясняется тем, что человек как индивид только в личности доходит до себя, обретает себя только в ней. Субстанция, чистый дух, разум, как таковой, не удовлетворяют его, кажутся ему слишком отвлеченными, то есть не выражающими его самого, не возвращающими его к себе самому. Человек чувствует себя довольным и счастливым только тогда, когда он бывает у себя, в своей сущности. Поэтому, чем индивидуальнее человек, тем сильнее его потребность в личном боге. Отвлеченно свободный дух не знает ничего выше свободы; ему незачем связывать ее с личным существом; свобода сама по себе, как таковая, имеет в его глазах значение действительной, истинной сущности. Математик или астроном, человек чистого рассудка, человек объективный, не замкнутый в себе, чувствующий себя свободным и счастливым только в созерцании объективно-разумных отношений, в разуме, лежащем в основе вещей, --- такой человек считает высшим существом субстанцию Спинозы или другую подобную идею и чувствует глубокую антипатию к личному, то есть субъективному, богу. Якоби был последователен, по крайней мере в этом отношении, и поэтому является классическим, согласным с собой философом. Его философия была так же лична и субъективна, как и его бог. Нельзя иначе научно обосновать личного бога, чем так, как обосновали его Якоби и его ученики. Личность утверждает себя только личным образом.

Конечно, можно и даже должно обосновать личность путем естественным, но это будет возможно только тогда, когда мы перестанем бродить в потемках мистицизма, выйдем на свет действительной природы и заменим понятие личного бога понятием личности вообще. Наша ошибка состоит в том, что мы соединяем природу с понятием личного бога, сущность которого составляет именно освобожденная, отрешенная от гнетущей силы природы личность. Такое сочетание равносильно тому, как если бы я смешал божественный нектар с брауншвейгским пивом, чтобы придать небесному напитку большую крепость. Конечно, составные части животной крови нельзя извлечь из небесного напитка, питающего богов. Однако продукт возгонки возникает только через улетучивание материи; как же можно искать в сублимированной субстанции ту именно материю, от которой она отделена? Во всяком случае безличную сущность природы нельзя объяснять из понятия личности. Объяснять --- значит обосновывать; но если личность есть истина, или, вернее, высшая, единственная истина, то значит природа не имеет существенного значения и, следовательно, не имеет никакого существенного основания. Единственным удовлетворительным объяснением является здесь творение из ничего, так как оно свидетельствует лишь о том, что природа есть ничто, и таким образом точно определяет то значение, которое имеет природа для абсолютной личности.




\chapter{Тайна промысла и творение из ничего}


Творение есть изречённое слово божие, слово творческое, слово внутреннее, тождественное с мыслью. Изречение есть акт воли, следовательно, творение есть продукт воли. Человек в слове божием утверждает божественность слова, а в творении утверждает божественность воли и при этом не воли разума, а воли воображения, воли абсолютно субъективной и неограниченной. Творение из ничего есть вершина начала субъективности. Вечность мира или материи свидетельствует только о существенности материи; а творение мира из ничего говорит о ничтожестве мира. Начало той или другой вещи непосредственно предполагает её конец, если не в смысле времени, то в смысле понятия. Начало мира есть начало его конца. Как нажито, так и прожито. Воля вызвала мир к существованию, воля же обратит его в ничто. Когда? --- Время безразлично. Бытие или небытие мира зависит только от воли. Воля, утверждающая бытие мира, есть, или, по крайней мере, может быть волей, отрицающей его. Поэтому бытие мира есть преходящее, произвольное, ненадежное, то есть ничтожное.

Творение из ничего есть высшее выражение всемогущества. Но всемогущество есть не что иное, как субъективность, отрешенная от всех объективных определений и ограничений и прославляющая свою свободу как наивысшую власть и сущность. Мощь есть способность полагать субъективно все действительное как недействительное, все представляемое как возможное; оно есть сила воображения или сила воли, тождественная с силой воображения, сила произвола*\let\svthefootnote\thefootnote\let\thefootnote\relax\footnotetext{*более глубокое происхождение творения из ничего лежит в душе, --- что будет прямо и косвенно выяснено и показано в этой книге. Произвол есть воля души, проявление силы души во вне.}\let\thefootnote\svthefootnote. Самое характерное, самое сильное выражение субъективного произвола есть прихоть, своенравие. <<Богу угодно было вызвать к бытию мир телесный и мир духовный>> --- неопровержимое доказательство того, что собственная субъективность, собственный произвол рассматриваются как наивысшая сущность, как всемогущий мировой принцип. Творение из ничего, как акт всемогущей воли, относится поэтому к категории чудес, или, вернее, оно есть первое чудо не только по времени, но и по значению --- принцип, из которого сами собой вытекают все дальнейшие чудеса. Доказательством этого служит история. Все чудеса оправдывались и объяснялись всемогуществом, создавшим мир из ничего. Тот, кто сотворил из ничего мир, разве не мог превратить воду в вино, заставить осла говорить человеческим языком, из камня извлечь воду? Но чудо, как мы увидим дальше, есть только дело и предмет воображения, следовательно, также творение из ничего есть первое чудо. Поэтому учение о творении из ничего объяснялось, как сверхъестественное, недоступное разуму; и в пример приводились языческие философы, утверждавшие, что мир создан божественным разумом из существовавшей ранее материи. Но этот сверхъестественный принцип есть не что иное, как принцип субъективности, достигший в христианстве неограниченного развития. Древние же философы не были настолько субъективны, чтобы абсолютно субъективную сущность понимать в смысле исключительно абсолютной сущности; ведь они ограничивали субъективность созерцанием мира или действительности мир же был для них истиной.

Более глубокое происхождение творения из ничего лежит в душе, --- что будет прямо и косвенно выяснено и показано в этой книге. Произвол есть воля души, проявление силы души во вне.

Творение из ничего тождественно с чудом, тождественно с промыслом. Ведь идея промысла, первоначально в её истинном значении, не стесненном и не ограниченном неверующим рассудком, --- тождественна с идеей чуда. Чудо есть доказательство промысла\dag\let\svthefootnote\thefootnote\let\thefootnote\relax\footnotetext{\dag<<Самым достоверным свидетельством божественного промысла служат чудеса>> (Г. Гроций, De verit. rel. christ. Lib. \rom{1}, par. 13).}\let\thefootnote\svthefootnote. Вера в промысл есть вера в силу, подчиняющую все своему произволу; по сравнению с ней всякая мощь действительности есть ничто. Провидение отменяет законы природы, нарушает естественный ход событий, уничтожает железную связь между следствием и причиной. Одним словом, промысл является той неограниченной, всесильной волей, которая создала мир из ничего. Чудо есть creatio ex nihilo, творение из ничего. Тот, кто превращает воду в вино, делает вино из ничего, потому что в воде нет материалов для вина; в противном случае превращение воды в вино было бы актом не чудесным, а естественным. Промысл божий оправдывает, доказывает себя только в чуде. Промысл выражает собой то же что и творение из ничего. Творение из ничего можно понимать и объяснять только в связи с промыслом, с чудом, так как чудо свидетельствует о том, что чудотворцем является тот самый, кто создал все из ничего, благодаря одной своей воле, --- бог, творец.


Провидение по существу относится к человеку. Ради человека бог поступает с вещами по своему произволу --- ради него он уничтожает силу ранее всемогущего закона. Прославление провидения в природе, особенно в мире животных, есть не что иное, как прославление природы, и поэтому оно свойственно только натурализму, хотя и религиозному. Ибо в природе открывается только естественный, а не божественный промысл, не тот промысл, который имеет предметом религия\ddag\let\svthefootnote\thefootnote\let\thefootnote\relax\footnotetext{\ddagрелигиозный натурализм есть во всяком случае момент религии христианской --- ещё более моисеевой, столь сочувственно относящейся к животным. Но он никоим образом не есть характерный, христианский момент христианской религии. Христианское религиозное провидение есть нечто совсем иное, чем то провидение, которое украшает лилию её одеждами и питает птиц. Естественный промысл позволяет утонуть человеку, если он не умеет плавать; но христианский, религиозный промысл, при помощи всемогущего, позволяет человеку пройти невредимым по поверхности воды.}\let\thefootnote\svthefootnote. Религиозный промысл открывается только в чуде, и прежде всего в чуде воплощения, средоточии религии. Но нигде не сказано, чтобы бог воплотился в животное ради животных. Такая мысль с религиозной точки зрения нечестива и безбожна. Вообще нигде не встречается, чтобы бог творил чудеса ради животных или растений, напротив: мы узнаем, что бедная смоковница, не приносящая плодов в неурочное время, проклинается только ради того, чтобы показать человеку преобладание веры над природой; бесы изгоняются из человека и вселяются в стадо свиней. Правда, говорится, что <<ни один воробей не упадет с кровли помимо воли отца>>; но эти воробьи имеют такую же цену и значение, как волосы на голове человека, которые все сочтены.

У животного нет другого ангела-хранителя, другого попечения, кроме инстинкта, внешних чувств или вообще органов. Птица, теряющая зрение, теряет своего ангела-хранителя и неизбежно погибает, если не спасется чудом. Но в Писании, где говорится, что ворон приносил пищу пророку Илие\footnote{см. Третью Книгу Царств, 17, 2--6.}, не упоминается (насколько мне известно), чтобы когда-нибудь животные кормились иным способом, кроме как естественным. Если человек надеется только на свои силы: на свои чувства и свой рассудок, то религия и её защитники считают его человеком безбожным, ибо он верит только в естественное попечение, не имеющее, однако, никакого значения в глазах религии. Поэтому попечение относится главным образом к человеку и притом к человеку религиозному. Бог <<есть спаситель всех человеков, а наипаче верных>>*\let\svthefootnote\thefootnote\let\thefootnote\relax\footnotetext{*Первое послание к Тимофею, 4:10.}\let\thefootnote\svthefootnote. Провидение, подобно религии распространяется только на человека; в нем выражается существенное различие между человеком и животным; провидение освобождает человека от власти природы. Иона во чреве кита, Даниил во рве львином\footnote{см. Книгу пророка Даниила, 6, 16--24; книгу пророка Ионы, 2, 1--11.} служат примерами, как провидение делает различие между (религиозным) человеком и животным. Поэтому если провидение, дающее зверям орудие для ловли добычи и для питания и вызывающее такое изумление благочестивых христианских естествоиспытателей, есть истина, то провидение Библии, провидение религии, есть ложь, и наоборот. Как жалко, смешно и лицемерно желание угодить одновременно --- природе и Библии! Природа противоречит Библии, Библия противоречит природе! Бог природы обнаруживает себя тем, что наделяет льва силой и соответствующими органами, годными для того, чтобы поддержать его жизнь, а в крайнем случае задушить и сожрать человека. Бог Библии освобождает человека из пасти льва\dag\let\svthefootnote\thefootnote\let\thefootnote\relax\footnotetext{\dagпри том противоположении религиозного, или библейского, и естественного промысла автор особенно имел в виду плоскую, ограниченную теологию английских естествоиспытателей.}\let\thefootnote\svthefootnote.




Провидение есть преимущество человека; в нем выражается ценность человека в отличие от других естественных существ и вещей; оно изъемлет человека из связи с мировым целым. Провидение есть убеждение человека в безграничной ценности его существования --- убеждение, при котором он отрекается от веры в реальность внешних вещей. Оно есть идеализм религии. Следовательно, вера в провидение тождественна с верой в личное бессмертие, с той только разницей, что здесь в отношении времени бесконечная ценность определяется как бесконечная продолжительность существования. Кто не имеет особенных притязаний, кто равнодушен к себе, кто не отделяет себя от природы, кто считает себя частицей преходящего целого, тот не верит в провидение, то есть в особое провидение. А между тем только особое провидение есть провидение в смысле религии. Вера в провидение есть вера в собственную ценность. Отсюда вытекают благотворные последствия этой веры, но в то же время и ложное смирение и религиозное высокомерие, которое хотя не надеется на себя, но зато возлагает заботу о себе на бога --- вера человека в самого себя. Бог заботится обо мне, стремится к моему счастью, к моему спасению; он хочет доставить мне блаженство. Но я сам хочу того же; следовательно, мой собственный интерес есть интерес бога, моя собственная воля --- его воля, моя конечная цель --- его цель, любовь же его ко мне --- мое обожествлённое себялюбие.

Но где есть вера в провидение, там вера в бога стоит в зависимости от веры в провидение. Кто отрицает провидение, --- отрицает бытие бога или, что то же, отрицает, что бог есть бог; ведь бог без заботы о человеке есть смехотворный бог, бог, лишенный наиболее божественного, наиболее существенного свойства, достойного поклонения. Следовательно, вера в бога есть не что иное, как вера в человеческое достоинство\ddag\let\svthefootnote\thefootnote\let\thefootnote\relax\footnotetext{\ddag<<Кто отрицает богов, отрицает благородство человеческого рода>>. Бэкон Веруламский (Serm. Fidei, 16).}\let\thefootnote\svthefootnote\footnote{Фейербах цитирует сочинение Ф. Бэкона <<Опыты или наставления нравственные и политические>>. См.: Бэкон Ф. Соч. в 2-х т. Т. 2. М., 1972 г., с. 382.}, вера в божественное значение человеческого существа. Но вера в (религиозное) провидение есть вера в творение из ничего, наоборот; таким образом, творение из ничего не может иметь другого значения, кроме только что рассмотренного значения провидения; и действительно, оно не имеет другого значения. Религия удовлетворительно выражает это тем, что делает человека целью творения. Все вещи существуют не ради себя, а ради человека. Всякий, кто, подобно благочестивым христианским естествоиспытателям, считает это учение высокомерным, объявляет высокомерным и само христианство. Ведь существование <<материального мира>> ради человека имеет бесконечно меньше значения, чем тот факт, что бог или, как выражается Павел, почти бог, существо, едва отличное от бога, становится человеком ради человека.



Но если человек есть цель творения, значит, он является также истинным основанием его, ибо цель есть принцип деятельности. Различие между человеком как целью творения и человеком как основанием его заключается лишь в том, что основание есть абстрактная сущность человека, а цель есть человек действительный, индивидуальный. Человек знает, что он ость цель творения, но не знает, что он является его основанием, ибо он отличает от себя основание, сущность как другое личное существо*\let\svthefootnote\thefootnote\let\thefootnote\relax\footnotetext{*у Климента Александрийского (Coh. ad gentes) встречается интересное место. Оно гласит в латинском переводе (по плохому вюрцбургскому изданию 1778 г.): At nos ante mundi constitutionem fuimus, ratione futurae nostrae productionis, in ipso Deo quodammodo tum praeexistentes. Divini igitur Verbi sive Rationis nos creaturae rationales sumus, et per eum primi esse dicimur, quoniam in principio erat Verbum. Ещё определеннее высказалась христианская мистика, что человеческая сущность есть творческое начало и причина мира. <<Человек, до времени сущий в вечности, вместе с богом творит все дела>>. <<Благодаря человеку создались все твари>>. <<Predigten vor und zu Tauleri Zeiten>> (Ed. cit., p. 5, p. 119).}\let\thefootnote\svthefootnote\footnote{<<Ибо мы некоторым образом уже предсуществовали в самом боге до сотворения мира благодаря промыслу о будущем нашем сотворении, таким образом, божественным словом, или промыслом, мы являемся разумными тварями и благодаря ему мы можем быть названы превоначальными, так как вначале было слово>> (лат.)}. Но это другое личное существо, эта творческая сущность есть на самом деле не что иное, как освобождённая от всякой связи с миром человеческая личность, утверждающая себя в своей действительности творением, то есть путем установления другого, объективного мира, как бытия несамостоятельного, преходящего и ничтожного. Творение обусловливает собой не истинность и реальность природы, или мира, а истинность и реальность личности, субъективности в отличие от мира. Здесь речь идет о личности бога, но личность бога есть освобожденная от всяких определений и ограничений природы личность человека. Этим объясняется искренняя симпатия к творению и отвращение к пантеистическим космогониям. Творение как личный бог вообще есть дело личного чувства, душевное дело, а не предмет науки и свободного разумения. Ведь творение служит гарантией, последним мыслимым утверждением личности, или субъективности, как сущности особой, сверх --- и внемировой, не имеющей ничего общего с сущностью природы\dag\let\svthefootnote\thefootnote\let\thefootnote\relax\footnotetext{\dagотсюда ясно, почему не удавались и не должны удаваться все попытки умозрительного богословия и одинаково с ним настроенной философии вывести мир из бога: именно потому, что они в самом корне своем ложны и извращены и не знают, о чем собственно идет речь в творении.}\let\thefootnote\svthefootnote.




Человек отличает себя от природы. Это его отличие есть его бог. Отличие бога от природы есть отличие человека от природы. Противоположность между пантеизмом и персонализмом заключается в следующем вопросе: есть ли сущность человека внемировая или внутримировая, сверхъестественная или естественная сущность? Поэтому все рассуждения и споры о личности или безличности бога бесплодны, суетны, некритичны и надоедливы. Спекулятивные философы, в особенности защищающие личность, не хотят называть вещи своими именами. Они созерцают самих себя, созерцают в интересах своего собственного стремления к блаженству и не желают сознаться, что только ломают себе голову, тогда как им кажется, что они открывают тайны другого существа. Пантеизм отождествляет человека с природой, с её внешними явлениями или отвлеченной сущностью. Персонализм изолирует, отделяет человека от природы, приписывает ему значение не части, а целого, делает его самодовлеющим абсолютным существом. В этом различие. Следовательно, если вы хотите надлежащим образом уяснить себе это, замените вашу мистическую, извращенную антропологию, называемую теологией, действительной антропологией и размышляйте при свете сознания и природы о различии или сходстве между сущностью человека и сущностью природы. Вы сами соглашаетесь, что сущность пантеистического бога есть сущность природы. Почему же вы, усматривая сучок в глазу вашего противника, не замечаете бревна в вашем собственном глазу? Почему вы хотите составлять исключение из всеобщего закона? Вы должны признать, что ваш личный бог есть ваша собственная личная сущность, что ваша вера в сверхъестественность и внеприродность вашего бога есть вера в сверхъестественность и сверхприродность вашего собственного <<Я>>.

В творении, как везде, истинная сущность его затемняется примесью общих метафизических или даже пантеистических определений. Но достаточно обратить внимание на ближайшее определение, чтобы убедиться, что сущность творения есть самоутверждение человеческой сущности в отличие от природы. Бог создает мир вне себя; сначала этот мир является только мыслью, планом, решением, затем он становится деянием и отделяется от бога, как отличное от него и до некоторой степени самостоятельное существо. Но также и человек, отличающий себя от мира и считающий себя отдельной от него сущностью, полагает мир вне себя как другую сущность; и такое самоотделение и самоотличение составляют единый акт. Поэтому, полагая мир вне бога, мы полагаем бога в себе самом, в отличие от мира. Итак, что такое бог, как не ваша собственная, субъективная сущность, если мир сказывается вне его?\ddag\let\svthefootnote\thefootnote\let\thefootnote\relax\footnotetext{\ddagпротив этого нельзя также возразить ссылкой на вездесущее божие, бытие бога во всех вещах, или бытие вещей в боге. Ведь независимо от того, что предстоящей действительной кончиной мира уже достаточно останавливается бытие мира вне бога, то есть его небожественность, бог только в человеке пребывает особым способом; но ведь я только тогда и бываю дома, когда я именно дома. <<Бог есть собственно бог только в душе. Во всех тварях есть немножко бога, но лишь в душе бог является вполне богом, ибо душа есть место его упокоения>> (Predigten etzlicher Lehrer etc., p. 19). И бытие вещей в боге, если не придавать ему пантеистического значения, в настоящем случае не имеющего места, точно так же есть представление, лишенное реальности и не выражающее специфического настроения религии.}\let\thefootnote\svthefootnote Конечно, когда вступается лукавая рефлексия, то можно отрицать различие между внешним и внутренним как конечное, человеческое различие. Но мы не должны считаться с отрицанием рассудка, неправильно толкующего или вовсе не понимающего религии. Отрицать это различие --- значит разрушать основу религиозного сознания, уничтожать возможность, даже сущность творения, основанного только на реальности этого различия. Кроме того, если мы будем понимать нахождение мира вне нас не в действительном смысле, то совершенно пропадет эффект творения, величие этого акта для сердца и фантазии. Делать, творить, производить --- значит объективировать, воплощать что-либо субъективное, невидимое, несуществующее, чтобы другие, отличные от меня, существа могли познавать его и наслаждаться им, то есть полагать нечто вне меня, делать его чем-то, отличным от меня? Где нет действительности или возможности бытия вне меня, там не может быть и речи о деянии, о творении. Бог вечен, а мир имеет начало; бог был, когда не было ещё мира; бог невидим и неосязаем; мир осязаем, материален, следовательно, он существует вне бога; ибо материя, как таковая, как масса, как вещество не может быть в боге. Мир находится в таком же смысле вне бога, в каком дерево, животное, мир вообще находится вне моего представления, вне меня, как отличная от моей субъективности сущность. Поэтому чистое, неподдельное учение религиозного сознания встречается только там, где принимается такое перенесение себя вне мира, как, например, у древних философов и теологов. Умозрительные теологи и философы нового времени, напротив, загрязняют религию разными пантеистическими определениями, хотя и отрицают самый принцип пантеизма, и таким образом пускают в оборот абсолютно противоречивую, несносную стряпню.



Итак, творец мира есть сам человек, убеждающийся в собственной ценности, реальности и бесконечности посредством доказательства или сознания того, что мир сотворён, что он есть акт воли, то есть бытие несамостоятельное, бессильное, ничтожное. Ничто, из которого возник мир, есть его собственное ничто. Говоря: мир сотворен из ничего, ты мыслишь самый мир как ничто, ты устраняешь все пределы твоей фантазии, твоего чувства, твоей воли, ведь мир есть предел твоей воли, твоей души; лишь мир стесняет твою душу, лишь он служит средостением между собой и богом --- твоим блаженным, совершенным существом. Следовательно, ты субъективно уничтожаешь мир. Ты мыслишь бога существующим только для себя, то есть безоговорочно неограниченной субъективностью, душой, которая наслаждается только собой, не нуждается в мире и не знакома с тяжелыми узами материи. В глубине своей души ты желаешь, чтоб не было мира; ведь где мир, там материя, а где материя, там утеснение и гнет, пространство и время, ограничение и необходимость. Но тем не менее мир и материя существуют. Как же выйти из рисков этого противоречия? Как забыть о мире, чтобы он не мешал тебе наслаждаться неограниченностью твоей души? Для этого ты делаешь самый мир продуктом воли и придаешь ему произвольное существование, колеблющееся между бытием и небытием и готовое уничтожиться в каждый момент. Во всяком случае, мир или материю --- они нераздельны --- нельзя объяснить актом творения; однако неразумно предъявлять к творению такое требование, --- ведь в основе творения лежит мысль: не должно быть ни мира, ни материи; поэтому конец мира ожидается ежечасно и с нетерпением. Здесь мир в своей истинности вовсе не существует; он служит только пределом для человеческой души и личности. Истинный, действительный мир не может основываться на принципе, отрицающем мир.

Чтобы признать правильным только что изложенное значение творения, надо серьезно обдумать то обстоятельство, что в творении главную роль играет не создание трав и животных, воды и земли, не знающих бога, а создание личных существ или, как обыкновенно говорят, --- духов. Бог есть понятие или идея личности; как лицо он есть в себе сущая, от мира отделенная субъективность, себе довлеющее, абсолютное бытие, <<Я>> без <<ты>>. Но так как абсолютное, себе довлеющее бытие противоречит понятию истинной жизни, понятию любви, так как самосознание тесно связано с сознанием некоего <<ты>>, так как продолжительное одиночество неизбежно влечет за собой чувство скуки и однообразия, то мы переходим от божественного существа к другим сознательным существам, и понятие личности, заключавшееся сначала в едином существе, распространяется на множество лиц*\let\svthefootnote\thefootnote\let\thefootnote\relax\footnotetext{*здесь пункт, где творение являет нам не только божественное всемогущество, но также и божественную любовь. <<Мы существуем по благости божией>> (Августин). Вначале, до сотворения мира бог существовал один для себя. <<Раньше всех вещей бог был един, он сам был для себя и миром, и местом, и всем. Но он был един потому, что вне его ничего не было>> (Тертуллиан). Но нет более высокого счастья, чем осчастливить других, блаженство содержится в акте общения. Но общаться может только радость, только любовь. Поэтому человек ставит общающуюся любовь как принцип бытия. <<Экстаз благости переносит бога вне себя>> (Дионисий Ареопагит). Всякая сущность основывается только на себе самой. Божественная любовь есть на себе основанная, себя утверждающая радость жизни. Но высшее ощущение жизни, высшая радость жизни есть любовь, осчастливливающая других. Бог, как существо благое, есть олицетворённое и объективированное счастье бытия.}\let\thefootnote\svthefootnote. Если личность понимается физически, как действительный человек, то есть как существо, имеющее потребности, то она выступает как конечная цель творения лишь в конце мира, когда имеются уже в наличности предпосылки её существования. Если же человек мыслится отвлеченно, как лицо (как это присуще религиозному умозрению), тогда этот окольный путь можно сократить, и речь идет уже прямо о самообосновании, о конечном самоутверждении человеческой личности. Правда, между божественной и человеческой личностью делается всяческое различие с целью скрыть их тождество, но эти различия являются либо чисто фантастическими, либо софистическими. Все существенные основания творения сводятся только к таким определениям, к таким основаниям, которые навязывают первому лицу сознание необходимости другого личного существа. Размышляйте сколько угодно, но вам не удастся вывести свою личность из бога, если вы уже прежде не вложили её в него, если сам бог не есть же ваша личная или субъективная сущность.







\chapter{Значение творения в иудействе}

Учение о творении исходит из юдаизма; оно есть характерное учение, основное учение иудейской религии. Лежащий в его основе принцип есть не столько принцип субъективности, сколько эгоизма. Учение о творении в его характерном значении возникает только тогда, когда человек смотрит на природу как на практическое средство удовлетворения своей воли и потребностей и низводит её в своем представлении на степень простого орудия, простого продукта воли. Бытие природы становится для него понятным, коль скоро он объясняет его самим собой, в своем смысле. Вопрос: откуда взялась природа или вселенная? --- предполагает, собственно, удивление, вызванное существованием мира, или другой вопрос: почему она есть? Но это удивление и этот вопрос возникают только там, где человек уже отделил себя от природы и сделал её простым объектом воли. Автор <<Книги премудрости>>\footnote{имеется в виду библейская книга <<Премудрости Иисуса сына Сирахова>>, 13, 1--3.} справедливо замечает, что <<язычники не возвысились до понятия творца вследствие восхищения своего красотою мира>>. Для кого природа --- прекрасное существо, тот видит цель и основание её существования в ней самой и не может задаться вопросом, почему она существует. В его сознании и миросозерцании понятие природы отождествляется с понятием божества. Он убежден, что природа, действующая на его чувства, возникла, произошла, но не была сотворена в собственном смысле, в религиозном смысле, то есть она не есть произвольный продукт, изделие. Самый факт возникновения не заключает в себе, по его мнению, ничего дурного, нечистого, небожественного; он и богов своих считает существами сотворёнными. Производительная сила является в его глазах первой силой, и поэтому он считает основанием природы действительную, в его чувственном созерцании проявляющуюся, силу природы. Так мыслит человек, относящийся к миру эстетически или теоретически, ибо теоретическое созерцание первоначально есть эстетическое, эстетика есть prima philosophia\footnote{первая философия (лат.).}, так он мыслит, отождествляя понятие мира с понятием космоса, красоты, божественности. Только там, где это созерцание одушевляло человека, могли зародиться и возникнуть мысли вроде следующего изречения Анаксагора: человек рожден для созерцания мира*\let\svthefootnote\thefootnote\let\thefootnote\relax\footnotetext{*у Диогена Лаэрция (\rom{2}, 11) дословно сказано: <<для созерцания солнца, луны и неба>>. Подобные мысли встречаем и у других философов. Так, стоики говорили: <<Человек рожден для созерцания мира и подражания ему>> (Цицерон. О природе богов, \rom{2}, 14, 37).}\let\thefootnote\svthefootnote\footnote{см.: Диоген Лаэртский.<<О жизни, учениях и изречениях знаменитых философов>>. М., 1979 г., с. 106.}\footnote{см.: Цицерон. <<Философские трактаты>>. М., 1985 г., с. 112}. Исходной точкой этой теории является гармония с миром. Субъективная деятельность, благодаря которой человек удовлетворяет себя и расширяет круг своих действий, сводится здесь к чувственной силе воображения. Удовлетворяя себя, человек оставляет здесь в покое природу, строя свои воздушные замки, свои поэтические космогонии исключительно на естественном материале. Если же человек, напротив, смотрит на мир с практической точки зрения и превращает эту практическую точку зрения даже в теоретическую, то он расходится с природой и превращает её в покорного слугу своих эгоистических интересов, своего практического эгоизма. Теоретическим выражением этого эгоистического, практического взгляда, утверждающего, будто природа --- ничто сама по себе, служит взгляд на природу или вселенную как на нечто сделанное, сотворённое, как на продукт веления. Бог сказал: да будет свет! и был свет, то есть бог приказал: да будет мир! и он тотчас предстал по этому велению\dag\let\svthefootnote\thefootnote\let\thefootnote\relax\footnotetext{\dag<<Евреи утверждают, что божество осуществляет все через слово и, что все сотворено по его повелению, чтобы показать, как легко оно осуществляет свою волю и как велико его могущество>>. Псалом 33,6: <<Словом господним созданы небеса>>. Псалом 148, 5: <<Он повелел, и они сотворились>> (J. Clericus, Comment. in Mosem. Genes, \rom{1}. 3).}\let\thefootnote\svthefootnote.





Высшим принципом для иудейства является утилитаризм, польза. Вера в особый божественный промысл есть характерная вера судейства; вера в промысл есть вера в чудо; а вера в чудо предполагает взгляд на природу, как на объект произвола, эгоизма, низводящего природу только на степень средства к достижению произвольных целей. Вода расступается и смыкается подобно твердой массе, пыль превращается во вшей, жезл в змия, река --- в кровь, скала --- в родник воды; свет и тьма появляются одновременно на том же месте; солнце замедляет и изменяет свое обращение. И все эти противоестественные явления совершаются на пользу Израиля, по одному велению Иеговы, пекущегося только об Израиле; и они олицетворяют собой эгоизм израильского народа, исключающий все другие народы, олицетворяющий абсолютную нетерпимость --- тайну монотеизма.

Греки смотрели на природу теоретически; в гармоническом течении звезд им слышалась небесная музыка; они видели, как из пены всепорождающего океана вышла природа в образе Венеры Анадиомены. Израильтяне, напротив, относились к природе с точки зрения гастрономии; всю прелесть природы они находили только в желудке; они познавали своего бога только в поедании манны. Грек занимался гуманитарными, свободными искусствами, философией; израильтянин не ушел дальше небезвыгодного изучения теологии. <<Вечером вкушайте рыбу, а по утрам насыщайтесь хлебом и проникайтесь сознанием, что я господь бог ваш>>\ddag\let\svthefootnote\thefootnote\let\thefootnote\relax\footnotetext{\ddag2. Моисей, 16, 12.}\let\thefootnote\svthefootnote\footnote{см.: Вторая книга Моисеева. Исход.}. <<И положил Яков обет, сказав: если бог будет со мною и сохранит меня в пути, в который я иду, и даст мне хлеб есть и одежду одеться, и я в мире возвращусь к отцу моему, тогда будет господь моим богом>>*\let\svthefootnote\thefootnote\let\thefootnote\relax\footnotetext{*1. Моисей, 28, 20.}\let\thefootnote\svthefootnote\footnote{см.: Первая книга Моисеева. Бытие.}. Еда есть самый торжественный акт, как бы начало иудейской религии. В еде израильтянин празднует и повторяет акт творения; в момент еды человек признает природу вещью, ничтожной самой по себе. Когда семьдесят старейшин с Моисеем во главе взошли на гору и <<увидели там бога, они стали пить и есть>>\dag\let\svthefootnote\thefootnote\let\thefootnote\relax\footnotetext{\dag2. Моисей, 24, 10, 11.<<Tantum abest, ut mortui sunt, ut contra convivium hilares celebrarint>> (Clericus).}\let\thefootnote\svthefootnote\footnote{<<Они не только не помышляли о смерти при видебожества, но даже стали пировать>> (лат.). Фейербах цитирует Ж. Леклерка.}. Таким образом, вид высшего существа возбудил в них только аппетит к еде.

Свои особенности евреи сохранили и до настоящего времени. Их принцип, их бог есть самый практический принцип в мире --- эгоизм и притом эгоизм в форме религии. Эгоизм есть бог, не дающий своих слуг на посрамление. Эгоизм есть по существу монотеизм, поскольку он имеет целью только одно --- себя самого. Эгоизм объединяет, сосредоточивает человека на самом себе, дает ему твердый, цельный принцип жизни, но ограничивает его теоретически, так как делает его равнодушным по всему, что не касается непосредственно его личного блага. Наука, как и искусство, возникает только из политеизма, ибо политеизм есть откровенное, прямодушное понимание всего прекрасного и доброго, понимание мира, вселенной, греки устремляли взор свой в необъятный мир, чтобы расширить свой кругозор; а евреи во время молитвы ещё теперь обращают взор к Иерусалиму. Одним словом, монотеистический эгоизм отнял у израильтян свободное теоретическое стремление и понимание. Соломон, несомненно, превосходил своим разумом и мудростью всех детей востока и говорил (рассуждал) <<о деревьях, о ливанском кедре, о вырастающем из стены иссопе, о животных, птицах, червях и рыбах>>\ddag\let\svthefootnote\thefootnote\let\thefootnote\relax\footnotetext{\ddag 3. Цар., 4, 33.}\let\thefootnote\svthefootnote\footnote{см. Третью Книгу Царств.}, но зато он служил Иегове не от всего сердца. Соломон благоволил к чужим богам и женщинам; он разделял политеистический образ мыслей и вкусы. Повторяю: политеистический уклон есть основа науки и искусства.



Взгляд еврея на природу объясняет и взгляд его на её происхождение. В том, как я объясняю себе происхождение какой-нибудь вещи, выражается только мое откровенное о ней мнение, мой взгляд на нее. Если я отношусь к ней презрительно, но и приписываю ей недостойное происхождение. Некогда люди думали, что черви и насекомые происходят от падали и всяких нечистот. Презрительное отношение к насекомым было не следствием, а причиной того, что им приписывали нечистое происхождение. В глазах евреев природа была простым средством к достижению эгоистических целей, простым объектом воли. Но идеалом, кумиром эгоистической воли является такая воля, которая неограниченно повелевает, не нуждается в средствах для достижения своей цели, для реализации своего предмета и призывает к существованию все, что ей угодно, непосредственно, само собой, то есть просто волей. Эгоист страдает, если не может удовлетворить своих желаний и потребностей непосредственно, то есть если между предметом и желанием, между целью в действительности и целью в представлении есть трещина. И вот, чтобы избавиться от этого страдания и освободиться от границ действительности, он признает истинным, высшим существом существо, создающее вещи одним словом <<я хочу>>. Поэтому еврей считал природу, мир продуктом диктаторского слова, категорического императива, волшебного заклинания.

Все, что не имеет для меня теоретического значения, все, что я не признаю существенным в теории или разуме, то лишено для меня теоретического, существенного основания. Волей я лишь усиливаю, реализую его теоретическое ничтожество. Мы не обращаем внимания на то, что презираем. Человек обращает внимание лишь на то, что он уважает. Созерцание есть признание. Все созерцаемое нами очаровывает кичливую волю, стремящуюся все подчинить себе, очаровывает своей скрытой притягательной силой, покоряет своим внешним видом. Все, что производит впечатление на теоретическую мысль, на разум, выходит из-под власти эгоистической воли, реагирует, оказывает сопротивление. Что разрушительный эгоизм обрекает смерти, то исполненная любви теория возвращает жизни.

Столь упорно отвергаемое учение языческих философов о вечности материи или мира имеет только тот смысл, что природа была для них теоретической истиной*\let\svthefootnote\thefootnote\let\thefootnote\relax\footnotetext{*впрочем они, как известно, думали о природе различно (см., например, у Аристотеля, De coelo, lib. \rom{1}, c. 10). Но их разномыслие являлось несущественным, так как творческое существо и у них представляется существом более или менее космическим.}\let\thefootnote\svthefootnote\footnote{см.: <<О небе>>, кн. \rom{1}, гл. 10 (лат.).}. Язычники были идолопоклонники, то есть они созерцали природу; они поступали так же, как поступают теперь глубоко христианские народы, делающие природу предметом своего восхищения, своего неустанного исследования. <<Но ведь язычники поклонялись предметам природы>>. Это правда, но поклонение есть только детская религиозная форма созерцания. Созерцание и поклонение --- по существу одно и то же. Я смиренно приношу в жертву тому, что созерцаю, лучшее, что я имею: мое сердце, мою мысль. Естествоиспытатель тоже преклоняется перед природой, когда он с опасностью для жизни извлекает из недр земли какой-нибудь лишай, какое-нибудь насекомое или камень, чтобы восславить их светом созерцания и увековечить в научной памяти человечества. Изучение природы есть служение природе, идолопоклонство в смысле иудейского и христианского бога, а идолопоклонство есть не что иное, как примитивное миросозерцание человека; ведь религия есть не что иное, как примитивное, поэтому детское народное, но ещё ограниченное и несвободное созерцание человеком природы и себя. Евреи, напротив, возвышались над идолопоклонством до служения богу, над тварью до созерцания творца, то есть возвышались над теоретическим созерцанием восхищавшей язычников природы до чисто практического созерцания, подчиняющего природу только целям эгоизма. <<Не поднимай очей твоих к небу, не смотри на солнце, месяц и звезды, не будь отступником, не поклоняйся и не служи этим небесным полчищам, созданным (то есть подаренным) господом богом твоим для всех народов под всем небом>>\dag\let\svthefootnote\thefootnote\let\thefootnote\relax\footnotetext{\dag<<Хотя небесные тела созданы не человеком, но все-таки они созданы ради людей. Поэтому никто не должен поклоняться солнцу, но мысленно возноситься к творцу солнца>> (Климент Александрийский, Coh. ad. gentes).}\let\thefootnote\svthefootnote\footnote{см.: Пятая книга Моисеева. Второзаконие, 4, 19.}. Итак, творение из ничего, то есть творение только как акт деспотический, объясняется неизмеримой глубиной и силой еврейского эгоизма.





На том же основании творение из ничего --- во всяком случае в том виде, как оно изложено здесь, --- не есть предмет философии, потому что оно в корне уничтожает всякое истинное умозрение и не представляет никакой точки опоры для мышления, для теории. Это для теории беспочвенное и необоснованное учение лишь подтверждает утилитаризм, эгоизм выражает только приказание делать природу предметом пользования и наслаждения, а не мышления и созерцания. Но, разумеется, тем бессодержательнее такое учение для естественной философии, тем глубже становится его <<умозрительное>> значение; ибо, не имея никакой теоретической точки опоры, оно доставляет для умозрения безграничный простор произвольных, необоснованных толкований и умствований.

История догматов и умозрений уподобляется истории государств. Древние обычаи, права и учреждения существуют долгое время после того, как они утратили всякий смысл. То, что некогда существовало, не желает отказаться от своих прав на вечное существование; то, что когда-то было хорошо, желает остаться хорошим навеки. Затем на сцену выступают толкователи и философы и начинают рассуждать о глубоком смысле, потому что истинный смысл им не ведом\ddag\let\svthefootnote\thefootnote\let\thefootnote\relax\footnotetext{\ddagразумеется, лишь в отношений абсолютной религии, так как относительно других религий они выставляют бессмысленными и смешными те или иные представления и обряды этих религий, обряды, нам чуждые, первоначальный смысл и цель которых нам неизвестны. Между тем почитание, например, мочи коров, которую пьют парсы и индусы, чтобы получить прощение грехов, на самом деле нисколько не смешнее, чем почитание гребня или обрывка риз матери божией.}\let\thefootnote\svthefootnote. Точно так же непоследовательно религиозное умозрение рассматривает и догматы вне той связи, при которой они только и имеют смысл; оно не сводит их критически к их истинному внутреннему источнику; оно приписывает второстепенному значению главного, и наоборот. Оно считает бога первым, а человека вторым. Таким образом извращается естественный порядок вещей! Первое, это --- как раз человек, второе --- объективированная сущность человека: бог. Только позднее, когда религия облеклась в плоть и кровь, можно сказать: человек таков, таков его бог; но и это положение выражает лишь тавтологию. Однако и первоначально было иначе, а ведь только изучая происхождение, можно познать его истинную сущность. Сперва человек бессознательно и непроизвольно создает по своему образу бога, а затем уже этот бог сознательно и произвольно создает по своему образу человека. Это прежде всего подтверждается ходом развития израильской религии. Отсюда возникло недоговорённое теологическое положение, будто откровение божие идет рука об руку с развитием человеческого рода. Это, конечно, так; ибо откровение божие есть не что иное, как откровение, самораскрытие человеческого существа. Супранатуралистический эгоизм евреев не вытекал из создателя, а наоборот: в творении только как бы оправдывал израильтянин свой эгоизм перед судом своего разума.



Разумеется, и израильтянин как человек, даже из практических оснований не мог уклониться от теоретического созерцания и преклонения перед природой. Но прославляя мощь и величие природы, он только прославлял мощь и величие Иеговы. Эта мощь Иеговы наиболее ярко проявилась в чудесных творениях, созданных им на благо Израиля. Следовательно, прославляя эту мощь, израильтянин опять-таки имеет в виду самого себя. Он прославляет величие природы по тем же соображениям, по которым победитель преувеличивает силу своего противника, чтобы тем в большей степени удовлетворить свое тщеславие и ещё более себя прославить. Велика и могущественна природа, созданная Иеговой, но ещё более велико и могущественно себялюбие Израиля. Ради него останавливается солнце; ради него, при объявлении закона, происходит землетрясение; одним словом, ради него изменяется все существо природы. <<Вся тварь снова свыше преобразовалась в своей природе, повинуясь особым повелениям, дабы сыны твои сохранились невредимыми>>. (Премудр. 19, 6). Бог, по Филону, дал Моисею власть над всей природой; каждая стихия повиновалась ему, как владыке природы. Потребности Израиля --- всесильный мировой закон, нужда Израиля --- судьба мира. Иегова есть сознание  Израилем святости и необходимости своего существования --- необходимости, перед которой бытие природы, бытие других народов исчезает в ничто. Иегова --- salus populi\footnote{благо народа (лат.).}, спасение Израиля, в жертву которому должно быть принесено все, стоящее на его дороге. Иегова --- исключительный, монархический эгоизм, истребительный гнев в пламенных очах мстительного Израиля; одним словом, Иегова есть <<Я>> Израиля, признавшего себя конечной целью и владыкой природы. Итак, в мощи природы израильтянин прославляет мощь Иеговы, а в мощи Иеговы --- мощь собственного самосознания. <<Слава тебе, боже! Бог --- помощь наша, бог --- спасение наше!>>, <<Иегова бог --- моя сила>>. <<Сам бог повиновался слову героя (Иешуа), ибо он, сам Иегова, сражался перед Израилем>>. <<Иегова --- бог войны>>*\let\svthefootnote\thefootnote\let\thefootnote\relax\footnotetext{*по Гердеру.}\let\thefootnote\svthefootnote.



Постепенно для отдельных умов идея Иеговы приобрела более широкое значение, и любовь его распространилась на людей вообще, как мы это, например, видим у автора книги Ионы. Но это не характерно для израильской религии. Бог отцов, с которым связываются самые дорогие воспоминания, древний исторический бог останется навсегда основанием религии\dag\let\svthefootnote\thefootnote\let\thefootnote\relax\footnotetext{\dagздесь необходимо заметить, что преклонение перед мощью и величием бога вообще, так же как Иеговы, в природе является лишь преклонением перед мощью и величием природы --- если и не в сознании израильтянина, то фактически (См. об этом <<P. Bayle>>, Feuerbachs Werke, 2. Aufl., Bd. \rom{6}, S. 38). Но доказывать это формально не входит в наш план, так как здесь мы останавливаемся только на христианстве, то есть на почитании бога в человеке. Впрочем, принцип такого доказательства достаточно выражен уже и в настоящем сочинении.}\let\thefootnote\svthefootnote\footnote{см.: Р. Кедворт. <<Истинная духовная система мира>>, гл. 5, разд. 5, \S 27 (англ.).}.







\chapter{Сила чувства или тайна молитвы}


Израиль есть историческое определение своеобразной природы религиозного сознания, причем здесь это сознание суживалось ещё национальными интересами. Если отбросить национальную ограниченность, то получится христианская религия. Иудейство есть мирское христианство, а христианство --- духовное иудейство. Христианская религия есть очищенная от национального эгоизма иудейская религия и во всяком случае одновременно новая, другая религия; ведь всякая реформация, всякое очищение религии влечет за собой её существенное изменение, потому что здесь даже незначительное имеет значение. Для еврея Израиль был посредником, связующим звеном между богом и человеком; в его отношениях к Иегове обнаруживалось его отношение к себе как Израилю. Иегова олицетворял собой единство, самосознание Израиля, объективированное как абсолютная сущность, национальную совесть, всеобщий закон, средоточие политики*\let\svthefootnote\thefootnote\let\thefootnote\relax\footnotetext{*<<Большая часть еврейской поэзии, которая считается духовной, носит характер политический>> (Гердер).}\let\thefootnote\svthefootnote. Отбросьте ограниченность национального самосознания, и вместо Израиля у вас получится человек. Израильтянин объективировал в Иегове свою национальную сущность, а христианин объективировал в боге освобождённую от национализма человеческую и при этом субъективно-человеческую сущность\dag\let\svthefootnote\thefootnote\let\thefootnote\relax\footnotetext{\dagсубъективно-человеческую потому, что человеческая сущность, соответствующая сущности христианства, есть сущность супранатуралистическая, исключающая природу, тело, чувственность, которые одни только и дают нам объективный мир.}\let\thefootnote\svthefootnote. Как Израиль возводил потребности, необходимость своего существования на степень мирового закона и в силу этой необходимости обожествлял свою политическую мстительность, так и христианин обратил в могущественные мировые законы потребности человеческой души. Чудеса христианства, которые так же хорошо характеризуют христианство, как чудеса Ветхого завета иудейство, способствуют не благу одной какой-нибудь нации, а благу человека, но только человека, непременно верующего во Христа; ведь христианство, становясь в противоречие с подлинным универсальным человеческим сердцем, признает человеком только христианина. Но об этом роковом ограничении будет речь ниже. В христианстве иудейский эгоизм одухотворился в субъективность, и цель иудейской религии --- желание земного благополучия, --- превратилась в цель христианскую, в стремление к небесному блаженству. Но христианская субъективность есть тот же эгоизм.





Высшее понятие, бог политической общины, народа, политика которого выражается в форме религии, есть закон, сознание закона как абсолютной божественной власти; высшее понятие, бог чуждого мира и политики человеческого чувства есть любовь --- любовь, приносящая в жертву своему возлюбленному все небесные и земные блага и сокровища, любовь, считающая законом желание возлюбленного, а силой --- неограниченную мощь фантазии, чудодейственную силу мысли.

Бог есть любовь, исполняющая наши желания, наши духовные потребности; он есть осуществлённое желание сердца, желание, достигшее несомненной уверенности в своем исполнении и преодолевающее преграды рассудка, опыта и внешнего мира. Уверенность является для человека высшей силой; то, в чем он уверен, кажется ему сущим, божественным. Бог есть любовь; это изречение, высшее изречение христианства, есть только выражение самоуверенности человеческого сердца, уверенности в себе как в исключительно полномочной, то есть божественной, силе, выражение уверенности в том, что заветные сердечные желания человека абсолютно ценны и истинны, что человеческое чувство не ограничено и не имеет в мире противовеса, что целый мир, со всем его великолепием и блеском, есть ничто в сравнении с человеческой душой\ddag\let\svthefootnote\thefootnote\let\thefootnote\relax\footnotetext{\ddag<<Нет ничего, чего не мог бы ожидать добрый и праведный человек от божественной благости; ворующий и бога может надеяться на все блага, какие только нужны человеческому существу, на все вещи, каких ещё не видел глаз, не слышало ухо и но постигал человеческий рассудок; ведь бесконечную надежду неизбежно питают те, кто верует, что существо бесконечной благости и всемогущества печется о человеческих делах и что души наши бессмертны. И такую надежду ничто не может ни поколебать, ни уничтожить, если только не предаваться порокам и но вести нечестивой жизни>> (Кэдворт, Syst. Intellect., cap. 5, sect. 5, par. 27).}\let\thefootnote\svthefootnote\footnote{см.: <<Пьер Бейль. К истории философии и человечества>> (1838 г.).}. Бог есть любовь; значит, чувство есть бог человека, бог --- безоговорочно абсолютное существо. Бог есть объективированная сущность чувства, неограниченное, чистое чувство; он есть вожделение человеческого сердца, обращённое в определенное блаженное бытие. Бог есть безотносительное всемогущество чувства, себе внимающая молитва, внемлющая себе душа, эхо наших стенаний. Скорбь просится наружу. Артист невольно берется еа лютню, чтобы излить свое горе в её звуках. Его грусть рассеивается, когда он доводит его до своего слуха и объективирует; тяжесть спадает с его сердца, когда он сообщает свое горе воздуху и делает его общей сущностью. Но природа не внемлет жалобам человека --- она безучастна к его страданиям. Поэтому человек отворачивается от природы и видимых предметов вообще --- он обращается внутрь себя, чтобы скрыться от равнодушных сил и найти сочувствие к своим страданиям. Здесь он открывает свои тяжелые тайны, здесь облегчает он свое угнетенное сердце. Это облегчение сердца, эта высказанная тайна, это обнаруженное душевное страдании есть бог. Бог есть слеза любви, пролитая в глубоком уединении над человеческими страданиями. <<Бог есть невыразимый вздох, скрытый в глубине душ>>*\let\svthefootnote\thefootnote\let\thefootnote\relax\footnotetext{*Себастиан Франк фон-Верд в Zinkgrefs Apophthegmata deutscher Nation.}\let\thefootnote\svthefootnote --- вот наиболее замечательное, глубокое, истинное выражение христианской мистики.





Сокровеннейшая сущность религии открывается в простейшем религиозном акте --- молитве. Этот акт имеет, если не большее, то по крайней мере такое же значение, как догмат воплощения, хотя религиозное умозрение и усматривает в нем величайшую тайну. Но здесь имеется в виду не сытая молитва эгоистов, произносимая до и после еды, а скорбная молитва, молитва безутешной любви, молитва, зарождающаяся в отчаянии и кончающаяся в блаженстве.

В молитве человек обращается к богу на ты, то есть громко и открыто объявляет бога своим вторым <<Я>>; он открывает богу, как ближайшему, самому доверенному существу, свои сокровеннейшие помыслы, свои тайные желания, которые он не решается высказать вслух. Он выражает эти желания в надежде, в уверенности, что они будут исполнены. Как мог бы он обратиться к существу, не внимающему его жалобам? Следовательно, молитва есть желание сердца, выраженное с уверенностью в его исполнении\dag\let\svthefootnote\thefootnote\let\thefootnote\relax\footnotetext{\dagбыло бы слабым возражением, если б кто сказал, что бог, исполняет только те желания и просьбы, которые исходят во имя его или в интересах церкви христовой, словом, только те желания, которые согласуются с его волею; ведь воля божия есть воля человека, или, лучше, богу принадлежит сила, а человеку воля. Бог делает человека блаженным, а человек хочет быть блаженным. То или другое единичное желание может, конечно, остаться неисполненным; но это не имеет большого значения, если только род, существенная тенденция в общем приемлются. Поэтому благочестивый человек, молитва которого осталась неисполненной, утешает себя тем, что исполнение её не было бы для него спасительно. См., например, Oratio de precatione, в Declamat. Melanchtнonis, G. \rom{3}.}\let\thefootnote\svthefootnote, а существо, исполняющее эти желания, есть не что иное, как внемлющее себе, одобряющее себя, безусловно себя утверждающее человеческое сердце. Человек, который не может отрешиться от реального представления о мире, от представления о том, что все имеет связь и естественную причину, что всякое желание достигается только, когда оно стало целью, при помощи соответствующих средств, --- такой человек не молится; он только работает, он обращает осуществимые желания в цели реальной деятельности; а остальные желания, признаваемые им за субъективные, он или отрицает совершенно, или рассматривает только как субъективные, благочестивые желания. Одним словом, он обусловливает свои желания представлением необходимости и ограничивает свое существо миром, членом которого он себя считает. В молитве же, напротив, человек отрешается от мира и вместе с ним от всяких мыслей о посредниках, зависимости и печальной необходимости; свои желания, движения своего сердца он превращает в объекты независимого, всемогущего, абсолютного существа, то есть утверждает их без всяких ограничений. Бог есть утверждение человеческого чувства, молитва --- безусловная уверенность человеческого сердца в абсолютном тождестве субъективного и объективного, уверенность, что сила сердца преобладает над силой природы, что потребность сердца есть абсолютная необходимость, судьба мира. Молитва изменяет естественный ход вещей, она побуждает бога совершать действия, противоречащие законам природы. Молитва есть отношение человеческого сердца к себе самому, к своей собственной сущности. В молитве человек забывает об ограниченности своих желаний, и в этом забвении заключается его блаженство.



Молитва есть саморасчленение человека на два существа --- беседа человека с самим собой, со своим сердцем. Действительность молитвы обусловливается её громким, ясным, выразительным произношением. Молитва выливается из уст непроизвольно: от избытка сердца уста глаголят. Но громкая молитва открывает только сущность молитвы; молитва по существу своему есть речь, даже когда она не произносится слух. Латинское слово oratis обозначает и то, и другое. В молитве человек откровенно высказывает все, что его тяготит, что его близко касается; в молитве он объективирует свое сердце --- в этом моральная сила молитвы. Говорят, что сосредоточенность является условием молитвы. Но это больше, чем условие. Молитва сама до себе есть собранность, устранение всяких отвлекающих представлений и внешних влияний, исключительное обращение к своей собственной сущности. Говорят, что только уверенная, откровенная, сердечная, искренняя молитва может помочь, но эта помощь заключается в молитве. Религия везде --- в субъективном, человеческом, подчиненном усматривает главное, prima causa\footnote{см. примеч. 22.}; так и здесь субъективные свойства кажутся ей объективной сущностью молитвы*\let\svthefootnote\thefootnote\let\thefootnote\relax\footnotetext{*из субъективных оснований молитва сообща действительнее чем молитва отдельного лица. Общение возвышает силу чувства, усиливает самочувствие. Чего не может сделать один, делается сообща с другими. Чувство одного есть чувство ограниченности, а чувство общения есть чувство свободы. Поэтому люди и собираются вместе, когда им угрожают силы природы. <<Немыслимо, --- говорит Амвросий, --- чтобы молитва многих не была услышана\dots В чем отказывают одному, не отказывают множеству людей>> (P. Paul Mezger, Sacra Hist. de gentis hebr. ortu. Aug. \rom{5}. 1700, p. 668--69).}\let\thefootnote\svthefootnote.



Только поверхностные люди могут усматривать в молитве одно выражение чувства зависимости. Правда, в молитве выражается зависимость, но зависимость человека от своего сердца, от своих чувств. Кто чувствует себя только зависимым, тот не может молиться: чувство зависимости отнимает у него необходимые для этого охоту и мужество; ведь чувство зависимости есть чувство необходимости. Молитва скорее коренится в безусловном, свободном от всякой принудительности сердечном доверии к тому, что его интересы служат объектом абсолютного существа, что всемогущее, неограниченное существо, отец людей, есть существо, полное участия, чувства, любви, что таким образом наиболее дорогие и святые чувства и желания человека являются божественной истиной. Ребенок не чувствует себя зависимым от отца как отца; скорее, в отце он видит свою опору, сознание своей ценности, залог своего существования, уверенность в исполнении своих желаний. Отец обременён заботами; ребенок, наоборот, беспечен и счастлив в своем доверии к отцу, к своему живому хранителю, который не желает ничего, кроме блага и счастья своего ребенка. Отец делает ребенка целью, а самого себя средством его существования. Ребенок, просящий о чем-нибудь своего отца, обращается к нему не как к отличному от него, самостоятельному существу, не как к господину и вообще лицу, а лишь поскольку сам отец зависит от своих отцовских чувств и определяется любовью к своему ребенку. Просьба есть только выражение власти ребенка над отцом, если только здесь применимо выражение власть, потому что власть ребенка есть не что иное, как власть отцовского сердца. Просьба и приказание выражаются одной и той же формой --- повелительным наклонением. Просьба есть императив любви. И этот приказ гораздо сильнее всякого приказа деспота. Любовь не повелевает; любви достаточно сделать только намек, чтобы желание её было исполнено; деспот должен говорить властным тоном, чтобы заставить равнодушных к нему людей исполнить его желание. Императив любви действует с электромагнетической силой, а деспотический --- механически, подобно деревянному телеграфу. Название отец является интимнейшим выражением бога в молитве --- интимнейшим, поскольку здесь человек относится к абсолютному существу, как к своему собственному. Ведь слово отец выражает полное внутреннее единство и служит непосредственно ручательством моих желаний, залогом моего спасения. Всемогущество, к которому человек обращается в молитве, есть только всемогущество благости, делающей возможным невозможное ради спасения человека; на самом деле она есть не что иное, как всемогущество сердца, чувства, преодолевающего все преграды разума, все границы природы и желающего, чтобы существовало одно только чувство и погибло все, противоречащее сердцу. Вера во всемогущество есть вера в ничтожество внешнего мира, объективности --- вера в абсолютную истинность и ценность чувства. Сущность всемогущества выражает собой только сущность чувства. Всемогущество есть сила, перед которой не может устоять ни один закон, ни одно определение природы; но эта сила есть само чувство, ощущающее и уничтожающее, как преграду, всякую необходимость, всякий закон. Всемогущество только исполняет, осуществляет сокровенную волю сердца. В молитве человек обращается к всемогуществу благости --- это значит, что в молитве человек поклоняется своему собственному сердцу, рассматривает сущность чувства как высшую, божественную сущность.





\chapter{Тайна веры --- тайна чуда}



Вера в силу молитвы, --- а молитва является религиозной истиной только тогда, когда ей приписывают силу и власть над предметами, окружающими человека, --- равносильна вере в силу чуда, а вера в чудо тождественна с сущностью веры вообще. Только вера способна молиться, только молитва веры имеет силу. Но вера есть не что иное, как совершенная уверенность в реальности, то есть в безусловной силе и истинности субъективного в противоположность границам, то есть законам природы и разума. Поэтому специальный объект веры есть чудо, вера есть вера в чудо, вера и чудо абсолютно нераздельны. Все, что является чудом или чудотворной силой в смысле объективном, есть вера в смысле субъективном; чудо есть внешнее проявление веры, вера есть внутренняя душа чуда, чудо духа; вера есть чудо чувства, лишь объективирующего себя во внешнем чуде. Для веры нет ничего невозможного, и это всемогущество веры осуществляется только во внешнем чуде. Чудо есть только осязательный пример того, что может сделать вера. Беспредельность, безмерность чувства, одним словом: супранатурализм, сверхъестественность есть сущность веры. Вера относится только к тому, что вопреки ограничениям, то есть законам природы и разума, объективирует всемогущество человеческого чувства, человеческих желаний. Вера освобождает желания человека от уз естественного разума; она одобряет то, что отрицается природой и разумом; она доставляет человеку блаженство, потому что удовлетворяет его субъективные желания. Истинная вера не смущается никаким сомнением. Сомнение возможно только там, где я выхожу за пределы моего существа, переступаю грань моей субъективности, приписываю реальность и право голоса чему-нибудь другому, отличному от меня, где я сознаю субъективность, то есть ограниченность моего существа, и стараюсь расширить ограничивающие меня пределы. Но вере чужд сам принцип сомнения, потому что она считает субъективное --- объективным, абсолютным. Вера есть не что иное, как вера в божественность человека.

<<Вера есть мужественное убеждение сердца, что бог наделяет тебя всякими благами. Такой веры, возлагающей все свое упование на бога, бог требует в первой заповеди, когда он говорит: Я господь бог твой\dots это значит: я хочу быть твоим единым богом, ты не должен искать другого бога; я помогу тебе во всякой нужде\dots Не думай, что я твой враг и не желаю помогать тебе. Думая так, ты считаешь меня не тем, что я есть. Будь уверен, что я хочу быть милостивым>>. <<Бог относится к тебе так, как ты относишься к нему. Если ты думаешь, что он гневается на тебя, значит, он гневается. Если ты думаешь, что он не жалеет тебя и хочет ввергнуть тебя в ад, значит, это действительно так. Бог таков, каким ты его себе представляешь>>. <<Во что ты веришь, тем ты и владеешь; то, чему ты не веришь, того и нет у тебя>>. <<Поэтому с нами происходит то, во что мы верим. Если мы считаем бога своим богом, то он, разумеется, не может быть чёртом. Если же мы не считаем его своим богом, значит, он не бог, а всепожирающий огонь>>. <<Благодаря неверию мы превращаем бога в чёрта>>*\let\svthefootnote\thefootnote\let\thefootnote\relax\footnotetext{*Лютер, ч. \rom{15}, стр. 282; ч. \rom{16}, стр. 491--492.}\let\thefootnote\svthefootnote. Итак, если я верю в бога, значит, у меня есть бог, то есть вера в бога есть бог человека. Если мой бог таков, какова моя вера, значит, существо божие есть только сущность веры. Ты не можешь верить в благого бога, если ты недобр по отношению к себе, если ты не веришь в человека, если он не имеет для тебя значения. Веруя, что бог существует для тебя, ты веришь в то, что ничто не может противиться и противоречить тебе. Но веруя, что ничто не может противиться тебе, ты веруешь в то, что ты --- ни много, ни мало, --- сам бог\dag\let\svthefootnote\thefootnote\let\thefootnote\relax\footnotetext{\dag<<Бог всемогущ; а кто имеет веру, тот есть бог>> (Лютер, ч. \rom{14}, стр. 320). В другом месте Лютер прямо называет веру <<творцом божества>>; конечно, он тут же, что необходимо с его точки зрения, ставит ограничение: <<не потому, чтобы вера придавала что-нибудь к вечной божественной сущности, а потому, что в нас творит она нечто новое>> (ч. \rom{11}, стр. 161).}\let\thefootnote\svthefootnote. Бог не есть другое существо: это только призрак, одно воображение. Бог есть твоя собственная сущность: это выражается уже тем, что бог, по-твоему, существует для тебя. Итак, вера есть только непоколебимая самоуверенность человека, его несомненная уверенность в то, что его собственная субъективная сущность есть объективная, абсолютная сущность, существо существ.







Представление мира, вселенной, неизбежности не ограничивает веры. Вера считается только с богом, то есть с неограниченной субъективностью. Если в человеке зарождается вера, значит, мир для него перестал или перестает существовать. Поэтому вера в истинную, близкую, ожидаемую кончину этого противоречащего христианским желаниям мира есть проявление сокровенной сущности христианской веры. Эта вера нераздельна с остальным содержанием христианской веры, и уничтожение её равносильно уничтожению, отрицанию истинного положительного христианства\ddag\let\svthefootnote\thefootnote\let\thefootnote\relax\footnotetext{\ddagэта вера так существенна для Библии, что без нее библия не может быть понята. Место из второго послания апостола Петра (3, 8) не говорит, как видно из текста всей главы, о скорой кончине мира, ибо у господа один день, как тысяча лет, и тысяча лет как один день; и мир поэтому уже завтра может перестать существовать. Вообще в Библии ожидается и предсказывается очень близкий конец мира, хотя и не определяется ни дня, ни часа; только лжец или слепец может отрицать это. Смотри об этом также соч. Люцельбергера. Поэтому религиозные христиане верили почти во все времена в близкую кончину мира --- Лютер, например, часто говорит <<что судный день близок>> (например, ч. \rom{16}, стр. 26) --- или они по крайней мере, в сердце своем желали конца мира, хотя из благоразумия и не старались определить, близок он или нет (см., например, Августин, De fine saeculi ad Hesychium, гл. 13).}\let\thefootnote\svthefootnote. Сущность веры, проявляющейся во всех своих частях, сводится к существованию того, что желательно человеку. А он желает быть бессмертным, и он бессмертен; он желает, чтобы было такое существо, власть которого превышала бы власть природы и разума, и оно существует. Он желает, чтобы существовал мир, отвечающий желаниям его сердца, мир неограниченной субъективности, то есть мир невозмутимого чувства, непрерывного блаженства; но тем не менее существует мир, противоположный этому сердечному миру, и поэтому он должен погибнуть, что так же неизбежно, как необходим бог, необходима абсолютная сущность человеческого чувства. Вера, любовь и надежда составляет христианскую троицу. Надежда относится к исполнению обетов --- тех желаний, которые еще не исполнены, но будут исполнены; любовь --- к существу, дающему и исполняющему эти обеты; вера --- к тем обещаниям, которые уже почти исполнились, стали историческими фактами.



Чудо есть существенная часть христианства, существенное содержание веры. Но что такое чудо? Это не что иное, как осуществившееся супранатуралистическое желание. Апостол Павел поясняет сущность христианской веры на примере Авраама. Авраам не мог рассчитывать на потомство естественным путем. Иегова обещал ему потомство в виде особой милости. И Авраам верил в это обещание вопреки природе. Поэтому его вера считается праведной заслугой; ведь необходима большая сила воображения, чтобы признать достоверным то, что противоречит опыту, по крайней мере разумному, закономерному опыту. Но что же было предметом этого божественного обетования? Потомство: предмет человеческого желания. Во что же верил Авраам, веруя в Иегову? Он верил во всесильное существо, могущее исполнить все человеческие желания. <<Есть ли что трудное для господа?>>*\let\svthefootnote\thefootnote\let\thefootnote\relax\footnotetext{*Первая книга Моисея, 8:14.}\let\thefootnote\svthefootnote.



Впрочем, зачем нам брать в пример Авраама. Очевидные доказательства этого встречаются и в позднейшее время. Чудо насыщает голодных, исцеляет слепорожденных, глухих и хромых, спасает людей от опасностей жизни, воскрешает мертвых по просьбе их родственников. Следовательно, чудо удовлетворяет человеческие желания, которые не всегда имеют в виду только себя, как, например, желание воскресить мертвого, но всегда претендуют на чудесную силу, чудесную помощь и поэтому являются сверхъестественными, супранатуралистическими желаниями. Но чудо отличается от естественного и разумного способа удовлетворения человеческих желаний и потребностей тем, что оно удовлетворяет желания человека способом, соответствующим сути его желания, достойным его. Желание не считается ни с пределом, ни с законом, ни со временем; оно хочет быть исполненным немедленно, мгновенно. И вот --- тотчас вслед за желанием происходит чудо! Сила чуда осуществляет человеческие желания моментально, непосредственно, не считаясь с препятствиями. Выздоровление больных не есть чудо; тайна чуда заключается в том, что они выздоравливают по одному слову. Таким образом, не созидаемым продуктом или объектом осуществляет чудо нечто абсолютно новое, невиданное, непредставляемое и немыслимое, в этом случае оно было бы существенно иной и вместе с тем объективной деятельностью, --- только своим способом, своим образом действия отличается чудо от деятельности природы и разума. Но деятельность, которая по существу, по содержанию есть естественная, чувственная и только по способу действия или по форме представляется сверхъестественной, сверхчувственной, --- такая деятельность есть только фантазия или сила воображения. Поэтому сила чуда есть только сила воображения.

Чудотворная деятельность есть деятельность целесообразная. Тоска по умершему Лазарю, желание родственников вернуть его к жизни были побудительной причиной его чудесного воскрешения, а самое дело воскрешения было удовлетворением их желания, было целью. Разумеется, чудо совершилось <<во славу божию, чтобы прославился сын божий>>\footnote{см.: Иоан., 11, 4.}. Но слова пославших за господом сестер Лазаря: <<видишь, друг твой болен>>\footnote{см.: Иоан., 11, 3.}, --- и слезы, пролитые Иисусом, придают чуду человеческое происхождение и цель. Смысл таков: сила, воскрешающая мертвых, может исполнить всякое человеческое желание\dag\let\svthefootnote\thefootnote\let\thefootnote\relax\footnotetext{\dag<<Для целого мира невозможно воскресить мертвого, но для господа Иисуса это не только возможно, но не требует ни усилия, ни труда\dots Христос сделал это во свидетельство, что он может и хочет спасти от смерти. Он не делает этого всегда и для всякого\dots Довольно и того что он совершил это единожды, а прочее бережёт для судного дня>>. (Лютер, ч. \rom{16}, стр. 518). Положительное существенное значение чуда состоит поэтому в том, что божественная сущность не что иное, как сущность человеческая. Чудеса подтверждают, скрепляют учение. Но какое учение? Именно то учение что бог есть спаситель людей, избавляющий их от всякой нужды то есть он есть существо, отвечающее потребностям и желаниям человека, следовательно, существо человеческое. Что богочеловек выражает лишь словами, то фактами наглядно подтверждает чудо.}\let\thefootnote\svthefootnote\footnote{см. примеч. 24.}. Слава сына именно в том и заключается, что он признается и почитается как существо, которое может творить то, чего не может, но хочет человек. Целесообразная деятельность, как известно, описывает друг: в конце она возвращается к своему началу. Но чудесная деятельность отличается от обыкновенного осуществления цели тем, что она осуществляет цель без помощи средств, создает непосредственное единство желания и исполнения и, следовательно, описывает круг не по кривой, а по прямой, то есть по кратчайшей линии. Круг по прямой линии есть математическое изображение и символ чуда. Насколько нелепа попытка построить круг по прямой линии, настолько же нелепо желание объяснить чудо путем философии. Чудо так же бессмысленно, недопустимо для разума, как немыслимо, например, деревянное железо или круг без окружности. Прежде чем толковать, возможно ли чудо, надо доказать, возможно ли мыслить чудо, то есть мыслить немыслимое.



Человек считает чудо возможным потому, что чудо носит характер чувственного события, и разум человека обольщается чувственными представлениями, скрывающими за собой противоречие. Например, чудесное превращение воды в вино свидетельствует, в сущности, о том, что вода есть вино, то есть тождество двух абсолютно противоречивых предикатов или субъектов; ведь в руке чудотворца нет различия между обеими субстанциями, и превращение есть только видимое проявление тождества вещей противоречивых. Превращение прикрывает противоречие, так как у нас является естественное представление изменения. Но это не есть постепенное, естественное, так сказать, органическое изменение, а изменение безусловное, невещественное, чистое творение из ничего. В таинственном, роковом акте чуда, в акте, делающем чудо чудом, вода становится вином внезапно, а это равносильно заявлению: железо есть дерево, или деревянное железо.

Действие чуда (а чудо мгновенно) не есть действие мыслимое, так как оно уничтожает начало мыслимости, но в то же время оно не есть объект чувства, объект действительного или только возможного опыта. Правда, вода и вино --- объекты чувства; я вижу и воду, и потом вино; но самое чудо, внезапно превращающее воду в вино, не есть естественный процесс, не есть объект действительного или только возможного опыта. Чудо есть дело воображения, поэтому оно так и удовлетворяет чувство; ведь фантазия есть деятельность, соответствующая сердцу, раз она устраняет все удручающие чувство границы и законы и объективирует непосредственное, безусловно неограниченное удовлетворение самых субъективных желаний человека\ddag\let\svthefootnote\thefootnote\let\thefootnote\relax\footnotetext{\ddagконечно, это удовлетворение и что подразумевается, впрочем, само собой --- является ограниченным постольку, поскольку оно связано с религией и верою в бога. Но это ограничение на самом деле не есть ограничение, так как сам бог есть неограниченная, абсолютно удовлетворенная, в себе насыщенная сущность человеческого чувства.}\let\thefootnote\svthefootnote. Сердечность составляет существенное свойство чуда. Чудо производит возвышенное, потрясающее впечатление, поскольку оно выражает силу, перед которой ничто не может устоять --- силу фантазии. Но такое впечатление заключается лишь в преходящем действии чуда, длительное существенное впечатление есть впечатление душевное. В момент воскрешения дорогого мертвеца окружающие родственники и друзья ужасаются, пораженные необычайной, всемогущей силой, возвращающей жизнь мертвым. Но впечатления чудесной силы быстро сменяются, и в то самое мгновение, когда мертвый воскресает, когда чудо свершается, родственники уже бросаются в объятия воскресшего и со слезами радости ведут его домой, чтобы отпраздновать там душевный праздник. Чудо исходит из глубины души и в нее возвращается. Происхождение чуда видно даже по его изображению. Только задушевный рассказ соответствует чуду. В рассказе о величайшем чуде --- воскрешении Лазаря --- нельзя отрицать задушевного сердечного тона легенды*\let\svthefootnote\thefootnote\let\thefootnote\relax\footnotetext{*легенды католицизма --- конечно, только лучшие и действительно душевные --- представляют как бы эхо основного тона, господствующего в этом новозаветном повествовании. --- Чудо можно бы ещё определить как религиозный юмор. Католицизм в особенности разработал чудо с этой его юмористической стороны.}\let\thefootnote\svthefootnote. Чудо задушевно, сердечно именно потому, что оно удовлетворяет желания человека без труда, без усилия. Бездушный, неверующий, рационалистический труд ставит существование человека в зависимость от целесообразной деятельности, обусловленной в свою очередь изучением объективного мира. Но чувство не интересуется объективным миром; оно не выступает дальше себя и выше себя, оно блаженно в себе самом. Своеобразие северной культуры, северного начала самоотчуждения одинаково чужды душе. Дух классицизма, дух культуры есть дух объективный, ограничивающий самого себя законами, определяющий чувство и фантазию созерцанием мира, необходимостью и истинной природой вещей. На место этого духа с появлением христианства выступила неограниченная, безмерная, исключительная, супранатуралистическая субъективность --- начало, по самому существу своему противоположное принципу науки, культуры\dag\let\svthefootnote\thefootnote\let\thefootnote\relax\footnotetext{\dagособенно характерно для христианства --- и это есть популярное доказательство всего сказанного --- что только язык Библии, а не язык Софокла или Платона, следовательно, только неопределенный, не подчиненный законам язык чувства, а не язык искусства и философии, до сих пор считается языком и откровением духа божия в христианстве.}\let\thefootnote\svthefootnote. С появлением христианства человек потерял способность и желание вдумываться в природу, вселенную. Пока существовало истинное, нелицемерное, неподдельное, искреннее христианство, пока христианство было живой, практической истиной, до тех пор и совершались действительные чудеса, и они совершались необходимо, так как вера в мертвые, исторические, прошлые чудеса есть вера мертвая, первое начало неверия, или, лучше, первый и потому зыбкий, непрочный, робкий признак, что неверие в чудеса уже начинает просачиваться. А там, где совершаются чудеса, все определенные образы расплываются в тумане фантазии и чувства, там мир и действительность перестают быть истиной; там лишь чудодейственная душевная, то есть субъективная, сущность считается за подлинную, действительную сущность.






Человек чувства непосредственно, помимо своей воли и мысли, считает силу воображения высшей деятельностью, деятельностью бога, творческой деятельностью. Собственное чувство кажется ему непосредственной истиной и авторитетом; он видит в нем нечто самое истинное, самое существенное; он не может ни отвлечься от своего чувства, ни возвыситься над ним. Воображение кажется ему такой же истиной, как и его чувство. Фантазия или сила воображения (их различие не играет здесь роли) кажутся ему иными, чем нам, людям рассудка, считающим это созерцание не объективным, а субъективным. Фантазия для него тождественна с ним самим, с его чувством, и как тождественная с его сущностью, кажется ему существенным, объективным, необходимым созерцанием. Мы считаем фантазию произвольной деятельностью, но человек, не усвоивший принципа культуры, основ миросозерцания, парящий и живущий исключительно в своем чувстве, видит в фантазии деятельность непосредственную, непроизвольную.

Объяснение чудес из чувства и фантазии кажется многим в наше время объяснением поверхностным. Но подумайте хорошенько о тех временах, когда люди верили в живые, настоящие чудеса, когда реальность и бытие внешних вещей ещё не являлись священным символом веры, когда люди были настолько далеки от созерцания действительного мира, что ежедневно ждали светопреставления и жили только надеждою на жизнь небесную, то есть воображением; ведь каково бы ни было небо, но пока они жили на земле, оно могло существовать для них только в их воображении. Но тогда это воображение было не воображением, а действительной, вечной, исключительной истиной, не пустым праздным средством самоутешения, а практическим, определяющим поступки нравственным принципом, в жертву которому люди с радостью приносили действительную жизнь, действительный мир со всеми их прелестями. Если перенестись мысленно в те времена, надо самому быть очень поверхностным, чтобы признать это психологическое объяснение чудес поверхностным. Нельзя возразить, что ведь эти чудеса совершались перед лицом целых собраний, ни один из присутствующих не был в полном сознании, все были насквозь проникнуты напряженными, супранатуралистическими представлениями и ощущениями, всех воодушевляла одинаковая вера, одинаковая надежда и фантазия. Кто не знает о существовании таких общих видений, присущих главным образом людям, замкнутым в себе живущим тесным кружком? Но пусть думают, как хотят. Если объяснение чудес из чувства и фантазии поверхностно, то в этом виноват не автор, а сам предмет --- чудо. Если мы внимательно рассмотрим чудо, мы убедимся, что оно выражает собою не что иное, как волшебную силу фантазии, которая без возражений исполняет все желания сердца\ddag\let\svthefootnote\thefootnote\let\thefootnote\relax\footnotetext{\ddagв основе некоторых чудес действительно лежит физическое или физиологическое явление. Но здесь речь идет только о религиозном значении и происхождении чуда.}\let\thefootnote\svthefootnote.







\chapter{Тайна воскресения и сверхъестественного рождения}


Качество душевности относится не только к практическим чудесам, где оно само собой бросается в глаза, потому что непосредственно касается блага и желаний человеческого индивида; оно относится также к теоретическим или собственно догматическим чудесам. Таково чудо воскресения и сверхъестественного рождения.

Человек, по крайней мере благоденствующий, желает бессмертия. Это желание было первоначально равносильно стремлению к самосохранению. Все живущее хочет утверждать себя, хочет жить, следовательно, не умирать. Это вначале отрицательное желание, под гнетом жизни, особенно гражданской и политической жизни, приобретает в уме и сердце позднейших поколений положительную окраску и превращается в желание жизни и притом лучшей жизни после смерти. Но в этом желании заключена также уверенность в исполнении этой надежды. Разум не может осуществить этой надежды. Поэтому говорят: все доказательства бессмертия недостаточны, или даже: разум не может привести их, тем менее доказать их. И это совершенно верно: доказательства разума носят всеобщий, отвлеченный характер; разум не может дать мне уверенности в продолжение моего личного существования, а я требую именно этой уверенности. Для такой уверенности необходимо непосредственное, чувственное удостоверение, фактическое подтверждение. А это последнее может быть дано мне только тогда, когда умерший, в смерти которого мы раньше были убеждены, восстанет из гроба, причем этот мертвец должен быть не простым смертным, а скорее прообразом для других, так что его воскресение должно служить прообразом, гарантией воскресения других. Поэтому воскресение Христа есть удовлетворенное желание человека непосредственно увериться в своем личном существовании после смерти --- в личном бессмертии как в чувственном, несомненном факте.

Языческие философы находили, что вопрос о бессмертии затрагивает интересы личности только мимоходом. Здесь речь шла главным образом о природе души, духа, жизненного начала. Мысль о бессмертии жизненного начала не заключает в себе непосредственно мысли о личном бессмертии, не говоря уже об уверенности в бессмертии поэтому древние высказывались об этом предмете так неопределенно, так противоречиво и неуверенно. Напротив, христиане, вполне уверенные в исполнении своих личных сердечных желаний, то есть уверенные в божественной сущности своей души, в истинности и святости своих чувств, превратили теоретическую проблему древних в непосредственный факт, а теоретический, свободный по существу вопрос --- в обязательное дело совести, непризнание которого уподоблялось тягчайшему преступлению атеизма. Отрицание воскресения равносильно было отрицанию воскресения Христа, отрицание воскресения Христа --- равносильно отрицанию бога. Таким образом, <<духовные>> христиане обратили в нечто бездушное предмет чисто духовный! Бессмертие разума, духа казалось христианам слишком <<абстрактным>> и <<отрицательным>>; они думали лишь о личном бессмертии залогом которогственный факт, а теоретический, свободный по существу вопрос --- в обязательное дело совести, непризнание которого уподоблялось тягчайшему преступлению атеизма. Отрицание воскресения равносильно было отрицанию воскресения Христа, отрицание воскресения Христа --- равносильно отрицанию бога. Таким образом, <<духовные>> христиане обратили в нечто бездушное предмет чисто духовный! Бессмертие разума, духа казалось христианам слишком <<абстрактным>> и <<отрицательным>>; они думали лишь о личном бессмертии залогом которого служило только воскресение плоти. Воскресение плоти есть высшее торжество христианства над возвышенной, но во всяком случае отвлеченной духовностью и объективностью древних. Поэтому мысль о воскресении не приходила в голову язычникам.

Воскресение, конец священной истории, имеющей, впрочем, значение не истории, а самой истины, является таким же осуществившимся желанием, как и начало ее, сверхъестественное рождество Христа, хотя последнее не относится к непосредственным личным интересам, а скорее к частному, субъективному чувству.

Чем больше человек отчуждается от природы, чем субъективнее, то есть сверхъестественнее или противоестественнее становится его созерцание, тем больше он боится природы или, по крайней мере, тех естественных вещей и процессов, которые не нравятся его фантазии и раздражают его*\let\svthefootnote\thefootnote\let\thefootnote\relax\footnotetext{*<<Если бы Адам не впал в грех, то мы не знали бы вреда от волков, львов, медведей и пр., и во всем творении не было бы для человека ничего неприятного или вредного\dots не было бы ни терний, ни сорных трав, ни болезней\dots Не было бы у человека морщин на лице; ноги, руки и другие члены тела не были бы слабы, вялы и болезненны>>. <<Только после грехопадения мы все почувствовали и узнали, какое зло скрывается в нашем теле и как похотлива наша плоть, и как нам бывает противно, когда мы пресыщены>>. <<Но виною всему является первородный грех, от которого запятналось все творение, и поэтому я утверждаю, что до грехопадения солнце было гораздо светлее, вода чище, а земля больше изобиловала всякими растениями>>. (Лютер, ч. \rom{1}, стр. 322--323, 329, 337).}\let\thefootnote\svthefootnote. Свободный, объективный человек тоже замечает отрицательные, даже отвратительные стороны природы, но считает их естественным, неизбежным следствием и потому подавляет свое чувство как чувство только субъективное, ложное. Напротив, субъективный человек, живущий только своим чувством и фантазией, подчеркивает эти стороны природы с особенным отвращением. Он уподобляется тому несчастному найденышу, который на самых красивых цветках замечал лишь маленьких <<черных жучков>>, отравлявших ему наслаждение созерцанием цветка. Субъективный человек делает свои чувства мерилом того, что должно быть. То, что ему не нравится, что оскорбляет его сверхъестественное или противоестественное чувство, не должно существовать. Субъективный человек не считается со скучными законами логики и физики, а только с произволом своей фантазии. А так как то, что ему нравится, не может существовать без того, что ему не нравится, то он устраняет то, что ему не нравится, и оставляет только то, что ему нравится. Так, ему нравится чистая, незапятнанная дева, и в то же время ему нравится мать, но только такая мать, которая, не испытав беременности, уже сносила бы ребенка на руках.



В глубине его души, его веры девственность, как таковая, является высшим нравственным понятием, рогом изобилия его супранатуралистических чувств и представлений, олицетворённым чувством его чести и стыда перед грубой природой\dag\let\svthefootnote\thefootnote\let\thefootnote\relax\footnotetext{\dag<<Наконец, отсутствие нецеломудренной страсти может быть таково, что для некоторых благонравная связь способствует стыдливости>> (<<Tantum denique abest incesti cupido, ut nonnulis rubori sit etiam pudica conjunctio>>. Munitius Felix, Octav., 31). Патер Гиль был до такой степени целомудрен, что не знал ни одной женщины в лицо, он даже боялся прикоснуться к самому себе. У патера Котона было в этом отношении столь тонкое обоняние, что он уже при одном приближении нецеломудренной личности ощущал невыносимое зловоние. (Бейль, Dict. Art. Mariana Rem. C.). Но высшим божественным началом этой сверхфизической утонченности является дева Мария; поэтому она у католиков и называется: слава девственниц; венец девства, образ девства и идеал чистоты, знаменосец девственниц, наставница девства, перводевственница, водительница девства.}\let\thefootnote\svthefootnote\footnote{<<Наконец, отсутствие нецеломудренной страсти может быть таково, что у некоторых даже благонравная связь вызывает стыдливость>>. Минуций Феликс. Октавий, гл. 31 (лат.).}\thinspace\footnote{см.: П. Бейль. <<Исторический и критический словарь>> (статья <<Мария>>, прим. С).}\thinspace\footnote{слава девственниц, венец девства, образ девства и идеал чистоты, знаменосица девственниц, наставница девства, перводевственница, предводительница девства (лат.).}. Но в то же время в его груди теплится и естественное чувство: сострадательное чувство материнской любви. Как же удовлетворить эту сердечную потребность, как помирить этот разлад между естественным и сверхъестественным или противоестественным чувством? Для этого супранатуралисту приходится связывать то и другое, соединять в одном и том же субъекте два взаимно исключающие себя свойства. О какая полнота сердечных, блаженных, сверхчувственно-чувственных ощущений лежит в этом соединении!



Этим объясняется противоречие в католицизме, который одновременно освящает брак и безбрачие. Догматическое противоречие девственной матери, или матери-девы, осуществляется здесь как практическое противоречие. И, однако, это удивительное, противоречащее природе и разуму, и особенно чувству и фантазии, сочетание девственности и материнства не есть продукт католицизма; оно лежит уже в той двусмысленной роли, какую брак играет в Библии и особенно в толковании апостола Павла. Учение о сверхъестественном рождении и зачатии Христа есть существенная часть учения христианства, выражающая его внутреннюю догматическую сущность и основанная на том же начале, как и все другие чудеса и верования. Философ, естествоиспытатель, свободный, объективный человек вообще считает смерть естественной необходимостью; а христиане были недовольны границами природы, которые являются для чувства оковами, а для разума разумными законами, и они устраняли их с помощью чудотворной силы. На том же основании они были недовольны и естественным процессом рождения и уничтожили его при помощи той же чудотворной силы. Как воскресение, так и сверхъестественное рождение одинаково выгодны для всех верующих: ведь зачатие Марии, незапятнанное мужским семенем, этим подлинным заразным началом первородного греха, было первым очистительным актом осквернённого грехами, то есть природой, человечества. Богочеловек не был заражен первородным грехом, и только эта чистота дала ему возможность очистить людей в глазах бога, питающего отвращение к естественному процессу деторождения, потому что сам он есть не что иное, как сверхъестественный дух.

Даже сухие, ортодоксально настроенные протестантские свободные критики смотрели на зачатие девы-богоматери как на великую, достойную почитания и удивления, священную, сверхразумную тайну веры\ddag\let\svthefootnote\thefootnote\let\thefootnote\relax\footnotetext{\ddagнапример, I. D. Winckler, Philolog. Lactant. s. Brunsvigae. 1754, p. 247--254.}\let\thefootnote\svthefootnote. Но так как протестанты снижали и ограничивали христианина только в смысле веры, а в жизни предоставляли ему оставаться человеком, то у них эта тайна имела не практическое, а лишь догматическое значение. Она не заставляла их отказываться от удовольствий брачной жизни. Но у католиков, вообще у древних, безусловных, некритических христиан тайна веры была также тайной жизни и нравственности. Католическая мораль носит христианский, мистический, а протестантская --- уже с самого начала рационалистический характер. Протестантская мораль всегда была плотским смешением христианина с человеком --- естественным, политическим, общественным, социальным человеком, или как угодно называйте его в отличие от христианина; а католическая мораль хранила в своем сердце тайну незапятнанной девственности. Католическая мораль была mater dolorosa\footnote{скорбящая богоматерь (лат.).}; протестантская --- дородная, благословенная детьми хозяйка дома. Протестантизм по самому существу есть полное противоречие между верой и жизнью, и поэтому он сделался источником или, по крайней мере, условием свободы. Тайна девыбогородицы признавалась протестантами только в теории или в догматике, а не в жизни, вследствие чего они относились к этому предмету чрезвычайно осторожно и сдержанно, не решаясь сделать его предметом умозрения. То, что отрицается практически, не находит и в человеке прочного основания и является лишь призраком представления. Поэтому его скрывают, изымают из сферы рассудка. Привидения не выносят дневного света.



Даже позднейшее, уже в одном письме к св. Бернарду выраженное, но им отвергнутое представление, что и сама Мария зачата непорочно, без наследственного греха, вовсе не есть <<странное мнение школы>>, как его называет один современный историк. Оно возникло вполне естественно, как благочестивое, благодарное размышление о матери божией. Порождающее чудо или бога, должно быть чудесно, божественно по своей сущности и происхождению. Св. дух не осенил бы деву Марию, если бы она не была чиста изначала. Св. дух не мог вселиться в тело, осквернённое первородным грехом. Если вы не находите странным самый принцип христианства --- благое и чудесное рождение спасителя, --- о, тогда вам нечего находить странными наивные, простодушные, задушевные рассуждения католицизма!



\chapter{Тайна христианского Христа или личного бога}


Основные догматы христианства --- осуществлённые желания сердца; сущность христианства есть сущность чувства. Приятнее страдать, чем действовать; приятнее быть спасённым и освобождённым другим, чем самому освобождаться; приятнее обусловливать свое спасение определенным лицом, чем силой самодеятельности; приятнее любить, чем добиваться; приятнее, чтобы бог любил нас, чем любить самого себя простой, естественной, эгоистической любовью, присущей всякому существу; приятнее смотреть в любящие глаза другого личного существа, чем в пустое зеркало собственного <<Я>> или в холодную глубь тихого океана природы; вообще приятнее определяться своим собственным чувством, считая его существом от нас отличным, чем подчиняться разуму. Чувство вообще есть casus obliquus\footnote{косвенный падеж (лат.).} нашего <<Я>>, есть <<Я>> в винительном падеже. Фихтевское <<Я>> бессердечно, потому что у него винительный падеж сходен с именительным, потому что оно несклоняемо. Но чувство есть <<Я>>, определенное самим собой, при этом это <<Я>> собой определено так, будто оно определено другим существом, это <<Я>> страдательное. Сердце обращает действительный залог в человеке в залог страдательный и страдательный --- в действительный. Для чувства мыслящее есть мыслимое, а мыслимое есть мыслящее. Душа по своей природе --- мечтательница, поэтому она не знает ничего блаженнее, глубже, чем мечта. А что такое мечта? Обратная сторона бодрствующего сознания. Во сне действующее является страдающим, а страдающее

– действующим; во сне я принимаю свои самоопределения за определения, идущие извне, душевные волнения --- за внешние события, свои представления и ощущения --- за нечто, находящееся вне меня, я переживаю свои отдельные действия. В сновидении лучи света преломляются вдвойне --- отсюда его неописуемая прелесть. Во сне, как и наяву, действует то же <<Я>>, то же существо; разница только в том, что наяву наше <<Я>> определяет себя само, а во сне определяется самим собой как другим существом. Выражение я мыслю себя --- слишком холодно, рационалистично; выражение обо мне думает бог и я мыслю себя лишь как существо, о котором мыслит бог, --- задушевно и религиозно. Чувство есть сон наяву; религия есть сон бодрствующего сознания; сон есть ключ к тайнам религии.

Высший закон чувства есть непосредственное единство воли и дела, желания и действительности. Этот закон исполнил спаситель. Внешнее чудо в противоположность естественной деятельности удовлетворяет физические потребности и желания человека непосредственно; спаситель, примиритель, богочеловек в противоположность нравственной самодеятельности естественного, или рационалистического, человека удовлетворяет внутренние, моральные потребности и желания тоже непосредственно, освобождая при этом человека от всякой помощи с его стороны. Всякое твое желание можно считать уже свершившимся. Ты желаешь получить, заслужить блаженство. Условием, средством для блаженства является нравственность. Но ты бессилен, то есть на самом деле тебе этого и не нужно. То, что ты ещё собираешься сделать, уже свершилось. Тебе остается лишь пассивно ждать: только верить, только наслаждаться. Ты хочешь расположить к себе бога, смягчить его гнев, примириться со своей совестью. Но этот мир уже заключен; этот мир есть посредник, богочеловек, он --- твоя облегченная совесть, он --- исполнение закона и вместе с тем исполнение твоего собственного желания и стремления.

Поэтому теперь уже не закон и исполнитель закона служит для тебя примером, правилом, законом твоей жизни. Всякий исполняющий закон сводит его на нет. Закон имеет авторитет, имеет значение лишь против беззакония. Но всякий, кто в совершенстве исполняет закон, как бы говорит закону: я сам хочу того же, чего ты хочешь, и утверждаю делом твои приказания; моя жизнь есть истинный, живой закон. Поэтому исполнитель закона неизбежно заступает место закона, и притом как новый закон, именно такой закон, иго которого легко и приятно. Ведь он заступает место закона лишь повелевающего, как пример, как предмет любви, восхищения и подражания, и через это становится избавителем от греха. Закон не дает мне силы, необходимой для исполнения закона. Нет! Он деспотичен, он только повелевает, не справляясь с тем, могу ли я исполнить его требования и как я должен их исполнять; он меня предоставляет себе самому, без совета и помощи. А тот, кто служит для меня примером, ведет меня под руки, уделяет мне частицу своей собственной силы. Закон не оказывает сопротивления греху, а пример творит чудеса. Закон мертв, а пример оживляет, вдохновляет и невольно увлекает за собой человека. Закон обращается только к рассудку и становится в прямое противоречие с нашими склонностями; пример, напротив, приспособляется к сложному чувственному импульсу, к невольному инстинкту подражания. Пример действует на чувство и фантазию. Короче, пример обладает магической, то есть чувственной, силой; ведь магическая, то есть невольная притягательная, сила есть существенное свойство материи вообще и чувственности в особенности.

Древние говорили, что если бы добродетель оказалась или могла оказаться видимой, она покорила бы и воодушевила бы всех своей красотой. Христиане были так счастливы, что дождались исполнения этого желания. У язычников был закон неписанный у иудеев писанный, а у христиан был пример, прообраз, видимый, личный живой закон, ставший плотью, --- человеческий закон. Отсюда бодрость первых христиан, отсюда слава христианства, что только оно имеет и дарует силу противиться греху. И этой славы --- здесь во всяком случае --- у него нельзя оспаривать. Следует только заметить, что сила примера добродетели есть не только сила добродетели, сколько сила примера вообще, подобно тому как сила религиозной музыки есть не сила религии, а сила музыки*\let\svthefootnote\thefootnote\let\thefootnote\relax\footnotetext{*в этом отношении интересна исповедь Августина (Confess. Lib. \rom{10}, c. 33): <<Итак, я колеблюсь между опасностью наслаждения и познанием благодати, и я все больше склоняюсь к тому, чтобы одобрить обычай пения в церкви, чтобы через наслаждение пением слабый дух мог подняться до благочестивого настроения. Однако случалось нередко со мной, что на меня больше действовало само пение, нежели его содержание, и тогда я раскаивался, что согрешил, и уже не хотел больше слушать певца>>.}\let\thefootnote\svthefootnote. Поэтому образец добродетели имеет следствием добродетельные поступки, а не добродетельные настроения и импульсы. Но этот простой и истинный смысл спасительной и примирительной силы примера в отличие от силы закона, к которому мы свели противоположность закона и Христа, далеко не исчерпывает религиозного значения христианского спасения и примирения. В последнем все сводится к личной силе того чудесного посредника, который не был исключительно ни богом, ни человеком, а человеком и в то же время богом и богом и в то же время человеком, вследствие чего его можно понять только в связи со значением чуда. В этом смысле чудесный спаситель есть не что иное, как осуществлённое желание сердца, стремящегося освободиться от законов морали, то есть от тех условий, с которыми связана добродетель на естественном пути, --- осуществленное желание спастись от нравственного зла мгновенно, непосредственно, по мановению волшебного жезла, то есть абсолютно субъективным, задушевным способом. Так, Лютер говорит: <<Слово божие исполняет все быстро, оно отпускает себе грехи и дарует вечную жизнь только за то, что ты слушаешь слово, а услышав его, веруешь. Как только ты поверил, то достигаешь всего немедленно, без всякого труда, усилия и напряжения>>\dag\let\svthefootnote\thefootnote\let\thefootnote\relax\footnotetext{\dag ч. \rom{16}, стр. 490.}\let\thefootnote\svthefootnote. Но и выслушивание слова божия, следствием которого является вера, есть тоже <<дар божий>>. Итак, вера есть не что иное, как психологическое чудо, чудесное дело бога в человеке, как говорит сам Лютер. Человек освобождается от греха и сознания вины только посредством веры, а очищается и облагораживается нравственно только посредством чуда. Нравственность зависит от веры: добродетели язычников --- только блестящие пороки.



Тождественность чудесной силы с понятием посредника подтверждается исторически уже тем, что ветхозаветные чудеса, законодательство, промысл, --- одним словом, все элементы, составляющие сущность религии, в позднейшем иудействе приурочивались к божественной мудрости, к логосу. Но у Филона этот логос парит ещё в воздухе между небом и землей то как нечто, только мыслимое, то как нечто действительное, то есть Филон колеблется между философией и религией, между метафизическим, абстрактным и собственно религиозным, подлинным богом. Только в христианстве этот логос укрепился и воплотился, сделавшись из мыслимого существа действительным существом, то есть религия сосредоточилась теперь исключительно на той сущности, на том объекте, который обосновывает её собственную природу. Логос есть олицетворённая сущность религии. Поэтому если бог определяется как сущность чувства, то только в логосе это определение обращается в свою полную истину.

Бог, как таковой, есть ещё замкнутое, скрытое чувство; а Христос есть открытое, объективированное чувство, или сердце. Только во Христе чувство окончательно убеждается и уверяется в самом себе, отрешается от всяких сомнений в истинности и божественности своей сущности; ведь Христос не отказывает чувству ни в чем; он исполняет все его просьбы. В боге чувство ещё молчит о том, что лежит у него в сердце; оно только вздыхает; во Христе оно высказывается совершенно, ничего не утаивая. Вздох есть ещё желание робкое; он больше выражает жалобу, что нет того, чего он желает; он не говорит открыто, определенно, чего он желает; вздох выражает сомнение души в правоте её желаний. Но во Христе пропадает уже всякая душевная робость; Христос есть вздох, перешедший в победную песнь в связи с исполнением желания; он есть ликующая уверенность чувства в истинности и действительности своих скрытых в боге желаний, фактическая победа над смертью, над всеми силами мира и природы, не только ожидаемое, но уже совершившееся воскресение из мертвых. Христос есть сердце, освободившееся от всех гнетущих препятствий и страданий, блаженная душа, видимое божество\ddag\let\svthefootnote\thefootnote\let\thefootnote\relax\footnotetext{\ddag<<Даровав нам сына своего, бог вместе с ним даровал нам и все остальное, то есть дьявола, грех, смерть, ад, небо, справедливость и жизнь; все, решительно все это должно быть вашим, потому что сын как дар принадлежит нам, а в нем все это совмещается>> (Лютер, ч. \rom{15}, стр. 311). <<Самое важное воскресение уже совершилось: Христос, слава всего христианства, победил смерть и восстал из мертвых. Вместе с Христом победила смерть и лучшая часть моего существа, моя душа. Может ли мне повредить теперь могила, и смерть?>>. (ч. \rom{16}, стр 235). <<Христианин имеет равную с Христом силу, составляет часть его и восседает на одном с ним седалище>> (ч. \rom{13}, стр. 648). <<Верующий в Христа так же могущественен, как он>> (ч. \rom{16}, стр. 574).}\let\thefootnote\svthefootnote.



Созерцать бога есть высшее желание, высшее торжество сердца. Христос есть это исполненное желание, эта победа. Бог, только мыслимый, взятый как исключительно мыслящее существо, то есть бог, как бог, есть существо далекое; отношение наше к нему всегда есть отношение абстрактное, подобное той дружбе, которую мы питаем к человеку далекому и лично нам незнакомому. Хотя олицетворением существа божия служат его деяния, доказывающие его любовь к нам, тем не менее всегда остается незаполненный пробел; это не удовлетворяет нашего сердца, и мы страстно хотим увидеть его лично. Пока мы не познакомились с кем-нибудь лично, мы всегда сомневаемся, действительно ли данное лицо существует и соответствует ли оно нашему представлению о нем; только личное знакомство служит залогом окончательной уверенности, окончательного успокоения. Христос есть лично знакомый бог, и поэтому Христос есть блаженная уверенность, что бог есть и именно таков, каким его желает видеть и утверждает наше сердце. Бог как объект молитвы есть уже существо человеческое, сочувствующее человеческим страданиям, внемлющее человеческим желаниям, тем не менее для религиозного сознания он не представляется как человек действительный. Поэтому только в Христе осуществляется конечное желание религии, разрешается тайна религиозной души --- разрешается на образном языке, свойственном религии; ведь сущность бога проявилась именно в Христе. В этом смысле христианскую религию по справедливости можно назвать абсолютной, совершенной религией. Цель религии заключается в том, чтобы бог, который сам по себе есть не что иное, как сущность человека, действительно сделался человеком и был объектом сознания как человек. И христианская религия достигла этого в вочеловечении бога, которое не есть акт преходящий, потому что Христос и после вознесения на небо остался человеком, человеком в смысле сердца и в смысле образа, с той только разницей, что теперь его тело уже перестало быть бренным телом, подверженным страданиям.

Вочеловечение бога у восточных народов, именно у индусов, не имело такого сильного значения, как христианское воплощение. Оно повторяется часто и потому теряет свое значение. Человечность бога есть его личность; бог есть существо личное --- значит бог есть существо человеческое, бог есть человек. Личность есть мысль, которая истинна лишь в действительном человеке*\let\svthefootnote\thefootnote\let\thefootnote\relax\footnotetext{*отсюда явствует лживость бесполезность современного умозрения о личности божией. Если вы не стыдитесь иметь личного бога, то не стыдитесь и приписать ему плоть. Абстрактная, бесцветная личность, личность без плоти и крови, есть пустой призрак.}\let\thefootnote\svthefootnote. Поэтому основной смысл вочеловечения бога достигается бесконечно лучше чрез одно вочеловечение, чрез одну личность. Где бог появляется последовательно в нескольких лицах, там эти личности пропадают. А между тем здесь необходима именно прочная, исключительная личность. Там, где много воплощений, может быть их ещё бесчисленное множество; фантазия не ограничена; там даже осуществившиеся воплощения попадают в категорию только возможных или воображаемых, в категорию фантазии или простых явлений. Но если в качестве воплощения божества рассматривается только одна личность, она тотчас же приобретает авторитет исторической личности; фантазии тут делать нечего, возможность ещё других воплощений отнята. Эта единая личность вынуждает меня верить в её действительность. Характерной чертой подлинной личности является именно её исключительность --- лейбницевский принцип различия, согласно которому существующие предметы никогда полностью не совпадают. Тон, выражение, характеризующие одну личность, производят такое впечатление на душу, что эта личность непосредственно представляется действительной и превращается из предмета фантазии в предмет обычного исторического созерцания.



Тоска есть потребность души, и душа томится по личному богу. Но это томление по личности бога только тогда подлинно, серьезно, глубоко, когда душа томится по одной личности и довольствуется ею одной. Множество лиц уничтожает истинность потребности и делает личность предметом роскоши для фантазии. Но сила необходимости действует на человека как сила действительности. Существо, необходимое для души непосредственно, кажется ей действительным существом. Тоска говорит: должен быть личный бог, то есть он не может не быть; а удовлетворенное чувство утверждает: он есть. Для души залог его существования лежит в необходимости его существования, а необходимость удовлетворения покоится на силе потребности. Необходимость не знает другого закона, кроме себя; нужда ломает железо. Чувство не знает другой необходимости, кроме потребности души --- тоски; она отвергает необходимость природы, необходимость разума. Для чувства необходим субъективный, душевный, личный бог, но ему необходима только одна личность, и притом эта личность непременно должна быть исторической, действительной личностью. Чувство успокаивается, сосредоточивается только на одной личности, множество дробит его.

Как истина личности есть единство, истина единства --- действительность, так истина действительной личности есть кровь. Последнее доказательство встречается между прочим в четвертом евангелии, автор которого особо подчеркивает, что видимая личность бога была не призраком, не иллюзией, а действительным человеком, потому что на кресте из его ребра текла кровь. Где личный бог есть подлинная потребность сердца, там он сам должен испытать страдание. Только в его страдании кроется залог его действительности, только на нем покоится прочное впечатление воплощения. Чувство не довольствуется созерцанием бога; глаза не служат достаточным ручательством. Истинность зрительного впечатления подтверждается только осязанием. Но чувство является последним критерием истины в субъективном смысле, а в объективном эту роль исполняет осязаемость, прикосновение, способность к страданию. Поэтому страдание Христа есть высшая уверенность, высшее самоуслаждение, высшая утеха сердцу: ведь лишь кровь Христа утоляет жажду в личном, то есть в человечном чувствующем, сострадающем боге.

<<Поэтому мы считаем вредным заблуждением мнение, будто человечность лишает Христа этого (то есть божественного) величия. Это заблуждение отнимает у христиан высшее утешение, которое они имеют в\dots обещании, что их глава, царь и первосвященник пребудет с ними не только как божество, которое действует на грешников, как всепожирающий огонь на сухие колосья, но и как человек, говоривший с нами, испытавший на себе все скорби наши и потому питающий сострадание к нам, людям и своим братьям, что пребудет он с нами во всех наших нуждах в том естестве, которое делает его нашим братом, а нас --– плотью от его плоти>>\dag\let\svthefootnote\thefootnote\let\thefootnote\relax\footnotetext{\dag Конкордации. Объясн., п. 8.}\let\thefootnote\svthefootnote.

Весьма поверхностно мнение, будто христианство есть религия не единого личного бога, а трех лиц. Эти три лица существуют только в догматике, но и здесь личность св. духа есть только произвольное дополнение, опровергаемое безличными определениями, вроде того, что св. дух есть donum\footnote{дар (лат.).} отца и сына\ddag\let\svthefootnote\thefootnote\let\thefootnote\relax\footnotetext{\ddagуже Фавст Социн прекрасно показал это. См. его Defens. Animadv. in Assert. Theol. Coll. Posnan, de trino et uno Deo. Irenopoli 1565, c. 11.}\let\thefootnote\svthefootnote. Уже одно исхождение святого духа делает его личность весьма сомнительной, так как личное существо производится только чрез рождение, а не чрез неопределенное исхождение или выдыхание (spiratio). И даже отец, как представитель строгого понятия божества, является личным существом более в воображении верующих и согласно заявлениям, чем по своим определениям; в действительности отец есть отвлеченное понятие, только мыслимое существо. Пластическая личность есть только Христос. Личность нераздельна с образом; образ есть реальность личности. Только Христос есть личный бог, он есть подлинный, действительный бог христиан, и это необходимо чаще повторять*\interfootnotelinepenalty=1\let\svthefootnote\thefootnote\let\thefootnote\relax\footnotetext{*см. в этом отношении особенно сочинения ортодоксальных христиан против еретиков, например против социниан. Новейшие богословы, как известно, толкуют даже церковную божественность Христа, как не библейскую; но она, бесспорно, составляет характерное начало христианства; если она и не так выражена в Библии, как в догматике, все-таки она есть необходимый вывод из Библии. Что такое существо, которое вмещает в себе всю полноту божества, которое знает все (Иоанн, 16, 30), которое всемогуще (воскрешает мертвых и творит чудеса), которое по времени и порядку явилось раньше всех вещей и существ, заключает в себе самом жизнь, подобно тому как и отец содержит в себе жизнь, --- чем же иным может быть это существо, как не богом? <<Христос в отношении воли един с отцом>>; но единство воли предполагает единство сущности.<<Христос есть посланный, есть заместитель бога>>; но бог может быть замещен только божественным существом. Только того, в ком я нахожу одинаковые или сходные качества, что и в себе, я могу избрать своим заместителем или своим посланным, иначе я опозорю самого себя.}\let\thefootnote\svthefootnote\interfootnotelinepenalty=10000. В нем одном сосредоточена христианская религия, сущность религии вообще. Только он отвечает стремлению к личному богу; только он есть существо, соответствующее сущности чувства; только он является средоточием всех радостей Фантазии и всех страданий чувства; только им исчерпывается чувство и исчерпывается фантазия. Христос есть единство чувства и фантазии.





Христианство тем и отличается от других религий, что в них сердце и фантазия идут врозь, а в христианстве совпадают. Здесь фантазия не предоставляется самой себе, а следует влечению сердца; она описывает круг, центром которого служит чувство. Здесь фантазия ограничивается потребностями сердца, исполняет только желания души, относится только к тому, что необходимо, одним словом, она --- по крайней мере в целом --- преследует практические, сосредоточенные, а не разбросанные, лишь поэтические цели. Чудеса христианства, зачатые в глубине страждущей, томящейся души, не являются продуктами одной только свободной, произвольной самодеятельности, а переносят нас непосредственно на почву обыденной, действительной жизни; они действуют на чувства человека с непреодолимой силой, потому что они опираются на потребность души. Одним словом, здесь сила фантазии есть вместе с тем сила сердца, фантазия есть только победоносное, торжествующее сердце. У народов восточных, у греков фантазия, не заботясь о потребностях сердца, наслаждалась земным великолепием и блеском; в христианстве она снизошла с божественного трона в жилища бедняков, где царила только необходимость и нужда, и подчинилась господству сердца. Но чем больше ограничивала она себя внешне, тем больше выигрывала она в силе. Веселие олимпийских богов разбилось о потребность сердца; сердце заключило могущественный союз с фантазией. И этим союзом свободы фантазии с потребностью сердца является Христос. Все подчиняется Христу; владыка мира и делает с ним, что только хочет. Но эта неограниченно повелевающая природой сила в свою очередь подчиняется власти сердца. Христос повелевает бушующей природе утихнуть, только чтобы лучше расслышать вопли страдальцев.




\chapter{Отличие христианства от язычества}


Христос есть всемогущество субъективности, освобождённое от всех уз и законов природы сердца, исключающее мир и сосредоточенное только в себе чувство, исполнение всех сердечных желаний, вознесение на небо фантазии, пасхальный праздник сердца, поэтому в Христе заключено отличив христианства от язычества.

В христианстве человек сосредоточивался только на себе, освобождался от связи с миром, становился самодовлеющим целым, существом абсолютным, внемировым, сверхмировым. Он не считал себя существом, принадлежащим миру; он порывал всякую связь с ним; поэтому он не имел больше основания сомневаться в истинности и законности своих субъективных желании и чувств и считал себя существом неограниченным: ведь граница субъективности есть именно мир, объективность. Язычники наоборот, не замыкались в себе самих и не удалялись от природы и потому ограничивали свою субъективность созерцанием мира. Древние преклонялись перед величием интеллекта, разума, но они были настолько свободомыслящи и объективны, что признавали право существования, и притом вечного существования, за оборотной стороной духа, за материей, и не только в теории, но и на практике. Христиане же простирали свою практическую и теоретическую нетерпимость до того, что ради утверждения своей вечной субъективной жизни уничтожали противоположность субъективности, природу, создавая веру в кончину мира*\let\svthefootnote\thefootnote\let\thefootnote\relax\footnotetext{*<<Язычники потому осмеивали христиан, что они угрожали гибелью небу и звездам, которые мы покидаем такими же, какими их нашли, а самим себе, то есть людям, имеющим не только свое начало, но и свой конец, они обещали вечную жизнь после смерти>> (Минуции Феликс, Октав., гл. 11, \S 2).}\let\thefootnote\svthefootnote. Древние были свободны от себя, но их свобода была свободой равнодушия к себе; а христиане были свободны от природы, но их свобода была не свободой разума, не истинной свободой (истинная свобода ограничивает себя созерцанием мира, природой), а свободой чувства и фантазии, свободой чуда. Древние восхищались природой до такой степени, что забывали о себе, терялись в целом; христиане презирали мир: что такое тварь в сравнении с творцом? что такое солнце, луна и земля в сравнении с человеческой душой? Мир прейдёт, а человек вечен. Христиане отрешали человека от всякого общения с природой и через это впадали в крайность чрезмерной щепетильности, усматривая даже в отдаленном сравнении человека с животным безбожное оскорбление человеческого достоинства; а язычники, напротив, впадали в другую крайность, часто не делая никакого различия между животным и человеком, или даже, как, например, Цельз, противник христианства, ставили человека ниже животного.



Но язычники рассматривали человека не только в связи со вселенной, они рассматривали человека, то есть индивида, отдельного человека, в связи с другими людьми, в связи с обществом. Они строго отличали, по крайней мере как философы, индивида от рода, смотрели на индивида, как на часть от целого человеческого рода, и подчиняли отдельное существо целому. <<Люди умирают, а человечество продолжает существовать>>, --- говорит один языческий философ. <<Как ты можешь жаловаться на потерю своей дочери? --- пишет Сульпиций Цицерону. --- Гибнут великие города и славные царства, а ты безутешно горюешь о смерти одного homunculi, человечка? Где же твоя философия?>> Понятие человека как индивида у древних обусловливалось понятием рода или коллектива. Они были высокого мнения о роде преимуществах человечества, возвышенно судили о разуме но были невысокого мнения об индивиде. Христианство, напротив, не считалось с родом и имело в виду только индивид. Христианство, --- разумеется, не современное христианство, воспринявшее культуру язычества и сохранившее лишь имя и некоторые общие положения христианства, есть прямая противоположность язычеству. Оно будет понято правильно и не будет искажено произвольной, умозрительной казуистикой, если будет рассматриваться как противоположность; оно истинно, поскольку ложна его противоположность, и оно ложно, поскольку истинна последняя. Древние поступались индивидом для рода; христиане жертвовали родом ради индивида иначе: язычники мыслили и понимали индивид только как часть в отличие от целого; христиане, напротив, видели его только в непосредственном, безразличном единстве с родом\interfootnotelinepenalty=1\dag\let\svthefootnote\thefootnote\let\thefootnote\relax\footnotetext{\dagАристотель, как известно, в своей <<Политике>> утверждает, что индивид, как не удовлетворяющий сам себя, относится к государству так, как часть к целому, и что поэтому государство по своей природе предшествует семье и индивиду, так как целое по необходимости появляется раньше части. Правда, христиане <<жертвовали>> также <<индивидом>>, то есть личностью как частью, целому, роду, общине. Часть, говорит св. Фома Аквинский, один из величайших христианских мыслителей и богословов, жертвует собой в силу естественного инстинкта интересам сохранения целого. <<Каждая часть по природе любит целое больше себя. И каждый отдельный человек по природе любит больше благо своего рода, чем свое личное благо или благополучие. Поэтому всякое отдельное существо по-своему любит бога, как всеобщее благо, больше, чем самого себя>> (Summae P. \rom{1}., Qu. 60, Art. \rom{5}). Поэтому христиане в этом отношении думали так же, как древние. Фома Аквинский восхваляет (de Regim. Princip., lib. \rom{3}, c. 4) римлян за то, что они превыше всего ставили свое отечество и своим благом жертвовали его благу. Но все эти мысли и настроения в христианстве встречаются только на земле, а не на небе, в морали, а не в догматике, в антропологии, а не в богословии. Как предмет богословия индивид есть сверхъестественное, бессмертное, самодовлеющее, абсолютное, божественное существо. Языческий мыслитель Аристотель считает дружбу (<<Этика>>, кн. 9, гл. 9) необходимой для счастья, а христианский мыслитель Фома Аквинский это отрицает. <<Общение друзей, --- говорит он, --- не есть необходимость для человеческого счастья, поскольку человек имеет уже всю полноту своего совершенства в боге>>. <<Поэтому душа, наслаждающаяся исключительно богом, все-таки блаженствует, даже если она не имеет возле себя ближнего, которого она любила бы>> (Prima Secundae, Qu. 4, 8). Следовательно, даже и в блаженстве язычник сознает себя одиноким как индивид и потому нуждается в другом существе, себе подобном, нуждается в роде; а христианин не нуждается в другом <<Я>>, раз он как индивид не есть индивид, а есть род, всеобщее существо, раз он обретает <<всю полноту своего совершенства в боге>>, то есть в себе самом.}\let\thefootnote\svthefootnote\interfootnotelinepenalty=10000.



Христианство считало индивида предметом непосредственного попечения, то есть непосредственным объектом божественного существа. Язычники верили в провидение индивида, обусловленное родом, законом, мировым порядком, то есть верили только в опосредствованное естественное, а не чудесное провидение; христиане, напротив, уничтожали всякое посредничество, становились в непосредственную связь с провидящим, всеобъемлющим, всеобщим существом, то есть они непосредственно отождествляли каждое отдельное существо с существом всеобщим.

Но понятие божества совпадает с понятием человечества. Все божественные определения, все определения, делающие бога богом, суть определения рода, определения, ограниченные отдельным существом, индивидом и не ограниченные в сущности рода и даже в его существовании, поскольку это существование соответственно проявляется только во всех людях, взятых как нечто собирательное. Мое знание, моя воля ограничены; но моя ограниченность не есть ограниченность для другого, не говоря уже о человечестве; то, что трудно для меня, легко для другого; то, что невозможно, непонятно для одной эпохи, понятно и возможно для другой. Моя жизнь связана с ограниченным количеством времени, жизнь человечества не ограничена. История человечества состоит не в чем ином, как в постоянной обеде над границами, которые в данное, определенное, время становятся границами человечества, то есть абсолютными, неодолимыми границами. Но будущее всегда оказывает, что мнимые границы рода были только границами индивидов. История наук, особенно философии и естествоведения, доставляет тому очень интересные данные. Было бы в высшей степени интересно и поучительно написать историю наук именно с этой точки зрения, чтобы показать всю несостоятельность тщетной мечты индивида ограничить свой род. Итак, род неограничен, ограничен только индивид\ddag\let\svthefootnote\thefootnote\let\thefootnote\relax\footnotetext{\ddagв смысле религии и богословия род не представляется, конечно, безграничным, всеведущим и всемогущим, но только потому, что божественные свойства существуют лишь в воображении и образуют только предикаты, только выражения человеческого чувства и способности представления, как это показано в настоящей книге.}\let\thefootnote\svthefootnote.



Но ощущать ограниченность тягостно; и индивид освобождается от нее в созерцании совершенного существа; это созерцание дает ему то, чего ему недостает. Бог у христиан есть не что иное, как созерцание непосредственного единства рода и индивида, всеобщей сущности и отдельного существа. Бог есть понятие рода как индивида, понятие или сущность рода; он как всеобщая сущность, как средоточие всех совершенств, всех качеств, свободных от действительных или мнимых границ индивида, есть в то же время существо отдельное, индивидуальное. <<Сущность и бытие в боге тождественны>>, то есть он есть не что иное, как родовое понятие, родовая сущность, признаваемая месте с тем за бытие, за отдельное существо. Высшая идея с точки зрения религии ли богословия такова: бог не любит, он сам есть любовь; бог не живет, он есть жизнь; бог не есть существо справедливое, он --- сама справедливость; бог не есть лицо, он --- сама личность, род, идея, являющаяся непосредственно действительностью.

Вследствие такого непосредственного единства рода и индивидуальности, такого сосредоточения всех общих свойств и сущностей в одном личном существе бог есть нечто глубоко задушевное, восхищающее фантазию, тогда как идея человечества есть идея бездушная, потому что идея человечества кажется нам чем-то отвлеченным в противоположность действительному человечеству, которое рисуется нам в образе бесчисленного множества отдельных, ограниченных индивидов. Напротив, в боге душа успокаивается непосредственно, потому что здесь все соединено в одном, все дано сразу, то есть здесь род является непосредственно бытием, отдельным существом. Бог есть любовь, добродетель, красота, премудрость, совершенная, всеобщая сущность, как единое существо, как бесконечного объема род, как сосредоточенная квинтэссенция. Бог есть собственная сущность человека, следовательно, христиане отличаются от язычников тем, что они непосредственно отождествляют индивид с родом, тем, что у них индивид имеет значение рода и считается сам по себе совершенным бытием рода, тем, что они обожествляют человеческий индивид, делают его абсолютным существом.

Особенно характерное отличие христианства от язычества представляет отношение индивида к интеллекту, к \textgreek{νόυs}\footnote{ум, разум (др.-гр.).}.

Язычники считали рассудок универсальной сущностью, христиане индивидуализировали его; язычники видели в рассудке сущность человека, христиане --- только часть самих себя. Поэтому язычники считали бессмертным, то есть божественным, только разум, род, а христиане --- индивид. Отсюда само собой вытекает дальнейшее различие между языческой и христианской философией.

Наиболее определенное выражение, наиболее характерный символ этого непосредственного единства рода и индивидуальности в христианстве есть Христос, действительный бог христиан. Христос есть прообраз, сущее понятие человечества, совокупность всех нравственных и божественных совершенств, понятие, исключающее все отрицательное и несовершенное, чистый, небесный, безгрешный человек, человек рода, Адам Кадмон, но рассматриваемый не как полнота рода человеческого, а непосредственно как один индивид, как одно лицо. Христос, то есть христианский Христос религии, есть не центральный пункт, а конец истории. Это вытекает как из понятия о нем, так и из истории. Христиане ждали конца мира, конца истории. Сам Христос ясно и определенно предсказывает в Библии близкий конец мира вопреки всем лживым софизмам наших экзегетов. История покоится только на отличии индивида от рода. Там, где прекращается это отличие, прекращается и история, разум, смысл истории. Человеку остается только созерцание и усвоение этого осуществившегося идеала и неприкрытое стремление к распространению его --- проповедь, что бог явился, и наступил конец мира.

Так как непосредственное*\let\svthefootnote\thefootnote\let\thefootnote\relax\footnotetext{*я умышленно говорю: непосредственное, то есть сверхъестественное, фантастическое, ибо посредственное, разумное, естественноисторическое единство рода и индивида основывается только на половом моменте. Я человек только как мужчина или как женщина или --- или, или свет или тьма, или мужчина или женщина таково творческое слово природы. Но для христианина действительный человек, женщина или мужчина, есть только <<животное>> его идеал его сущность есть кастрат --- человек бесполый; ведь для него человек в смысле рода, есть не что иное, как олицетворённое бесполое существо, противоположность мужчины и женщины, так как и то и другое --- люди.}\let\thefootnote\svthefootnote единство рода и индивида простирается за пределы разума и природы, этот универсальный, идеальный индивид вполне естественно и неизбежно стал считаться сверхъестественным, небесным существом. Поэтому нелепо выводить из разума тождество рода и индивида; ведь только фантазия осуществляет это единство, фантазия, для которой нет ничего невозможного, та самая фантазия, которая творит чудеса; в самом деле, индивид есть величайшее чудо; будучи индивидом, он в то же время является идеей, родом, человечеством во всей полноте его совершенств и бесконечности. Поэтому так же нелепо отвергать чудеса, но принимать библейского или догматического Христа. Приняв принцип нельзя отрицать его неизбежных следствий.



О полном отсутствии в христианстве понятия рода особенно свидетельствует характерное учение его о всеобщей греховности людей. Это учение основано на требовании, чтобы индивид не был индивидом, а это требование в свою очередь коренится в предположении, что индивид сам по себе есть совершенное существо, исчерпывающее выражение или бытие рода. Здесь совершенно отсутствует объективное созерцание, сознание того, что <<ты>> относишься к совершенству <<Я>>, что люди только совместно образуют человека и являются тем, чем может и должен быть человек. Все люди грешны. Я допускаю это; но все они грешат по-разному между ними наблюдается очень большое, существенное различие. Один человек имеет склонность ко лжи\dag\let\svthefootnote\thefootnote\let\thefootnote\relax\footnotetext{\dagтак, например, у сиамцев ложь и обман составляют врожденные пороки, но им же присущи и добродетели, которых нет у других народов, свободных от этих пороков сиамцев.}\let\thefootnote\svthefootnote, другой --- нет; он скорее пожертвует своей жизнью, чем нарушит свое слово или солжет; третий любит выпить, четвертый любит женщин, пятый свободен от всех этих недостатков или по милости природы или благодаря энергии своего характера. Таким образом, люди взаимно дополняют один другого не только в физическом и интеллектуальном, но и в моральном отношении, благодаря чему в целом они являются тем, чем они должны быть, и представляют собой совершенного человека.



Поэтому общение облагораживает и возвышает; в обществе человек невольно, без всякого притворства, держит себя иначе, чем в одиночестве. Любовь, особенно половая любовь, творит чудеса. Муж и жена взаимно исправляют и дополняют друг друга и, только соединившись, представляют собой род, то есть совершенного человека\ddag\let\svthefootnote\thefootnote\let\thefootnote\relax\footnotetext{\ddagу индусов (по закону Ману) <<почитается совершенным человеком тот, кто состоит из трех соединенных лиц: из своей жены, себя самого и своего сына. Ибо муж и жена, отец и сын суть едино>>. Также и ветхозаветный земной Адам сознает себя несовершенным без жены и стремится к ней. Но Адам новозаветный, христианский, рассчитывающий на кончину этого мира, не имеет уже половых стремлений и функций.}\let\thefootnote\svthefootnote. Любовь немыслима вне рода. Любовь есть не что иное, как самоощущение рода, выраженное в половом различии. Реальность рода, служащая вообще только объектом разума, предметом мышления, становится в любви объектом чувства, истиной чувства, ибо в любви человек выражает недовольство своей индивидуальностью, постулирует существование другого как потребность сердца и причисляет другого к своему собственному существу, признает жизнь, связанную с ним любовью, жизнью истинно человеческой, соответствующей понятию человека, то есть рода. Личность недостаточна, несовершенна, слаба, беспомощна; а любовь сильна, совершенна, удовлетворена, спокойна, самодовольна, бесконечна, так как в любви самоощущение индивидуальности обращается в самоощущение совершенства рода. Но и дружба действует так же, как любовь, по крайней мере там, где она является истинной, искренней дружбой и носит характер религии, как это было у древних. Друзья дополняют друг друга; дружба есть залог добродетели, даже больше: она есть сама добродетель, но добродетель общественная. Дружба возможна только между людьми добродетельными, как говорили ещё древние. Но для нее не нужно совершенного сходства или равенства, а скорее она требует различия, так как дружба покоится на стремлении к пополнению себя. Благодаря другу человек дополняет то, чего ему недостает. Дружба искупает недостатки одного добродетелями другого. Друг испрашивает оправдания для друга перед богом. Как бы ни был порочен человек сам по себе, его хорошие задатки обнаруживаются в том, что он ведет дружбу с людьми достойными. Если я сам не могу быть совершенным существом, то я по крайней мере ценю добродетель и совершенство в других. Поэтому, если когда-нибудь бог пожелает судить меня за мои грехи, слабости и ошибки, я выставлю ему в качестве защитника и посредника добродетели моего друга. Бог оказался бы существом деспотическим и неразумным, если бы осудил меня за грехи, которые хотя я и совершил, но сам же осудил их, любя своих друзей, свободных от этих грехов.


Но если уже дружба, любовь делают из несовершенного существа существо хотя бы относительно совершенное, то тем более грехи и ошибки отдельного человека должны исчезнуть в самом роде, который только в человечестве в целом получает надлежащее бытие*\let\svthefootnote\thefootnote\let\thefootnote\relax\footnotetext{*<<Только все люди в совокупности, --- говорит Гете слова, которые однажды уже где-то я цитировал, но здесь не могу воздержаться, чтоб не повторить их, --- познают природу; только все люди в совокупности любят человеческое>>.}\let\thefootnote\svthefootnote~и лишь поэтому является предметом разума. Жалобы на грехи раздаются только там, где человеческий индивид в своей индивидуальности считает себя существом по себе совершенным, абсолютным, не нуждающимся в другом существе для реализации рода, для реализации совершенного человека, где место сознания рода заступило исключительное самосознание индивида, где индивид перестал смотреть на себя, как на часть человечества, не отличает себя от рода и потому свои грехи, свою ограниченность и свои слабости считает грехами, ограниченностью и слабостями самого человечества. Но тем не менее человек не может совершенно утратить сознание рода, потому что его самосознание существенно связано с сознанием других людей. Поэтому там, где род не является человеку как род, он является ему как бог. Человек возмещает отсутствие понятия рода понятием бога, как существа, свободного от всех ограничений и недостатков, которые удручают индивида и, по его мнению, даже самый род, так как здесь индивид отождествляется с родом. Но такое свободное от индивидуальной замкнутости, неограниченное существо есть не что иное, как род, открывающий бесконечность своей сущности в том, что он осуществляет себя в бесчисленном множестве разнообразных индивидов. Если бы все люди были абсолютно равны, то между родом и индивидом, разумеется, не было бы различия. Но тогда существование множества людей было бы чистой роскошью; цель рода достигалась бы при помощи одного лица; все человечество могло бы быть заменено одним человеком, наслаждающимся счастием бытия.



Разумеется, сущность человека есть нечто единое. Но эта сущность бесконечна; поэтому её действительное бытие является бесконечным, взаимно дополняющим себя разнообразием, в котором открывается богатство сущности. Единство в сущности есть многообразие в бытии. Между мною и другим, --- а другой есть представитель рода, и, даже будучи один, он заменяет мне потребность во многих других, имеет для меня универсальное значение, является как бы уполномоченным человечества и говорит мне, одинокому, как бы от его имени, поэтому я даже в обществе одного лица веду общественную, человеческую жизнь, --- между мною и другим имеется существенное, качественное различие. Другое есть мое <<ты>> --- хотя это относится к обеим сторонам --- мое второе <<Я>>, объектированный для меня человек, мое вскрытое внутреннее <<Я>>, око, видящее самого себя. Благодаря другому я сознаю впервые человечество, узнаю и чувствую, что я человек; любовь к нему доказывает мне, что он необходим мне, а я необходим ему, что мы оба не можем существовать друг без друга, что только общение создает человечество. Кроме того, между <<Я>> и <<ты>> существует также качественное, критическое различие в моральном смысле. Другой есть моя объективированная совесть: он укоряет меня моими недостатками, даже когда не называет их открыто: он --- мое олицетворенное чувство стыда. Сознание нравственного закона, права, приличия, истины тесно связано с сознанием другого. Истинно только то, в чем другой соглашается со мной --- единомыслие есть первый признак истины, но только потому, что род есть последнее мерило истины. Если я мыслю только в меру моей индивидуальности, мое мнение не обязательно для другого, он может мыслить иначе, мое мнение есть случайное, субъективное. Но если я мыслю согласно мерилу рода, значит я мыслю так, как может мыслить человек вообще и, стало быть, должен мыслить каждый в отдельности, если он хочет мыслить нормально, закономерно и, следовательно, истинно. Истинно то, что соответствует сущности рода; ложно то, что ему противоречит. Другого закона для истины не существует. Но другой в отношении меня есть представитель рода, уполномоченный множества других; его суждение может иметь для меня даже большее значение, чем суждение бесчисленной толпы. <<Пусть мечтатель приобретает столько учеников, сколько песчинок в море, но песок остается песком; а жемчужиной мне будешь ты, мой разумный друг!>> Поэтому согласие другого служит для меня признаком закономерности, всеобщности, истинности моих мыслей. Я не могу настолько отрешиться от себя, чтобы судить о себе совершенно свободно и беспристрастно, а суждение другого беспристрастно; благодаря ему я исправляю, дополняю, расширяю свое собственное суждение, свой собственный вкус, свое собственное знание. Одним словом, между людьми существует качественное, критическое различие. Но христианство уничтожает это качественное различие, оно подгоняет всех людей под одну мерку, рассматривает их как один и тот же индивид, потому что не знает различия между родом и индивидом; христианство признает для всех людей без различия одно и то же средство спасения и видит во всех один и тот же основной и наследственный грех.

Благодаря исключительной субъективности христианство не признает рода, в котором именно и заключается разрешение, оправдание, примирение и спасение от грехов и недостатков индивидов. Христианству для победы над грехом понадобилась сверхъестественная, особая, опять-таки личная, субъективная помощь. Если я один составляю род, если кроме меня нет других, качественно отличных от меня людей или, что то же, если нет различия между мной и другим, если все мы совершенно равны, если мои грехи не нейтрализуются и не парализуются противоположными качествами других людей, тогда, конечно, мой грех есть вопиющий позор, возмутительное преступление, которое можно искупить только необычайным, нечеловеческим, чудесным средством. Но, к счастью, существует путь естественного примирения: другой индивид сам по себе есть посредник между мной и священной идеей рода. <<Человек человеку бог>>. Мои грехи уже потому оказываются введенными в свои границы и обращаются в ничто, что они только мои грехи и не являются сверх того грехами других людей.





\chapter{Христианское значение свободного безбрачия и монашества}



Понятие рода и значение жизни рода исчезло с появлением христианства. Этим лишний раз подтверждается приведенное выше положение, что христианство не заключает в себе принципа культуры. Если человек непосредственно отождествляет род с индивидом и считает это тождество своим высшим существом, своим богом, если, таким образом, идея человечества служит для него объектом только как идея божества, то пропадает потребность в культуре: человек вмещает все в себе, в своем боге, и поэтому не чувствует потребности восполнять себя другим индивидом, представителем рода, созерцанием мира вообще --- потребности, на которой только и покоится стремление к культуре. Но тем не менее человек достигает своей цели --- он достигает её в боге. Бог есть именно эта достигнутая цель, осуществлённая высшая цель человечества. Но каждый индивид думает, что бог существует только ради него. Только бог и нужен христианам. Им не нужны ни другой индивид, ни род, ни мир; внутренняя потребность в другом отсутствует. Бог заменяет мне род, заменяет другого. Только отрекаясь от мира, обособляясь от него, я сильнее чувствую потребность в боге, живее ощущаю присутствие бога и чувствую, что такое бог и чем он должен быть для меня. Разумеется, и религиозный человек нуждается в общении, в общественном назидании. Но потребность в другом лице сама по себе --- вещь второстепенная. Спасение души есть основная идея, главная задача христианства, но это спасение заключается только в боге, только в сосредоточении на нем. Требуется деятельность для других как условие спасения, но основа спасения есть бог, непосредственное отношение к богу. И поэтому деятельность для других имеет только религиозное значение, основой и целью её является отношение к богу, в силу чего она есть в сущности деятельность для бога, для прославления его имени, для распространения его славы. Но бог есть абсолютная субъективность, отрешенная от мира, сверхмировая, освобождённая от материи, жизни рода и, тем самым, от половых различий. Поэтому отчуждение от мира, от материи, от жизни рода есть существенная цель христианина*\let\svthefootnote\thefootnote\let\thefootnote\relax\footnotetext{*<<Жизнь для бога не есть естественная жизнь, которая подвержена тлению\dots Не должны ли мы томиться по будущему и относиться враждебно ко всему преходящему?.. Поэтому мы должны презирать эту жизнь и этот мир и всем сердцем стремиться к будущей славе и радости вечной жизни>> (Лютер, ч. \rom{1}, стр. 466, 467).}\let\thefootnote\svthefootnote. И эта цель реализуется чувственным образом в монашестве.



Утверждение, будто монашество появилось впервые на Востоке, --- самообман. Во всяком случае, защищая это положение, надо оставаться справедливым и объяснять тенденцию, противоположную монашеству, не характером самого христианства, а духом, природой Запада вообще. Но чем тогда объяснить восторженное отношение Запада к монашеству? Нет, монашество вытекает из самого христианства; оно было необходимым следствием веры в небо, которое христианством обещано человечеству. Где небесная жизнь есть истина, там земная жизнь есть ложь; где фантазия есть все, там действительность --- ничто. Земная жизнь теряет свою цену для того, кто верит в вечную, небесную жизнь. Или, вернее, она уже потеряла для него всякую цену: вера в небесную жизнь есть вера в ничтожество и бренность этой жизни. Я не могу представить себе загробной жизни без того, чтобы не томиться по ней, --- чтобы не окидывать жалкую земную жизнь взглядом сострадания или презрения. Небесная жизнь не может быть предметом, законом веры, не будучи в то же время моральным законом. Она должна определять мои поступки\dag\let\svthefootnote\thefootnote\let\thefootnote\relax\footnotetext{\dag<<Туда нужно устремлять дух свой, куда некогда он вознесется>> (Meditat. sacrae Ioh. Gerhardi. Med. 46).}\let\thefootnote\svthefootnote для того, чтобы моя жизнь соответствовала моей вере: я не должен привязываться к преходящим благам этой земли. Я не должен и не хочу, ибо что такое все эти блага в сравнении со славой небесной жизни?\ddag\let\svthefootnote\thefootnote\let\thefootnote\relax\footnotetext{\ddag<<Кто стремится к небесному, тому не нравится земное. Кто жаждет вечного, тому преходящее тягостно>> (Бернард, Epist. Ex persona Heliae monachi ad parentes). <<Нет ничего в этой жизни ближе нашему сердцу, как умереть возможно скорее>> (Тертуллиан, Apol. adv. Gentes, гл. 41). Древние христиане праздновали поэтому не день рождения, как теперь, а день смерти. (см. примеч. к Мин. Феликсу e rec Gronovii. Lugd. Bat. 1719, стр. 332). <<Поэтому скорее следует советовать человеку-христианину, чтобы он терпеливо переносил болезнь, даже желал смерти чем скорее, тем лучше. Ибо, как говорит святой Киприан, ничто так не полезно христианину, как близкая смерть. Но мы предпочитаем слушать язычника Ювенала, который говорит: <<Orandum est ut, sit mens sana in corpore sano>> (<<Нужно просить о том, чтобы здоровый ум был в здоровом теле>>) (Лютер, ч. \rom{4}, стр. 15).}\let\thefootnote\svthefootnote\footnote{<<Надо молить чтобы ум был здравым в теле здоровом>> (лат.). Выражение, заимствованное из <<Сатир>> Ювенала.}


Правда, качество будущей жизни зависит от качества, от моральных свойств этой жизни, но нравственность сама определяется верой в вечную жизнь. И эта соответствующая загробной жизни нравственность заключается в отвращении от этого мира, в отрицании этой жизни. Чувственное проявление этого духовного отвращения есть монастырская жизнь. Все должно в конце концов проявляться внешне, чувственно*\let\svthefootnote\thefootnote\let\thefootnote\relax\footnotetext{*<<Тот совершенен, кто духовно и телесно отрешился от мира>>. De modo bene vivendi ad Sororem S. \rom{7} (из апокрифического сочинения Бернарда).}\let\thefootnote\svthefootnote. Монастырская и вообще аскетическая жизнь есть жизнь небесная, насколько она проявляется и может проявиться здесь, на земле. Если моя душа принадлежит небу, то мое тело не может и не должно принадлежать земле. Душа оживляет тело. Когда душа улетает на небо, покинутое тело умирает и вместе с ним умирает соединительный, связующий орган между миром и душой. Смерть, отделение души от тела, по крайней мере от этого грубого материального, греховного тела, есть переход на небо. Но если смерть --- условие блаженства и нравственного совершенства, то единственным законом морали неизбежно является укрощение, умерщвление плоти. Нравственная смерть есть необходимое предвосхищение естественной смерти --- необходимое потому, что было бы в высшей степени безнравственно достижение неба предоставить смерти чувственной, которая не есть акт моральный, но акт естественный, общий человеку и животному. Поэтому смерть должна возвыситься до акта морального, стать актом самодеятельности. <<Я каждый день умираю>>\footnote{см.: <<Первое послание к коринфянам ап. Павла>> (15,31).}, --- говорит апостол, --- и это изречение основатель монашества святой Антоний\dag\let\svthefootnote\thefootnote\let\thefootnote\relax\footnotetext{\dagсм., однако, Иеронима, De vita Pauli Eremitae.}\let\thefootnote\svthefootnote\footnote{<<О жизни Павла, первого отшельника>> (лат.). Имеется в виду соч. Иеронима Стридонского, посвященного ап. Павлу.} сделал девизом своей жизни.





Мне могут возразить, что христианство стремилось только к духовной свободе. Пусть так; но что такое духовная свобода, не выражающаяся в поступках, не проявляющаяся в чувственной форме? Или ты думаешь, что твоя свобода зависит только от тебя, от твоей воли, от твоих мыслей? О, тогда ты сильно ошибаешься; значит, ты никогда не переживал истинного акта освобождения! Пока ты связан известным положением, известной профессией, известными отношениями, твои поступки невольно определяются ими. Твоя воля, твои убеждения освобождают тебя только от сознательных, а не от скрытых, бессознательных границ и впечатлений, заключающихся в самой природе вещей. Поэтому нам не жутко, мы не подавлены только тогда, когда мы пространственно, чувственно отдаляемся от того, с чем мы порвали внутренним образом. Только чувственная свобода есть залог истинной духовной свободы. Человек, действительно переставший духовно интересоваться земными благами, готов совершенно отказаться от них, чтобы окончательно облегчить свое сердце. Все то, чего я не признаю в силу своих убеждений, но от чего я ещё не отрешился, тяготит меня, потому что противоречит моим убеждениям. Долой все это! Рука не должна удерживать того, что вылетело из головы. Только убеждение сообщает силу рукопожатию; только убеждение освящает обладание. Тому, кто должен обладать женой так, как если бы он не обладал ею, лучше не иметь её вовсе. Обладать так, как будто не обладаешь, значит обладать без убеждения в обладании, то есть в сущности не обладать вовсе. И кто поэтому говорит: обладайте этой вещью так, как если бы вы ею не обладали, только деликатным, вежливым образом внушает вам не обладать ею вовсе. То, что становится чуждым моему сердцу, перестает принадлежать мне и свободно улетучивается. Св. Антоний решил отречься от мира, как только услышал слова: <<Если хочешь быть совершенным, иди; продай имение свое и раздай нищим, и будешь иметь сокровище на небесах, и иди вслед за мной>>. Св. Антоний понял истинный смысл этих слов. Он пошел, продал свое имущество и роздал его нищим. Только этим он доказал свою духовную свободу от сокровищ этого мира\ddag\let\svthefootnote\thefootnote\let\thefootnote\relax\footnotetext{\ddagестественно, что христианство обладало такой силой только до тех пор, пока, как пишет Иероним к Деметрии, кровь нашего господа была ещё тепла и вера таилась ещё в свежей крови (см. об этом предмете у Арнольда, Von der ersten Christen Genuegsamkeit und Verschmaehung alles Eigennutzes (1. c. B. \rom{4}, гл. 12, пар. 7--16).}\let\thefootnote\svthefootnote.



Разумеется, такая свобода, такая истина противоречит современному христианству, утверждающему, будто господь хотел только духовной свободы, то есть свободы, не требующей ни жертв, ни энергии, свободы иллюзорной, свободы самообмана --- такой свободы от земных благ, которая заключается в обладании и наслаждении этими благами. Поэтому господь будто и говорил: <<Иго мое легко>>. Как дико, как нелепо было бы христианство, если бы оно заставляло людей жертвовать сокровищами этого мира! Тогда оно и не годилось бы для этого мира. В действительности дело обстоит иначе. Христианство в высшей степени практично и житейски мудро: оно предоставляет освобождение от сокровищ и удовольствий этого мира естественной смерти, --- самоумерщвление монахов есть не христианское самоубийство, --- а обладание земными сокровищами и наслаждение ими предоставляет нашей самодеятельности. Истинные христиане, правда, не сомневаются в истинности небесной жизни, сохрани боже! В этом они вполне согласны и теперь ещё с древними монахами; но они ожидают этой жизни терпеливо, преданные воле божией, то есть воле своекорыстия, стремлению к наслаждению благами этого мира*\let\svthefootnote\thefootnote\let\thefootnote\relax\footnotetext{*Совсем иначе думали древние христиане. <<Трудно, даже невозможно одновременно наслаждаться настоящими и будущими благами>> (Иероним, Epist. Juliano). <<Ты слишком хитроумен, брат мой, если предполагаешь теперь участвовать в наслаждениях мира, а потом царствовать вместе со Христом>> (он же, Epist. ad Heliodorum). <<Вы хотите нераздельно обладать и богом и тварью, но это невозможно. Нельзя одновременно радоваться и богу и тварям>> (Таулер, Ed. c. p. 334). Но, разумеется, они были абстрактными христианами. А мы теперь живем в век примирения. О, конечно!}\let\thefootnote\svthefootnote. Но я гадливо и с презрением отворачиваюсь от современного христианства, где невеста Христова клянется вечным связующим, неопровержимым, священным, истинным словом божиим и при этом --- о позорное лицемерие! --- сама охотно исповедует полигамию, по крайней мере последовательную полигамию, которая, однако, в глазах истинных христиан мало чем отличается от полигамии одновременной, и я возвращаюсь с священным трепетом к непризнанной истине целомудренной монастырской кельи, где душа, вверенная небу, не имела греховной связи с чуждой ей земной плотью!



Жизнь вне мира, сверхъестественная жизнь, есть жизнь безбрачная по существу. Безбрачие --- разумеется не в качестве закона --- заключается во внутренней сущности христианства. Достаточным доказательством этого служит сверхъестественное происхождение спасителя. Этой верой христиане освятили непорочную девственность, как спасительный принцип, как принцип нового, христианского мира. Нельзя ради нерасторжимости брака ссылаться на такие места из Библии, как <<размножайтесь>>, или <<что бог сочетал, человек да не разлучает>>. Первое изречение, как заметили ещё Тертуллиан и Иероним, относится только к незаселённой земле, к началу мира, а не к концу его, наступившему после непосредственного пришествия бога на землю. Второе изречение относится только к браку, как установлению ветхозаветному. Иудеи интересовались вопросом может ли человек разводиться со своей женой, и вышеприведенный ответ служил самым целесообразным решением этого вопроса. Кто раз вступил в брак, должен считать его священным. Взгляд на другую есть уже измена. Брак сам по себе есть индульгенция, обусловленная слабостью, или скорее энергией чувственности, зло, требующее возможного ограничения. Нерасторжимость брака есть ореол святости, выражающий как раз противоположное тому, чего ищут в нем ослепленные этим ореолом, сбитые с толку люди. Брак, как таковой, то есть в смысле совершенного христианства, есть грех
\dag\let\svthefootnote\thefootnote\let\thefootnote\relax\footnotetext{\dag
<<Грешить --- значит не желать совершенства>>. Иероним. Замечу кстати, что приведенную здесь цитату из Библии о браке я толкую в том смысле, в каком её понимает вся история христианства.}\let\thefootnote\svthefootnote или, по меньшей мере, слабость, которая допускается и извиняется только с тем условием, чтобы ты ограничился одной --- заметь! --- одной женой навсегда. Короче говоря, брак освящается не в Новом, а только в Ветхом завете. Новый завет знает более возвышенное, сверхъестественное начало --- тайну незапятнанной девственности\ddag\let\svthefootnote\thefootnote\let\thefootnote\relax\footnotetext{\ddag<<Брачное состояние не есть нечто новое или необычное, и даже язычники, руководствуясь суждением разума, считали брак благим и похвальным делом>>. (Лютер. Ч. 2. Стр. 337а).}\let\thefootnote\svthefootnote. <<Если кто может вместить это, да вместит>>. <<Чада века сего женятся и выходят замуж, а сподобившиеся достигнуть того века и воскресения из мертвых не женятся и замуж не выходят. И умереть уже не могут, ибо они равны ангелам и суть сыны божии, будучи сынами воскресения>>*\let\svthefootnote\thefootnote\let\thefootnote\relax\footnotetext{*Евангелие от Луки, 20:34 --- 36}\let\thefootnote\svthefootnote. Итак, на небе люди не женятся; небо исключает начало половой любви как начало земное, мирское. Но небесная жизнь есть истинная, совершенная, вечная жизнь христианина. Если я предназначен для неба, я не буду связывать себя узами, которые уничтожаются моим истинным назначением. Если я сам по себе в возможности являюсь существом небесным, то я постараюсь осуществить эту возможность уже на земле\dag\let\svthefootnote\thefootnote\let\thefootnote\relax\footnotetext{\dag<<Кто желает войти в рай, должен отрешиться от всего, что не свойственно раю>>. Тертуллиан. <<Безбрачие есть подражание ангелам>>. Иоанн Дамаскин.}\let\thefootnote\svthefootnote. Брак, отвергнутый небом, существенным предметом моей веры, надежды и любви, изгнан также и из моей головы и сердца. Мое сердце так переполнено небом, что в нем нет уже места земной женщине. Я не могу разделить свое сердце между богом и человеком\ddag\let\svthefootnote\thefootnote\let\thefootnote\relax\footnotetext{\ddag <<Незамужняя помышляет только о боге и у нее только эта мысль, а замужняя разделяет свою жизнь между богом и мужем>>. (Климент Александрийский, Paedag., lib. \rom{2}, гл. 10). <<Избравший жизнь одинокую помышляет только о божественном>> (Феодорит, Haeretic. Fabul., lib. \rom{5}, 24).}\let\thefootnote\svthefootnote. Любовь христианина к богу не есть любовь абстрактная, или всеобщая, подобно любви к истине, справедливости, науке; это любовь к субъективному, личному богу, и потому это любовь субъективная, личная. Существенное свойство этой любви заключается в том, что она --- любовь исключительная, ревнивая, так как её предметом является существо личное и притом наивысшее, не имеющее себе подобных. <<Будь верен Иисусу (а Иисус Христос есть бог христианина) в жизни и смерти; положись на его верность; он один может тебе помочь, когда все тебя покинут. Твой возлюбленный не хочет видеть около тебя никого другого; он один желает обладать твоим сердцем и господствовать в твоей душе, подобно царю на троне>>. <<Что значит мир без Иисуса? Без Христа всюду адская мука, с Христом везде небесное блаженство>>\dots <<Ты не можешь жить без друга, но если ты не ставишь выше всего дружбу Христа, твоя жизнь будет исключительно печальна и безутешна>>. <<Люби всех ради Иисуса, а Иисуса --- ради него самого. Один Иисус Христос достоин любви>>. <<Бог мой, любовь моя, ты весь мой, а я весь твой>>. <<Любовь\dots надеется и полагается на бога, даже если он не относится к ней благосклонно (или если это горько --- nonsapit), ведь любовь нераздельна со страданием\dots>> <<Ради своего возлюбленного любящий должен переносить все, даже тяжкое и скорбное>>. <<Мой бог, мое все\dots в твоем присутствии мне все приятно, в своем отсутствии все противно\dots Без тебя мне ничто не мило>>. <<О, когда же наступит наконец тот блаженный, желанный час, когда ты окончательно сольешься со мной и станешь для меня всем! Пока я не удостоюсь этого, радость моя будет только частичной>>. <<Когда мне было хорошо без тебя и дурно с тобой? Я скорее соглашусь быть бедным ради тебя, чем богатым без тебя. Я скорее соглашусь быть темным странником с тобой, чем владыкой неба без тебя. Где ты --- там небо; где тебя нет --- там смерть и ад. Я тоскую только по тебе>>.



<<Ты не можешь одновременно служить богу и радоваться тому, что происходит. Удались от всех друзей и знакомых и не думай о временных утехах*\let\svthefootnote\thefootnote\let\thefootnote\relax\footnotetext{*Фома Кемпийский, De imit., lib. \rom{2}, гл. 7, гл. 8; гл. 5; гл. 34, 53, 59. <<О, как блаженна та дева, в груди которой кроме любви ко Христу нет другой любви!>> (Иероним, Demetriadi, virgini Deo consecratae). Но, конечно, это опять слишком абстрактная любовь, которая в век примирения, когда Христос и дьявол стали жить душа в душу, не приходится по вкусу. О, как горька бывает истина!}\let\thefootnote\svthefootnote. Верующие во Христа должны следовать заповеди св. апостола Петра и смотреть на себя только как на странников и гостей в этом мире>>. Итак, любовь к богу, как к личному существу, есть настоящая, форменная, личная, исключительная любовь. Как я могу любить бога как бога и в то же время любить смертную женщину? Могу ли я ставить бога на одну доску с женщиной? Нет, душа, истинно любящая бога, не может любить женщину --- это измена. <<Всякий, имеющий жену, --- говорит апостол Павел, --- думает только о жене; всякий, не имеющей жены, думает только о боге. Женатый думает о том, как угодить жене, а неженатый --- как угодить богу>>\footnote{см.: <<Первое послание к коринфянам ап. Павла>> (7,32--33).}.






Истинный христианин одинаково не чувствует потребности ни в образовании --- принципе мирском, противном душе, ни в любви (естественной). Бог вознаграждает его за скудость, за неудовлетворенную потребность в образовании, в любви, в женщине и семье. Христианин непосредственно отождествляет индивид с родом: поэтому он отвергает половое различие, как тягостный, случайный придаток\dag\let\svthefootnote\thefootnote\let\thefootnote\relax\footnotetext{\dag<<Женщина отличается от девы>>. <<Посмотрим, как блаженна та, которая утратила даже имя своего пола. Дева не называется больше женщиной>> (Иероним, Adv. Helvidium de perpet. virg., p. 14, t. \rom{2}. Erasmus).}\let\thefootnote\svthefootnote. Только муж и жена вместе образуют действительного человека; муж и жена вместе есть бытие рода, ибо их союз есть источник множества, источник других людей. Поэтому человек, сознающий свою мужественность, чувствующий себя мужчиной и считающий это чувство естественным и закономерным, сознает и чувствует себя существом частичным, которое нуждается в другом частичном существе для создания целого, истинного человечества. Христианин, напротив, в своей исключительной, природной субъективности, считает себя существом, совершенным в самом себе. Но это воззрение исключает половой инстинкт, как противоречащий его идеалу, его высшему существу. Поэтому христианин и должен был подавлять в себе этот инстинкт.



Разумеется, христианин чувствовал потребность в половой любви, но считал её противоречащей своему небесному назначению, считал её только естественной потребностью в том обыденном, презрительном смысле, какой придает этому слову христианство, а не потребностью моральной, внутренней, так сказать, метафизической, то есть существенной потребностью, которую человек может ощущать только тогда, когда не гонит от себя половой инстинкт, а скорее относит его к своей внутренней сущности. Поэтому в христианстве брак не считается священным, по крайней мере, он считается таковым только мнимо, иллюзорно, ибо естественное начало брака, половая любовь, есть начало несвященное, отрицаемое небом\ddag\let\svthefootnote\thefootnote\let\thefootnote\relax\footnotetext{\ddagМожно ещё и так сформулировать: брак имеет в христианстве только нравственное, а не религиозное значение, не есть религиозный принцип и прообраз. Иначе было у греков, где, например, <<Зевс и Гера считались великим прообразом всякого брака>> (Крейцер, Symb.), или у древних парсов, у которых деторождение как <<умножение человеческого рода и умаление царства Аримана>> считалось религиозным долгом и актом (Зенд-Авеста), наконец, у индусов сын был возродившимся отцом.

<<Если к жене приближается муж, он сам ещё раз рождается от той, что делается матерью через него>> (Фр. Шлегель).

У индусов ни один возрождённый не мог получить звания саниасси, то есть погруженного в бога отшельника, если предварительно не исполнил трех обязанностей, и между прочим, не произвел на свет законного сына. У христиан, наоборот, по крайней море у католиков, наступал настоящий религиозный праздник, если супруги или обручённые --- предполагая, что то, что происходило по взаимному согласию --- покидали брачное состояние и приносили брачную любовь в жертву любви религиозной.}\let\thefootnote\svthefootnote. А то, что человек исключает из своего неба, он исключает из своей истинной сущности. Небо --- его сокровищница. Не верь тому, что он устанавливает на земле, что он тут разрешает и санкционирует: здесь он должен приспособляться; здесь встречается много неожиданного, что не укладывается в его систему; здесь он ускользает от твоего взора, ибо он находится среди чуждых существ, внушающих ему страх. Но наблюдай его тогда, когда он отбрасывает свое инкогнито и выступает в своем истинном достоинстве, как гражданин небесного царства. На небе он говорит так, как думает; там ты услышишь его истинное мнение. Где его небо, там его сердце --- небо есть его открытое сердце. Небо есть понятие всего истинного, благого, законного, того, что должно быть; земля есть понятие всего ложного, незаконного, того, чего не должно быть. Христианин исключает из своего неба жизнь рода; там род прекращается, там существуют только чистые, бесполые индивиды, <<духи>>; там царит абсолютная субъективность. Следовательно, христианин исключает из своей истинной жизни жизнь рода; он отрицает начало брака, как начало греховное, зазорное, потому что безгрешной, истинной жизнью является только жизнь небесная*\let\svthefootnote\thefootnote\let\thefootnote\relax\footnotetext{*так как религиозное сознание в конце концов восстанавливает то, что оно вначале отвергало, и так как потусторонняя жизнь на самом деле есть не что иное, как восстановленная земная жизнь, то должен быть восстановлен также и пол. <<Они будут подобны ангелам и, следовательно, не перестанут быть людьми, так что апостол останется апостолом, а Мария --- Марией>> (Иероним, Ad Theodoram viduam). Но как в будущей жизни наши тела будут бестелесными, призрачными телами, так и пол неизбежно будет там бесполым, только мнимым полом.}\let\thefootnote\svthefootnote.









\chapter{Христианское небо или личное бессмертие}


Безбрачная, вообще аскетическая жизнь есть прямой путь к небесной бессмертной жизни, так как небо есть не что иное, как сверхъестественная, абсолютно субъективная жизнь, не знающая ни рода, ни пола. Вера в личное бессмертие коренится в вере, что половое различие есть только внешний придаток индивидуальности, что индивид сам по себе есть бесполое, по себе совершенное, абсолютное существо. Но бесполый не принадлежит ни к какому роду. Половое различие есть связующее звено между индивидом и родом, а тот, кто не принадлежит ни к какому роду, принадлежит только себе, как существо божественное, абсолютное, безусловно чуждое потребностей. Поэтому в небесную жизнь верит только тот, у кого исчезло сознание рода. Живущий в сознании рода и, следовательно, его истинности, живет также в сознании истинности полового различия. Он видит в нем не случайный, механически появившийся на его пути камень преткновения, а внутреннюю, химическую составную часть своего существа. Он сознает себя человеком, но вместе с тем сознает в себе и половой момент, который не только пропитывает его до мозга костей, но и определяет его внутреннее <<Я>>, существенные свойства его мышления, воли и чувствования. Поэтому живущий в сознании рода, ограничивающий, определяющий свое чувство и свою фантазию созерцанием действительной жизни, действительного человека, не может представить себе жизни, в которой отсутствовала бы жизнь рода и вместе с нею половое различие; он считает бесполого индивида, небесного духа, только приятным образом фантазии.

Настоящий человек не может абстрагировать не только от полового различия, но и от своей нравственной, или духовной, определенности, которая тесно связана с его естественной определенностью. Ввиду того, что он живет в созерцании целого, он смотрит на себя только как на частичное существо, которое есть только то, что оно есть, благодаря определенности, делающей его частью целого или относительным целым. Поэтому каждый справедливо считает свою профессию, свое состояние, свое искусство или науку наивысшими, ведь дух человека есть существенное свойство его деятельности. Кто в своем положении, в своем искусстве представляет нечто значительное, кто, как принято говорить, свято исполняет свой долг, отдается своему призванию телом и душой, тот непременно считает свое призвание чем-то высоким и прекрасным. Его ум не может отрицать, не может унижать того, что он превозносит своим делом, чему он с радостью посвящает свои силы. Я не стану отдавать свое время, свои силы тому, что я недостаточно высоко ценю. Если же я принуждён это делать, то моя деятельность печальна, потому что я противоречу самому себе. Работать --- значит служить. Но я не могу служить, подчиняться предмету, который я недостаточно высоко ценю. Одним словом, занятия определяют суждения, образ мыслей, настроение человека. Чем выше род занятия, тем более человек отождествляет себя с ним. Вообще то, что человек ставит существенной целью своей жизни, он называет своей душой, так как оно делается в нем движущим началом. Но благодаря своим целям, своей деятельности, посредством которой он осуществляет эти цели, человек живет не только для себя, но и для других, для целого, для рода. Поэтому кто живет в сознании рода как истины, тот считает свое бытие для других, свое общественное, общеполезное бытие бессмертным, тождественным с бытием его сущности. Он живет для человечества всей душой, всем сердцем. Он не может иметь в виду ещё другое, особое бытие, не может отделять себя от человечества. Он не может отрицать смертью того, что утверждал он жизнью.


Небесная жизнь, или, что то же, личное бессмертие, есть характерное учение христианства. Правда, отчасти оно встречается и у языческих философов, но имеет в них лишь значение фантазии, потому что не соответствует их основному мировоззрению. Как противоречиво говорят об этом предмете, например, стоики! только у христиан личное бессмертие стало опираться на такое начало, из которого оно вытекает с необходимостью, как сама собой разумеющаяся истина. Древних постоянно затрудняло созерцание мира, природы, рода; они отличали начало жизни от живущего субъекта, отличали душу, дух от себя самих, тогда как христиане уничтожили различие между душой и лицом, родом и индивидом и поэтому перенесли непосредственно на себя то, что принадлежит только целому роду. Непосредственное единство рода и индивидуальности есть высшее начало, бог христианства, --- благодаря ему индивид приобретает значение абсолютного существа, --- и необходимое последствие этого начала есть личное бессмертие.

Или, выражаясь точнее: вера в личное бессмертие совершенно тождественна с верой в личного бога, то есть вера в небесную бессмертную жизнь личности выражает то же самое, что и бог, как объективируют его христиане, --- сущность абсолютной, неограниченной личности. Неограниченная личность есть бог, а небесная, бессмертная личность есть не что иное, как неограниченная личность, освобождённая от всех земных тягостей и ограничений, --- с той только разницей, что бог есть духовное небо, а небо есть чувственный бог, что в боге мыслится то, что в небе предполагается как объект фантазии. Бог есть только неразвернувшееся небо, а действительное небо есть развернувшийся бог. Теперь бог представляет царство небесное, а в будущей жизни небо будет богом. Бог есть порука, наличие и существование --- пока ещё только отвлеченные --- будущей жизни, предвосхищённое небо, контур его. Наше собственное будущее, но отличное от того, какими мы существуем теперь, в этом мире, в этом теле, то есть наша лишь идеально объективированная сущность, есть бог; бог есть понятие рода, которое впервые осуществится, индивидуализируется на том свете. Бог есть небесная, чистая, свободная сущность, которая там будет существовать в виде небесных, чистых существ, то блаженство, которое там проявится во множестве блаженных индивидов. Итак, бог есть только понятие или сущность абсолютной, блаженной, небесной жизни, которая сосредоточивается теперь в единой идеальной личности. Это достаточно ясно выражено в веровании, что блаженная жизнь есть единение с богом. Здесь, на земле, мы отделены и отмежеваны от бога, там --- преграды падут; здесь мы люди, там --- боги; здесь божественность составляет монополию, там --- общее состояние; здесь преобладает абстрактное единство, а там оно делается конкретным множеством*\let\svthefootnote\thefootnote\let\thefootnote\relax\footnotetext{*<<Прекрасно говорит Писание (Первое послание Иоанна 3:2), что мы тогда увидим бога, когда будем подобны ему, то есть будем тем, что он есть сам; ведь кому дано быть сыном божьим, тому дана также и власть, если и не быть богом, то, по крайней мере, быть тем, что есть бог>> (De vita solit. --- Псевдо-Бернард). <<Цель благой воли есть блаженство, а вечная жизнь --- сам бог>>. Августин. (у Петра Ломб., lib. \rom{2}, dist. 38, гл. 1). <<Блаженство есть сама божественность, следовательно, всякий блаженный есть бог>> (Боэций, De consol. Phil., lib. \rom{3}, Prosa 10). <<Блаженство и бог --- одно и то же>> (Фома Аквинский, Summa cont. Gentiles, \rom{1}. 1, гл. 101). <<Человек обновится для будущей жизни, он будет равен богу в жизни, справедливости, славе и мудрости>> (Лютер, ч. 1, стр. 324).}\let\thefootnote\svthefootnote.



Усвоение этого предмета затрудняется фантазией, которая разделяет единство понятия тем, что представляет себе, с одной стороны, личность и самостоятельность бога, а с другой --- рисует себе множество личностей, которыми заселяет царство небесное, обыкновенно изображаемое в чувственных красках. В действительности между абсолютной жизнью, которая мыслится как бог, и абсолютной жизнью, которая мыслится как небо, нет никакого различия, так как все, что в себе распространяется в длину и ширину, в боге сосредоточено в одной точке. Вера в бессмертие человека есть вера в божественность человека, и наоборот, вера в бога есть вера в чистую, ничем не ограниченную и, стало быть, бессмертную личность. Различие, предполагаемое между бессмертной душой и богом, носит или софистический, или фантастический характер, как если бы, например, стали утверждать, что блаженство небожителей имеет свои ограничения и делится по степеням, чтобы установить различие между богом и небесными существами.

Единство божественной и небесной личности обнаруживается даже в популярных доказательствах бессмертия. Если нет другой, лучшей жизни, значит, бог не справедлив и не благ. Справедливость и благость божии ставятся в зависимость от продолжительности жизни индивидов; но бог без справедливости и благости не есть бог, следовательно, божественность, бытие бога обусловливаются бытием индивидов. Если я не бессмертен, я не верую в бога; кто отрицает бессмертие, --- отрицает бога. Но я не могу этому поверить: как достоверно бытие бога, настолько же достоверно мое блаженство. Бог для меня есть достоверность моего блаженства. Желание, чтобы существовал бог, равносильно желанию собственного вечного существования. Бог есть мое обеспеченное, мое достоверное существование; он есть субъективность субъектов, личность лиц. Поэтому как может не принадлежать лицам то, что подобает личности? В боге мое будущее становится настоящим, или, вернее, глагол --- существительным; как можно было бы отделить одно от другого? Бог есть бытие, соответствующее моим желаниям и чувствам; он справедлив и благ, он исполняет мои желания. Природа, этот мир есть бытие, противоречащее моим желаниям, моим чувствам. Здесь все не так, как должно быть, --- этот мир преходящ, а бог есть бытие, отвечающее тому, что должно быть. Бог исполняет мои желания --- это только популярное олицетворение фразы: бог есть исполнитель, то есть действительность, исполнение моих желаний\dag\let\svthefootnote\thefootnote\let\thefootnote\relax\footnotetext{\dag<<Если нетленное тело есть благо для нас, то как сомневаться в том, что бог сотворит нам такое тело?>> (Августин, Opp. Antverp., 1700, t. \rom{5}, стр. 698).}\let\thefootnote\svthefootnote. А небо тоже есть бытие, соответствующее моим желаниям, моей тоске\ddag\let\svthefootnote\thefootnote\let\thefootnote\relax\footnotetext{\ddag<<Небесным телом называется духовная плоть, так как она будет подчиняться воле духа. Ничто в тебе не будет противиться тебе и ничто в тебе не будет возмущаться против тебя. Где ты пожелаешь быть, там ты и будешь мгновенно>> (Августин, 1. с., стр. 705, 703). <<Там не будет ничего противного, враждебного, нечистого, безобразного, ничего оскорбляющего взор>> (он же, 1. с., р. 707). <<Только блаженный живет так, как он хочет>> (он же, De Civit. Dei. lib. 14, гл. 25).}\let\thefootnote\svthefootnote, следовательно, между богом и небом нет различия. Бог есть сила, посредством которой человек осуществляет свое вечное блаженство; бог есть абсолютная личность, на которую у всех отдельных лиц опирается уверенность в своем блаженстве и бессмертии; бог есть высшая, конечная уверенность человека в абсолютной истинности своего существа.





Учение о бессмертии есть завершительное учение религии --- её завет, в котором она выражает свою последнюю волю. Поэтому здесь она открыто говорит о том, о чем в других местах умалчивает. В иных случаях речь идет о существовании другого существа, но здесь, очевидно, говорится только о собственном существовании; если во всех других отношениях человек ставит в религии свое бытие в зависимость от бытия бога, то здесь он обусловливает бытие бога своим собственным существованием; то, что обыкновенно является первой, непосредственной истиной, становится здесь истиной производной, вторичной: если я не вечен, то бог не есть бог; если нет бессмертия, то нет и бога. И к этому заключению пришел ещё апостол. Если мы не воскреснем, то и Христос не воскресал и все есть ничто\footnote{см.: там же, (15,16).}. Edite, bibite!\footnote{ешьте, пейте (лат.). Фейербах приводит часть известного латинского изречения, которое вошло в старинную студенческую песнь: <<Ешьте, пейте, после смерти нет никакого наслаждения>>.} Разумеется, из популярных доказательств можно устранить все мнимо или действительно предосудительное, избегая заключительной формы, но для этого надо и бессмертие обратить в аналогичную истину, так, чтобы понятие бога, как абсолютной личности или субъективности, само собой стало понятием бессмертия. Бог есть порука моего будущего существования, потому что он есть достоверность и истина моего настоящего существования, мое спасение, мое утешение, моя защита от насилий внешнего мира; поэтому мне незачем специально выдвигать бессмертие как особую истину; имея бога, я обладаю также бессмертием. Так думали и глубокие христианские мистики, для них понятие бессмертия заключалось в самом понятии бога; бог был для них их бессмертною жизнью, субъективным блаженством, следовательно, он был для них, для их сознания тем, чем он является сам по себе, то есть в существе религии.

Таким образом, доказано, что бог есть небо, что то и другое тождественно. Ещё легче было бы доказать обратное>> именно, что небо есть подлинный бог людей. Человек мыслит своего бога таким, каким он мыслит свое небо; содержание его неба есть содержание его бога, с той только разницей, что в небе чувственно рисуется, изображается то, что в боге является эскизом, наброском. Поэтому небо есть ключ к сокровенным тайнам религии. Как в объективном смысле небо есть раскрытая сущность божества, так субъективно оно есть откровенное выражение глубинных мыслей и воззрений религии. Поэтому религии столь же различны, как и небесные царства; а небесные царства столь же многочисленны, как и существенные отличия людей.

Сами христиане представляют себе небо в весьма разнообразных видах*\let\svthefootnote\thefootnote\let\thefootnote\relax\footnotetext{*И столь же различен и их бог. Так, у благочестивых христианских германофилов есть <<германский бог>>, у благочестивых испанцев --- свой испанский бог, у французов --- французский бог. У французов даже есть пословица: бог-француз (Le bon Dieu est Francais). Но на самом деле многобожие будет существовать до тех пор, пока человечество будет состоять из разных народов. Действительный бог какого-нибудь народа есть показатель чести (point d'honneur) его национальности.}\let\thefootnote\svthefootnote\footnote{боженька-француз (фр.).}\footnote{дело чести (фр.).}. Только наиболее хитрые из них не думают и не говорят ничего определенного о небе и вообще о загробной жизни, заявляя, что это непостижимо и что будущая жизнь должна мыслиться по образцу настоящей жизни. Они утверждают, что все представления загробной жизни носят характер только образов, в которых человек воплощает незнакомую ему по существу, но достоверную по бытию загробную жизнь. Здесь происходит то же, что и по отношению к богу: бытие бога признается достоверным, а то, что такое бог, каков он, --- признается непостижимым. Но кто так говорит, уже перестал думать о загробной жизни; и он ещё удерживает её или потому, что вовсе не думает о таких вещах, или потому, что она ещё составляет для него лишь потребность сердца; но, отдаваясь всецело вещам действительным, он гонит эту мысль как можно дальше от себя; его голова отрицает то, что утверждает его сердце; ведь он, несомненно, отрицает загробную жизнь, отнимая у нее те свойства, которые только и делают её действительным и действенным объектом для человека. Качество не отличается от бытия --- оно и есть не что иное, как действительное бытие. Бытие, лишенное качества, есть химера, призрак. Бытие обусловливается качеством, а не так, будто вперед проступает бытие, а сзади плетется качество. Поэтому учение о непознаваемости и неопределимости бога, так же как и учение о непостижимости загробной жизни, не есть первоначальное христианское учение. Оба они являются скорее продуктами безбожия, находящегося ещё под обаянием религии, или вернее, прикрывающегося ею, и потому именно, что первоначально бытие бога связывалось с определенным представлением о боге, а бытие загробной жизни --- с определенным представлением о последней. Так, христианин уверен только в существовании своего рая, того рая, который носит христианский характер, и не признает рая магометан или элизиума греков. Первой достоверностью всегда бывает качество. Если качество достоверно, бытие разумеется само собой. В Новом завете нет таких доказательств или общих положений, которые указывали бы прямо на существование бога или небесной жизни; там упоминаются только свойства небесной жизни: <<там не женятся>>. Это вполне естественно, возразят нам, так как бытие предполагается само собой. Но это значило бы приписывать религиозному сознанию некоторый элемент рефлексии, первоначально ему несвойственный. Разумеется, бытие предполагается, но потому лишь, что качество уже есть бытие, потому, что цельная религиозная душа живет только в качестве, подобно тому как человек естественный видит действительное бытие, вещь в себе, только в качестве, которое он ощущает. Приведенные выше слова из Нового завета свидетельствуют о том, что девственная, или, вернее, бесполая жизнь считается истинной жизнью, которая, однако, невольно становится будущей жизнью, поскольку настоящая противоречит идеалу истинной жизни, Но достоверность этой будущей жизни заключается только в достоверности её свойств, сообщающих ей характер истинной, высшей, соответствующей идеалу жизни.



Кто действительно верит в будущую жизнь и считает её несомненной жизнью, тому она является определенной жизнью именно потому, что она несомненна. Если я не знаю, как и чем я буду некогда, если между моим будущим и настоящим есть существенное, безусловное различие, значит, некогда я не буду знать, чем я был прежде, значит, единство сознания уничтожается, и мое место заступает в нем другое существо, следовательно, мое будущее бытие на самом деле ничем не отличается от небытия. Если же, напротив, нет существенного различия, то и будущая жизнь представляется мне предметом определенным и познаваемым. И это действительно так: я --- пребывающая сущность в смене свойств, я --- субстанция, соединяющая воедино настоящую и будущую жизнь. Как же я могу не знать будущей жизни? Напротив, жизнь в этом мире представляется мне темной и непонятной, и она станет для меня ясной и понятной только благодаря будущей жизни; здесь я --- таинственное, замаскированное существо; там спадет с меня маска, и я явлюсь таким, каков я в действительности. Поэтому утверждение, что другая, небесная жизнь, несомненно, существует, но что свойства её здесь непостижимы, есть только продукт религиозного скептицизма, зависящего от абсолютного непонимания религии, поскольку скептицизм совершенно чужд её сущности. То, из чего безбожная религиозная рефлексия делает только известный символ непостижимой, но тем не менее достоверной вещи, было вначале, в первоначальном истинном смысле религии, не символом, а вещью, самой сущностью. Неверие, являющееся в то же время ещё верой, ставит вещь под сомнение; оно не обладает достаточной глубиной мысли и мужеством, чтобы прямо отрицать её: но, подвергая сомнению, оно объявляет образ только образом, или символом. Ложность и ничтожество этого скептицизма подтверждаются историей. Кто сомневается в истинности образов бессмертия, сомневается в возможности такого существования, о котором говорит вера, например существования без материального, действительного тела или без полового различия, вот уже скоро усомнится и в самом загробном существовании. Вместе с образом рушится сама вещь --- именно потому, что образ и есть сама вещь.

Вера в небо или вообще в загробную жизнь основана на суждении. Оно выражает хвалу и порицание; оно носит критический характер; оно составляет онтологию флоры этого мира; и эта критическая онтология есть небо. Человек находит, что должно существовать только то, что кажется ему прекрасным, добрым, приятным, а все, что он считает дурным, скверным, неприятным, составляет для него то бытие, которое не должно существовать; а так как оно все-таки существует, --- оно обречено на гибель и ничтожество. Там, где жизнь не противоречит чувству, представлению, идее, или где это чувство, эта идея не считаются абсолютно истинными, там не может возникнуть веры в другую, небесную жизнь. Другая жизнь есть не что иное, как жизнь в соответствии с чувством, с идеей, которым противоречит эта жизнь. Загробная жизнь имеет своей целью уничтожить этот разлад и добиться такого состояния, которое соответствовало бы чувству и при котором человек был бы в согласии с самим собой. Неведомая загробная жизнь есть смешная химера: загробная жизнь есть осуществление известной идеи, удовлетворение известного требования, исполнение желания\dag\let\svthefootnote\thefootnote\let\thefootnote\relax\footnotetext{\dag<<Ibi nostra spes erit res>>. --- (Августин, где-то). <<Поэтому мы, первенцы бессмертной жизни, надеемся, что совершенство наступит в день судный, и мы ощутим и увидим ту жизнь, в которую мы верили и на которую надеялись>> (Лютер, ч. \rom{1}, стр. 459).}\let\thefootnote\svthefootnote\footnote{там наша надежда станет фактом (лат.).}; она лишь устраняет те ограничения, которые мешают здесь осуществлению идеи. Где бы было утешение, какой бы был смысл загробной жизни, если бы она была непроницаема, как ночь? Нет! там все сияет блеском самородного металла, а здесь все окрашено в тусклые краски ржавого железа. Значение будущей жизни, причина её существования только в том и заключаются, что она очищает металл от посторонних, чуждых примесей, отделяет хорошее от дурного, приятное от неприятного, похвальное от недостойного. Загробная жизнь есть свадьба, знаменующая союз человека со своей возлюбленной. Он давно уже знал свою невесту, давно томился по ней; но внешние обстоятельства, бездушная действительность препятствовали ему соединиться с ней. На свадьбе его возлюбленная не становится другим существом; иначе человек не мог бы так горячо к ней стремиться. Но теперь она принадлежит только ему, теперь она перестает быть только целью стремления и становится предметом действительного обладания. Здесь, на земле, загробная жизнь, конечно, есть только образ, но не образ далекого неведомого предмета, а портрет существа, пользующегося особой любовью и предпочтением человека. То, что человек любит, есть его душа. Язычник хранил в урнах прах дорогих мертвецов; у христианина небесное царство есть мавзолей, в который он заключает свою душу.



Для понимания веры, религии вообще нужно учитывать самые низшие, грубые ступени религии. Религию надо рассматривать не только по восходящей линии, но и во всю ширину её существования. При обсуждении абсолютной религии мы должны иметь также налицо многообразие других религий, извлекая их из мрака прошлого, если мы хотим надлежащим образом понять и оценить ту и другие. Страшные <<заблуждения>>, самые дикие крайности религиозного сознания нередко обнаруживают перед нами глубочайшие тайны абсолютной религии. На вид самые грубые представления оказываются часто детскими, невинными и правдивыми представлениями. Это относится также к представлениям загробной жизни. <<Дикарь>>, сознание которого не выходит за пределы его страны, переносит свою страну в загробную жизнь, причем он или оставляет природу такой, какова она есть, или исправляет её и таким образом преодолевает невзгоды своей жизни в представлении жизни будущей\ddag\let\svthefootnote\thefootnote\let\thefootnote\relax\footnotetext{\ddagДревние путешественники упоминают о народах, представляющих себе будущую жизнь но только не тождественной с настоящей и не лучшей, но даже ещё худшей. Парни (Oeuvres choisies, t. \rom{1}, Melang.) рассказывает об одном умиравшем негре-невольнике, который на увещания его креститься, чтобы получить бессмертие, ответил: <<Je ne veux point d'une autre vie, car peut-etre y serais-je encore votre esclave>>.}\let\thefootnote\svthefootnote
\footnote{я не стремлюсь к другой жизни, потому что, быть может, и в ней я все еще буду вашим невольником (фр.).}. В этой ограниченности некультурных народов заключается одна очень трогательная черта. В будущей жизни здесь выражается тоска по родине. Смерть разлучает человека с его родными, с его народом, с его отчизной. Но человек с неразвитым сознанием не может вынести этой разлуки, он должен вернуться в свое отечество. В Вест– Индии негры умышленно лишали себя жизни, чтобы воскреснуть у себя на родине. Эта ограниченность представляет резкий контраст с фантастическим спиритуализмом, который делает из человека какого-то бродягу, равнодушного к земле и перебегающего с одной звезды на другую. И в этой ограниченности, несомненно, кроется истина. Человек становится тем, что он есть, не только в силу своей самодеятельности, но и благодаря природе, тем более, что самодеятельность человека сама коренится в природе, именно в его природе. Будьте благодарны природе! Человека нельзя отделить от нее. Немец, божеством которого является самодеятельность, обязан своим характером своей природе в такой же мере, как восточный человек. Порицая индийское искусство, индийскую религию и философию, мы порицаем природу Индии. Вы негодуете на рецензента, который выхватывает из вашей статьи отдельные места, чтобы иметь возможность осмеять их. Но зачем вы сами делаете то, что порицаете у других? Зачем вы отделяете религию индусов от целого? В целом она не менее разумна, чем ваша абсолютная религия.



Вера <<диких>> народов в потустороннее, в будущую жизнь есть в сущности прямая, непосредственная, непоколебимая вера в жизнь земную. Эта жизнь, даже со всеми её местными условностями, имеет в их глазах абсолютную ценность; они не могут отвлечься от нее, не могут представить себе её конца, то есть верят в бесконечность, в непрерывность этой жизни. Вера в жизнь посмертную становится верой в другую жизнь только тогда, когда вера в бессмертие становится критической верой, когда люди начинают отличать то, что должно остаться здесь, от того, что должно перенести туда, --- преходящее --- от вечного. Но такая критика, такое различение относится и к этой жизни. Так, христиане делают различие между естественной и христианской, чувственной, мирской и духовной, святой жизнью. Небесная, другая жизнь есть та духовная жизнь, которую мы отличаем от всецело естественной жизни, но которая вместе с тем здесь, на земле, ещё не освободилась из-под её влияния. Все, что уже здесь исключает христианин, например, половую жизнь, он исключает и из другой жизни, с той только разницей, что там он освобождается от всего того, от чего здесь он только желает и волей, благочестием и умерщвлением плоти старается на самом деле освободиться. Эта жизнь тяжела и мучительна для христианина, потому что здесь ему приходится бороться с вожделениями плоти и с искушениями дьявола, и сам он обременён противоречиями.

Таким образом, вера культурных народов отличается от веры народов некультурных только тем, чем культура вообще отличается от некультурности, --- тем, что вера культурных народов есть вера, различающая, выделяющая, абстрактная. Различение нераздельно связано с суждением, а суждение влечет за собой различие между положительным и отрицательным, между добром и злом. Вера диких народов есть вера без суждения. Суждение свойственно только культуре: культурный человек считает истинной только культурную жизнь, христианин --- христианскую. Грубый человек природы вступает в будущую жизнь, не соблюдая приличий, таким, каков он есть: будущая жизнь есть его естественная нагота. Культурный человек, напротив, отталкивается от подобной необузданной жизни после смерти, так как и здесь, на земле, он останавливается перед нею. Вера в потустороннюю жизнь есть только вера в истинную земную жизнь: существенное содержание земной жизни есть существенное содержание потусторонней; поэтому вера в будущую жизнь не есть вера в другую, неведомую жизнь, а есть вера в истинность, бесконечность и, следовательно, непрерывность той жизни, которая уже и здесь, на земле, считается настоящей жизнью.

Как бог есть только сущность человека, очищенная от того, что человек чувствует или мыслит как ограничение, как зло, так и потусторонний мир есть не что иное, как настоящий мир, освобождённый от того, что представляется ограничением, злом. Чем яснее и определеннее индивид сознает ограничение как ограничение и зло как зло, тем яснее и определеннее он сознает потустороннее, где все эти ограничения исчезают. Потустороннее есть чувство, представление освобождения от тех ограничений, которые здесь угнетают самоощущение, бытие индивида. 

Развитие религии отличается от развития естественного, или разумного, человека только тем, что религия вместо кратчайшей прямой линии описывает кривую, а именно окружность. Естественный человек остается на своей родине потому, что она ему здесь нравится и совершенно удовлетворяет его; религия, возникающая на почве неудовлетворенности и раскола, покидает родину и удаляется на чужбину, чтобы вдали живее почувствовать сладость отчизны. Человек в религии разлучается с самим собой, чтобы снова вернуться к своей исходной точке. Человек отрицает себя, только чтобы потом снова утвердить себя, но теперь уже в преображённом виде. Таким же образом он отвергает и настоящую жизнь, чтобы в конце концов восстановить её как потустороннюю*\let\svthefootnote\thefootnote\let\thefootnote\relax\footnotetext{*поэтому там будет все восстановлено; даже не пропадет <<ни единый зуб или ноготь>> (см. Aurelius Prudent, Apotheos. de resurr. carnis hum.). И эта, в ваших глазах грубая, плотская и потому неугодная вам вера есть на самом деле единственно последовательная, честная, единственно истинная вера. Тождество личности предполагает тождество плоти.}\let\thefootnote\svthefootnote\footnote{см.: Аврелий Пруденций.<<Апофеоз воскрешения тела человека>> (лат.).}. Загробная жизнь есть потерянная и вновь обретённая и потому ярче сияющая земная жизнь. религиозный человек отказывается от радостей этого мира, чтобы обрести небесные радости или, вернее, потому, что он уже идеально обладает небесными радостями. А небесные радости ничем не отличаются от земных, они только свободны от ограничений и зла этой жизни. Таким образом, религия окольным путем приходит к той же цели, к какой естественный человек стремится по прямой линии, то есть к радости. Образность есть сущность религии. Религия жертвует вещью ради образа. Загробная жизнь есть жизнь земная, отраженная в зеркале фантазии, --- чарующий образ, а в смысле религии --- прообраз настоящей жизни: эта жизнь есть только тень, мерцание другой, духовной жизни в образах. Жизнь потусторонняя есть созерцаемая в образе, очищенная от всякой грубой материи --- украшенная посюсторонняя жизнь.



Улучшение, исправление предполагает порицание, недовольство. Но это недовольство есть нечто поверхностное. Я не отнимаю у вещи её ценности; она только не нравится мне такой, как она есть; я отрицаю только свойства, а не сущность; иначе я стал бы настаивать на её уничтожении. Дом, который мне совсем не нравится, я не улучшаю, а ломаю. Вера в потустороннее отрицает мир, но не его сущность; мир не нравится мне лишь в том виде, в каком он существует. Радость нравится каждому, верующему в будущую жизнь, --- кто же не испытывает радости, как него-то подлинного, существенного? --- но ему не нравится, что здесь за радостью следуют противоположные ощущения, что она преходяща. Поэтому он переносит радость в потустороннюю жизнь, как вечную, непрерывную, божественную радость; потустороннее поэтому называется царством радости, а на земле он переносит эту радость на бога; ведь бог по существу есть вечная, непрерывная радость. Ему нравится индивидуальность, но только без объективных целей; поэтому он уносит с собой только чистую, абсолютно субъективную индивидуальность. Ему нравится свет, но не нравится тяжесть, ибо для индивида она является ограничением; ему не нравится ночь, так как ночью человек подавлен природой. Поэтому в будущей Жизни есть свет, но нет ни тяжести, ни ночи --- только чистый, невозмутимый свет\dag\let\svthefootnote\thefootnote\let\thefootnote\relax\footnotetext{\dag<<После воскресения мертвых время уже не будет измеряться днями и ночами. Будет единый день без вечера>>. (Иоанн Дамаскин, Orth. fidei, lib. \rom{2}, гл. \rom{1}).}\let\thefootnote\svthefootnote.



Удаляясь от себя и сосредоточиваясь в боге, человек постоянно возвращается к самому себе, он только вращается вокруг себя: удаляясь от настоящей жизни, человек, в конце концов, снова приходит к ней. Чем выше и внечеловечнее кажется бог в начале, тем человечнее оказывается он в самом процессе и в конце. Точно так же: чем сверхъестественнее кажется в начале или издали жизнь небесная, тем яснее обнаруживается в конце или вблизи её тождество с естественной жизнью --- тождество, простирающееся даже на плоть, на тело. Сначала речь идет об отделении души от тела, как в созерцании бога речь идет об отделении сущности от индивида --- индивид умирает духовной смертью, остающееся мертвое тело есть человеческий индивид, а отделившаяся от него душа есть бог. Но отделение души от тела, сущности от индивида, бога от человека должно быть опять уничтожено. Всякая разлука близких друг другу существ мучительна. Душа тоскует по своей утраченной части, по своему телу; как и бог, отошедшая душа стремится к действительному человеку. Поэтому бог становится снова человеком, а душа снова возвращается в свое тело --- и восстанавливается полное единство этой и будущей жизни. Правда, это обновленное тело есть уже тело лучезарное, просветленное, чудесное, а главное, оно есть другое и вместе с тем то же самое тело\ddag\let\svthefootnote\thefootnote\let\thefootnote\relax\footnotetext{\ddag<<Ipsum (corpus) erit et non ipsum erit>> (Августин, v. \rom{1}. Сн. Doederlein, Inst. Theol. Cнrist. Altdorf 1781 par. 280).}\let\thefootnote\svthefootnote\footnote{тело будет то же и вместе с тем иное (лат.).}, подобно тому как бог есть другое и в то же время тождественное с человеком существо. Здесь мы снова возвращаемся к понятию чуда, которое соединяет противоречивое. Сверхъестественное тело есть тело фантастическое, и потому оно вполне отвечает чувству человека, так как оно есть тело чисто субъективное и ничуть не тяготит человека. Вера в будущую жизнь есть не что иное, как вера в истинность фантазии, подобно тому как вера в бога есть вера в истинность и бесконечность человеческой души. Иначе говоря: вера в бога есть вера в абстрактную сущность человека, а вера в потустороннюю жизнь есть только вера в абстрактную посюстороннюю жизнь.



Содержание будущей жизни есть блаженство, вечное блаженство личности, которая здесь, на земле, ограничивается, стесняется природой. Поэтому вера в потустороннее есть вера в освобождение субъективности от границ природы, следовательно, вера в вечность и бесконечность личности, и притом не в смысле её рода, постоянно развивающегося в новые индивиды, а в смысле уже существующих индивидов, следовательно, вера человека в самого себя. Но вера в царство небесное совпадает с верой в бога --- они однородны по содержанию; бог есть чистая, абсолютная, свободная от природных границ личность; он есть уже то, чем человеческие индивиды только должны быть и некогда будут --- поэтому вера в бога есть вера человека в бесконечность и истинность своего собственного существа --- божественная сущность есть человеческая, и притом субъективно-человеческая, сущность во всей своей абсолютной свободе и безграничности.

Главная наша задача выполнена. Мы свели внемировую, сверхъестественную и сверхчеловеческую сущность бога к составным частям существа человеческого как к его основным элементам. В конце мы снова вернулись к началу. Человек есть начало, человек есть середина, человек есть конец религии.





\part{Ложная, то есть богословская сущность религии}


\chapter{Исходная точка зрения религии}


Исходная точка зрения религии --- практическая, то есть в данном случае субъективная. Целью религии является благо, спасение, блаженство человека; отношение человека к богу есть не что иное, как отношение человека к своему спасению: бог есть осуществлённое спасение души или неограниченная сила, способная осуществить спасение, блаженство человека*\let\svthefootnote\thefootnote\let\thefootnote\relax\footnotetext{*<<Твое спасение да будет своей единственной мыслью, а бог --- твоей единственной заботой>>. (Фома Кемпийский, De imit., lib. \rom{1}, гл. 23. <<Не помышляй ни о чем вопреки собственному спасению>>. Нет, это слабо сказано: вместо вопреки мне следовало сказать кроме. (Бернард, De consid. ad Eugenium, lib. \rom{2}). <<Кто ищет бога, тот заботится о своем собственном спасении>>. (Климент Александрийский, Coh. ad gentes).}\let\thefootnote\svthefootnote. Христианская религия отличается от других религий именно тем, что ни одна из них не выдвигает спасение человека так настойчиво, как она. Поэтому она и называется не богословием, а сотериологией. Напротив, наиболее вдумчивые, правоверные христиане утверждали, что земное счастье отвлекает человека от бога, а земное несчастье, страдания и болезни возвращают его к богу\dag\let\svthefootnote\thefootnote\let\thefootnote\relax\footnotetext{\dagкто исключительно на основании несчастия доказывает реальность религии, доказывает также реальность суеверия.}\let\thefootnote\svthefootnote, поэтому только они приличествуют христианину. Почему? Потому, что в несчастьи человек настраивается только практически или субъективно; в несчастьи он помышляет только о том, что ему необходимо; в несчастьи бог ощущается как потребность человека. Наслаждение, радость ширят душу человека; несчастье, страдание гнетут, стесняют её --- страдание заставляет человека отрицать истину мира; все, чем очаровывается фантазия художника и разум мыслителя, утрачивает тогда в его глазах свою прелесть, свою силу; он замыкается в самом себе, в своей душе. Такое погруженное в себя, сосредоточенное на себе, успокаивающееся только в себе и отрицающее мир существо, идеалистическое по отношению к миру и природе и реалистическое по отношению к человеку, помышляющее только о необходимой внутренней потребности в спасении, есть бог. Бог, как бог, в смысле собственного имени, а не в смысле общего, метафизического существа, является богом лишь постольку, поскольку он по существу есть предмет религии, а не философии, чувства, а не разума, потребности сердца, а не свободы мысли, --- одним словом, он выражает не сущность теоретической, а сущность практической точки зрения.





%Сотериологией называется богословское учение о Христе как спасителе и искупителе человечества. Но это спасение не есть мирское, земное счастье и благо. --- Ред.



Религия связывает свои учения с проклятием и благословением, наказанием и блаженством. Блажен, кто верует, неверующего же ждет несчастье, гибель и осуждение. Следовательно, религия взывает не к разуму, а к чувству, к исканию блаженства, к аффектам страха и надежды. Она не стоит на теоретической точке зрения; иначе она должна была бы высказывать свое учение свободно, не связывая с ним практических следствий, не заставляя верить почти насильно; ведь утверждение, что неверующие осуждены на гибель, есть косвенное насилие над моей совестью. Страх перед адом заставляет меня верить. Даже в том случае, если моя вера свободна по происхождению, к ней все-таки примешивается страх, моя душа всегда охвачена беспокойством; сомнение, принцип теоретической свободы, кажется не преступлением. Высшее понятие, высшее существо религии есть бог, поэтому величайшим преступлением считается сомнение в боге или, по крайней мере, в том, что бог существует. А то, в чем я не дерзаю, не могу сомневаться, не беспокоясь о душе и не считая себя преступником, есть не предмет теории, а предмет совести, не сущность разума, а сущность чувства.

Но так как практическая или субъективная точка зрения является исключительно точкой зрения религии, то она считает законченным, цельным человеком только практического, предусмотрительного человека, действующего согласно сознательно поставленным физическим или нравственным целям и созерцающего мир с точки зрения этих целей и потребностей, а не существу изучающего свою природу; поэтому все то, что лежит позади практического сознания и составляет существенный предмет теории --- теории в первоначальном и наиболее общем смысле, в смысле объективного созерцания и опыта, разума и вообще науки, --- выносится религией за пределы человека и природы и воплощается в особое личное существо\ddag\let\svthefootnote\thefootnote\let\thefootnote\relax\footnotetext{\ddagздесь и в других местах этой книги теория понимается как источник истинной объективной практики, ибо человек может лишь постольку, поскольку он знает: tantum potest quantum scit. Поэтому выражение --- <<субъективная точка зрения>> означает, что точка зрения необразованности и невежества есть точка зрения религии.}\let\thefootnote\svthefootnote. Все доброе, особенно же то, что невольно производит впечатление на человека, что не соответствует расчету и преднамеренности, что выступает за пределы практического сознания, исходит от бога, а все дурное, злое, бедственное и главным образом все то, что непроизвольно рушит все его моральные и религиозные намерения или увлекает с невероятной силой, исходит от дьявола. Для ознакомления с сущностью религии необходимо познакомиться с понятием дьявола, сатаны, злого духа*\let\svthefootnote\thefootnote\let\thefootnote\relax\footnotetext{*относительно библейских представлений о сатане, его мощи и влиянии см. у Люцельбергера (Gundzuege der Paulinischen Glaubenslehre) и у Кнаппа (Vorles. ueber die christl. Glaubensleнre, par 62--65). Сюда не относятся и демонические болезни и одержимость бесом. Даже для этих болезней подведено основание в Библии (см. Кнапп, пар. 65, \rom{3}.2, 3).}\let\thefootnote\svthefootnote\footnote{<<Основы учения ап. Павла о вере>>; <<Лекции о христианском учении о вере>> (нем.).}. Отрицание этих существ влечет за собой искажение религии. Благодать и её действие противополагаются действиям дьявола. Все невольные, исходящие из глубины природы, чувственные импульсы, вообще все подлинные или воображаемые необъяснимые проявления нравственного и физического зла религия истолковывает влиянием злого существа, а непроизвольный экстаз, воодушевление --- влиянием доброго существа, бога, святого духа, благодати. Этим объясняется произвол благодати --- жалобы благочестивых людей на то, что благодать то освещает и посещает их, то оставляет и отвергает их. Жизнь, сущность благодати, есть жизнь, сущность непроизвольных душевных движений. Чувство есть утешитель христиан. Провалы чувства и воодушевления суть моменты, не осенённые божией благодатью.





По отношению к внутренней жизни благодать можно назвать религиозным гением, а по отношению к внешней жизни --- религиозным случаем. Человек становится добрым или злым не только благодаря самому себе, своей силе, своей воле, но также и под влиянием множества тайных и явных предопределений, которые мы приписываем, как выражался Фридрих Великий, силе <<его величества случая>>, так как они вовсе не покоятся на какой-либо абсолютной, или метафизической, необходимости\dag\let\svthefootnote\thefootnote\let\thefootnote\relax\footnotetext{\dagШеллинг в своем трактате о свободе объясняет эту загадку самоопределением, состоявшимся в вечности, то есть до начала настоящей жизни. Какое фантастическое, иллюзорное предположение! Но фантастика, беспочвенная, детская фантастика, составляет самую сокровенную тайну так называемых положительных философов, этих <<глубоких>>, даже слишком глубоких религиозных мыслителей. Чем нелепее, тем глубже.}\let\thefootnote\svthefootnote. Божия благодать есть мистифицированная сила случая. Этим ещё раз подтверждается то, что мы признали существенным законом религии. Религия отрицает, отбрасывает случайность и ставит все в зависимость от бога, объясняя все из него; но она отрицает случайность только кажущимся образом; она только заменяет его божественным произволом. Божья воля, которая по необъяснимым причинам, то есть, говоря откровенно и честно, по беспричинному, абсолютному произволу, по капризу божества, предопределяет одних на гибель, зло и страдание, а других к спасению, добру и блаженству, --- такая воля в сущности ничем не отличается от силы <<его величества случая>>. Тайна благодатного избрания есть тайна, или мистика, случая. Я говорю мистика случая, так как на самом деле случай есть мистерия, хотя и игнорируемая нашей умозрительной философией религии, которая из-за иллюзорных мистерий абсолютного существа, то есть из-за богословия, забыла подлинные мистерии мышления и жизни, а из-за тайны божией благодати или свободы выбора упустила из виду тайну случая\ddag\let\svthefootnote\thefootnote\let\thefootnote\relax\footnotetext{\ddagэто вскрытие тайны благодатного избрания, несомненно, будет названо нечестивым, безбожным или дьявольским. Пусть так; но я предпочитаю быть дьяволом в союзе с истиной, чем ангелом в союзе с ложью.}\let\thefootnote\svthefootnote.





Но возвратимся к нашему предмету! Дьявол олицетворяет собой начало отрицательное, злое, исходящее из сущности, а не из воли; бог есть начало положительное, доброе, исходящее из сущности, а не из сознательной воли. Дьявол есть непроизвольное, необъяснимое зло, вред, бедствие; а бог --- непроизвольное, необъяснимое добро. Оба имеют один и тот же источник, только качества их различны или даже противоположны. Поэтому вплоть до наших дней вера в дьявола была тесно связана с верой в бога и отрицание дьявола считалось таким же атеизмом, как и отрицание бога. И это не без основания. Ведь там, где явления зла выводятся из естественных причин, там и явления добра, проявления божества, выводятся из самой природы вещей, а не из сверхъестественного существа, и дело кончается либо совершенным отрицанием бога, либо верой в другого, религией непризнаваемого бога, либо обычно превращением бога в праздное, пассивное существо, бытие которого уподобляется небытию; поэтому оно не оказывает никакого влияния на жизнь и только связывается с началом мира, как его первопричина. Бог сотворил мир; вот все, что ещё остается от бога. Прошедшее совершенное здесь необходимо; ведь с этого момента мир продолжает свой путь наподобие машины. Добавление: он творит постоянно, творит и поныне, является только добавлением формальной рефлексии; прошедшее совершенное вполне выражает здесь религиозный смысл: ведь дух религии есть дух прошедшего, и деятельность бога обратилась в прошедшее совершенное. Но если религиозное сознание говорит: прошедшее совершенное есть и поныне ещё настоящее, то дело резко меняется: хотя и это положение есть продукт рефлексии, но оно имеет закономерный смысл, так как здесь бог вообще мыслится как существо деятельное.

Религия вообще упраздняется там, где между богом и человеком возникает представление мира и так называемых посредствующих причин. Как только сюда замешивается чуждое религии начало --- принцип разума, --- тотчас нарушается мир, гармония религии, основанная только на непосредственной связи человека с богом. Посредствующая причина есть капитуляция верующего сердца перед неверующим рассудком. Согласно религии, бог действует на человека посредством других вещей и существ. Но и в данном случае бог есть единственная причина, единственное действующее и деятельное существо. По смыслу религии, все, что делают тебе другие, исходит не от них, а от бога. <<Другой>> есть только призрак, средство, орудие, а не причина. Но посредствующая причина есть нечто среднее между самостоятельным и несамостоятельным существом. Бог дает первый толчок, а затем начинается и самодеятельность*\let\svthefootnote\thefootnote\let\thefootnote\relax\footnotetext{*сюда относится также нелепое или скорее софистическое учение concursus dei, согласно которому бог даст не только первый толчок, но и соучаствует в действии causa secunda. Впрочем, это учение есть только особый вид противоречивого дуализма между богом и природой, который проходит через всю историю христианства (по поводу этого примечания, как и вообще всей главы, см. также у Штрауса, Христианское вероучение, кн. \rom{2}, пар. 75 и 76).}\let\thefootnote\svthefootnote\footnote{содействие бога (лат.).}\footnote{вторая причина (лат.), в смысле посредствующей причины.}.



Религия вообще по существу своему не знает никаких посредствующих причин; они скорее служат для нее камнем преткновения; потому что царство посредствующих причин, чувственный мир, природа отделяют человека от бога, хотя бог, как бог действительный, сам в свою очередь есть чувственное существо\dag\let\svthefootnote\thefootnote\let\thefootnote\relax\footnotetext{\dag<<Доколе мы облечены плотью, мы далеки от бога>> (Бернард, Epist. 18, в базельском издании 1552 г.). Поэтому понятие загробной жизни есть не что иное, как понятие истинной, совершенной, свободной от земных ограничений и стеснений религии, и потустороннее, как уже сказано, есть не что иное, как истинное мнение и настроение, открытое сердце религии. Здесь мы верим, а там мы узрим, то есть там нет ничего, кроме бога, следовательно, нет ничего между богом и душою, да и не должно быть, ибо непосредственное единство бога и души есть истинная вера и настроение религии. Для нас бог неизменно сокрыт, и мы не можем соприкасаться с ним лицом к лицу. Всякая тварь теперь есть лишь пустая личина, под которой скрывается и через которую действует бог>> (Лютер, ч. \rom{11}, стр. 70). <<Если б ты был одинок и не носил образа твари, ты мог бы беспрерывно обладать богом>> (Таулер, L. c., стр. 313).}\let\thefootnote\svthefootnote. Поэтому религия верит в падение этой преграды. Придет время, когда не будет ни природы, ни материи, ни тела, но крайней мере такого тела, которое отделяет человека от бога: придет время, когда останутся только бог и благочестивые души. Религия черпает сведения о посредствующих причинах, о преграде между богом и человеком, из чувственного, естественного, следовательно, безбожного, или во всяком случае нерелигиозного, созерцания, которое она тотчас же отвергает, обращая действие природы в деятельность бога. Но эта религиозная идея противоречит естественному рассудку и смыслу, приписывающему вещам природы действительную самодеятельность. Религия разрешает это противоречие между чувственным и религиозным созерцанием тем, что приписывает неопровержимую действенность вещей богу, действующему посредством этих вещей. Здесь сущностью, главным предметом является бог, а предметом ничтожным, второстепенным --- мир.



С другой стороны, где посредствующие причины становятся активными, где они, так сказать, эмансипируются, там, наоборот, природа становится сущностью, а бог --- отрицанием сущности. Мир является самостоятельным в своем бытии, в своем существовании; он ещё зависим лишь по своему началу. Здесь бог оказывается не первоначальным, абсолютно необходимым существом, а существом только гипотетическим, производным, возникшим из потребности нашего ограниченного рассудка, для которого без самодвижущего начала необъяснимо существование мира, им же обращённого в машину. Бог существует не ради себя, а только ради мира, для того, чтобы им как первопричиной объяснить мировой механизм. Ограниченный человеческий рассудок становится в тупик перед первоначально самостоятельным бытием мира, потому что он смотрит на мир лишь с субъективно практической точки зрения, видя в нем только простую, грубую машину, а не величественный, прекрасный космос. Таким образом, мир потрясает его голову. Удар сотрясает его мозг, и при этом сотрясении он объективирует вне себя собственный толчок, обращает его в первоначальный толчок, положивший начало миру, подобно тому, как математический толчок приводит в движение материю, так и мир продолжает свое вечное движение. Следовательно, чело век приписывает миру механическое происхождение. Всякая машина должна иметь начало; это содержится в её понятии, ибо причина её движения заключается не в ней самой.

Из этого примера очевидно, что всякая религиозно умозрительная космогония есть тавтология. В космогонии человек только выясняет или объективирует то понятие, которое он имеет о мире. Так и тут: если мир есть машина, то, разумеется, он не сам себя сделал, а сделан, то есть возник путем механическим. В этом пункте религиозное сознание совпадает с механическим воззрением: оба считают мир простым изделием, продуктом воли. Но они согласны друг с другом только на одно мгновение, только на момент сотворения или создания мира --- как только это творческое мгновение исчезло, исчезла и гармония. Механику бог нужен только для создания мира; как только мир создан, он поворачивается к богу спиной и от души радуется безбожной самостоятельности мира. Но религия создает мир только для того, чтобы всегда сознавать его ничтожество, его зависимость от бога. У механика творение является последней тонкой нитью, связывающей его с религией; религия, считающая ничтожество мира настоящей истиной (ибо всякая сила и деятельность есть сила и деятельность бога), является для него не более, как воспоминанием юности; поэтому творение мира, акт религиозный, обусловливающий ничтожество мира, --- ибо первоначально, до творения не было мира, а был один только бог, --- он относит в даль прошедшего, тогда как самостоятельность мира, поглощающая все его мысли и чувства, действует на него всей мощью своего реального наличия. Механик прерывает и сокращает деятельность бога деятельностью мира. В его глазах бог имеет лишь историческое право, противоречащее его естественному праву, поэтому он по возможности урезывает это ещё признаваемое за богом право, чтобы ещё более расширить и очистить поле деятельности для своих естественных причин и для собственного рассудка.

Человек механического мировоззрения относится к творению так же, как к чудесам, которые он может допустить и действительно допускает, раз они уже существуют, по крайней мере в религиозном мнении. Но помимо того, что он объясняет чудеса естественным, то есть механическим, путем, он закрывает глаза на чудеса лишь потому, что переносит их в прошедшее; в настоящем же он требует только одного естественного. Когда разум и чувство человека отрешаются от чего-либо, когда чему-нибудь мы верим не добровольно, а лишь потому, что другие верят, или что нужно почему-либо верить, --- словом, когда внутренняя вера уходит в область прошедшего, то и внешний предмет веры также переносится в прошедшее. Это разбивает оковы неверия, но вместе с тем оставляет вере, по крайней мере, историческое право. Прошедшее является здесь хорошим средством соединения веры с неверием: я верю в чудеса, но, nota bene\footnote{хорошо заметьте (лат.).}, не в те чудеса, которые совершаются теперь, а в те, которые совершились когда-то и теперь, слава богу, относятся к далекому прошлому. Так и здесь. Творение есть непосредственное дело или действие рук божиих --- чудо, потому что сначала ничего не было, кроме бога. В представлении творения человек выходит за пределы мира, отвлекается от него; он представляет его себе несуществовавшим в момент творения; таким образом, он закрывает глаза на то, что между ним и богом стоит чувственный мир; он становится в непосредственное соприкосновение с богом. А человек механического мировоззрения боится этого непосредственного соприкосновения с богом, и в лучшем случае он настоящее обращает в прошедшее; между своим естественным, или материалистическим, миросозерцанием и идеей непосредственного действия божьего он вдвигает преграду тысячелетий.

По смыслу религии, бог есть единственная причина всякого положительного, благого действия\ddag\let\svthefootnote\thefootnote\let\thefootnote\relax\footnotetext{\ddagсобственно, он является также причиной и всех отрицательных, злых, вредных, человеконенавистнических действий; ибо и они совершаются, по словам софистической теологии, лишь с соизволения божьего, даже дьявол, виновник всего злого и дурного, есть, собственно, не что иное, как злой бог, гнев божий, олицетворённый, представленный как особое существо; поэтому гнев божий есть причина всякого зла. <<Ужасные сцены истории (например, в Иерусалиме и Утике) должны напоминать нам о гневе божьем и побуждать нас к тому, чтобы искренним покаянием и сердечной молитвой смягчить бога>>. (Меланхтон, Declam., t. \rom{3}, стр. 29).
}\let\thefootnote\svthefootnote, бог есть последний и единственный довод, которым она разрешает, или вернее, отклоняет все задаваемые теорией, или разумом, вопросы; ведь религия на все вопросы отвечает <<нет>>; она дает ответ, не говорящий ничего, потому что она отвечает на самые различные вопросы одним и тем же доводом, что все явления природы суть непосредственные действия бога, то есть целеполагающего, личного сверхъестественного или внеприродного, существа. Бог есть понятие, возмещающее недостаток теории. Бог есть объяснение необъяснимого, ничего не объясняющее, потому что оно должно объяснять все без различия; бог есть ночь теории, делающая все ясным сердцу, потому что в ней пропадает последний различающий луч рассудка; бог есть положительное неведение, разрешающее все сомнения путем их полного отрицания, --- всеведующее и ничего определенного не знающее, ибо все, что производит впечатление на разум, исчезает, теряет свою индивидуальность перед религией, становится ничем перед очами божественного всемогущества. Ночь есть мать религии.


Характерным актом религии, подтверждающим её сущность, является молитва. Молитва всемогуща. Бог исполняет то, о чем его просит благочестивый человек в молитве. А он просит не только о духовных вещах*\let\svthefootnote\thefootnote\let\thefootnote\relax\footnotetext{*только неверие в молитву лукаво ограничило молитву духовным элементом.}\let\thefootnote\svthefootnote, он просит также о вещах, которые находятся вне человека и зависят от власти природы, которую он хочет победить в молитве; в молитве он прибегает к сверхъестественному средству, чтобы достигнуть естественных самих по себе целей. Бог является для него не отдаленной, первой, а непосредственной, ближайшей действующей причиной всех естественных явлений. В молитве им игнорируются все так называемые посредствующие силы и причины; иначе молитва его утратила бы всю свою силу, весь свой жар. Эти силы не могут быть его объектом; иначе он добивался бы своей цели окольным путем. А он хочет непосредственной помощи. Он прибегает к помощи молитвы в уверенности, что эта молитва поможет ему больше, чем все усилия разума и природы, и в уверенности, что молитва обладает сверхчеловеческой, сверхъестественной силой\dag\let\svthefootnote\thefootnote\let\thefootnote\relax\footnotetext{\dagпоэтому в грубо чувственном представлении молитва есть принуждающее или чудодейственное средство. Но это представление то есть христианское (хотя у многих христиан встречается утверждение, что молитва принуждает бога), ибо в христианстве бог есть сам по себе самоудовлетворённый дух, всемогущество благости, которое ни в чем не отказывает (религиозно настроенной) душе. Но представление принуждения соответствует лишь бессердечному богу.}\let\thefootnote\svthefootnote. В молитве он обращается непосредственно к богу. Следовательно, бог является для него непосредственной причиной, исполнением молитвы, осуществляющей молитву силой. Но непосредственное действие бога проявляется в чуде --- чудо составляет поэтому существенный элемент религиозного созерцания. Религия объясняет все путем чуда. Само собой разумеется, что чудеса совершаются не всегда, подобно тому как человек не всегда молится. Что чудеса не всегда бывают, это заключается не в сущности религии, а в естественном, или чувственном, созерцании. Религия же нераздельна с чудом. Всякая истинная молитва есть чудо, акт чудодейственной силы. Внешнее чудо только обнаруживает внутренние чудеса, то есть в нем фактически проявляется, в пространстве и времени, то, что составляет основное воззрение религии, а именно, что бог есть непосредственная сверхъестественная причина всех вещей. Фактическое чудо есть аффективное выражение религии --- момент возбуждения. Чудеса совершаются только в необычайных случаях, когда душа приходит в экстаз, поэтому случаются и чудеса гнева. Хладнокровие не творит чудес. Зато душа открывается именно в аффекте. Человек не всегда молится с одинаковой теплотой и силой. Поэтому иногда молитва остается без последствий. Только горячая молитва обнаруживает сущность молитвы. Молится только тот, кто считает молитву священной, божественной силой. То же можно сказать о чуде. Чудеса совершаются часто или редко --- это все равно --- только там, где чудом определяется миросозерцание. Чудо не есть принадлежность теоретического, или объективного, созерцания мира и природы; чудо удовлетворяет практические потребности, и притом вопреки законам, импонирующим разуму. В чуде человек подчиняет природу, как бытие, ничтожное само по себе, по своим целям; чудо есть высшая степень духовного, или религиозного, эгоизма; в чуде все идет на пользу страждущего человека. Следовательно, основой религиозного миросозерцания, с практической или субъективной точки зрения, является бог (ведь сущность чудотворной силы совпадает с сущностью бога) --- существо чисто практическое, или субъективное, возмещающее недостаток, потребность теоретического миросозерцания, следовательно, не служащее объектом мышления и познания, подобно чуду, обязанному своим происхождением отсутствию мышления. Если я стану на точку зрения мышления, исследования, теории и буду рассматривать вещи только по отношению к ним самим, то чудотворная сущность, чудо обращается в ничто --- я строго отличаю здесь религиозное чудо от чуда естественного, хотя их постоянно смешивают между собой, чтобы обольстить разум и под видом естественности перенести религиозное чудо в царство разума и действительности.





Религии неведомы ни точка зрения, ни сущность теории, поэтому скрытая от нее, очевидная только для теории, истинная, всеобщая сущность природы и человечества представляется ей другим, чудесным, сверхъестественным существом: понятие рода заменяется понятием бога, который в свою очередь является личным существом, отличающимся от человеческих индивидов только тем, что ему присущи качества целого рода. Поэтому в религии человек неизбежно ставит свою сущность вне себя и воплощает её в другое существо, --- неизбежно потому, что ему неведома сущность теории, потому, что все его сознательное существо претворяется в практическую субъективность. Бог есть его другое <<Я>>, его вторая, утраченная половина --- в боге он дополняет себя; в боге он впервые является совершенным человеком. Он чувствует потребность в боге; ему недостает чего-то, но чего --- он не знает; бог есть это недостающее нечто; бог необходим человеку, неразделен с его сущностью. Мир для религии ничто\ddag\let\svthefootnote\thefootnote\let\thefootnote\relax\footnotetext{\ddag<<Вне божественного промысла и всемогущества природа есть ничто>> (Лактанций, Div. Inst. lib. 3, c. 28). <<Хотя все созданное богом прекрасно, но по сравнению с творцом --- несовершенно и даже ничтожно, ибо он приписывает лишь себе бытие в высшем и собственном смысле, когда говорит: Я есть сущий>> (Августин, De perfect. just. hom., c. 14).}\let\thefootnote\svthefootnote\footnote{см.: Вторая книга Моисеева. Исход (3,14).}. Мир как совокупность действительности во всем её великолепии открывает только теория; теоретические радости суть высшие духовные радости жизни; но религии неведомы радости мыслителя, естествоиспытателя, художника. Ей чуждо созерцание вселенной, сознание действительной бесконечности, сознание рода. Только в боге дополняет она недостаток жизни, недостаток существенного её содержания, которое в бесконечном изобилии дает действительная жизнь теоретическому созерцанию. Бог заменяет религии утраченный мир. Бог для нее есть чистое созерцание, жизнь теории.



Практическое созерцание есть созерцание нечистое, запятнанное эгоизмом, ибо оно не позволяет мне относиться к вещи бескорыстно; оно не есть самодовлеющее созерцание, так как я не отношусь здесь к объекту, как к равноценному со мной. Теоретическое созерцание, напротив, есть радостное, самодовлеющее, блаженное созерцание; ведь для него всякий объект является объектом любви и восхищения, для него он сверкает дивным блеском, как алмаз, в свете свободного исследования, прозрачный, как горный кристалл. Теоретическое созерцание есть созерцание эстетическое; а практическое, напротив, есть созерцание неэстетическое. Поэтому в боге религия возмещает недостаток эстетического созерцания. Религия считает мир сам по себе ничтожным, а восторженное созерцание его кажется ей идолопоклонством, ибо мир для нее лишь жалкое изделие*\let\svthefootnote\thefootnote\let\thefootnote\relax\footnotetext{*<<Глаз любит наслаждаться прекрасными формами, блеском и гармонией красок. Но эти вещи не должны сковывать моей души; ею владеет лишь бог, её создавший; хотя они прекрасны, ибо они им созданы, но только он сам есть мой бог, а не эти вещи>> (Августин, Confess., 1. \rom{10}, c. 34). <<Писание запрещает нам (Второе послание к Коринфянам 4: 18) обращать свои помыслы на видимое. Поэтому должно любить бога и презирать весь этот мир, то есть все чувственное, хотя им и должно пользоваться для потребностей этой жизни>> (он же, De Moribus Eccl. cathol., lib. 1, c. 20).}\let\thefootnote\svthefootnote. Поэтому бог представляется ей чистым, незапятнанным, то есть теоретическим или эстетическим созерцанием. Бог есть объект, к которому религиозный человек относится объективно ради самого объекта. Бог есть самоцель. Бог для религии имеет то же значение, какое для теории --- предмет вообще. Всеобщая сущность теории составляет для религии особую сущность.







\chapter{Противоречие в бытии бога}

Религия есть отношение человека к своей собственной сущности --- в этом заключается её истинность и нравственная спасительная сила, --- но не как к своей сущности, а как к другому, отличному от него и даже противоположному ему существу; в этом заключается её ложь, её ограниченность, её противоречие разуму и нравственности, в этом --- пагубный источник религиозного фанатизма, высший, метафизический принцип кровавых человеческих жертв, одним словом, в этом заключается первопричина всех ужасов, всех потрясающих сцен в трагедии истории религии.

Созерцание человеческой сущности как другого, самодовлеющего существа носит в первоначальном понятии религии непроизвольный, наивный, непосредственный характер; оно непосредственно отличает бога от человека и в то же время отождествляет его с ним. Но когда религия с годами приобретает большую рассудочность, когда внутри религии пробуждается рефлексия о религии, тогда начинает тускнеть сознание тождественности божественного существа с человеческим, --- одним словом, когда религия превращается в богословие, тогда первоначально непроизвольное, невинное отграничение бога от человека становится преднамеренным, вымученным разграничением, имеющим целью вытеснить из сознания укоренившееся в нем представление тождества бога и человека.

Поэтому чем ближе религия к своему источнику, тем она правдивее и искреннее, тем меньше она скрывает от себя эту свою сущность. Другими словами, в момент возникновения религия не знает качественного, или существенного, различия между богом и человеком. И это тождество не смущает религиозного человека, так как его рассудок находится ещё в гармонии с его религией. Древнееврейский Иегова лишь по бытию отличался от человеческого индивида, но качественно, по своему внутреннему существу, он вполне уподоблялся человеку, имел те же страсти, те же человеческие, даже телесные свойства. Только позднейшее иудейство провело резкую грань между Иеговой и человеком и прибегло к помощи аллегории, чтобы придать антропопатизмам другой, чем первоначально, смысл.

То же было и в христианстве. В древнейших источниках божественность Христа не была так резко выражена, как впоследствии. В изображении Павла Христос является ещё существом неопределенным, колеблющимся между небом и землей, между богом и человеком, вообще существом, подчиненным богу, --- первым из ангелов, первосозданным, но все же сотворённым и, быть может, даже рождённым, но тогда и ангелы и люди тоже рождены, а не сотворены, потому что бог также и их отец. Только церковь выразительно отождествила Христа с богом, сделала его исключительным сыном божьим, определила его отличие от людей и ангелов и предоставила ему, таким образом, монополию вечного, несотворённого существа.

Первый по понятию способ, каким религиозная рефлексия, богословие, обращает божественное существо в другое существо, стоящее вне человека, есть бытие божие, ставшее предметом особой, формальной аргументации.

Доказательства бытия божия считаются противоречащими сущности религии. Они действительно являются такими, но лишь по форме доказательства. Религия непосредственно представляет внутреннюю сущность человека, как объективированное, другое существо. И аргументация имеет только одну цель --- доказать, что религия права. Совершенное существо есть такое существо, выше которого немыслимо никакое другое существо, --- бог есть наивысшее, что только мыслит и может мыслить человек. Эта предпосылка онтологического доказательства --- самого интересного доказательства, так как оно исходит изнутри --- выражает самую внутреннюю, самую сокровенную сущность религии. То, что является наивысшим для человека, от чего он не может больше отвлечься, что составляет существенную границу его разума, его чувства, его настроения, это и есть для него бог --- id quo nihil majus cogitari potest\footnote{то, выше чего нельзя помыслить}. Но это наивысшее существо не было бы наивысшим, если бы оно не существовало; тогда мы могли бы представить себе другое, более высокое существо, преимуществом которого было бы бытие; но такое предположение недопустимо в силу вперед уже установленного понятия совершеннейшего существа. небытие есть недостаток; бытие есть совершенство, счастье, блаженство. Существо, которому человек отдает, приносит в жертву все, что ему священно и дорого, должно обязательно обладать благом, счастьем бытия. Несоответствие религиозному смыслу заключается только в том, что бытие мыслится обособленным, и чрез то возникает иллюзия, что бог есть существо только мыслимое, существующее лишь в представлении, иллюзия, которая, впрочем, тотчас же уничтожается; ведь доказательство бытия божия именно то и доказывает, что бог есть бытие, отличное от бытия только мыслимого, бытие вне человека, вне мышления, действительное бытие, бытие в себе.

Доказательство лишь тем и отличается от религии, что оно скрытую энтимему религии облекает в формальное умозаключение, развивает и тем самым различает то, что религия непосредственно соединяет, ибо что для религии есть наивысшее, есть бог, то представляется ей не как мысль, а непосредственно как истина и действительность. Что каждая религия делает такое скрытое недоразвитое умозаключение, это лучше всего обнаруживается в полемике против других религий. Вы, язычники, не знали ничего выше своих богов, потому что вы погрязали в пороках. Ваши боги покоились на выводе, предпосылками которого были ваши чувственные стремления и страсти. Вы считали лучшей жизнью жизнь, дающую полную волю чувственным побуждениям, и поэтому сделали её своим богом. Вашим богом было ваше чувственное стремление, вашим небом --- лишь широкое поприще страстей, ограниченных только общественной, вообще действительной жизнью. Но по отношению к самой себе религия, естественно, не делает никакого сознательного заключения, так как высшая доступная ей мысль есть её предел и имеет для нее силу необходимости, --- следовательно, является для нее не мыслью, не представлением, а непосредственной действительностью.

Доказательства бытия божия имеют целью обнаружить, отделить от человека его внутреннюю сущность*\let\svthefootnote\thefootnote\let\thefootnote\relax\footnotetext{*вместе с тем они имеют целью утвердить сущность человека. Различные доказательства суть не что иное, как различные в высшей степени интересные формы самоутверждения человеческой сущности. Таково например, физикотеологическое доказательство, которое есть самоутверждение целесообразно действующего рассудка.}\let\thefootnote\svthefootnote. Благодаря бытию бог становится вещью в себе. Бог есть существо не только для нас, существо для нашей веры, для нашего чувства, для нашего существа, --- он есть также существо для себя, существо вне нас, --- одним словом, он есть не только вера, чувство, мысль, но и действительное бытие, отличное от веры, чувства и мышления. А такое бытие есть не что иное, как чувственное бытие.



Понятие чувственности заключается уже в характерном выражении: бытие вне нас. Софистическая теология понимает слова <<вне нас>> не в собственном смысле, а заменяет их неопределенным выражением: бытие, независимое и отличное от нас. Но если это бытие вне нас есть бытие не в собственном смысле, то так же надо понимать и бытие божие. А между тем здесь речь идет именно только о бытии в собственном смысле; и вполне определенным, не вызывающим сомнения выражением для такого бытия, отличного от нас, и является бытие вне нас.

Действительное, чувственное бытие есть такое бытие, которое не зависит от моего самоопределения, от моей деятельности, а само меня определяет помимо моей волн и которое существует также и тогда, когда меня нет, когда я не мыслю и не чувствую его. Поэтому бытие божие должно бы быть чувственно определенным бытием. Но бога нельзя ни видеть, ни слышать, ни воспринимать чувственно. Он вовсе не существует для меня, если я для него не существую; если я не верю в бога и не мыслю о нем, то и бог для меня не существует. Следовательно, он существует лишь постольку, поскольку я о нем мыслю и верую в него --- добавление <<для меня>> не нужно. Итак, его бытие есть действительное и в то же время недействительное; это --- духовное бытие, вот выход из затруднения. Но духовное бытие есть только мыслимое бытие, предмет веры. Следовательно его бытие есть нечто среднее между чувственным бытием и мыслимым бытием, нечто противоречивое. Иначе говоря: оно есть чувственное бытие, но отделенное от всех определений чувственности, следовательно, нечувственно-чувственное бытие, противоречащее понятию чувственности, вообще неопределенное бытие, которое в основании своем чувственно, но лишено всех предикатов действительного, чувственного бытия с целью замаскировать это основание. Но такое бытие противоречит себе. Бытию присуща полная, определенная действительность.

Необходимым следствием этого противоречия является атеизм. Бытие бога по своему существу есть эмпирическое или чувственное бытие, но оно не имеет его признаков; оно является само по себе делом опыта, но в действительности не является предметом опыта. Оно требует от человека, чтобы тот искал его в действительности; оно наполняет его чувственными представлениями и притязаниями; поэтому если они не удовлетворяются, если опыт скорее противоречит этим представлениям, то он имеет полное право отрицать подобное бытие.

Кант, как известно, в своей критике доказательств бытия божьего утверждает, что это бытие нельзя доказать из разума. Поэтому Кант не заслужил порицания, которое ему высказал Гегель. Напротив, Кант совершенно прав: из одного только понятия нельзя выводить бытия. Он не прав лишь постольку, поскольку он хотел сказать этим нечто особенное и как бы упрекнуть разум. Это само собой понятно. Разум не может сделать свой объект объектом чувств. То, что я мыслю, я не могу представить в то же время вне себя, как чувственный предмет. Доказательство бытия божия выходит за пределы разума; верно, но в том же смысле, в каком и зрение, слух, обоняние выходят за пределы разума. Нелепо упрекать разум в том, что он не удовлетворяет требованию, которое можно предъявить только к чувствам. Бытие эмпирическое, действительное бытие, дают мне только чувства. И бытие в вопросе о бытии божьем имеет значение не внутренней реальности, истины, а только формального, внешнего бытия, такого бытия, которое присуще всякому чувственному существу, пребывающему вне человека и не зависящему от его настроения и мысли.

Поэтому религия, опирающаяся на бытие бога как на эмпирическую, внешнюю истину, становится для внутреннего настроения безразличной вещью. Таким образом, вера только в бытие бога становится главным предметом религии, независимо от внутреннего качества, от духовною содержания, подобно тому как в культе религии церемония, обряд, таинство сами по себе обращаются в предмет религии, помимо духа и настроения. Вера в бога, в то, что он существует, есть уже залог спасения. Ты можешь представлять этого бога благим существом или чудовищем, Нероном или Калигулой, образом твоей страсти, твоего мщения, твоего тщеславия --- это безразлично; --- главное, чтоб ты не был атеистом. История религии достаточно доказывает это. Если бы бытие божье само по себе укреплялось в умах как религиозная истина, то люди не имели бы о боге позорных, нелепых и ужасных представлений, пятнающих историю религии и богословия. Бытие бога было чем– то обыкновенным, внешним и в то же время священным, --- что же удивительного в том, что на этой почве возникли только крайне пошлые, грубые, нечестивые представления и мысли.

Атеизм считался и до сих пор ещё считается отрицанием всех моральных принципов, всех нравственных основ и связей: если нет бога, нет никакого различия между добром и злом, добродетелью и пороком. Следовательно, это различие зависит только от бытия бога, и истина добродетели заключается не в ней самой, а вне её. Таким образом, существование добродетели связывается с бытием бога, а не с добродетельным настроением, не с убеждением во внутренней ценности и содержательности добродетели. Напротив, вера в бога, как в необходимое условие добродетели, есть вера в ничтожество добродетели самой по себе.

Замечательно то, что понятие эмпирического существования бога окончательно образовалось лишь в новейшее время, когда вообще эмпиризм и материализм достигли полного расцвета. Правда, и по первоначальному, наивному представлению религии бог был существом эмпирическим, находящимся где-то, но не на земле. Но это ещё не имело обнаженно прозаического значения; сила воображения отождествляла внешнего бога с душой человека. Сила воображения вообще есть истинное место пребывания отсутствующего, неосязаемого, но чувственного по существу бытия\dag\let\svthefootnote\thefootnote\let\thefootnote\relax\footnotetext{\dag<<Христос вознесся на небо\dots Это означает, что он не только находится там, но и пребывает на земле, он вознесся затем, чтобы исполнить все земное и быть повсеместно, а этого он не мог бы сделать, оставаясь на земле, ибо здесь его не могли бы увидеть все телесные глаза. Поэтому он воссел там, на небесах, чтобы всякий мог его видеть и он сам мог общаться со всеми>> (Лютер, ч. \rom{13}, стр. 643). Это значит: Христос, или бог, есть объект, есть существо, созданное силой воображения; в воображении он не ограничен местом, он налицо и объективирован для всех. Бог существует на небе, а потому и вездесущ; но это небо есть фантазия, плод воображения.}\let\thefootnote\svthefootnote. Только фантазия разрешает противоречие между чувственным и в то же время нечувственным бытием; только фантазия предохраняет от атеизма. В воображении такое бытие может совершать чувственные действия --- проявлять себя как силу; сила воображения приобщает к сущности чувственного бытия ещё и проявления его. Где бытие бога есть живая истина, дело воображения, там возникает вера в явления бога\ddag\let\svthefootnote\thefootnote\let\thefootnote\relax\footnotetext{\ddag<<Тебе нечего жаловаться, что ты взыскан менее, чем были взысканы Авраам или Исаак. Тебе также является бог\dots У тебя есть таинства крещения, причащения, когда под формулой, образом и видом хлеба и вина бог общается с тобой и становится слышимым, видимым и воспринимаемым сердцем. Он является тебе и в крещении и есть тот, который тебя крестит и говорит к тебе\dots Все сущее исполнено явлениями бога и беседами с ним>>. (Лютер, ч. \rom{2}, стр. 446; см. о том же, ч. \rom{19}, стр. 407).}\let\thefootnote\svthefootnote. Напротив, где огонь религиозного воображения угасает, где исчезают нераздельные с чувственным бытием осязаемые действия или явления, там бытие становится мертвым, противоречивым бытием, не способным отразить нападения атеизма.





Вера в бытие божие есть вера в особое бытие, отличное от бытия человека и природы. Особое бытие может проявляться лишь особым образом; следовательно, эта вера является истинной и живой лишь постольку, поскольку она верит в особые действия, непосредственные явления бога, в чудеса. Там, где вера в бога отождествляется с верой в мир, а не является особой верой, где всеобщая сущность мира овладевает всем человеком, там, естественно, исчезает также и вера в особые действия и явления бога. Вера в бога терпит крушение, разбивается о веру в мир, в естественные явления как явления единственно действительные. Как вера в чудеса становится здесь только верой в исторические, минувшие чудеса, так и бытие божие обращается здесь лишь в историческое, само по себе атеистическое представление.





\chapter{Противоречие в откровении божием}


Понятие бытия тесно связано с понятием откровения. Откровение есть самоудостоверение бытия, документальное свидетельство, что бог существует. Доказательства бытия божья от разума суть только субъективные доказательства, а откровение божие есть единственное истинное, объективное доказательство бытия божия. Откровение есть слово божие --- бог говорит к человеку, издает звуки, произносит слова, овладевающие сердцем человека и вселяющие в него радостную уверенность в действительном существовании бога. Слово есть евангелие жизни --- отличительный знак бытия и небытия. Вера в откровение есть высшая точка религиозного объективизма. Благодаря ей субъективная уверенность в бытии божием становится несомненным, внешним, историческим фактом. Бытие божие само по себе, как бытие, есть внешнее, эмпирическое бытие, но в то же время только мыслимое, представляемое и потому подверженное сомнению; --- отсюда утверждение, что все доказательства не дают достаточной уверенности. Это мыслимое, представляемое бытие как действительное бытие, как факт есть откровение. Бог открыл себя, показал себя. Кто же может ещё сомневаться в нем. Достоверность бытия заключается в достоверности откровения. Бог, только существующий и не открывающий себя, существующий только чрез меня и для меня, --- такой бог есть только отвлеченный, мыслимый, субъективный бог. Только такой бог, который сам дает мне познать себя, есть действительно существующий и бытие свое проявляющий объективный бог. Вера в откровение есть непосредственная уверенность религиозной души в существовании того, во что она верит, чего желает, что представляет. Религиозная душа не делает различия между субъективным и объективным --- она не сомневается; она обладает чувствами не для того, чтобы видеть другие предметы, а лишь затем, чтобы смотреть на свои представления как на внешние существа. Для религиозного чувства всякий теоретический предмет --- дело практики, дело совести --- факт. Факт есть то, что перестает быть предметом разума и становится делом совести; факт есть то, чего нельзя касаться и критиковать, не совершая святотатства*\let\svthefootnote\thefootnote\let\thefootnote\relax\footnotetext{*отрицание какого-нибудь факта не есть нечто невинное, по себе безразличное, но оно имеет дурное нравственное значение. В том обстоятельстве, что христианство обратило свой символ веры в чувственные, то есть в неоспоримые, неприкосновенные факты, то есть чувственными фактами поработило разум, мы видим истинное, конечное и первичное основание объяснения, почему и каким образом в христианстве, и не только в католическом, но и в протестантском формально и торжественно был высказан и установлен принцип, что ересь, то есть отрицание какого-нибудь вероучительного факта или представления, есть преступление, наказуемое светской властью. Чувственный факт в теории обратился на практике в чувственное насилие. В этом отношении христианство стоит гораздо ниже магометанства, по крайней мере Корана, которому неведомо преступление ереси.}\let\thefootnote\svthefootnote; факт есть то, во что надо верить nolens volens\footnote{волей-неволей (лат.).}; факт есть чувственное насилие, а не основание; факт подходит к разуму, как к корове седло. О вы, близорукие немецкие богословствующие философы, вы, забрасывающие нас фактами религиозного сознания, чтобы отуманить наш разум и сделать нас рабами вашего ребяческого суеверия, разве вы не видите, что эти факты столь же относительны, различны и субъективны, как и представления религии. Олимпийские боги также были некогда фактами, свидетельствовавшими о себе самих\dag\let\svthefootnote\thefootnote\let\thefootnote\relax\footnotetext{\dag<<Боги нередко проявляют свое присутствие>> (Цицерон, De nat. deorum. \rom{1}, \rom{2}). Сочинения Цицерона <<De nat. deor.>> и <<De divinatione>> интересны также особенно потому, что в них истина языческого вероучения обосновывается теми же аргументами, которые ещё и теперь приводятся богословами и положительными мыслителями в защиту христианского вероучения.}\let\thefootnote\svthefootnote\footnote{Фейербах ссылается на сочинения Цицерона <<О природе богов>> и <<О гадании>> (лат.).}. Разве нелепейшие рассказы язычников о чудесах не считались также когда-то фактами? Ангелы и демоны разве не были когда-то историческими лицами и не являлись людям? Разве в один прекрасный день не заговорила Валаамова ослица? И не далее, как в прошлом столетии даже просвещенные ученые считали говорящую ослицу таким же действительным чудом, как чудо воплощения и всякое другое чудо. О вы, великие, глубокомысленные философы, вам следует прежде всего изучить язык Валаамовой ослицы! Только невежде он кажется таким непонятным; что же касается вас, то я ручаюсь, что при более близком знакомстве вы узнаете в этом языке ваш родной язык и убедитесь, что эта ослица тысячу лет тому назад выболтала все глубочайшие тайны вашей умозрительной премудрости. Повторяю, господа, факт есть представление, в истинности которого нельзя сомневаться, потому что его предмет есть объект не теории, а чувства, желающего существования того, чего оно хочет, во что верит; факт есть то, что запрещено отрицать если и не внешним, то внутренним образом; факт есть всякая возможность, считающаяся действительностью, всякое представление, бывшее некогда фактом, выражавшее потребность своего времени, а теперь служащее непреодолимой границей для духа; факт есть любое желание, представляющееся осуществлённым, --- одним словом, факт есть все то, в чем мы не сомневаемся, не потому ли просто, что никто в нем не сомневается, что сомневаться в нем запрещено.





Религиозная душа, согласно её раскрытой выше природе, непосредственно уверена в том, что её произвольные движения и определения суть впечатления извне, суть проявления другого существа. Религиозная душа считает себя пассивным, а бога деятельным существом. Бог есть деятельность; но то, что вызывает его деятельность, что обращает его деятельность, прежде всего его всемогущество, в действительную деятельность, подлинный мотив, причину, есть не он сам, --- ему ничего не нужно, у него нет потребностей, --- а человек, религиозный субъект, или чувство. Но в то же время человек определяется богом, становится существом пассивным; он принимает от бога известные откровения, определенные доказательства его бытия. Таким образом, в откровении человек определяется собой, как основанием, определяющим бога, то есть откровение есть только самоопределение человека, причем посредником между человеком в качестве определяемого и человеком в качестве определяющего является другое существо, другой объект, бог. Посредством бога человек объединяет себя со своей собственной сущностью --- бог есть олицетворённый союз между сущностью, родом и индивидом, между человеческой природой и человеческим сознанием.

Вера в откровение всего отчетливее обнаруживает характерную иллюзию религиозного сознания. Общей предпосылкой этой веры служит следующее: человек сам по себе не может знать о боге ничего; все его знание носит суетный, земной, человеческий характер. Но бог есть существо сверхчеловеческое: бог познает лишь сам себя. Итак, мы не знаем о боге ничего, кроме того, что он нам открывает. Сообщённое нам богом содержание носит божественный, сверхчеловеческий, сверхъестественный характер. В откровении мы познаем бога благодаря ему самому; ведь откровение есть слово божие, сам о себе высказывающийся бог. Поэтому в вере в откровение человек отрицает себя, выходит за пределы своего существа; он противопоставляет откровение человеческому знанию и мнению; в нем заключается скрытое знание, полнота всех сверхчувственных тайн; здесь разум должен молчать. Но в то же время божественное откровение есть откровение, определяемое человеческой природой. Бог обращается не к животным или ангелам, а к людям; следовательно, ему свойственны человеческая речь и человеческие представления. Человек был объектом для бога ещё прежде, чем бог внешним образом вступил в общение с человеком; бог думает о человеке; он определяет себя его природой, его потребностями. Воля бога, разумеется, свободна; он может открывать и не открывать себя; но он не свободен в сфере мысли: он не может открывать человеку все, что только ему заблагорассудится; он может открывать ему только то, что соответствует человеку и его природе, только то, что он должен открывать, --- если только его откровение есть откровение для людей, а не для других существ. Следовательно, то, что бог мыслит ради человека, он мыслит под влиянием идеи человека и рефлексии о человеческой природе. Бог переселяется в человека и в нем мыслит о себе так, как это другое существо может и должно мыслить о нем. Он мыслит о себе не своими, а человеческими мыслительными способностями. План откровения божия зависит не от бога, а от мыслительной способности человека. Все, что из бога переходит в человека, переходит в человека из человека, заключенного в боге, то есть переходит из сущности человека в сознание человека, из рода в индивид. Итак, между божественным откровением и так называемым человеческим разумом или природой существует только иллюзорное различие --- содержание, божественного откровения имеет человеческое происхождение, так как оно произошло не от бога, как бога, а от бога, определяемого человеческим разумом, человеческими потребностями, то есть просто из человеческого разума, человеческих потребностей. Следовательно, в откровении человек удаляется от себя только затем, чтобы снова вернуться к себе окольным путем! Это служит новым блестящим доказательством того, что тайна теологии есть не что иное, как тайна антропологии\ddag\let\svthefootnote\thefootnote\let\thefootnote\relax\footnotetext{\ddagв чем же заключается существенное содержание откровения? В том, что Христос и есть бог, то есть что бог есть человеческое существо. Язычники обращались к богу со своими потребностями, но они сомневались, услышит ли бог молитвы людей, милосерден ли он, человечен ли он. Христиане же твердо уверены в любви бога к человеку: бог открыл себя как человек (см. об этом, например, Or. de vera dei invocat. Меланхтон, Decl., т. \rom{3}, и Лютер, например, ч. \rom{3}, стр. 538, 539). Стало быть, откровение божие есть уверенность человека в том, что бог есть человек, а человек есть бог. Уверенность есть факт.}\let\thefootnote\svthefootnote.



Впрочем само религиозное сознание в отношении прошедших времен признает божественное откровение преисполненным человеческим содержанием. Но религиозное сознание позднейшего времени уже не удовлетворяется Иеговой, который является человеком с головы до ног и безбоязненно выставляет свою человечность. То были только представления, в которых бог приспособлялся к силе разумения человека того времени, то есть это были исключительно человеческие представления. Но по отношению к своему теперешнему содержанию религиозное сознание этого уже не допускает, будучи поглощено этим содержанием. Тем не менее всякое откровение божие есть только откровение человеческой природы. В откровении объективируется человеку его скрытая природа. Человек находится под воздействием своей сущности, он определяется ею, словно другим существом; он получает из рук бога то, что ему его же собственная, ему неведомая сущность навязывает как необходимость при известных условиях времени.

Вера в откровение есть ребяческая вера и заслуживает уважения, пока она остается ребяческой. Но ребенок определяется извне. А откровение имеет целью достичь с божеской помощью того, чего человек не может достичь своими силами. В этом смысле откровение называется воспитанием человеческого рода. И это вполне верно; только не надо откровение отделять от человеческой природы. Поскольку человек побуждается изнутри облекать нравственные и философские учения в форму рассказов и басен, постольку он неизбежно считает откровением то; что дается изнутри. Баснописец преследует только одну цель --- сделать человека добрым и разумным; он преднамеренно выбирает форму басни как наиболее целесообразный, наглядный метод, причем его любовь к басне, его собственная внутренняя природа влечет его к этой форме поучения. То же бывает и с откровением, исходящим от определенного индивида. Он преследует известную цель, но в то же время сам живет в тех представлениях, посредством которых осуществляет свою цель. Человек непроизвольно силой воображения наглядно созерцает свою внутреннюю сущность, он ставит её вне себя. Эта наглядно созерцаемая, олицетворённая, действующая на него с непреодолимой силой воображения сущность человеческой природы как закон его мышления и действия --- есть бог.

В этом заключается благотворное нравственное влияние на человека веры в откровение; ведь собственная сущность влияет на некультурного, субъективного человека лишь тогда, когда он представляет себе эту сущность как другое, личное существо, как существо имеющее власть наказывать и взор, которого нельзя избегнуть.

Подобно тому как природа <<бессознательно создает вещи, которые кажутся произведёнными сознательно>>, так и откровение родит нравственные поступки, не вытекающие, однако, из нравственности, --- нравственные поступки, а не нравственные настроения. Моральные заповеди исполняются, но они остаются чуждыми внутреннему настроению души уже по одному тому, что представляются как заповеди какого-то во вне стоящего законодателя, и относятся к категории произвольных, полицейских законов. Я поступаю известным образом не потому, что считаю такой поступок справедливым и хорошим, а потому, что так велел бог. Всякое повеление бога должно считаться справедливым, независимо от его содержания*\let\svthefootnote\thefootnote\let\thefootnote\relax\footnotetext{*<<Что жестоко, когда люди делают это без приказания божия, то должны были делать евреи, так как они вели войну по приказанию бога, верховного владыки над жизнью и смертью>> (Clericus, Comm. in Mos. Num. гл. 31, 7). <<Многое совершил Самсон, чего нельзя было бы извинить, если бы он не был орудием бога, от которого зависят люди>> (он же, Comment. in Iudicum, гл. 14, 19; см. об этом также у Лютера, например, ч. \rom{1}, стр. 339; ч. \rom{16}, стр. 495).}\let\thefootnote\svthefootnote. Если эти заповеди соответствуют разуму, этике, то это есть счастье, хотя и случайное, для понятия откровения. Торжественные заповеди евреев были законами божественного откровения, но сами по себе являлись случайными, произвольными законами. Иудеи даже получали от Иеговы снисходительную заповедь, разрешившую им красть, правда, лишь в особом случае.



Вера в откровение не только портит моральный вкус и чувство, эстетику добродетели; она отравляет, даже убивает в человеке наиболее божественное чувство --- чувство правды, смысл истины. Божественное откровение есть определенное, временное откровение: божественное откровение произошло раз навсегда, тогда-то, в таком-то году, в известных пределах и притом не для человека всех времен и местностей, не для разума, рода, а для определенного, ограниченного индивида. Такое ограниченное пространством и временем откровение необходимо было сохранить в письменах, чтобы и другие в неискажённом виде могли пользоваться его благами. Следовательно, вера в откровение есть вместе с тем вера в письменное откровение, по крайней мере для позднейших поколений. А неизбежное следствие и результат веры, приписывающей исторической, временной конечной книге значение вечного, абсолютного, всеобщего закона, --- это суеверие и софистика.

Вера в писанное откровение только тогда ещё является действительной, истинной, нелицемерной и, стало быть, достойной уважения верой, когда люди верят, что каждое слово священного писания имеет серьезное, истинное, священное, божественное значение. Напротив, где существует различие между человеческим и божеским, относительно и абсолютно ценным, историческим и вечным, где не каждое слово священного писания признается безусловно истинным, там сомнение в божественности Библии как бы вносится в самую

Библию, и она утрачивает, по крайней мере косвенным образом, характер божественного откровения. Божественность характеризуется единством, безусловностью, цельностью, непосредственной уверенностью. Книга, заставляющая меня делать различие, критиковать и отделять божеское от человеческого, вечное от временного, перестает быть божественной, достоверной, истинной книгой и относится к разряду светских книг; ведь всякая светская книга заключает в себе одновременно божеское и человеческое, всеобщее, вечное и индивидуальное. Истинной, прекрасной или, вернее, божественной можно назвать только такую книгу, где доброе не чередуется с дурным, вечное --- с временным, где все без исключения вечно, истинно и прекрасно. Могу ли я довольствоваться таким откровением, где я должен прослушать сначала апостола Павла, потом Петра, затем Иакова, Иоанна, потом Матфея, Марка и Луку, чтобы найти такое место, где, наконец, моя ищущая бога душа может воскликнуть: <<Нашел! здесь говорит сам св. дух; вот что мне необходимо, вот что важно для всех времен и народов!>>. Староверы были совершенно правы, считая боговдохновенным не только каждое слово, но и каждую букву. Слово для мысли не безразлично; определенную мысль можно выразить только определенным словом. Одно слово, одна буква нередко изменяют весь смысл. Такая вера есть, разумеется, суеверие; но это суеверие есть не что иное, как истинная, правдивая, откровенная, не стыдящаяся своих последствий вера. Если бог считает волосы на голове человека, и ни один воробей не падает с кровли без его воли, захочет ли он предоставить на неразумие и произвол пишущего свое слово --- слово, от которого зависит вечное блаженство человека и которое он может предохранить от искажения, продиктовав свои мысли пишущему. <<Но если человек есть только орудие св. духа, то значит, человеческой свободы не существует!>>\dag\let\svthefootnote\thefootnote\let\thefootnote\relax\footnotetext{\dagвесьма справедливо возражали ещё янсенисты против иезуитов: <<Желание усматривать в писании человеческую слабость и естественный разум человека равносильно нраву критиковать Писание и отвергать в нем все неугодное, что зависит от слабости человеческой, а не от божественного ума>>.}\let\thefootnote\svthefootnote Какой жалкий вывод! Неужели человеческая свобода дороже божественной истины? Или человеческая свобода состоит лишь в искажении божественной истины?



Вера в определенное историческое откровение, как в абсолютную истину, необходимо связана с суеверием и софистикой. Библия противоречит морали, противоречит разуму, противоречит себе бесчисленное число раз; но она есть слово божие, вечная истина, а <<истина не может и не должна противоречить себе>>\ddag\let\svthefootnote\thefootnote\let\thefootnote\relax\footnotetext{\ddag<<В св. писании не следует предполагать противоречий>> (Петр Ломб., Lib. \rom{2}, dist. II, гл. I). Подобные же мысли встречаются у отцов церкви и у реформаторов, так, например, у Лютера. Следует ещё заметить, что ареной для софистики католические иезуиты избрали главным образом мораль, а протестантские иезуиты, не составляющие, насколько мне известно, формально организованной корпорации, обратились преимущественно к Библии и экзегетике.}\let\thefootnote\svthefootnote. Как же верующему в откровение разрешить противоречие между идеей откровения как божественной, гармонической истины и якобы действительным откровением? Для этого ему приходится прибегать к самообману, к нелепым отговоркам, к самым дурным, живым софизмам. Христианская софистика есть продукт христианской веры, в особенности веры в Библию как божественное откровение.


Истина, абсолютная истина дана объективно в Библии, субъективно --- в вере, так как к тому, что говорит сам бог, я могу относиться только с верой, преданностью и послушанием. Рассудок, разум играет здесь только формальную, подчиненную роль; он занимает ложное, противоречащее его сущности положение. Разум сам по себе является здесь равнодушным к истине, к различению между правдой и ложью; он не имеет критерия в себе; что стоит в откровении, то и есть истина, даже если оно прямо противоречит рассудку; он отдан во власть случайностям самой дурной эмпирии. Я должен верить во все, что только нахожу в божественном откровении, и мой рассудок в случае надобности должен защищать его; рассудок есть Canis Domini\footnote{пес Господен (лат.).}, он должен принимать за истину все без различия --- различение было бы сомнением, было бы преступлением. Следовательно, ему остается только случайное, безразличное, то есть лживое, софистическое, хитрое мышление --– мышление, опирающееся исключительно на всякие необоснованные, двусмысленные определения, на всякие постыдные уловки и увертки. Но, по мере того как человек перестает считаться с откровением, по мере того как рассудок становится самостоятельнее, противоречие между рассудком и верой в откровение выступает все резче. В наше время верующий может защищать святость и божественность откровения только в сознательном противоречии с самим собой, с истиной, с рассудком, только ценой наглого произвола и бесстыдной лжи, только ценой хулы против духа святого.





\chapter{Противоречие в существе божием вообще}


Понятие бога есть высший принцип, центральный пункт христианской софистики. Бог есть человеческое существо, и в то же время он должен быть другим, сверхчеловеческим, существом. Бог есть всеобщее, чистое существо, идея безусловного существа, и в то же время он должен быть личным, индивидуальным существом. Или: <<Бог есть лицо, и в то же время он должен быть богом, всеобщим, то есть не личным существом. Бог существует; его существование достоверно, более достоверно, чем наше собственное; ему присуще особое, отличное от нас и вещей, то есть индивидуальное, бытие, но при этом его бытие должно быть духовным, то есть не заключать в себе осязаемых особых свойств. В слове ,,должен`` всегда отрицается то, что утверждается в слове есть>>. Основное понятие является противоречием, прикрытым софизмами. Бог, о нас не пекущийся, не внемлющий нашим молитвам, нас не замечающий, нас не любящий, не есть бог; следовательно, существенным свойством бога является человечность; но, с другой стороны, говорят: бог, не существующий для себя, вне человека, над человеком, как другое существо, есть призрак; значит, существенное свойство бога есть его нечеловечность, сверхчеловечность. Бог, не похожий на нас, не обладающий, подобно субстанции Спинозы, ни сознанием, ни разумением, то есть лишенный личного рассудка, личного сознания, не есть бог. Главное условие божества есть существенное тождество с нами; понятие божества ставится в зависимость от понятия личности, сознания, как наивысшего, что мы можем мыслить. Но наряду с этим бог, существенно не отличающийся от нас, не признается богом.

Характерной чертой религии является непосредственное, непроизвольное, бессознательное созерцание человеческой сущности в качестве другого существа. Но как только это объективно созерцаемое существо становится предметом рефлексии, богословия, оно превращается в неисчерпаемый источник лжи, обмана, заблуждений, противоречий и софизмов.

Особенно характерную уловку и выгоду христианской софистики составляет непознаваемость и непостижимость существа божия. Но тайна этой непостижимости, как мы сейчас увидим, заключается в том, что определенное, естественное качество обращается в неопределенное и неестественное, и тем самым создается видимость, иллюзия, будто существо божье отличается от человеческого и поэтому является непостижимым.

В первоначальном смысле религии непостижимость бога имеет лишь значение приподнятого выражения. При виде какого-нибудь необычайного явления мы нередко восклицаем: это непостижимо, это выше моего понимания! хотя позднее, когда мы приходим в себя, мы признаем, что предмет нашего удивления вполне понятен. Религиозная непостижимость не есть бездушное положение, обусловленное ограниченностью рассудка; это --- просто патетическое выражение впечатления, произведённого фантазией на чувство. Фантазия есть основной орган и сущность религии. В первоначальном смысле религии между богом и человеком существовало, с одной стороны, лишь различие в отношении бытия, поскольку бог противополагается человеку в качестве самостоятельного существа, а с другой стороны --- различие количественное, то есть различие по объему фантазии, так как различие фантазии бывает только количественное. Бесконечность бога в религии есть количественная бесконечность; бог обладает всем, чем обладает человек, но в бесконечно большем масштабе. Сущность бога есть объективированная сущность фантазии*\let\svthefootnote\thefootnote\let\thefootnote\relax\footnotetext{*это особенно обнаруживается, между прочим, в превосходной степени и в приставке <<наи>>, которыми обозначались божественные предикаты и которые прежде, как, например, у неоплатоников, --- этих христиан среди языческих философов, --- играли в богословии главную роль.}\let\thefootnote\svthefootnote. Бог есть чувственное существо, но свободное от границ чувственности --- неограниченное чувственное существо. А что такое фантазия? Безграничная, неограниченная чувственность. Бог есть вечное бытие, то есть непрестанно сущее, бытие во все времена; бог есть бытие вездесущее, то есть не ограниченное местом; бог есть существо всеведущее, то есть его объектом является все единичное, все чувственное, без различия, без ограничения местом и временем.



Вечность и вездесущие суть чувственные свойства, так как они не отрицают бытия во времени и пространстве, а отрицают только бытие, ограниченное определенным местом и временем. Всеведение есть также чувственное качество, чувственное знание. Религия, не стесняясь, приписывает богу внешние чувства; бог видит и слышит все. Но божественное всеведение есть чувственное знание, лишенное главного свойства, существенной определенности действительного чувственного знания. Мои чувства позволяют мне представлять себе чувственные предметы только отдельно друг от друга и в последовательном порядке, а бог представляет себе все чувственное сразу, все пространственное --- вне пространства, все временное --- вне времени, все чувственное --- внечувственным образом\dag\let\svthefootnote\thefootnote\let\thefootnote\relax\footnotetext{\dag<<Следовательно, бог знает, как велико количество блох, козявок, комаров и рыб; он знает, сколько их родится и умирает, но знает это не в раздельности и по порядку, но все одновременно, сразу>> (Петр Ломб., lib. \rom{1}, dist. 39, гл. 3).}\let\thefootnote\svthefootnote. Другими словами: я фантазией расширяю свой чувственный горизонт, я объединяю все вещи, не исключая даже и пространственно отсутствующих, в смутном представлении всеобщности и олицетворяю это возвышающее меня над ограниченной чувственной точкой зрения и благотворно действующее на меня представление как божественную сущность. Мое знание, связанное местом и чувственным опытом, кажется мне ограниченным, и я уничтожаю это ограничение при помощи фантазии, дающей полный простор моим чувствам. Это отрицание при помощи фантазии есть утверждение всеведения как божественной силы и сущности. Тем не менее всеведение отличается от моего знания только количественным образом; качество знания одинаково и там и здесь. Я не мог бы на самом деле приписывать всеведения объекту или существу, вне меня находящемуся, если бы оно существенно отличалось от моего знания и не отвечало моим представлениям, моему воображению. Божественное всеведение и мое знание носят одинаково чувственный характер, как в отношении объекта, так и содержания. Фантазия устраняет границу только количества, а не качества. Наше знание ограничено; это значит: мы знаем лишь кое-что и немногое, а не все.



Благотворное влияние религии основано на этом расширении чувственного сознания. В религии человек чувствует себя как бы под открытым небом, sub divo\footnote{под богом (лат.).}; в чувственном сознании он замкнут в своем узком, ограниченном помещении. Религия по существу, по своему источнику (а только источник религии свят, истинен, чист и прекрасен) относится только к непосредственно чувственному, неоформившемуся сознанию, она есть устранение чувственных границ. Замкнутые в себе, ограниченные люди и народы сохраняют религию в её первоначальном смысле, поскольку они сами остаются ещё при начале или у источника религии. Чем ограниченнее кругозор человека, чем меньше он знаком с историей, природой и философией, тем искреннее его привязанность к своей религии.

Поэтому религиозный человек не чувствует в себе потребности в образовании. Почему у евреев не было ни искусства, ни науки, как у греков? Потому, что они не чувствовали потребности в них. А чем объяснить отсутствие у них этой потребности? Иегова возмещал им эту потребность. Божественное всеведение возвышает человека над границами его знания\ddag\let\svthefootnote\thefootnote\let\thefootnote\relax\footnotetext{\ddag<<Кто сознал всеведущего, тот не может не быть всезнающим>> (Liber meditat, гл. 26 --- Псевдо-Августин).}\let\thefootnote\svthefootnote; божественная вездесущность --- над границами пространства; божественная вечность --- над границами его эпохи. Религиозный человек чувствует себя счастливым благодаря своей фантазии; он потенциально обладает всем in nuce\footnote{букв. в орехе, т.е. в зародыше, в основе (лат.).}: его багаж всегда при нем. Иегова сопровождает меня повсюду; мне не нужно преодолевать свою ограниченность; мой бог есть совокупность всех сокровищ и ценностей, всего, что нужно знать и помнить. Образование зависит от внешних условий, имеет разнообразные потребности, так как оно преодолевает границы чувственного сознания и жизни опять-таки чувственной, действительной деятельностью, а не волшебной силой религиозной фантазии. Поэтому, как сказано выше, христианская религия по существу не заключает в себе принципа культуры, образования, так как она преодолевает границы и трудности земной жизни только фантазией, богом и на небесах. Бог есть все, чего жаждет и требует сердце, он совмещает в себе все вещи, все блага. <<Ищешь ли ты любви или верности, истины, утешения или постоянной помощи, --- все это в нем содержится беспредельно и безмерно. Жаждешь ли ты красоты, --- он есть совершенная красота. Хочешь ли ты богатства, --- он богаче всех. Желаешь ли ты могущества, --- он всемогущ. Все, чего только пожелает твое сердце, все это во множестве ты находишь в нем, в том едином благе, каким является бог>>*\let\svthefootnote\thefootnote\let\thefootnote\relax\footnotetext{*J. Tauler, I, c., p. 312.}\let\thefootnote\svthefootnote. Но тот, для кого все исчерпывается богом, кто уже наслаждается небесным блаженством в своем воображении, тот не чувствует нищеты и бедности, которые вызывают стремление к культуре. Единая цель культуры --- осуществить земное блаженство, а небесное блаженство достигается только религиозной деятельностью.



Первоначально исключительно количественное различие между божеским и человеческим существом, благодаря рефлексии, превращается в различие качественное, вследствие чего душевный аффект, непосредственное выражение восторга, восхищения, впечатление, произведённое на душу фантазией, фиксируется теперь как объективное свойство, как действительная непостижимость. В этом отношении богословы обыкновенно ссылаются на то, что мы можем только знать то, что бог есть, а не как это возможно. Так, например, для нас вполне ясно, достоверно и несомненно, что бог есть творец, что он создал мир из ничего без наличия материи; но как это возможно, этого наш ограниченный разум постичь, разумеется, не может. Это значит: понятие рода ясно и достоверно, а понятие вида неясно и недостоверно.

Понятие деятельности, творчества, созидания само но себе есть понятие божественное, поэтому оно бесспорно применяется и к богу. В деятельности человек чувствует себя свободным, неограниченным, счастливым, а в состоянии страдания --- ограниченным, угнетенным, несчастным. Деятельность есть положительное ощущение. Положительным вообще является все то, что вызывает в человеке радость, поэтому бог, как уже упоминалось выше, есть понятие чистой, беспредельной радости. Нам удается только то, что мы делаем охотно; радость все преодолевает. Но деятельность радостна только тогда, когда она соответствует нашему существу, когда она не ограничивает, не связывает нас. Наиболее радостной блаженной деятельностью является созидающая деятельность. Например, чтение --- приятно; но чтение есть пассивное занятие, а творить достойное прочтения --- ещё приятнее. В данном случае давать --- приятнее, чем принимать. Итак, родовое понятие творческой деятельности прилагается и к богу, то есть доподлинно созерцается, объективируется как божественная деятельность и сущность. Однако сохраняется лишь основное, существенно человеческое основное определение: творение вне себя, и всякое специальное определение, способ деятельности устраняется. Бог создал но ту или другую вещь, не нечто частное, как человек, а все; его деятельность безоговорочно универсальна, безгранична. Поэтому само собой разумеется, способ божественного творчества непостижим, потому что деятельность бога не есть способ деятельности, вследствие чего и сам вопрос о способе творения есть вопрос неуместный, опровергаемый основным понятием безграничной деятельности. Всякая особая деятельность особым образом производит свое действие, потому что она сама есть определенный способ деятельности; и здесь необходимо возникает вопрос: как создала она это? Но на вопрос, каким образом бог сотворил мир, можно ответить только отрицательно, потому что деятельность, создавшая мир, отрицает всякую определенную деятельность, всякий способ деятельности, связанный с определенным содержанием или материей. В этом вопросе между субъектом --- созидающей деятельностью, и объектом --- продуктом деятельности, незаконно вклинивается лишний, ненужный посредник: понятие своеобразия. Деятельность относится только к коллективу: ко всему, к миру. Бог создал все, а не нечто, сотворил неопределенное целое --- все, как обнимает его фантазия, а не что-либо определенное, своеобразное, служащее в своей отдельности предметом чувств, а в своей совокупности как вселенная --- предметом разума. Всякое нечто возникает естественным путем, носит определенный характер и, как таковое, имеет определенное основание, определенную причину. Бриллиант создан не богом, а углеродом; он обязан своим происхождением не богу, а только соединению определенных кислот с определенным радикалом. Бог создал все в целом и без различия.

Конечно, по представлению религии, бог создал все в отдельности, так как отдельное включено в понятие всего, но только косвенным образом; ведь он не творил отдельное и определенное отдельным и определенным образом, иначе он был бы существом определенным. Непостижимо лишь одно: как из этой всеобщей, неопределенной деятельности явилось своеобразное, определенное; но это непонятно только потому, что я здесь тайком протаскиваю своеобразное, предмет чувственного естественного созерцания, что я божественной деятельности подсовываю другой, не подобающий ей объект. Религия не знает физического воззрения на мир. Она не интересуется естественным объяснением, основанным на понятии происхождения. Но происхождение есть теоретическое, натурфилософское понятие. Языческие философы занимались происхождением вещей. А христианско-религиозное сознание отвергало это понятие, как языческое, безбожное, и заменило его практическим или субъективно– человеческим понятием творения, которое есть не что иное, как запрет объяснять себе вещи естественным путем, интердикт на всякую физику и натурфилософию. Религиозное сознание непосредственно соединяет мир с богом; оно выводит все из бога, так как для него ничего не существует в своей отдельности и действительности как объект разума. Все исходит от бога --- этого достаточно, это вполне удовлетворяет религиозное сознание. Вопрос, как бог создал мир, есть косвенное сомнение в том, что мир создан богом. Этот вопрос привел человека к атеизму, материализму и натурализму. Кто так ставит вопрос, для того мир есть уж предмет теории, физики, то есть предмет во всей действительности, во всей определенности его содержания. Но это содержание противоречит представлению неопределенной, нематериальной, невещественной деятельности. А это противоречие приводит к отрицанию основного представления.

Творчество всемогущества только там уместно, только там истинно, где все события и явления вытекают из бога. Но оно, как уже сказано, становится мифом далекого прошлого, как только на сцену является физика, как только человек определенные основания, причину явлений делает предметом своего исследования. религиозное сознание не видит в творении ничего непостижимого, то есть неудовлетворительного; оно становится непостижимым разве только в моменты атеистических сомнений, когда человек отворачивается от бога и обращает взор свой к вещам; оно непостижимо также для рефлексии, для богословия, которое одним глазом заглядывает в небо, а другим --- в землю. Следствие обусловливается причиной. Флейта издает звуки только флейты, а не фагота или трубы. Если ты никогда не видел и не слышал других духовых инструментов, кроме флейты, и услышишь впервые звуки фагота, тебе покажется непостижимым, как может флейта издавать подобный звук. То же и здесь. Только сравнение это несколько неудачно, потому что сама флейта есть все-таки определенный инструмент. Но представь себе, если можешь, просто универсальный инструмент, соединяющий в себе все инструменты без того, что бы быть определенным инструментом, и ты поймешь, как нелепо требовать определенного звука, свойственного только определенному инструменту, от инструмента, от которого ты же сам отнял характерные свойства всех определенных инструментов.

Непостижимость имеет целью отделить божественную деятельность от человеческой, устранить их сходство, подобие, или, вернее, существенное единство, сделать божественную деятельность иной по существу. Это различие между божественной и человеческой деятельностью есть ничто. Бог творит --- он творит нечто вне себя, подобно человеку. Созидание есть истинно и глубокочеловеческое понятие. Природа родит, производит, человек созидает. Созидание есть такая деятельность, от которой я могу отказаться, преднамеренная, сознательная, внешняя деятельность, --- деятельность, в которой не участвует непосредственно мое собственное внутреннее существо, в которой я не играю страдательной, пассивной роли. Для меня не безразлична только такая деятельность, которая, подобно духовному творчеству, тождественна с моим существом, которая является для меня необходимой, внутренней потребностью и поэтому всецело овладевает моей душой, действует на меня патологически. Духовные произведения не создаются --- созидание здесь только внешняя деятельность --- они возникают в нас\dag\let\svthefootnote\thefootnote\let\thefootnote\relax\footnotetext{\dagпоэтому в новейшее время деятельность гения действительно стала приравниваться к деятельности творца мира, и чрез это были открыты новые горизонты для религиозно-философского воображения. Интересным предметом критики был бы тот способ, каким издавна религиозное умозрение пытается примирить свободу, или, вернее, произвол, то есть отсутствие необходимости творения, противоречащее разуму, с необходимостью творения, то есть с разумом. Но эта критика лежит вне нашей задачи. Мы критикуем умозрение лишь с помощью критики религии и потому ограничиваемся только первоначальным и основным. Критика умозрения является лишь выводом.}\let\thefootnote\svthefootnote. Созидание есть деятельность безразличная, то есть свободная и произвольная. Итак, бог тождествен с человеком и ничем не отличается от него, пока он созидает, но в его созидании религия особенно подчеркивает то обстоятельство, что созидание бога свободно и произвольно, что он созидает, как ему угодно! Богу было угодно сотворить мир. Так обожествляет здесь человек довольство своим собственным необоснованным, беспричинным произволом. Благодаря представлению произвола глубоко человеческое определение божественной деятельности становится грубо человеческим --- бог из зеркала человеческого существа превращается в зеркало человеческого тщеславия и самодовольства.



Но затем гармония вдруг становится дисгармонией; дотоле единый в себе человек раздваивается; --- бог делает мир из ничего; он творит; делание из ничего есть творение --- в этом заключается различие. Существенное определение есть человеческое, но оно утрачивает свой человеческий характер благодаря рефлексии, уничтожающей определенность этого основного определения. Вместе с этим уничтожением исчезает понятие, смысл; остается только пустое, бессодержательное представление, ведь мыслимость, представимость уже исчерпана, то есть различие между божественным и человеческим определением есть на самом деле ничто, nihil negativum\footnote{отрицательное ничто (лат.).} рассудка. Наивным самопризнанием этого является ничто как объект.

Бог есть любовь, но любовь не человеческая; рассудок, но не человеческий, нет! --- а существенно иной. Но в чем же состоит это различие? Я не могу мыслить или представить себе другого рассудка, кроме того определенного рассудка, который проявляется в нас. Я не могу разрезать рассудок на две или четыре части, чтобы у меня получилось несколько рассудков; я могу себе представлять только один и тот же рассудок. Я могу себе представить рассудок как таковой, то есть свободным от случайных ограничений; но я не опускаю здесь его существенной определенности. А религиозная рефлексия уничтожает именно ту определенность, которая делает предмет тем, что он есть. Только то, в чем божественный рассудок тождествен с человеческим, только это есть нечто, есть разум, есть подлинное понятие; а то, что делает его другим, даже существенно другим, есть объективно ничто, а субъективно --- простое воображение.

Другим характерным примером является непостижимая тайна рождения сына божьего. Разумеется, рождение бога отличается от обыкновенного естественного рождения; это рождение сверхъестественное, то есть на самом деле только иллюзорное, воображаемое; ему недостает той определенности, которая делает рождение рождением, так как ему недостает момента полового различия; это рождение противоречит природе и разуму, и именно в силу этого противоречия, в силу того, что здесь не высказывается ничего определенного, что здесь нет материала для мысли, рождение бога дает широкий простор фантазии и производит на душу впечатление чего-то глубокого. Бог есть отец и сын --- бог, подумай только! Бог! Сознание тождества с богом приводит человека в восторг; его мыслью овладевает аффект; далекое оказывается близким, чужое --- своим, высокое --- глубоким, сверхъестественное --- естественным, то есть сверхъестественное предполагается естественным, божественное --- человеческим; отрицается различие между божеским и человеческим. Но это тождество божеского и человеческого тотчас же отрицается. Все, что в боге есть общего с человеком, должно иметь в боге совершенно иное значение, чем в человеке; таким образом свое собственное опять становится чужим, знакомое --- неведомым, близкое --- далеким. Бог не рождает, как природа; бог не есть ни отец, ни сын в человеческом смысле. А в каком же смысле? Это и есть непостижимая, неизреченная глубина божественного рождения. Таким образом, религия или, вернее, теология снова полагает в боге все естественное, человеческое, что отрицалось ею, но теперь уже в противоречии с сущностью человека и природы, ведь все это должно быть в боге совершенно иным; но на самом деле ничего иного нет.

Во всех других определениях божественной сущности это отсутствие различия представляется скрытым; в творении, напротив, оно обращается в открытое, ясно выраженное, объективное ничто --- отсюда официальное, заведомое ничто теологии в отличие от антропологии.

Основное определение, посредством которого человек делает свою собственную обособленную сущность другим, непостижимым существом, есть понятие, представление самостоятельности индивидуальности или, выражаясь отвлеченно, личности. Понятие бытия осуществляется впервые в понятии откровения, а понятие откровения как самоутверждения бога --- в понятии личности. Бог есть личное существо --- вот то волшебное заклятие, которое мгновенно превращает представляемое в действительность, субъективное в объективное. Все предикаты, все определения божественного существа носят глубоко человеческий характер; но как определения личного, то есть другого, отличного от человека и независимо существующего бога, они кажутся непосредственно и действительно иными определениями, сохраняя при этом существенное единство. Отсюда возникает для рефлексии понятие так называемых антропоморфизмов. Антропоморфизмы суть виды сходства между богом и человеком. Определения божественного и человеческого существа хотя не одни и те же, но они сходны между собой.

Поэтому личность является противоядием от пантеизма; другими словами, представление личности позволяет религиозной рефлексии отрешиться от тождества божественного и человеческого существа. Грубое, но все же характерное определение пантеизма гласит: человек есть истечение или частица божественного существа; а определение религиозное гласит: человек есть образ божий, или существо, родственное богу; ведь религия приписывает человеку не естественное, а божественное происхождение, божественное начало. Но <<родство>> есть неопределенное, уклончивое выражение. Родство бывает близкое и отдаленное. Какое же родство подразумевается в данном случае? Для отношения человека к богу в смысле религии подходит единственное отношение родства --- самое близкое, интимное, священное, какое только можно себе представить, --- отношение ребенка к отцу. Таким образом, бог и человек отличаются тем, что бог есть отец человека, а человек --- сын или чадо бога. Здесь самостоятельность бога и зависимость человека останавливаются одновременно, при этом непосредственно, как объект чувства; тогда как в пантеизме часть столь же самостоятельна, как и целое, ибо целое представляется составленным из частей. Но и это различие есть не более как иллюзия. Отец не бывает отцом без сына; они оба вместе составляют одно общее существо. В любви человек отрекается от своей самостоятельности, становится частью. Это самоунижение и самоумаление искупается только тем, что и другой также становится частью, и оба подчиняются некоторой высшей силе --- силе семейного духа и любви. В данном случае устанавливается то же отношение между богом и человеком, как и в пантеизме; только здесь это отношение носит личный, патриархальный, а там безличный, всеобщий характер, причем в пантеизме логически, поэтому определенно, прямо выражается то, что в религии благодаря фантазии остается в стороне. Взаимная связанность, или, вернее, тождество бога и человека, в религии маскируется тем, что оба представляются как личности, или индивиды, но бог независимо от своего отцовства --- ещё и как существо самостоятельное, хотя и эта самостоятельность не более, как иллюзия, ибо кто, подобно богу религии, является отцом в силу глубокого чувства, тот живет и существует лишь своим сыном.

Взаимная тесная зависимость между богом как отцом и человеком как сыном нисколько не ослабляется разграничением, будто люди суть только усыновлённые чада божий, а родным сыном является лишь Христос и что поэтому бог находится в существенной зависимости только от Христа, как единородного сына, а не от человека. Ведь это различие только теологическое, то есть иллюзорное. Бог усыновляет только людей, а не животных. Причина усыновления заключается в человеческой природе. Человек, усыновлённый божией благодатью, сознает свою божественную природу и достоинство. Кроме того, единородный сын божий есть не что иное, как идея человечества, предвосхищённый человек, в боге скрывающийся от себя самого и от мира. Логос есть тайный, скрытый человек; а человек есть открытый, ясно выраженный логос. Логос есть только avant-propos\footnote{пролог, предисловие (фр.).} человека. Все, что сказано о логосе, относится и к существу человека\ddag\let\svthefootnote\thefootnote\let\thefootnote\relax\footnotetext{\ddag<<Великое единение, существующее между Христом и отцом, возможно и для меня, только бы я смог отрешиться от того, что исходит от того или иного определенного человека, и приобщился бы к общечеловеческому, ибо что даровал господь своему единородному сыну, он даровал также и мне>> (<<Predigten etzlicher Lehrer vor und zu Tauleri Zeiten>>. Hamburg, 1621, стр. 14). <<Между единородным сыном и душою нет различия>> (там же, стр. 68).}\let\thefootnote\svthefootnote. Но между богом и его единородным сыном нет никакого существенного различия --- кто знает сына, знает и отца, --- следовательно, между богом и человеком также нет различия.



То же можно сказать и о подобии божьем. Здесь образ --- не мертвое, а живое существо. Человек есть образ бога, значит, человек есть существо, подобное богу. Сходство между живыми существами сводится к естественному родству; человек подобен богу, потому что бог --- его отец. Сходство есть наглядное родство; по первому мы постоянно судим о втором.

Но сходство есть такое же обманчивое, иллюзорное, неопределенное представление, как и родство. Естественное единство устраняется только представлением личности. Подобие есть единство, не желающее называться единством и скрывающееся под смутным обликом посредника в тумане фантазии. Как только туман рассеивается, мы находим обнаженное единство. Чем больше сходства между существами, тем меньше между ними различия; если я знаю одного, значит, я знаю и другого. Есть разные степени сходства. Это относится и к подобию между богом и человеком. Добрый, благочестивый человек уподобляется богу в большей степени, чем тот, чье сходство с богом исчерпывается только его человеческой природой вообще. Поэтому можно допустить высшую степень сходства, хотя она будет достигнута только в будущей жизни. Но все, чем человек сделается впоследствии, до некоторой степени принадлежит ему и теперь, по крайней мере в возможности. Высшая степень подобия состоит в том, что два индивида или существа говорят и выражают одно и то же, так что все различие между ними ограничивается тем, что это --- два индивида. Существенные качества, благодаря которым мы различаем одну вещь от другой, одинаково присущи им обеим. Поэтому их можно различать не посредством мысли и разума, которым здесь не на что опереться, а лишь посредством чувственного представления или созерцания. Если бы мои глаза не говорили мне, что это --- два существа, действительно различные в смысле бытия, то мой разум принял бы их за одно и то же существо. Поэтому и мои глаза также смешивают их друг с другом. Вообще смешивать можно только то, что является различным не для разума, а для чувства, или, вернее, --- различное не в смысле сущности, а в смысле бытия. Поэтому вполне подобные друг другу лица имеют исключительную прелесть как для самих себя, так и для фантазии. Подобие дает повод ко всевозможным мистификациям и иллюзиям, потому что мой глаз смеется над моим разумом, для которого понятие самостоятельного бытия всегда связывается с понятием определенного различия.

Религия есть свет духовный, преломляющийся в среде фантазии и чувства и показывающий одно и то же существо в двойном виде. Подобие есть единство разума, которое в сфере действительности прерывается благодаря непосредственно чувственному представлению, а в сфере религии --- благодаря воображению. Другими словами, оно есть тождество разума, раздвоенное представлением индивидуальности, или личности. Я не могу установить действительного различия между отцом и сыном, первообразом и отображением, богом и человеком, если я отрешаюсь от представления личности. Подобие есть единство, утверждаемое разумом, чувством истины, и отрицаемое воображением; оно есть единство, кажущееся различием, призрачное представление, не говорящее прямо ни да, ни нет.





\chapter{Противоречие в умозрительном учении о боге}


Итак, личность бога есть средство, позволяющее человеку превращать определения и представления своего собственного существа в определения и представления другого существа, существа вне его. Личность бога есть не что иное, как отделенная, объективированная личность человека.

На этом процессе самоотчуждения основано умозрительное учение Гегеля, которое сознание человека о боге делает самосознанием бога. Бог мыслится, сознается нами. Согласно умозрению, это наше мышление о боге есть мышление бога о себе самом; умозрение соединяет обе стороны, разъединяемые религией. В этом случае умозрение рассуждает глубже, чем религия, так как бога мы мыслим иначе, чем другой, внешний предмет. Бог есть внутренняя, духовная сущность; а мышление, сознание, есть внутренний, духовный акт; поэтому мышление о боге есть утверждение того, что есть бог, --- акт, в котором проявляется сущность бога. Что бог мыслится, сознается, это для него существенно и необходимо; а что мыслится данное дерево, это для дерева --- случайно и несущественно. Но возможно ли, чтобы эта потребность была только субъективной и не выражала одновременно потребности объективной? Возможно ли, чтобы бог, --- если он должен существовать для нас, быть для нас объектом, объектом нашего мышления, --- был, как пень, равнодушен к тому, мыслим, сознаем мы его, или нет? Нет! Это невозможно. Мы принуждены обратить мыслимость бога в мышление бога о самом себе.

Религиозный объективизм допускает две пассивные формы, два рода мыслимости. С одной стороны, мы мыслим бога, с другой --- он сам мыслит себя. Бог мыслит себя независимо от того, мыслим ли мы его. Его самосознание независимо, отлично от нашего сознания. Это так и должно быть, если мы представляем бога действительной личностью; ведь действительная, человеческая личность мыслит себя и мыслится другими; мое мышление о ней имеет для нее безразличное, внешнее значение. Это высшая точка религиозного антропопатизма. Чтобы освободить бога от всего человеческого и сделать его самостоятельным, его делают настоящей, действительной личностью, включая в него мышление и исключая из него мыслимость, относимую к другому существу. Это равнодушие к нам, к нашему мышлению свидетельствует о его самостоятельном, то есть в нашем, личном бытии. Правда, религия обращает мыслимость бога в его мышление о себе самом; но так как этот процесс совершается вне ее сознания, причем бог непосредственно предполагается существом, себе довлеющим, личным, то и в сознании религии лишь отражается это безразличие обеих сторон.

Впрочем, религия ни в какой мере не ограничивается этим безразличием обеих сторон. Бог творит, чтобы открыть себя, --- творение есть откровение божие. Но бог существует не для камней, растений или животных, а ради человека, а человек --- для бога. Бог прославляет себя в человеке --- человек есть гордость бога. Бог познает себя помимо человека; но пока у него нет второго Я, он остается только возможной, только воображаемой личностью. Бог познает себя только тогда, когда установлено существо, отличное от бога, небожественное: чтобы познать блаженство своей божественности, чтобы знать, что значит бытие бога, он должен усвоить, что значит не быть богом. Только с установлением другого существа, мира, бог устанавливает себя как бог. Всемогущ ли бог без творения? Нет, всемогущество осуществляется, утверждается впервые в творении. Что значит сила, себя не проявляющая, свет, ничего не освещающий, мудрость, не знающая ничего, то есть никакой действительности? К чем всемогущество, к чему все другие божественные определения, если нет человека? Человек ничто без бога, но и бог ничто без человека*\let\svthefootnote\thefootnote\let\thefootnote\relax\footnotetext{*<<Бог наш не может обойтись без нас, как и мы без него>> (Predigten etzlicher Lehrer etc., S. 16); см. об этом также у Штрауса (Christl. Glaubensl., Bd. \rom{1}, par. 47 и немецкую Теологию, гл. 49.}\let\thefootnote\svthefootnote; ведь впервые в человеке бог объективируется, становится богом. Впервые различные качества человека устанавливают различие, основание действительных состояний в боге. Физические свойства человека делают бога физическим существом --- богом-отцом, творцом природы, то есть олицетворенной, очеловеченной сущностью природы\dag\let\svthefootnote\thefootnote\let\thefootnote\relax\footnotetext{\dag<<Эту временную, преходящую жизнь в этом мире это есть естественную жизнь. Мы обрели от бога, всемогущего творца неба и земли. Но вечную, нескончаемую жизнь мы обретаем через страдание и воскресение господа нашего Иисуса Христа\dots Иисус Христос --- владыка будущей жизни>> (Лютер, ч. \rom{16}, стр. 459).}\let\thefootnote\svthefootnote; интеллектуальные свойства человека делают бога интеллектуальным, а нравственные --- нравственным существом. Человеческое бедствие есть торжество для божественного милосердия; мучительное чувство греха есть радостное чувство божественной святости. Жизнь, огонь, аффект переходят через человека к богу. Бог гневается на нераскаянного грешника, радуется при виде раскаяния. Человек есть раскрывшийся бог --- божественная сущность осуществляется, проявляется только в человеке. В создании природы бог выходит за пределы собственного существа, относится к себе как к другому существу, а в человеке он снова возвращается в себя: человек познает бога, так как бог обретает и познает себя в человеке, бог чувствует себя богом. Где нет гнета, нет нужды, там нет и чувства, а чувство есть действительное познание. Можно ли познать милосердие, не зная потребности в нем, справедливость, не чувствуя несправедливости, можно ли познать блаженство, не испытывая скорби? Чтобы познать ту или иную вещь, необходимо почувствовать, что она такое. Божественные свойства становятся чувствами, ощущениями впервые в человеке; другими словами, человек есть самочувствие бога --- прочувствованный бог есть действительный бог; ведь свойства божии впервые становятся действительностью лишь как прочувствованные человеком, лишь как патологические и психологические определения. Если бы ощущение человеческого горя находилось вне бога, в существе, лично обособленном от него, то не было бы и милосердия в боге, и мы опять имели бы перед собой бескачественное существо, такое же ничто, каким был бог без человека или до человека. Пример. Я не знаю, доброе и общительное ли я существо --- так как добрым можно назвать только то, что отдается, что сообщается другому, bonus est communicativum sui\footnote{добру свойственно стремление поделиться (лат.).}, --- пока мне не представится случай сделать другому добро. Только в акте сообщения я узнаю счастье благотворительности, радость щедрости, снисходительности. Но разве радость дающего отличается от радости получающего? Нет; я радуюсь потому, что он радуется. Я чувствую несчастье другого, я разделяю его страдания; мне самому становится легче по мере того, как я облегчаю его горе, --- сознание горя есть тоже горе. Радостное чувство того, кто дает, есть только отражение чувства радости того, кто получает. Их радость есть общее ощущение, которое поэтому и проявляется внешним образом в рукопожатии, в поцелуе. То же происходит и здесь. Ощущение божественной благости носит такой же человеческий характер, как и ощущение человеческого горя. Блаженное сознание бесконечности нераздельно с горестным сознанием ограниченности. Где нет одного, там нет и другого. Ощущение бога как бога и ощущение человека как человека нераздельны; неотделимо и самосознание бога от познания человека. Бог становится самим собой только в человеческом <<Я>> --- только в человеческой способности различения, во внутренней двойственности человеческого существа. Так, милосердие ощущается, как <<Я>>, как нечто самостоятельное, как сила, то есть как нечто особое, только её противоположностью. Бог есть бог только чрез то, что не есть бог, только в отличие от его противоположности. В этом заключается тайна учения Якова Бёме. Необходимо только заметить, что Яков Бёме, как мистик и теолог, отделяет от человеческих ощущений, ставит вне человека (так ему во всяком случае кажется), объективирует в форме естественных качеств те ощущения, в которых впервые проявляется божественная сущность, и бог из ничего становится качественным существом. Но Бёме делает это таким образом, что качества эти не перестают производить впечатления на душу. Далее, не следует упускать из виду, что то, что эмпирическое религиозное сознание связывает впервые с действительным творением природы и человека, мистическое сознание приписывает богу, существовавшему ещё до сотворения мира, и этим уничтожает смысл творения. Ведь если бог имеет уже позади себя свою противоположность, то ему не нужно иметь её впереди себя; если бог уже заключает в себе эту противоположность, то ему не нужно, чтобы быть богом, полагать существующим это небожественное. Творение действительного мира есть в данном случае чистая роскошь, или, вернее, нечто невозможное: такой бог не достигает действительности от избытка действительности; он настолько преисполнен этого мира, настолько пресыщен земною пищей, что бытие, сотворение действительного мира, можно объяснить только обратным motus peristalticus\footnote{червеобразное движение кишечника (лат.).} в пожирающем мир желудке бога, как бы божественной рвотой. Это особенно относится к богу Шеллинга, который хотя и составлен из бесчисленных потенций, тем не менее является совершенно импотентным богом. Поэтому эмпирическое религиозное сознание гораздо разумнее; согласно ему, бог впервые открывает себя, как бог, только в связи с появлением действительного человека, действительной природы, так как человек сотворён во славу и хвалу божью. Другими словами, человек представляет уста божьи, в которых качества божьи выражаются как человеческие ощущения. Бог хочет, чтобы ему поклонялись, чтобы его прославляли. Почему? Потому что лишь чувство человека к богу есть самочувствие бога. Но, тем не менее, религиозное сознание разделяет эти две нераздельные стороны, делая при помощи представления личности и бога и человека существами самостоятельными. Гегелевское умозрение отождествляет обе стороны, не уничтожая при этом прежнего противоречия, --- оно есть последовательное проведение и завершение религиозной истины. Ученая толпа была сильно ослеплена своей ненавистью к Гегелю, почему и не понимала, что его учение, по крайней мере в этом отношении, не противоречит религии; оно противоречит лишь постольку, поскольку вообще развитая, последовательная мысль противоречит неразвитому, непоследовательному, но выражающему то же самое представлению.



Но если, как значится в учении Гегеля, сознание человека о боге есть самосознание бога, то, стало быть, человеческое сознание per se\footnote{само по себе (лат.).} есть божественное сознание. Зачем же ты отнимаешь у человека его сознание и делаешь его самосознанием другого, отличного от него существа, объекта? Зачем ты присваиваешь богу сущность, а человеку лишь сознание? Сознание бога заключается в человеке, а сущность человека --- в боге. Познание человека о боге есть познание бога о себе самом? Какой разлад и противоречие! Переверни это положение, и получится истина: знание человека о боге есть знание человека о своей собственной сущности. Только единство сущности и сознания есть истина. Сознание бога нераздельно с его сущностью --- то и другое обретается в человеке; в существе бога объективируется твоя же собственная сущность, перед твоим сознанием встает лишь то, что лежит позади него. Но если определения божественной сущности суть определения человеческие, то, стало быть, человеческим определениям присуща божественная природа.

Только таким путем мы получаем истинное, самодовлеющее единство божественной и мы ещё можем допустить. Мы мыслим три и больше лиц, тождественных по своей человеческой сущности --- единство человеческого существа с самим собой. Но для сущности. Так, например, мы, люди, отличаемся друг от друга личными признаками, этого мы должны отрешиться от особой, отличной от психологии и антропологии, но по существу, по человечеству, мы все в равной мере люди. И это отождествление религиозной философии, или теологии и признать в качестве теологии антропологию.

В основе всякого тождества, которое не есть истинное тождество, не есть единство в этом чувстве исчезают все другие различия --- богатство или бедность, ум или с самим собой, имеется разлад, раздвоение, которым оно уничтожается, или глупость, виновность или невинность. Поэтому чувство сострадания, участие, есть вернее, должно быть уничтожено. Подобное тождество есть противоречие с самим чувство субстанциальное, существенное, философское. Но три или больше собой и рассудком, как половинчатость, фантазия, извращение, нелепость, человеческих индивида существуют отдельно друг от друга, даже если они которая, однако, кажется тем глубже, чем лживее и извращеннее она на самом деле.






\chapter{Противоречие в троице}


Религия, или вернее, богословие не только объективирует в личное существо человеческую или божественную сущность вообще, но представляет также в образе лиц и основные определения или основные её различия. Поэтому троица первоначально есть не что иное, как средоточие существенных основных различий, которые человек усматривает в человеческой сущности. В зависимости от понимания этой сущности различно понимаются и основные определения, на которых основывается троица. Эти различия одной и той же человеческой сущности представляются как субстанции, как божественные личности. В боге они становятся ипостасями, субъектами, существами, и в этом заключается различие между определениями бога и такими же определениями человека, в силу установленного закона, что в представление о личности человеческая личность отчуждает только свои собственные определения. Но личность бога существует только в воображении, поэтому основные определения являются здесь ипостасями, лицами, только для воображения, а для разума, для мышления --- только определениями. Троица есть противоречие между политеизмом и монотеизмом, между фантазией и разумом, между воображением и действительностью. Фантазия есть троичность, а разум --- единство лиц. Разум усматривает в различных существах различия, а фантазия в различиях усматривает различные существа, уничтожающие, в связи с этим, единство божественной сущности. Для разума божественные лица --- фикции, а воображение усматривает в них существа. Троичность заставляет человека мыслить противоположное тому, что он воображает, и воображать противоположное тому, что он мыслит, то есть усматривать в фикциях существа*\let\svthefootnote\thefootnote\let\thefootnote\relax\footnotetext{*странно, что умозрительная философия религии берет под свою защиту троицу против безбожного разума и в то же время устраняет личные субстанции, объясняя, что отношение между отцом и сыном есть только несоответственный, заимствованный из органической жизни образ, вырывающий у троицы её душу, её сердце из тела. Действительно, если ухищрения каббалистического произвола, применяемые умозрительными философами религии к <<абсолютной>> религии, применить также и к <<конечным>> религиям, то будет нетрудно из рогов египетского Аписа сделать ларчик Пандоры христианской догматики. Для этого достаточно допустить сомнительное обособление рассудка от умозрительного разума, что всегда способствует оправданию всякой бессмыслицы.
}\let\thefootnote\svthefootnote.


Есть три лица, но они различны не по сущности. Tres personae, но una essentia\footnote{три лица, одна сущность (лат.).}. Это мы ещё можем допустить. Мы мыслим три и больше лиц, тождественных по своей сущности. Так, например, мы, люди, отличаемся друг от друга личными признаками, но по существу, по человечеству, мы все в равной мере люди. И это отождествление производит не только философствующий ум, но и чувство. Этот индивид такой же человек, как и мы; punctum satis\footnote{точка, довольно (лат.).}; в этом чувстве исчезают все другие различия --- богатство или бедность, ум или глупость, виновность или невинность. Поэтому чувство сострадания, участие, есть чувство субстанциальное, существенное, философское. Но три или больше человеческих индивида существуют отдельно друг от друга, даже если они осуществляют, удостоверяют единство сущности ещё и путем взаимной любви. Они любовно обосновывают нравственную личность, но все-таки каждому из них присуще особое физическое существование. Хотя бы они взаимно дополняли друг друга и не могли обходиться один без другого, все-таки каждый из них неизменно обладал бы устойчивым в-себе-бытием. В-себе-бытие и отдельное бытие --- тождественны и составляют существенный признак личности, субстанции. Иное мы видим в боге, и то по необходимости, так как хотя в нем содержание то же, что и в человеке, но оно другое в силу постулата, по которому оно должно быть другим. Три лица в боге ее имеют нераздельного существования, иначе с небесных высот христианской догматики на нас взирали бы, благодушно и открыто, правда не в большом количестве, но во всяком случае как боги Олимпа три божественных лица в индивидуальном образе, три бога. Олимпийские боги были действительными лицами в своей индивидуальности; по существу, по божественности они были подобны друг другу; но богом был каждый из них в отдельности; они были истинными божественными лицами. Три лица христианской троицы, напротив, суть лица только воображаемые, выдуманные, вымышленные --- во всяком случае отличные от действительных лиц --- именно потому, что они только воображаемые, только кажущиеся личности, которые однако в то же время хотят и должны быть действительными лицами. Существенный признак индивидуального бытия, политеистический элемент, в них исключён и отрицается как безбожный. Но именно благодаря этому отрицанию их личность обращается в призрак воображения. Только в истине плюрализма лежит истина лиц. Три лица христианской троицы должны бы во всяком случае быть не tres dii, не тремя богами, а единым богом, unus deus. Три лица кончаются не множественным числом, как следовало ожидать, а единственным; они представляют не только unum, одно, --- таковы и боги политеизма, --- но и единого, unus. Здесь единство простирается не только на сущность, но и на бытие; единство есть форма существования бога. Три составляют единицу, множественное число обращается в единственное. Бог есть личное существо, которое состоит из трех лиц\dag\let\svthefootnote\thefootnote\let\thefootnote\relax\footnotetext{\dag<<Единство не имеет значения рода, это не Unum, a Unus>>. См. Августин и Петр Ломб. (lib. \rom{1}, dist. 19, гл. 7, 8, 9). <<Таким образом, эти трое, благодаря непостижимым узам божественности, которые их соединяют неизреченным образом, представляют единого бога>>. Петр Ломб., 1, \rom{1}, d. 19, c. 6). <<Может ли разум постичь или поверить, что три равно единице и единица --- трем>>. (Лютер, ч. \rom{14}, стр. 13).}\let\thefootnote\svthefootnote.

Разум считает эти три лица только фикциями, потому что условия, или определения, которые должны утвердить их личность, уничтожаются заповедью монотеизма. Единство отрицает личность; самостоятельность лиц тонет в самостоятельности единства; они становятся простыми отношениями. Без отца нет сына; без сына нет отца; святой дух, нарушающий вообще симметрию, выражает собой лишь отношение их друг к другу. Божественные лица отличаются друг от друга только своим взаимоотношением. Существенное свойство отца, как лица, выражается в том, что он есть отец, а сына --- что он сын. Все другие свойства отца не касаются его личности; в них он является богом, и, как бог, он тождествен с сыном, как богом. поэтому и говорится: бог-отец, бог-сын, бог-дух святой; бог одинаков, един во всех трех лицах. <<Отец, сын и святое дух отличаются друг от друга, но по существу отец есть то же, что сын и святой дух, не составляя ничего иного>>. Другими словами: они --- различные лица, но без различия их существа. Итак, личность исчерпывается здесь отношением отцовства, то есть понятие личности является здесь относительным понятием, понятием отношения. Человек --- отец именно в своем качестве отца не самостоятелен как раз в отношении к сыну; он без сына не отец; сделавшись отцом, человек снижается до степени относительного, несамостоятельного, безличного существа. Но не следует заблуждаться из-за действительных человеческих отношений. Человек-отец не перестает быть во всем остальном самостоятельным, личным существом; он, во всяком случае, обладает устойчивым в-себе-бытием, бытием помимо сына; он не есть только отец с исключением всех других предикатов действительного личного существа. Дурной человек нередко ограничивает свои отцовские обязанности чисто внешними отношениями, не затрагивающими его личного существа. Но в боге-отце нет различия между богом-отцом и богом-сыном, как богом; личность первого лица основывается только на отвлеченном понятии отца, отличающем его от сына; а личность последнего точно так же основывается только на отвлеченном понятии сына.

Но в то же время эти отношения, как уже сказано, должны быть не просто отношениями, не чем-то несамостоятельным, а действительными лицами, существами, субстанциями. Этим вновь подтверждается истина плюрализма, истина политеизма\ddag\let\svthefootnote\thefootnote\let\thefootnote\relax\footnotetext{\ddagесли отец есть бог, и сын есть бог, и св. дух есть бог, то почему они не называются тремя богами? Слушай, что отвечает на этот вопрос Августин: <<Если б я сказал три бога, то я противоречил бы писанию, которое говорит: Слушай, Израиль: твой бог есть единый бог. Поэтому мы предпочитаем говорить три лица, а не три бога, ибо это не противоречит священному писанию>> (Петр Ломб, lib. \rom{1}, dist. 23, гл. 3). Как часто и католицизм опирался на св. писание!}\let\thefootnote\svthefootnote, и отрицается истина монотеизма. Таким образом в священной тайне троицы --- долженствующей представлять истину, отличную от человеческого существа, --- все сводится к иллюзиям, призракам, противоречиям и софизмам*\let\svthefootnote\thefootnote\let\thefootnote\relax\footnotetext{*прекрасное изображение тех разрушительных противоречий, в которые ввергает тайна троицы искренне религиозную душу, можно найти в книге моего брата Фридриха. <<Theanthropos>>, Zirich, 1838.}\let\thefootnote\svthefootnote.




\chapter{Противоречие в таинствах}


Как объективная сущность религии --- сущность бога, --- так и субъективная сущность её, по вполне понятным основаниям, сводится к целому ряду противоречий.

Существенные субъективные моменты религии с одной стороны, это --- вера и любовь, а с другой --- поскольку религия внешним образом выражается в культе --- таинства крещения и причащения. Крещение есть таинство веры, причащение --- таинство любви. Строго говоря, есть только два таинства, как и два существенных субъективных момента религии: вера и любовь; ведь надежда есть та же вера, только отнесённая к будущему; поэтому она столь же нелогично, как и святой дух, превращена в особую сущность.

Совпадение таинств с указанной выше особой сущностью религии выражается прежде всего и независимо от других отношений в том, что основанием их служат естественные предметы или вещи, которым, однако, приписывается значение и действие, противоречащее их природе. Так, например, субъектом или материей крещения является обыкновенная, естественная вода, подобно тому как вообще материей религии является сущность человека. Но религия, отнимающая, отчуждающая от нас нашу собственную сущность, превращает и воду крещения в совершенно особую, отличную от обыкновенной, воду, которой присущи не физическая, а сверхфизическая сила и значение; это --- lavacrum regenerationis\footnote{баня (ванна) возрождения (лат.).}, она очищает человека от скверны первородного греха, изгонят врожденного дьявола, примиряет с богом. Таким образом, эта вода только кажется естественной, а в действительности имеет сверхъестественную силу. Другими словами, крестильная вода производит сверхъестественное действие, а все, что действует сверхъестественным образом, есть сверхъестественная сущность, хотя бы только в представлении, в воображении.


Но, тем не менее, естественная вода есть необходимый материал крещения. Без воды крещение не имеет своего значения и действия. Следовательно естественное качество имеет здесь само по себе ценность и значение, потому что сверхъестественное действие крещения сверхъестественным образом связывается только с водой, а не с каким-либо другим материалом. Бог, благодаря своему всемогуществу, мог бы связать то же самое действие с любой вещью. Но он не делает этого; он приспособляется к естественному качеству; он выбирает соответствующее своему действию вещество. Следовательно, естественное качество не отвергается совершенно; по крайней мере, остается некоторая аналогия, видимость естественности. Вино представляет собой кровь, а хлеб --- тело*\let\svthefootnote\thefootnote\let\thefootnote\relax\footnotetext{*<<Таинство имеет сходство с предметом, символом которого оно служит>>. (Петр Ломб., lib. \rom{4}, dist. \rom{1}, гл. \rom{1}).}\let\thefootnote\svthefootnote. Чудо тоже считается с подобием; оно превращает воду в вино или кровь, одно вещество в другое, сохраняя при этом неопределенное родовое понятие жидкости. То же и здесь. Вода есть наиболее чистая, светлая и прозрачная жидкость, и благодаря этому своему естественному свойству она служит образом незапятнанной сущности божественного духа. Одним словом, вода сама по себе, как вода, имеет значение; благодаря своему естественному качеству вода освящается, становится избранным органом или орудием святого духа. В этом отношении крещение имеет прекрасный, глубокий, естественный смысл. Но этот прекрасный смысл тотчас же утрачивается, так как воде приписывается действие, ей не присущее и зависящее не от нее самой, а от сверхъестественной силы святого духа. Её естественное качество снова становится безразличным: тот, кто превращает воду в вино, может произвольно связать действие крестильной воды с любым веществом.



Поэтому крещение не может быть понято вне понятия чуда. Само крещение есть чудо. Та самая сила, которая производила чудеса и ими, как фактическими доказательствами божественности Христа, обращала иудеев и язычников в христиан, та же самая сила определила крещение и действует в нем. Чудесами началось христианство, чудесами оно и продолжается. Кто вздумает отрицать чудесную силу крещения, тот должен отрицать и чудеса вообще. Естественным источником чудодейственной силы крестильной воды является вода, превращённая в вино на свадьбе в Кане Галилейской.

Вера, порождаемая чудесами, не зависит от меня, от моей самодеятельности, от свободы убеждения и суждения. Чудо, совершающееся на моих глазах, заставляет меня веровать, если только я не окончательно закоснел. Чудо принуждает меня верить в божественность чудотворца\dag\let\svthefootnote\thefootnote\let\thefootnote\relax\footnotetext{\dagв отношении к чудотворцу вера (уверенность в помощи божией) во всяком случае является причиной, действующей причиной, causa efficiens чуда (ср. Матфей, 17: 20. Деян. Апост., 6: 8). Но в отношении к очевидцам чуда --- о них и идет здесь речь --- чудо есть causa efficiens веры.}\let\thefootnote\svthefootnote. Правда, иногда чудо предполагает веру, особенно, когда чудо является наградой; но такая вера есть не истинная вера, а только её смысл, расположение, готовность веровать, преданность вере в противоположность недобросовестности фарисеев. Чудо должно доказывать, что чудотворец действительно есть тот, за кого он себя выдает. Доказана, обоснована и объективна только та вера, которая опирается на чудо. Вера, предполагаемая чудом, есть только вера в Мессию, в Христа вообще, но веру в то, что именно этот человек есть Христос --- а это главное --- порождает впервые чудо. Впрочем, предполагать и эту неопределенную веру вовсе не необходимо. Многие стали верующими только благодаря чуду; следовательно, чудо было причиной их веры. Поэтому, если чудеса не противоречат христианству, --- а каким образом они могут ему противоречить? --- значит, и чудотворное действие крещения тоже ему не противоречит. Напротив, если мы хотим придать крещению христианский смысл, мы непременно должны придать ему значение сверхъестественное. Павел, исполненный ненависти к христианам, был обращён в христианство благодаря неожиданному чудесному явлению. Христианство насильственно им овладело. Здесь нельзя отделаться возражением, что то же самое чудо произвело бы на других иное впечатление и что поэтому его результат следует отчасти приписать самому Павлу. В самом деле, если бы другие удостоились такого же явления, то они несомненно стали бы такими же христианами, как Павел. Ведь божия благодать всемогуща. Неверие и упрямство фарисеев также не служат возражением, ведь они лишены были благодати божией. Согласно божественному определению, Мессию должны были предать, мучить и распять. Поэтому должны были существовать и лица, которые его мучили и распяли; а эти лица должно было заранее лишить божией благодати. Впрочем, они были лишены её не вполне, и не для того, чтобы они действительно обратились ко Христу, а только ради усугубления их вины. Разве можно было противиться воле божией, если то действительно была его воля, а не мимолетный каприз? Сам Павел считал свое обращение и перерождение не заслуженным им лично делом благодати божией\ddag\let\svthefootnote\thefootnote\let\thefootnote\relax\footnotetext{\ddag<<Это есть величайшее чудо, сотворённое Христом, что он так милостиво обратил в веру своего злейшего врага>>. (Лютер, ч. \rom{16}, стр. 560).}\let\thefootnote\svthefootnote. И это верно. Непротивление благодати есть уже само по себе нечто благое и, следовательно, результат действия благодати святого духа. Нет ничего нелепее желания примирить чудо со свободой мышления, а благодать со свободой воли. Религия отделяет человека от его сущности. Деятельность и благодать бога есть отчужденная самодеятельность человека, объективированная свободная воля*\let\svthefootnote\thefootnote\let\thefootnote\relax\footnotetext{*уму и искренности Лютера делает большую честь, что он особенно в своем сочинении против Эразма, безусловно отрицал свободу человеческой воли ради благодати божией. <<Название свободной воли, --- совершенно правильно с точки зрения религия говорит Лютер, --- есть божественный титул и имя которым никто не может и не должен пользоваться, кроме его божественного величества>> (ч. \rom{19}, стр. 28).}\let\thefootnote\svthefootnote.







В высшей степени непоследовательно поступает тот, кто против веры в чудесное действие крещения выставляет, как аргумент, свидетельство опыта, что крещение не делает людей святыми и не перерождает их; так поступали, например, ортодоксальные рационалисты --- богословы\dag\let\svthefootnote\thefootnote\let\thefootnote\relax\footnotetext{\dagдаже древних, безусловно правоверных, богословов опыт заставил признать, что действие крещения, по крайней мере в этой жизни, очень ограничено. <<Baptismus non aufert omnes poenalitates hujus vitae>>. Мецгер, Theol. schol., t. \rom{4}, p. 251; см. также у Петра Ломб. (lib. \rom{4}, dist. 4; lib. \rom{2}, dist. 32, гл. \rom{1}).}\let\thefootnote\svthefootnote\footnote{крещение не отсекает всех грехов этой жизни (лат.).}. Но ведь чудеса, объективная сила молитвы и вообще все сверхъестественные религиозные истины противоречат опыту. Кто ссылается на опыт, должен отказаться от веры. Где опыт признается высшей инстанцией, там уже нет религиозной веры и чувства. Неверующий отрицает объективную силу молитвы только потому, что она противоречит опыту; а атеист идет ещё дальше, он отрицает самое бытие бога, так как оно не подтверждается опытом. Внутренний опыт не смущает его; ведь то, что ты познаешь в себе самом о другом существе, доказывает только, что в тебе есть нечто, что не есть ты сам, что действует на тебя помимо твоей личной воли и сознания, так что ты даже не знаешь, в чем состоит это таинственное нечто. Но вера сильнее опыта. Никакие противоречивые случаи не смущают веру в её вере; она блаженна в себе; она видит только себя, все остальное для нее не существует.



Правда, даже с точки зрения своего мистического материализма, религия всегда требует одновременно момента субъективности, духовности, --- так и при таинствах; но в этом обнаруживается её внутреннее противоречие. И это противоречие выступает особенно ярко в таинстве причащения; ведь крещение идет на пользу уже детям, хотя и здесь, как залог чудотворного действия, имеет значение момент духовности, хотя этот момент странным образом перенесён в веру других лиц, в веру родителей или их заместителей, или церкви вообще\ddag\let\svthefootnote\thefootnote\let\thefootnote\relax\footnotetext{\ddagдаже в абсурдной фикции лютеран, что <<в крещение веруют даже дети>>, момент субъективности сводится к воле других, так как веру детей <<бог принимает благодаря ходатайству и молитве восприемников в вере церкви христианской>> (Лютер, ч. \rom{13}, стр. 360, 361). <<Следовательно, чужая вера так сильна, что я и сам начинаю веровать>> (он же, ч. \rom{14}, стр. 347).}\let\thefootnote\svthefootnote.



Предмет таинства причащения есть тело христово --- действительное тело; но ему недостает необходимых предикатов действительности. Здесь на осязательном примере мы имеем то же, что находили вообще в сущности религии. Объект или субъект в религиозном синтаксисе есть всегда действительно человеческий или естественный субъект, или предикат; но ближайшее определение, существенный предикат этого субъекта, отрицается. Субъект есть нечто чувственное, а предикат --- нечувственное, то есть противоречащее этому субъекту. Действительное тело отличается от воображаемого только тем, что первое производит на меня физическое, непроизвольное впечатление. Следовательно, если бы хлеб был действительным телом божиим, то вкушение его должно было бы непосредственно, невольно производить на меня святое действие; мне незачем было бы особо приготовляться к причащению или приходить в благоговейное настроение. Когда я ем яблоко, я непосредственно ощущаю его вкус. Чтобы опознать яблоко, как яблоко, мне ничего не нужно, кроме здорового желудка. Католики требуют от тела воздержания, как условия вкушения даров. Этого достаточно. Своими губами я беру тело в рот, своими зубами я разжевываю и через свой пищевод я отправляю его в желудок. Я усваиваю его себе не духовно, а телесно*\let\svthefootnote\thefootnote\let\thefootnote\relax\footnotetext{*<<В итоге, --- говорит Лютер, --- мнение мое таково, что действительно в хлебе съедается тело Христово, и все, что претерпевает хлеб, претерпевает и тело Христово, так что оно разрезается и пережевывается зубами благодаря священному единству (propter unionem sacramentalem)>> (Планк, Geschichte der Entst. der protest. Lehrbeg. B. VIII, S. 369). Но в другом месте тот же Лютер отрицает, чтобы тело Христово <<раскусывалось зубами и переваривалось в желудке, как кусок говядины>> (ч. \rom{19}, стр. 429). Поэтому нет ничего удивительного, что, по мнению Лютера, вкушаемое есть предмет без предметности, тело без телесности, плоть без плоти, т. е. <<духовная плоть>>, или воображаемая. Заметим ещё, что и протестанты причащаются натощак, но это не в силу закона, а по обычаю (Лютер, ч. \rom{18}, стр. 200, 201). \rom{1}.}\let\thefootnote\svthefootnote. Почему же действие его должно быть не телесное? Почему это тело, будучи одновременно телесного и небесного, сверхъестественного происхождения, должно производить на меня нефизическое и в то же время священное, сверхъестественное действие? Если мое настроение, моя вера обращает тело в освящающее меня тело, превращает сухой хлеб в духовнотелесную субстанцию, то при чем тут ещё внешний предмет? Ведь я сам есть причина воздействия тела на меня, я сам создаю его действительность; я сам действую на себя. Где же здесь объективная сила и истина? Кто недостойно вкушает причастие, вкушает лишь физический хлеб и вино. Кто ничего не приносит, ничего и не выносит. Поэтому существенное отличие этого хлеба от обыкновенного, естественного хлеба основано только на различии настроения за столом у господа от настроения за другими столами. <<Кто ест и пьет недостойно, тот ест и пьет в осуждение себе, не рассуждая о теле господнем>>\dag\let\svthefootnote\thefootnote\let\thefootnote\relax\footnotetext{\dagКоринф., 11, 29}\let\thefootnote\svthefootnote. Но это настроение зависит опять-таки от того значения, какое я придаю этому хлебу. Если он для меня не хлеб, а тело Христово, о он и не производит на меня действия обыкновенного хлеба. В значении содержится действие. Я ем не для того, чтобы насытиться, поэтому я вкушаю небольшую частицу хлеба. Следовательно, значение обыкновенного хлеба внешним образом устраняется даже в отношении количества, играющего существенную роль при всякой материальной еде.



Но сверхъестественное значение хлеба и вина существует только в фантазии, для чувства вино остается вином, а хлеб-хлебом. Схоластики прибегали в этом случае к утонченному различению между субстанцией и акциденцией. Все акциденции, составляющие природу хлеба и вина, здесь налицо; но только то, что выявляет эти акциденции, --- субстанция, сущность --- отсутствует, оно превращено в тело и кровь. Но совокупность свойств, единство, есть сама субстанция. Чем станут хлеб и вино, если я отниму у них свойства, делающие их тем, что они есть? Ничем. Поэтому тело и кровь не имеют объективного существования; иначе они должны быть объектом также и для неверующего чувства. Наоборот: единственные достоверные свидетели объективного бытия --- вкус, обоняние, осязание, зрение --- единогласно подтверждают только действительность вина и хлеба. Вино и хлеб в действительности --- естественные субстанции, а в воображении --- божественные.

Вера есть сила воображения, обращающая действительное в недействительное и недействительное в действительное --- прямое противоречие истине чувств, истине разума. Вера отрицает то, что утверждает разум, и утверждает то, что он отрицает\ddag\let\svthefootnote\thefootnote\let\thefootnote\relax\footnotetext{\ddag<<Мы видим формы вина и хлеба, по не верим в существование субстанции хлеба и вина. Напротив, мы верим, что существует субстанция хлеба и крови христовой. Но не видим их формы>> (Divus Bernardus, Ed. Basil. 1552, стр. 189--191).}\let\thefootnote\svthefootnote. Тайна причащения есть тайна веры*\let\svthefootnote\thefootnote\let\thefootnote\relax\footnotetext{*ещё и в другом отношении, здесь не указанном, но очень любопытном, именно в следующем. В религии, в вере человек ставит себя объектом, т. е. целью, бога. Человек является целью в боге и через бога. Бог есть средство для человеческого существования и блаженства. Эта религиозная истина, установленная как предмет культа как чувственный объект, представляет таинство причащения в дарах человек снедает бога --- творца неба и земли --- как плотскую снедь, обращает бога в простое продовольственное средство человека актом <<устного снедания и успивания>>. Здесь человек является богом бога --- поэтому причащение есть наивысшее услаждение человеческой субъективности. Даже протестант обращает здесь если не на словах, то на деле, бога во внешний предмет, подчиняя его себе как объект чувственного наслаждения.}\let\thefootnote\svthefootnote, поэтому причащение является самым высоким, увлекательным, упоительным моментом верующего сердца. Отрицание неудобной истины, истины действительности, объективного мира и разума --- отрицание, составляющее сущность веры, --- достигает своей высшей точки в причащении, так как здесь вера отрицает непосредственно сущий, очевидный, несомненный объект, утверждая, что он есть не то, что есть по свидетельству чувств и разума, утверждая, что он только кажется хлебом, а в действительности есть тело.

Положение схоластиков: по акциденциям хлеб, а по субстанции тело, есть только абстрактное пояснительное выражение того, что принимает и исповедует вера, и потому не имеет другого смысла, как: по свидетельству чувств или по обычному взгляду --- хлеб, а воистину --- тело. Поэтому не удивительно, что там, где воображение имеет такую власть над чувствами и разумом, что отрицает самую очевиднейшую чувственную истину, там экзальтация верующих может дойти до того, что вино действительно покажется им кровью. Католицизм может указать много таких примеров. То, что принимается за действительность верой, воображением, вовсе не нуждается в санкции внешних чувств.




Пока вера в таинство причащения царила над человечеством, как священнейшая, высочайшая истина, до тех пор господствующим принципом человечества была сила воображения. Исчезли все признаки отличия действительного от недействительного, разумного от неразумного --- все, что только можно было вообразить, считалось реальной возможностью. Религия оправдывала всякое противоречие разуму и природе вещей. Не смейтесь над нелепыми вопросами схоластиков! Они были неизбежным следствием веры. Предметы чувства обращались в предмет разума; что противоречило разуму, не считалось безрассудным. Это было основное противоречие схоластики, из которого происходили все другие противоречия.

Не важно, верую ли я в протестантское или католическое причастие. В протестантизме хлеб и вино лишь в самый момент вкушения\dag\let\svthefootnote\thefootnote\let\thefootnote\relax\footnotetext{\dag<<Не возражай, что Христос слова: ,,сие есть тело мое``, сказал раньше, чем ученики его стали вкушать хлеб, и что таким образом обратился в тело Христово ещё до вкушения (ante usum)>>. Буддеус (\rom{1}. c. lib. \rom{5}, гл. \rom{1}, par. 13, 17). См., с другой стороны, Concil. Trident. Sessio 13, гл. 3, гл. 8, Пр. 4.}\let\thefootnote\svthefootnote чудесным образом соединяются с телом и кровью; а в католицизме хлеб и вино действительно превращаются в тело и кровь прежде вкушения, властью священника, который, однако, действует лишь во имя всемогущего. Протестант благоразумно уклоняется от определенного объяснения; он не открывает своего слабого места, как то делает благочестивый, простодушный католик, --- бога которого, как внешнюю вещь, могут съесть даже мыши; протестант хранит своего бога у себя в недосягаемом тайнике, оберегая его от несчастных случаев и насмешек; все же, подобно истому католику, в хлебе и вине он вкушает действительное тело и кровь. Сначала протестанты мало чем отличались от католиков в учении о причащении. В Ансбахе возник даже спор по вопросу о том, неужели и тело Христово, подобно всякой другой пище, проходит в желудок, переваривается там и опять извергается естественным путем?>>\ddag\let\svthefootnote\thefootnote\let\thefootnote\relax\footnotetext{\ddagАпология Меланхтона, Strodel. Nuernb. 1783, стр. 127.}\let\thefootnote\svthefootnote





Но хотя сила воображения веры превращает объективное существование в простой призрак, а душевное, воображаемое бытие --- в истину и действительность, тем не менее действительно объективным является только естественное вещество. Даже дары в дарохранительнице католического священника только для веры являются божиим телом; тот внешний предмет, в который он превращает божественную сущность, есть только предмет веры, ибо тело, как тело, здесь невидимо, неосязаемо и лишено вкусовых качеств. Это значит: хлеб только по значению является телом. Правда, для веры это значение имеет смысл действительного бытия --- подобно тому как вообще в молитвенном экстазе означающее становится означаемым: --- этот хлеб должен не означать тело, а быть им. Но такое бытие не есть бытие телесное; оно есть только представляемое, воображаемое бытие, то есть оно имеет только ценность, только качество значения*\let\svthefootnote\thefootnote\let\thefootnote\relax\footnotetext{*<<Если мечтатель верит, что это есть только тленный хлеб и вино, то оно так и есть, и он вкушает только хлеб и вино>> (Лютер, т. \rom{19}, стр. 432). Это значит: если ты веришь и представляешь себе, что хлеб не есть хлеб, а тело, то оно и не есть хлеб; если ты думаешь иначе, то хлеб есть хлеб. Чем хлеб представляется тебе, то он и есть.}\let\thefootnote\svthefootnote. Вещь, имеющая для меня особое значение, в моем представлении иная, чем в действительности. Означающее не есть то, что им означается. Что оно есть, отражается в чувстве; а что оно означает, устанавливают мое настроение, представление, фантазия, и значение предмета существует только для меня, не для другого, не объективно. Так и здесь. Поэтому, когда Цвингли сказал, что причащение имеет лишь субъективное значение, он только повторил то, что было сказано другими; он только разрушил иллюзию религиозного воображения; ведь наличное в причастии есть только воображение, воображающее, что оно не есть воображение. Цвингли выразил в простой, откровенной, прозаической, рационалистической и поэтому оскорбительной форме то же самое, что до него мистическим, косвенным образом высказывалось другими\dag\let\svthefootnote\thefootnote\let\thefootnote\relax\footnotetext{\dagдаже католиками. <<Действие этого таинства, достойным образом принятого, состоит в соединении человека со Христом>> Concil. Florent. de S. Euchar.}\let\thefootnote\svthefootnote, которые признавали, что действие причастия зависит только от подобающего настроения или от веры, то есть что хлеб и вино суть тело и кровь господа, или сам господь, только для тех, кто придает им сверхъестественное значение божественного тела, поскольку только от этого зависит подобающее настроение и религиозное одушевление\ddag\let\svthefootnote\thefootnote\let\thefootnote\relax\footnotetext{\ddag<<Если тело заключается в хлебе и вкушается с верой, то оно укрепляет душу верой, что уста вкусили тело Христово>>. Лютер, ч. \rom{19}, стр. 433; см. также стр. 205). <<Ибо когда мы получили нечто, то оно так и есть>> (он же, ч. \rom{17}, стр. 557).}\let\thefootnote\svthefootnote.








Но если причастие не оказывает действия, следовательно, есть ничто, --- ибо существует только то, что действует, --- есть ничто без настроения и веры, то, стало быть, только в них и заключается его значение, и весь процесс совершается в душе. Если представление, что я вкушаю действительное тело Спасителя, действует на религиозное чувство, значит, само это представление вытекает из чувства; оно порождает благочестивые помыслы, так как и само оно есть благочестивое представление.

Таким образом, и здесь религиозный субъект воздействует на себя сам, в качестве другого существа, посредством представления воображаемого объекта.

Поэтому я мог бы отлично совершить акт причащения в себе самом, в своем воображении, без посредства вина и хлеба, без всякой церковной церемонии. Есть бесчисленное множество благочестивых стихов, единственным содержанием которых является кровь Христова. В них мы имеем чисто поэтическое прославление тайной вечери. В живом представлении страдающего, истекающего кровью Спасителя сердце человеческое соединяется с ним; здесь благочестивая душа в поэтическом воодушевлении пьет чистую кровь, не смешанную с противоречивым чувственным веществом; здесь между представлением крови и самой кровью нет никакого средостения.

Но хотя причастие и всякое таинство вообще есть ничто без настроения, без веры, тем не менее религия смотрит на таинство как на нечто действительное само по себе, как на нечто внешнее, отличное от человеческого существа, так что в религиозном сознании суть дела, --- вера, настроение, --- становится второстепенной вещью, условием, а мнимый, воображаемый предмет --- главным делом. И естественным, неизбежным следствием этого религиозного материализма, этого подчинения человеческого --- мнимо-божественному, субъективного --- мнимо-объективному, истины --- воображению, нравственности --- религии, естественным, неизбежным следствием этого является суеверие и безнравственность. Суеверие, потому что предмету приписывается действие, противоречащее его природе, потому что не позволяют вещи быть тем, что она есть, и продукт воображения принимают за действительность. Безнравственность, потому что в сердце святость действия, как такового, обособляется от нравственности, и вкушение даров, даже независимо от настроения, становится священным, спасительным актом. Так, по крайней мере, дело обстоит на практике, которая не считается с богословской софистикой. Религия противоречит нравственности тем самым, чем она противоречит разуму. Чувство добра тесно связано с чувством истины. Испорченность рассудка влечет за собой испорченность сердца. Кто обманывает свой рассудок, не может обладать искренним, честным сердцем; софистика портит всего человека. А учение о причастии есть именно софистика. Истинность настроения неразрывно связана с отрицанием физического присутствия божия, и наоборот: истинность объективного бытия божия неразрывно связана с отрицанием истинности и необходимости настроения.






\chapter{Противоречие веры и любви}


Таинства делают наглядным противоречие между идеализмом и материализмом, субъективизмом и объективизмом, противоречие, составляющее сокровенную сущность религии. Но таинства суть ничто без веры и любви.

Поэтому противоречие в таинствах заставляет нас вернуться к противоречию между верой и любовью.

Сокровенная сущность религии есть тождество существа божия и человеческого, а форма религии или очевидная, сознанная её сущность есть различие между богом и человеком. Бог есть человеческая сущность, но сознаваемая как другое существо. Любовь обнаруживает сокровенную сущность религии, а вера составляет её сознательную форму. Любовь отождествляет человека с богом, бога с человеком и следовательно, человека с человеком; а вера отделяет бога от человека и, следовательно, человека от человека; ведь бог есть не что иное, как мистическое понятие рода человеческого, поэтому отделение бога от человека есть отделение человека от человека, уничтожение их связи. Благодаря вере религия становится в противоречие с нравственностью, разумом и простым человеческим инстинктом правды; а благодаря любви она противится этому противоречию. Вера изолирует бога, делает его особым, другим существом; а любовь делает бога всеобщим существом, любовь к которому тождественна с любовью к человеку. Вера способствует внутреннему и, следовательно, внешнему раздвоению человека с самим собою; любовь исцеляет раны, наносимые верою сердцу человека. Вера делает веру в своего бога законом; любовь есть свобода, она не осуждает даже атеиста, потому что она сама атеистична и отрицает, если не всегда теоретически, то практически, существование особого, противоречащего человеку бога.

Вера устанавливает разграничение, что истинно и что ложно. И считает истинной только себя. Содержанием веры служит определенная, особая истина, которая поэтому необходимо связана с отрицанием. Вера исключительна по своей природе. Истина только одна, бог только один, монополия сына божия принадлежит только одному; все другое есть ничто, есть заблуждение и призрак. Один Иегова есть истинный бог; все другие боги --- ничтожные идолы.

Вера имеет свою особую сферу; она опирается на особое откровение божие; она добилась своего достояния не обычным путем, не тем путем, который доступен всем людям без различия. Что доступно всем, есть нечто обыденное и потому не составляет объекта веры. Бог есть творец --- это могли познать все люди путем изучения природы; но что такое этот бог индивидуально сам по себе --- есть особый вопрос благодати, содержание особой веры. Объект этой веры открывается особым образом, и потому он является особым существом. Христианский бог есть также языческий бог; но здесь есть и большая разница, такая же разница, как между мною, каким я представляюсь другу, и мною, каким я представляюсь чужому человеку, знающему меня только издали. Бог, как объект христиан, отличается от бога, как объекта язычников. Христиане знают бога индивидуально, лицом к лицу. Язычники в лучшем случае знают только, что есть бог, а не знают, кто есть бог, вследствие чего они и впали в идолопоклонство. Поэтому равенства язычников и христиан перед богом почти не существует; и если есть нечто общее между христианами и язычниками и обратно, --- будем настолько свободомыслящими, что допустим это, --- то оно не касается ни собственно христианства, ни того, что составляет веру. В чем христиане являются христианами, это же самое отличает их от язычников*\let\svthefootnote\thefootnote\let\thefootnote\relax\footnotetext{*<<Если я хочу быть христианином, я должен веровать и поступать не так, как другие люди>> (Лютер, ч. \rom{16}, стр. 569).}\let\thefootnote\svthefootnote; а христианами они являются в силу своего особого познания бога; следовательно, здесь признаком отличия является бог. Особенность есть соль, сообщающая вкус обыкновенному существу. Сущность каждой вещи заключается в её особенностях: знает меня только тот, кто знает меня специально или лично. Поэтому специальный бог, тот бог, который в особенности является объектом христиан, личный бог, только он и есть бог. И этот бог, неведомый язычникам и вообще неверующим, существует не для них. Он может стать богом и для язычников, но не непосредственным путем, а только когда они сами перестанут быть язычниками и обратятся в христиан. Вера ограничивает, сужает горизонты человека; она отнимает у него свободу и способность подвергать оценке то, что от него отличается. Вера замыкается в себе. Правда, философ, вообще ученый догматик, тоже ограничивает себя определенностью своей системы. Но теоретическое ограничение, как бы несвободно, узко и близоруко оно ни было, все-таки носит более свободный характер, так как область теории свободна сама по себе, ведь здесь суждение обусловливается только предметом, причиной, разумом. А вера делает свое содержание предметом совести, личного интереса и стремления к блаженству, ибо объект веры есть существо особое, личное, требующее признания и делающее это признание условием блаженства.



Вера сообщает человеку особое чувство тщеславия и эгоизма. Верующий выделяет себя среди других людей, ставит себя выше обыкновенного человека; он мнит себя лицом привилегированным, пользующимся особыми правами; верующие --- аристократы, а неверующие --- плебеи. Бог есть это олицетворённое отличие и преимущество верующего перед неверующим\dag\let\svthefootnote\thefootnote\let\thefootnote\relax\footnotetext{\dag<<Цельз упрекает христиан, что они хвалятся тем, что они первые после бога>>. Est Deus et postillum nos. (Origenes adv. Cels. ed. Hoeschelius. Aug. Vind. 1605, стр. 182).}\let\thefootnote\svthefootnote\footnote{есть бог, а после него мы (лат.).}. Но так как вера представляет собственную сущность как другое существо, то верующий относит свое достоинство не непосредственно к себе, а к этому другому лицу. Сознание своего преимущества есть в верующем человеке сознание этого лица, и ощущение самого себя он относит к этой другой индивидуальности\ddag\let\svthefootnote\thefootnote\let\thefootnote\relax\footnotetext{\ddag<<Я горжусь и хвалюсь своим блаженством и отпущением грехов, но почему? Потому что я горжусь чужой славою и высокомерием, именно славою господа Христа>> (Лютер, ч. \rom{2}, стр. 344. <<Хвалящийся, хвались господом>> (Первое послание к Коринфянам, 1, 31).}\let\thefootnote\svthefootnote. Верующий подобен слуге, который чувствует себя участником в достоинствах своего господина и ставит себя выше человека свободного и самостоятельного, но по положению менее высокого, чем его господин*\let\svthefootnote\thefootnote\let\thefootnote\relax\footnotetext{*один из бывших адъютантов русского генерала Миниха сказал: <<Когда я был его адъютантом, я чувствовал себя более важной особой, чем теперь, когда я командую отдельной частью>>.}\let\thefootnote\svthefootnote. Он отказывается от всех заслуг, чтобы предоставить честь этих заслуг своем господину, но только потому, что эти заслуги опять возвращаются к нему, и в этой чести своего господина он удовлетворяет свое собственное честолюбие. Вера высокомерна, но она отличается от естественного высокомерия тем, что она чувство своего превосходства, свою гордость переносит на другое лицо, которое наделяет верующего разными преимуществами и вместе с тем является его скрытой сущностью, его олицетворённым и удовлетворенным стремлением к блаженству; ведь назначение этого лица быть благодетелем, освободителем и спасителем человека и вести верующего к его собственному вечному спасению. Словом, отличительный признак религии сводится к тому, что она превращает действительный залог в страдательный. Язычник возвышается сам, христианин --- при помощи другого лица. Христианин превращает в дело приязни и чувства то, что для язычника является делом самодеятельности. Смирение верующего есть обратное высокомерие --- высокомерие, утратившее видимость, внешние признаки высокомерия. Он чувствует себя избранником; но это превосходство не есть результат его деятельности, а дело благодати; он стал избранником помимо своей воли, он ничего не сделал для этого. Он перестает вообще быть целью своей собственной деятельности, а становится целью, объектом бога.







Вера по существу есть определенная вера. Бог только в этой определенности есть истинный бог. Этот Иисус и есть Христос, истинный, единственный пророк, единородный сын божий. Если ты хочешь достигнуть блаженства, ты должен верить в эту определенную истину. Вера повелевает\dag\let\svthefootnote\thefootnote\let\thefootnote\relax\footnotetext{\dag<<Люди обязываются законом божиим к правой вере. Раньше всех других заповедей богом установлена правая вера в следующей заповеди: ,,Слушай, Израиль, господь бог наш есть единый владыка``. Этой заповедью исключается заблуждение утверждающих, что для спасения человека безразлично, какой верой служит он богу>> (Фома Аквинский, Summa cont. Gentiles, lib. \rom{3}, c. 118, par 3).}\let\thefootnote\svthefootnote. Сущность веры такова, что она должна гасить характер догмата. Догмат только выражает то, что уже давно находится на языке или на уме у веры. Установление какого-нибудь основного догмата влечет за собой возникновение более специальных вопросов, которые тоже приходится облекать в догматическую форму. Отсюда возникает обременительное изобилие догматов, что не исключает, однако, их необходимости, потому что они дают нам возможность точно знать, во что мы должны верить и каким образом мы можем достичь блаженства.



Многое, что в наши дни опровергается, осмеивается, признается ошибкой, недоразумением или преувеличением даже с точки зрения правоверного христианства, является неизбежным следствием внутренней сущности веры. Вера по своей природе несвободна и ограничена, так как она имеет дело с собственным блаженством и славой божией. Как мы с тревогой заботимся о том, чтобы воздать высокопоставленному лицу подобающую ему честь, --- о том же заботимся мы и в вере. Апостол Павел заботится исключительно о славе, чести и заслугах Христа. Догматическая, исключительная, скрупулезная определенность коренится в сущности веры. По отношению к пище и другим безразличным для веры вещам вера вполне либеральна, но иначе относится она к предметам веры. Кто не за Христа, тот против Христа; все не --- христианское есть антихристианское. Но что же является христианским? Это должно быть точно установлено, это нельзя предоставить усмотрению. Содержание веры изложено в книгах, составленных различными авторами; вера изложена в форме случайных, противоречивых, отдельных изречений, поэтому догматическое определение и истолкование является внешней необходимостью. Христианство обязано своим продолжительным существованием только церковной догматике.

Только бесхарактерность и правоверное неверие нашего времени стараются спрятаться за библию и противопоставить догматическим определениям библейские изречения, с целью освободить произвол экзегезы из оков догматики. Но веры уже нет, она стала безразличным делом там, где определения веры ощущаются как оковы. Это просто религиозный индифферентизм, который под видом религиозности делает исключительным мерилом веры неопределенную по своей природе и происхождению библию. Этот индифферентизм прикрывается желанием верить только в существенное, а на самом деле он не верит ни во что, заслуживающее имени веры; например, он заменяет определенный, выразительный образ сына божия, созданный церковью, неясным, ничего не говорящим определением безгрешного человека, который более всех других имел бы право называться сыном божиим, то есть понятием человека, которого нельзя назвать ни человеком, ни богом. Доказательством того, что за библией действительно кроется только религиозный индифферентизм, служит тот факт, что здесь отрицают, считают необязательным даже содержащееся в библии, но находящееся в противоречии с современной научной точкой зрения, и даже называют нехристианскими такие, например, чисто христианские, неизбежно из веры вытекающие поступки, как обособление верующих от неверующих.

Церковь совершенно справедливо осуждает всякое неверие и иноверие\ddag\let\svthefootnote\thefootnote\let\thefootnote\relax\footnotetext{\ddagдля веры, если у нее есть еще огонь в крови и сила, всякий иноверец есть неверующий --- атеист.}\let\thefootnote\svthefootnote, так как это осуждение лежит в сущности веры. Вначале вера представляется лишь невинным отделением верующих от неверующих; но на самом деле это отделение носит в высшей степени резкий характер. Верующий имеет бога за себя, а неверующий --- против себя. Только возможность обращения неверующего в верующего примиряет его с богом, и этим объясняется требование отказаться от неверия. Все, что имеет бога против себя, --- ничтожно, отвергнуто, осуждено; ведь всякий, вооружающий против себя бога, сам восстает против него. Веровать --- значит быть добрым, а не веровать --- быть злым. Вера, ограниченная и узкая, все объясняет настроением. Неверующий кажется ей неверующим по злобе и упрямству и является врагом Христа*\let\svthefootnote\thefootnote\let\thefootnote\relax\footnotetext{*уже в Новом завете с неверием связывается понятие непослушания. <<Главное зло есть неверие>> (Лютер, ч. \rom{17}, стр. 647).}\let\thefootnote\svthefootnote. Поэтому вера ассимилирует только верующих и отталкивает неверующих. Она добра по отношению к верующим и зла по отношению к неверующим. В вере лежит злое начало.





Христиане настолько эгоистичны, тщеславны и самодовольны, что видят сучки в вере нехристианских народов и не замечают бревен в своей собственной вере. Но вероисповедные отличия у христиан иные, чем у других народов. Лишь климатические различия или различия народных темпераментов создают отличия. Народ воинственный или вообще пылкий, чувственный, естественно проявляет свое религиозное различие в чувственных поступках, в силе оружия. Но природа веры, как таковой, одинакова везде. Вера везде осуждает и проклинает. Всякое благословение, всякое благо сосредоточивает она на себе и своем боге, как любовник на своей возлюбленной, а всякое проклятие, бедствие и зло оставляет на долю неверующего. Верующий участвует в божьем благословении, благоволении и вечном блаженстве; неверующий предается проклятию, отвергается богом и людьми; ведь человек не должен принимать тех, от кого отрёкся бог; это было бы критикой божественного приговора. Магометане истребляют неверующих огнем и мечом, христиане --- геенной огненной. Но пламя потустороннего мира врывается и в этот мир, чтобы осветить мочь, окружающую неверующих. Верующие предвкушают небесное блаженство ещё на земле, и поэтому неверующие тоже должны уже заранее испытывать адские муки, по крайней мере в моменты религиозного воодушевления\dag\let\svthefootnote\thefootnote\let\thefootnote\relax\footnotetext{\dagсам бог не всегда откладывает наказание богохульников, неверующих и еретиков на будущее время, а наказывает их нередко уже теперь, в этой жизни <<во славу христианства и для укрепления веры>>; так, например, он покарал еретика Керипфа и еретика Ария (Лютер, ч. \rom{14}, стр. 13).}\let\thefootnote\svthefootnote. Правда, христианство не заповедует ни преследования еретиков, ни, тем менее, обращения в веру посредством оружия; но так как вера обрекает их на гибель то она неизбежно вызывает враждебные настроения, те настроения, которые порождают преследования еретиков. Любить человека, не верующего во Христа, --- значит грешить против Христа и любить его врага\ddag\let\svthefootnote\thefootnote\let\thefootnote\relax\footnotetext{\ddag<<Имеющий в себе дух божий пусть вспоминает стих (псалом 39:21): Мне ли не возненавидеть ненавидящих тебя>> (Бернард, Epist. 193 ad magist. Ivonem Card).}\let\thefootnote\svthefootnote. Человек не должен любить того, чего не любит бог или Христос, иначе его любовь будет противоречить божественной воле и станет грехом. Правда, бог любит всех людей, но лишь тогда и потому, что они христиане или, по крайней мере, желают и могут быть ими. Быть христианином --- значит пользоваться божественной любовью; а не быть христианином --- значит быть предметом ненависти и гнева божия*\let\svthefootnote\thefootnote\let\thefootnote\relax\footnotetext{*<<Кто отрекается от Христа, того отвергнет и Христос>> Киприан (Epist., E. 73, par. 18, ed. Gersdorf).}\let\thefootnote\svthefootnote. Следовательно, христианин может любить только христианина или того, кто может сделаться христианином; он может любить только то, что освящает и благословляет вера. Вера есть крещение любви. Любовь к человеку, как к человеку, есть любовь только естественная. Христианская любовь есть любовь сверхъестественная, преображённая, освящённая, но христианская любовь любит только все христианское. Заповедь <<любите врагов ваших>> относится только к личным врагам, а не к врагам общественным, к врагам бога и веры, к неверующим. Кто любит человека, отрицающего Христа, не верующего в него, тот сам отрекается от своего господа бога; вера уничтожает естественные узы человечества; она замещает всеобщее, естественное единство единством обособленным.







Пусть не указывают на то, что в библии сказано: <<не судите, да не судимы будете>>, что, следовательно, суд и осуждение вера исключительно предоставляет богу. Как это, так и другие подобные изречения относятся только к области христианского частного, а не христианского публичного права и имеют в виду только мораль, а не догматику. Только религиозный индифферентизм перенес эти моральные изречения в область догматики. Различие между неверующим и человеком есть продукт современной гуманности. Для веры человек исчерпывается верой; для веры существенное отличие человека от животного покоится только на религиозной вере. Одна лишь вера вмещает в себе все добродетели, делающие человека угодным богу; бог есть мерило, и его благоволение есть наивысшая норма; следовательно, только верующий человек есть законный, нормальный человек, такой человек, каким он должен быть и каким признает его бог. Там, где делается различие между просто человеком и верующим, там человек уже отделил себя от веры; там уже человек ценит самого себя независимо от веры. Поэтому вера является истинной, нелицемерной только там, где резко устанавливается вероисповедное различие. Как только это различие притупляется, вера естественно становится индифферентной и бессильной. Вера либеральна только в вещах, которые безразличны сами по себе. Либерализм апостола Павла предполагает принятие основных догматов веры. Где все сводится к основным догматам веры, там возникает различие между существенным и несущественным. В области несущественных вещей закон не действует, там вы свободны. Но, разумеется, только при том условии, что вы не ограничите у веры её прав, она предоставляет вам известные права и вольности.

Поэтому было бы совсем неосновательно ссылаться на то, что вера предоставляет суд богу. Она предоставляет ему только суд моральный по отношению к вере, только суд над нравственными свойствами её, над лицемерной или искренней верой христиан. Вера знает, кто станет одесную и ощущаю бога. Она не может назвать отдельных лиц, но нисколько не сомневается в том, что верующие вообще унаследуют вечное царство\dag\let\svthefootnote\thefootnote\let\thefootnote\relax\footnotetext{\dag<<Современные вольнодумцы пытаются установить опасный принцип, будто ошибки творения, каковы бы они ни были, но могут служить основанием для осуждения>>, --- говорит Жюрье (т. 4, Papisme, c. \rom{2}); ведь невозможно, чтобы тот, кто верует во единую спасительную) веру (Послание к Ефесянам 4:5) и знает, какая вера истинна и спасительна, не знал, что такое неправая вера и что такое еретик>> (Das Ebenbild Christ. Thomasii durch S. Bentzen Pastorn, 1692, стр. 57). <<Мы судим и рассуждаем по Евангелию, --- говорит в своих застольных речах Лютер по поводу перекрещенцев; --- кто не верует, тот уже осужден. Поэтому мы должны быть уверены, что они заблуждаются и прокляты>>.}\let\thefootnote\svthefootnote. Но и помимо этого бог, делающий различие между верующими и неверующими, бог, награждающий одних и осуждающий других, есть не что иное, как сама вера. Вера осуждает все, что осуждено богом, и наоборот. Вера есть огонь, беспощадно пожирающий свою противоположность. Этот огонь веры как объективная сущность есть гнев божий или, что то же, ад; ведь ад очевидно имеет свое основание в гневе божием. Но этот ад заключается в самой вере, в её праве осуждения. Адское пламя есть только отблеск того уничтожающего, гневного взора, который вера бросает на неверующих.



Вера по существу партийна. Кто не за Христа, тот против Христа. Либо за меня, либо против меня. Вера знает только врагов или друзей; она не может быть беспристрастной; она имеет в виду только себя. Вера по существу нетерпима --- по существу потому, что вера тесно связана с иллюзией, будто её дело есть дело бога, её честь есть честь бога. Бог веры есть не что иное, как объективированная сущность самой веры. Поэтому религиозное чувство и сознание отождествляют дело веры с делом бога. Бог является заинтересованной стороной; интерес верующих есть сокровенный интерес самого бога. <<Касающийся вас касается зеницы ока господа>>, --- говорится у пророка Захарии\ddag\let\svthefootnote\thefootnote\let\thefootnote\relax\footnotetext{\ddag<<Он указал на самую нежную часть человеческого тела, чтобы мы ясно видели, что бог столь же сильно задевается малейшей хулой на его святых, как человек страдает от малейшего прикосновения к его глазу>>. Сальвиан (L. \rom{8} de gebern. Dei.). <<Господь бдительно охраняет пути святых своих, дабы они не преткнулись о камень>>. Кальвин (Inst. Rel. chr., lib. \rom{1}, c. 17, sect. 6).}\let\thefootnote\svthefootnote. Все, что оскорбляет или отрицает веру, оскорбляет и отрицает самого бога.



Вера не знает другого различия, кроме различия между служением богу и идолопоклонством. Одна только вера воздает честь богу, а неверие лишает бога того, что ему подобает. Неверие есть оскорбление бога, преступление против высшей власти. Язычники молятся демонам: их боги суть бесы. <<Язычники, принося жертвы, приносят их бесам, а не богу; но я не хочу, чтобы вы были в общении с бесами>>\footnote{см.: <<Первое послание к коринфянам ап. Павла>> (10,20).}. Бес есть отрицание бога, он ненавидит бога, не хочет, чтобы бог существовал. Поэтому вера слепа к добру и истине, лежащим в основе идолопоклонства; она видит идолопоклонство во всем, что не служит её богу, т. е. ей самой, а в идолопоклонстве --- только дело дьявола. Поэтому вера даже по своему настроению должна относиться к такому отрицанию бога только отрицательно: она по существу нетерпима к своей противоположности и вообще ко всему, что с ней не согласно. Её терпимость была бы нетерпимостью к богу, имеющему право на неограниченное господство. То, что не признает бога и веры, не должно быть, не должно существовать. <<Дабы во имя Иисуса преклонились колени всех тех, кто находится на небе, земле и в преисподней, и всякий бы язык исповедывал, что Иисус Христос --- господь во славу бога-отца>>*\let\svthefootnote\thefootnote\let\thefootnote\relax\footnotetext{*Послание к Филиппийцам 2:10, 11. <<При имени Христа да устрашится всякое неверие и безбожие на небе и на земле>>. (Лютер, ч. \rom{16}, стр. 322). <<Христианин гордится убийством язычника, поскольку этим он прославляет Христа>> (Divus Bernardus, Sermo exhort. ad Milites Templi).}\let\thefootnote\svthefootnote. Поэтому вера требует потусторонней жизни, требует такого мира, где противоположность веры не существует вовсе или существует только для того, чтобы усилить самочувствие торжествующей веры. Ад услаждает собою радость блаженных верующих. <<Избранные будут смотреть на муки безбожников, не проникаясь состраданием; напротив, эти невыразимые муки заставят их восторженно благодарить бога за свое спасение>>\dag\let\svthefootnote\thefootnote\let\thefootnote\relax\footnotetext{\dagПетр Ломб. (lib. \rom{4}, dist. 50, c. 4). Но это изречение никоим образом не принадлежит самому Петру Ломб. Петр Ломб. лично слишком скромен, робок и слишком преклоняется перед авторитетами христианства, чтобы решиться на утверждение чего-либо за собственный риск. Нет! Это утверждение есть общее утверждение, характерное для христианской верующей любви. --- Учение некоторых отцов церкви, так, например, Оригена и Григория Нисского, будто показания осужденных некогда прекратятся, исходит не из христианского или церковного учения, а из платонизма. Показательно и то, что учение о конечности адских страданий было отвергнуто не только католической, но и протестантской церковью (Augsbg. Konfess. Art.). Драгоценным примером исключительной, человеконенавистнической узости христианской любви является цитируемое Штраусом (Christl. Glaubensl. B. \rom{2}, S. 547) из Буддеуса место, согласно которому не дети людей вообще, а только дети христиан удостоятся благодати божией и блаженства, если умрут некрещеными.}\let\thefootnote\svthefootnote.





Вера есть противоположность любви. Любовь умеет находить добродетель в грехе и истину в заблуждении. Только недавно, когда сила веры уступила место естественному единству человечества, силе разума и гуманности, люди стали замечать истину в политеизме, в идолопоклонстве вообще или, по крайней мере, попытались объяснить человеческими, естественными причинами то, что замкнутая в себе вера приписывает исключительно дьяволу. Поэтому любовь тождественна только с разумом, а не с верой; ведь разум и любовь носят свободный, всеобщий, а вера --- узкий, ограниченный характер. Где разум, там царит всеобщая любовь; разум есть не что иное, как универсальная любовь. Ад изобретён верой, а не любовью, не разумом. Для любви ад есть ужас, а для разума --- бессмыслица. Ад нельзя считать только религиозным заблуждением и видеть в нем ложную веру. О нем упоминается ещё в библии. Вера всегда верна самой себе, по крайней мере вера положительной религии, вера в том смысле, в каком она рассматривается здесь и должна рассматриваться, если мы не хотим смешивать с верой элементы разума и культуры, что только затемняет истинную природу веры.

Итак, если вера не противоречит христианству, то не противоречат ему и те настроения, которые вытекают из веры, и те поступки, которые обусловливаются этими настроениями. Вера осуждает; все поступки, все настроения, противоречащие любви, гуманности и разуму, не противоречат вере. Все ужасы истории христианской религии, о которых верующие говорят, что они не вытекали из христианства, возникли из веры, следовательно, из христианства. Даже это их отрицание является неизбежным следствием веры; ибо вера присваивает себе только все хорошее, а все дурное оставляет на долю неверия, ереси, или на долю человека вообще. Но отрекаясь от того, что она виновница зла в христианстве, вера лишний раз убедительно доказывает нам, что она есть истинная виновница этого зла, так как этим она свидетельствует о своей ограниченности, пристрастии и нетерпимости, благодаря чему она желает добра только себе и своим приверженцам и зла --- всем другим. Вера приписывает добро, сделанное христианами, не человеку, а христианину, а дурные поступки христиан не христианину, а человеку. Итак, злые деяния христианской веры соответствуют сущности веры --- той веры, как она выражена в древнейшем и самом священном источнике христианства --- библии. <<Кто благовествует вам не то, что вы приняли, да будет отлучён>>\ddag\let\svthefootnote\thefootnote\let\thefootnote\relax\footnotetext{\ddagизбегайте, страшитесь такого проповедника! --- Но почему же я должен его избегать? --- Ибо гнев, то есть проклятие божие, тяготеет над головой его.}\let\thefootnote\svthefootnote. (Послание к Галатам, 1:9). <<Не впрягайтесь в чужое ярмо с неверными, ибо что общего между справедливостью и беззаконием? Что общего у света с тьмою? У Христа с Велиалом? Что общего у верного с неверным? Можно ли сравнивать храм божий с идолами? Ведь мы --- храм бога живого, как сказал бог: вселюсь в них и буду ходить в них; и буду их богом, и они будут моим народом. И потому выйдите от них и отделитесь, говорит господь, и не прикасайтесь к нечистым, и я приму вас>> (Второе послание к коринфянам, 6:14--17). <<В явлении господа Иисуса с неба, с ангелами силы его, в пламенеющем огне совершающего отмщение непознавшим бога и непокоряющимся евангелию господа нашего Иисуса Христа, которые подвергнутся наказанию, вечной погибели от лица господа и от славы могущества его, когда он приидет прославиться во святых своих и явиться дивным во всех веровавших>> (Второе послание Фессалоникийцам, 1:7--10). <<Без веры угодить богу невозможно>> (Послание к Евреям, 11:6). <<Ибо так возлюбил бог мир, что отдал сына своего единородного, дабы всякий, верующий в него, не погиб, но имел жизнь вечную>>. (Иоанн, 3:16). <<Всякий дух, который исповедует Иисуса Христа, пришедшего во плоти, есть от бога. А всякий дух, который не исповедует Иисуса Христа, пришедшего во плоти, не есть от бога, но это дух антихриста>> (Первое послание Иоанна, 4:2, 3). <<Кто лжец, если не тот, кто отвергает, что Иисус есть Христос? Это --- антихрист, отвергающий отца и сына>> (Первое послание Иоанна, 2:22). <<Всякий, преступающий учение Христа и не пребывающий в нем, не имеет бога; пребывающий в учении Христовом имеет и отца и сына. Кто приходит к вам и не приносит сего учения, того не принимайте в дом и не приветствуйте его; ибо приветствующий его участвует в злых делах его>> (Второе послание Иоанна, 9--11). Так говорит апостол любви. Но любовь, которую он прославляет, есть только христианская, братская любовь. <<Бог есть спаситель всех человеков, а наипаче верных>> (Первое послание к Тимофею, 4:10). Роковое слово! <<Будем делать добро всем, а наипаче своим по вере!>>. (Послание к Галатам, 6:10). Опять роковое слово <<наипаче>>! <<Еретика после первого и второго вразумления отвращайся, зная что таковой развратился и грешит, будучи самоосужден>> (Послание к Титу, 3:10,11). <<Таковы Гименей и Филет, которых я предал сатане, чтобы они научились больше не богохульствовать>> (Первое послание к Тимофею, 1:20; Второе послание к Тимофею,:17,18). --- Вот места, на которые католики ссылаются ещё и теперь, чтобы оправдать нетерпимость церкви в отношении еретиков. <<Кто не любит господа Иисуса Христа, --- анафема>> (Первое послание к Коринфянам 16:22). <<Верующий в сына имеет жизнь вечную; а неверующий в сына не увидит жизни, но гнев божий пребывает на нем>>. (Иоанн, 3:36)*\let\svthefootnote\thefootnote\let\thefootnote\relax\footnotetext{*отсюда по необходимости вытекает настроение, высказываемое, например, Киприаном: <<Если еретики повсюду называются врагами и антихристами и характеризуются как люди проклятые, которых надо избегать, то почему же и нам не осуждать и не проклинать тех, кто, по свидетельству апостолов, сами себя осудили?>> (Epist. 74, Edit. cit.).}\let\thefootnote\svthefootnote\dag\let\svthefootnote\thefootnote\let\thefootnote\relax\footnotetext{\dagэто место у Луки (9:56) и как параллельное ему у Иоанна (3, 17) сопровождается следующим добавлением (стих. 18): <<Верующий в него не судится, а неверующий уже осужден>>.}\let\thefootnote\svthefootnote. <<А кто соблазнит одного из малых сих, верующих в меня, тому лучше было бы, если бы повесили ему мельничный камень на шею и бросили в море>> (Марк, 9:42; Матфей, 18:6). <<Кто будет веровать и креститься, спасён будет, а кто не будет веровать и креститься, осужден будет>> (Марк, 16, 16). Различие между верой, выраженной в библии, и верой позднейших времен подобно различию между зародышем и растением. Зародыш представляет лишь неясное очертание того, что бросается в глаза в созревшем растении; и однако в зародыше содержалось уже растение. Но того, что бросается в глаза, софисты не желают признавать; они держатся только различия между развившимся и неразвившимся существованием; они забывают о единстве.







Вера неизбежно переходит в ненависть, а ненависть --- в преследование, если сила веры не встречает противодействия, не разбивается о другую, чуждую вере силу, о силу любви, гуманности и чувства справедливости. Вера неизбежно считает себя выше законов естественной морали. Учение веры есть учение об обязанностях по отношению к богу --- высший долг есть вера. Обязанности по отношению к богу превосходят обязанности по отношению к человеку настолько же, насколько бог превосходит человека. Обязанности по отношению к богу неизбежно сталкиваются с общечеловеческими обязанностями. Бог не только мыслится и представляется как существо универсальное, отец людей, любовь, --- такая вера есть вера любви, --- он ещё представляется как личное существо, как существо само по себе. Следовательно, как бог, в качестве существа себе довлеющего, отличается от человека, так и обязанности по отношению к богу отличаются от обязанностей по отношению к людям, и в душе вера обособляется от морали и любви\ddag\let\svthefootnote\thefootnote\let\thefootnote\relax\footnotetext{\ddagхотя веры без добрых дел не бывает, и даже, по выражению Лютера, невозможно отделить дела от веры, как пламя от огня, но, тем не менее, --- и в этом вся суть, --- добрые дела не служат к оправданию перед богом, то есть оправдаться перед богом и достичь блаженства можно <<помимо дел, в силу одной лишь веры>>. Следовательно, вера решительным образом отделяется от добрых дел: только вера имеет значение перед богом, а не добрые дела; только вера доставляет блаженство, а не добродетель; таким образом, только вера имеет субстанциальное, а добродетель лишь акцидентальное значение, то есть только вера имеет религиозное значение, божественный авторитет, а не мораль. Как известно, некоторые даже утверждали, что добрые дела не только не нужны, но даже <<вредны для блаженства>>. Что правда, то правда.}\let\thefootnote\svthefootnote. Нельзя указывать на то, что вера в бога есть вера в любовь, в добро, что вера есть выражение доброго чувства. Нравственные определения исчезают в понятии личности; они становятся второстепенным делом, простыми акциденциями. Сутью дела является субъект, божественное я. Любовь к богу, как к существу личному, носит не нравственный, а личный характер. Множество благочестивых песен дышат любовью только к господу, но в этой любви не обнаруживается ни малейшей искры какой-либо высокой нравственной идеи или настроения.



Для веры нет ничего выше её самой, ибо её объектом является божественная личность. Поэтому она ставит вечное блаженство в зависимость от себя, а не от исполнения общих человеческих обязанностей. Но все, что имеет своим последствием вечное блаженство, неизбежно становится в глазах человека главным делом. Поэтому мораль, которая внутренне подчиняется вере, может и должна подчиняться ей и во внешнем, практическом отношении. Поэтому неизбежны и такие поступки, в которых обнаруживается не только различие, но и противоречие между верой и моралью --- поступки дурные в нравственном отношении, но похвальные в отношении веры, целям которой они наилучшим образом служат. Все спасение заключается в вере; поэтому все зависит от спасительности веры. При опасности для веры подвергаются опасности и вечное блаженство и слава божия. Поэтому вера разрешает все, что способствует её утверждению; ведь вера, в строгом смысле, есть единственное благо в человеке, подобно тому как сам бог есть единственное благое существо, почему первая, высшая заповедь гласит: <<веруй!>>*\let\svthefootnote\thefootnote\let\thefootnote\relax\footnotetext{*см. об этом, например, у Бемера (Ius. Eccles., lib. \rom{5}.Tit. \rom{7}, par. 32, par. 44).}\let\thefootnote\svthefootnote.



Но так как между верой и нравственным настроением нет никакой естественной, внутренней связи и так как вера по существу равнодушна к нравственным обязанностям\dag\let\svthefootnote\thefootnote\let\thefootnote\relax\footnotetext{\dag<<Плачетта де-Фиде говорит: истинную причину неотделимости веры от благочестия нельзя искать в природе вещей. Её надо искать, если я не ошибаюсь, единственно в воле божией>>. Он прав и думает, как и мы, когда выводит эту связь (то есть святости или благочестивого настроения с верою) из благодатной воли божией. Но эта мысль не нова и её уже высказывали наши древние богословы (Эрнести, Vindiciae arbitrii div. Opusc. Theol., p. 297). <<Если кто утверждает, что тот не христианин, кто верует, но не любит, да будет проклят!>> (Concil. Trid., Sess. \rom{6}, de justif. can. 28).}\let\thefootnote\svthefootnote и приносит любовь к человеку в жертву славе божией, то именно поэтому от веры и требуется, чтобы она сопровождалась добрыми делами и проявляла себя актами любви. Вера, безразличная в отношении любви или бессердечная, противоречит разуму, естественному чувству справедливости человека, нравственному сознанию, из которого непосредственно возникает любовь, как закон и истина. Поэтому вера, в противоречии со своей сущностью, в себе самой ограничивается моралью: вера, не творящая добра, не проявляющая себя актами любви, не есть истинная, живая вера. Но это ограничение не вытекает из самой веры. Независимая от веры сила любви предписывает ей законы; ибо здесь критерием истинности веры становится нравственное качество; истина веры делается зависимой от истинности морали --- отношение, противоречащее вере.



Вера делает человека блаженным; но несомненно она не внушает ему действительно нравственных настроений. Если она исправляет человека и имеет своим результатом моральные настроения, то это исходит лишь из внутреннего, не зависящего от веры убеждения в неопровержимой истинности морали. Мораль, а никоим образом не вера, взывает к совести: твоя вера ничто, если она не способна тебя исправить. Правда, нет спора, уверенность в вечном блаженстве, в прощении грехов, в благодати, спасающей от всех наказаний, может побудить человека делать добро. Человек, у которого есть эта вера, обладает всем; он блажен\ddag\let\svthefootnote\thefootnote\let\thefootnote\relax\footnotetext{\ddagсм. об этом у Лютера, например, ч. \rom{14}, стр. 286.}\let\thefootnote\svthefootnote; он становится равнодушным к благам этого мира; он не знает зависти, стяжания, тщеславия, чувственных желаний; все земное исчезает перед божественной благодатью и вечным неземным блаженством. Но добрые дела исходят у веры не из любви к самой добродетели. Не любовь, не объект любви, не человек, основа всякой морали, является пружиной его добрых дел. Нет! Он делает добро не ради добра, не ради человека, а ради бога --- из благодарности к богу, который все для него сделал и для которого он, в свою очередь, должен сделать все, что только находится в его власти. Он перестает грешить, ибо, грех оскорбляет бога, его спасителя, его господа и благодетеля*\let\svthefootnote\thefootnote\let\thefootnote\relax\footnotetext{*<<Поэтому добрые дела должны сопутствовать вере из благодарности к богу>> (Apol. der Augsb. Konf. Art., 3). <<Чем могу я воздать тебе за твои благодеяния? есть нечто, что тебе угодно, а именно --- чтобы я обуздывал вожделения плоти, дабы они не разжигали сердце мое новыми грехами>>. <<Когда грехи начинают смущать меня, я не теряюсь, ибо взор на распятого Иисуса убивает их силу>> (Gesangbuch der evangel. Bruedergemeinen).}\let\thefootnote\svthefootnote. Понятие добродетели заменяется здесь понятием искупительной жертвы. Бог принес себя в жертву человеку, поэтому и человек должен жертвовать собою богу. Чем крупнее жертва, тем лучше и деяние. Чем больше деяние противоречит природе человека, чем больше самоотречение тем выше добродетель. Особенно католицизм развил и осуществил это исключительно отрицательное понятие добра. Его высшее моральное понятие есть понятие жертвы, отсюда высокое значение отрицания половой любви --- девство. Целомудрие, или, вернее девство, есть характерная добродетель католической веры. Оно не основано на природе и есть чрезвычайная, самая трансцендентная, фантастическая добродетель, добродетель супранатуралистической веры; оно есть высшая добродетель для веры, но не добродетель сама по себе. Следовательно, вера считает добродетелью то, что само по себе, по своему содержанию, не есть добродетель; ей, стало быть, неведомо чувство добродетели; она необходимо должна снижать истинную добродетель, потому что превозносит мнимую добродетель и не руководится никаким иным понятием, как только понятием отрицания, противоречия человеческой природе.





Итак, деяния истории христианской религии соответствуют христианству, хотя и противоречат любви; поэтому противники догматического христианства совершенно правы, когда они винят его за жестокие поступки христиан; однако они в то же время противоречат и христианству, так как христианство есть не только религия веры, но и религия любви, и обязывает нас не только верить, но и любить. Значит, бессердечные деяния, внушённые ненавистью к еретикам, одновременно соответствуют и противоречат христианству? Как же это возможно? Да, христианство санкционирует в одно и то же время как поступки, вытекающие из любви, так и поступки, вытекающие из веры без любви. Если бы христианство сделало законом только любовь, то приверженцы его были бы правы, и христианство нельзя было бы обвинять во всех ужасах истории христианской религии; если бы оно сделало законом только веру, то и упреки людей неверующих были бы справедливы безусловно, без всяких ограничений. Но христианство не отдалось всецело любви; оно не поднялось до той высоты, чтобы понимать любовь абсолютно. И оно не могло достичь этой свободы, раз оно есть религия --- и поэтому любовь осталась в подчинении у веры. Любовь есть экзотерическое, а вера --- эзотерическое учение христианства --- любовь есть только мораль, а вера --- религия христианской религии.

Бог есть любовь. Это положение есть высший принцип христианства. Но противоречие между верой и любовью заключено уже и в этом положении. Любовь есть только предикат, а бог --- субъект. Чем же является этот субъект в отличие от любви? Я должен по необходимости так ставить вопрос и делать это различие. необходимость различия отпала бы лишь в том случае, если бы имело силу обратное положение: любовь есть бог, любовь есть абсолютное существо. В положении <<бог есть любовь>> субъект является тьмою, в которой прячется вера; а предикат --- светом, которым впервые освещается сам по себе темный субъект. В предикате я проявляю любовь, а в субъекте --- веру. Любовь не наполняет всего моего духа: я оставляю ещё место и для нелюбви, когда я мыслю бога как субъект в отличие от предиката. Поэтому я не могу не терять из виду или мысль о любви, или мысль о субъекте и должен жертвовать то любовью ради личности бога, то личностью бога ради любви. История христианства достаточно подтверждает это противоречие. Католицизм с особенным воодушевлением превозносил любовь как божественную сущность, так что у него в этой любви совершенно исчезала личность бога. Но в то же время в одной и той же душе он жертвовал любовью ради величия веры. Вера зиждется на самостоятельности бога, а любовь уничтожает её. <<Бог есть любовь>>, что значит: бог есть ничто сам по себе; кто любит, тот поступается своей эгоистичной самостоятельностью; он обращает то, что любит, в неотъемлемую сущность своего бытия. Но когда я погружаюсь в глубину любви, во мне опять всплывает мысль о субъекте и нарушает гармонию божественной и человеческой сущности, которую установила любовь. Выступает вера со своими притязаниями и оставляет на долю любви только то, что принадлежит вообще предикату в обыкновенном смысле. Она не позволяет любви свободно и самостоятельно развиваться; она делает себя сущностью, главным делом, фундаментом. Любовь веры есть только риторическая фигура, поэтическая фикция веры --- вера в экстазе. Когда же вера начинает приходить в себя, тогда и от любви ничего не остается.

Это теоретическое противоречие должно было неизбежно проявиться и практически. Неизбежно, --- ведь любовь в христианстве замарана верою, она не берется свободно и в чистом виде. Любовь, ограниченная верой, не подлинная любовь\dag\let\svthefootnote\thefootnote\let\thefootnote\relax\footnotetext{\dagединственное не противоречащее существу любви ограничение есть самоограничение любви разумом, интеллектом. Любовь, отвергающая суровый закон разума, есть любовь ложная в теоретическом и гибельная в практическом отношении.}\let\thefootnote\svthefootnote. Любовь не знает закона вне себя самой; она божественна сама по себе; она не нуждается в благословении веры; она может быть обоснована только самой собой. Любовь, скованная верой, есть любовь узкая, ложная, противоречащая понятию любви, т.е. себе самой, любовь лицемерная, поскольку она в себе прячет зародыш религиозной ненависти; она добра только до тех пор, пока не задевается вера. В этом противоречии с собой она оказывается во власти дьявольских софизмов, чтобы сохранить вид любви, к каким прибегал, например, Августин в своей Апологии гонения на еретиков. Любовь, ограниченная верой, не находит для себя противоречия в тех деяниях, в которых нет любви и которые разрешает себе вера; она толкует акты ненависти, совершающиеся из-за веры, как акты любви. И она по необходимости подпадает под действие этих противоречий, так как противоречием представляется уже сама любовь, ограниченная верой. Мирясь с этим ограничением, она утрачивает свой собственный критерий и свою самостоятельность суждения; она в бессилии поддается внушениям веры.



Здесь мы опять находим пример тому, что многое, о чем буквально в Библии не говорится, тем не менее по духу содержится в ней. Мы находим те же самые противоречия, какие видим у Августина и вообще в католицизме, но только здесь они более определенно высказаны и получили очевидное и поэтому возмутительное выражение. Библия осуждает из-за веры, милует из-за любви. Но она знает только одну, основанную на вере, любовь. Следовательно, здесь мы имеем любовь проклинающую, ненадежную любовь, которая не дает мне никакой гарантии, что она не превратится в ненависть, ведь если я не признаю символа веры, то я выпадаю из сферы царства любви, делаюсь предметом проклятия и гнева божия, так как существование неверных оскорбляет бога и является как бы сучком в его глазу. Христианская любовь не преодолела ада, так как она не преодолела веры. Любовь сама по себе находится вне сферы веры, а вера --- вне сферы любви. Но любовь является не верующей потому, что она не знает ничего более божественного, чем она сама, потому что она верит только в самое себя, как абсолютную истину.

Христианская любовь уже потому есть любовь своеобразная, что она есть любовь христианская и называется христианской. Но в существе любви лежит универсальность. Пока христианская любовь не отрешится от христианства и не признает высшим законом любовь вообще, до тех пор она будет оскорблять чувство правды, --- ведь любовь именно и уничтожает всякое различие между христианством и так называемым язычеством, --- до тех пор она будет любовью ненормальной, противоречащей вследствие своего своеобразия существу любви, будет любовью лишенной любви, которая давно уже по справедливости сделалась предметом иронии. Истинная любовь себе довлеет; она не нуждается ни в особом титуле, ни в авторитете. Любовь есть универсальный закон разума и природы --- она есть не что иное, как осуществление единства рода через единомыслие. Любовь, основанная на имени какого-нибудь лица, возможна только под условием, что с этой личностью связаны суеверные представления, все равно, будут ли они религиозного или умозрительного порядка. Но с суеверием всегда бывает связан дух сектантства и сепаратизма, а с сепаратизмом --- фанатизм. Любовь может корениться только в единстве рода, в единстве интеллекта и в природе человечества; только тогда она есть основательная, принципиально выдержанная, свободная и надежная любовь, ведь тогда она опирается на источник любви, из которого исходила и любовь Христа. Любовь Христа была сама любовью производной. Он любил нас не по собственному произволу и побуждению, а в силу природы человеческой. Если любовь опирается на личность Христа, то эта любовь есть особая, обусловленная признанием его личности, а не та, которая покоится на своем собственном основании. Потому ли мы должны любить друг друга, что Христос нас любил? Но такая любовь была бы эффектацией и подражанием. Тогда ли любовь наша искренна, когда мы любим Христа? Но Христос ли причина любви? Или он скорее апостол любви? Не есть ли основа его любви единство человеческой природы? Должен ли я любить Христа больше, чем человечество? Но не будет ли такая любовь призрачной? Могу ли я преодолеть сущность рода: любить нечто более высокое, чем человечество? Любовь облагородила Христа; чем он был, тем его сделала только любовь; он не был собственником любви, каким он является во всех суеверных представлениях. Понятие любви есть понятие самостоятельное, которое я не заимствую из жизни Христа; напротив, я признаю эту жизнь только потому и в той мере, в какой она совпадает с законом, с понятием любви.

Исторически это доказывается уже тем, что идея любви вовсе не возникла впервые с христианством и не вошла вместе с ним в сознание человечества, и потому не есть исключительно христианская идея. Царство политики, объединявшее человечество несвойственным ему способом, должно было распасться. Политическое единство есть единство насильственное. Деспотизм Рима должен был обратиться на самого себя и разрушиться. Но именно благодаря этому гнету политики человек совершенно освободился из тисков политики. На место Рима стало понятие человечества, и вместе с тем понятие любви заняло место понятия господства. Даже иудеи смягчили свой полный ненависти религиозный фанатизм под влиянием гуманного начала греческой культуры. Филон восхваляет любовь, как наивысшую добродетель. В понятии человечества лежало начало разрешения национальных разногласий. Мыслящий дух ещё раньше преодолел проблему гражданской и политической дифференциации человечества. Аристотель, правда, отличает человека от раба, но раба, как человека, уже ставит на одну ступень с господином, допуская между ними даже дружбу. Среди рабов были даже философы. Эпиктет, раб, был стоиком; Марк Аврелий, император, также был стоиком. Так сближала людей философия. Стоики учили, что человек рожден не ради себя, а ради других, т. е. рожден для любви\ddag\let\svthefootnote\thefootnote\let\thefootnote\relax\footnotetext{\ddagтому же учили и перипатетики, но они обосновывали любовь ко всему человечеству не особым, религиозным началом, а естественным, то есть всеобщим, разумным принципом.}\let\thefootnote\svthefootnote, --- изречение бесконечно более содержательное, чем знаменитые слова Марка Аврелия, предписывающие любить врагов. Практическим принципом стоиков является начало любви. Мир представлялся им как общий город, а люди как сограждане. Например, Сенека в самых возвышенных изречениях восхваляет любовь, милосердие, гуманность, особенно по отношению к рабам. Так исчезли политический ригоризм, равно патриотическая узость и ограниченность.



Своеобразным проявлением этих гуманных стремлений --- простонародным, популярным и потому религиозным и притом наиболее напряженным проявлением этого нового начала было христианство. Что в других местах определилось на пути культуры, то здесь получило выражение в религиозном чувстве, как деле веры. Этим христианство опять превратило всеобщее единство в частное, любовь --- в дело веры, и тем самым поставило себя в противоречие со всеобщей любовью. Единство не было сведено к своему первоисточнику. Национальные различия исчезли; но вместо них появилось теперь различие веры, противоположность христианского и нехристианского, и эта противоположность раскрылась в истории резче и с большей ненавистью, чем национальная рознь.

Всякая любовь, основанная на сепаратизме, противоречит, как сказано, сущности любви, которая не терпит никаких ограничений и преодолевает всякую обособленность. Мы должны любить человека ради человека. Человек является предметом любви, потому что он есть самоцель, разумное и способное к любви существо. Это есть закон рода, закон разума. Любовь должна быть непосредственной любовью, и только непосредственная любовь есть любовь. Но если я между другим и мною, осуществляющим род в своей любви, вклиниваю представление личности, в которой уже осуществлён род, то этим я уничтожаю сущность любви и нарушаю единство представлением третьего существа, находящегося вне нас; ведь это другое существо является объектом моей любви не ради себя, т. е. не ради своей сущности, а потому только, что имеет сходство или нечто общее с этим прообразом, здесь снова выступают на первый план все противоречия, какие мы находим в личности бога, где понятие личности устанавливается в сознании и чувстве само по себе, вне того качества, которое обращает её в личность, достойную любви и почитания. Любовь есть субъективное существование рода, подобно тому как разум является его объективным существованием. В любви, в разуме исчезает потребность иметь посредника. Сам Христос есть не что иное, как только символ, под которым народному сознанию представлялось единство рода. Христос любил людей: он хотел всех их осчастливить и объединить без различия пола, возраста, состояния и национальности. Христос есть любовь человечества к самому себе, как образ --- согласно развитой природе религии --- или как лицо, но такое лицо, которое понимается как религиозный объект и имеет лишь значение образа, --- лицо только идеальное. Поэтому отличительным признаком его учеников служит любовь. Но любовь, как сказано, есть не что иное, как проявление, осуществление единства рода в единодушии. Род не есть только мысль; он существует в чувстве, в настроении, в энергии любви. Род возбуждает во мне любовь. Исполненное любви сердце есть сердце рода. Итак, Христос есть сознание любви, сознание рода. Все мы должны быть едины во Христе. Христос есть осознание нашего единства. Таким образом, кто любит человека ради человека, кто возвышается до любви рода, до всеобщей любви, соответствующей сущности рода*\let\svthefootnote\thefootnote\let\thefootnote\relax\footnotetext{*действенная любовь есть и, естественно, должна быть всегда любовью особой, ограниченной, то есть направленной на нечто ближайшее. Но она по природе своей есть любовь универсальная: она любит человека ради человека, во имя рода. Напротив, христианская любовь, как таковая, уже по природе своей является исключительной любовью.}\let\thefootnote\svthefootnote, тот --- христианин, даже сам Христос. Он делает, что делал Христос, что делало Христа Христом. Следовательно, где сознание рода возникает как род, там уже нет Христа, но остается его истинная сущность, ибо он был лишь заместителем, образом сознания рода.







\chapter{Заключение}


Рассмотренное нами противоречие между верою и любовью вынуждает нас практически, наглядно возвыситься над христианством и над специфической сущностью религии вообще. Мы доказали, что содержание и предмет религии совершенно человеческие, доказали, что теологическая тайна есть антропология, а тайна божественной сущности есть сущность человеческая. Но религия не сознает человеческого характера своего содержания; она даже противополагает себя началу человеческому, или, по крайней мере, она не признает, что её содержание человечно. Поэтому необходимый поворотный пункт истории сводится к открытому признанию, что сознание бога есть не что иное, как сознание рода, что человек может и должен возвыситься над пределами своей индивидуальности или личности, но не над законами и существенными определениями своего рода, что человек может мыслить, желать, представлять, чувствовать, верить, хотеть и любить, как абсолютное, божественное существо, --- только человеческое существо*\let\svthefootnote\thefootnote\let\thefootnote\relax\footnotetext{*со включением природы; ведь как человек принадлежит к существу природы, --- это важно для опровержения вульгарного материализма, --- так и природа принадлежит к существу человека; это служит опровержением субъективного идеализма, этой тайны нашей <<абсолютной>> философии, по крайней мере в отношении к природе. Только чрез тесное соединение человека с природой можем мы преодолеть супранатуралистический эгоизм христианства.}\let\thefootnote\svthefootnote.



Поэтому наше отношение к религии не является только отрицательным, но критическим; мы лишь отделяем истинное от ложного, --- хотя, конечно, отделенная от лжи истина является всегда, как истина новая, существенно отличная от старой истины. Религия есть первое самосознание человека. Религии потому и священны, что они --- предания первоначального сознания. Но мы уже показали, что то, что для религии является первым, то есть богом, на самом деле, согласно свидетельству истины, является вторым, так как бог есть только объективированная сущность человека; а что религия признает вторым, то есть человека, мы должны установить и признать как первое. Любовь к человеку не должна быть производной; она должна стать первоначальной. Только тогда любовь будет истинной, священной, надежной силой. Если человеческая сущность есть высшая сущность человека, то и практически любовь к человеку должна быть высшим и первым законом человека. Человек человеку бог --- таково высшее практическое основа начало, таков и поворотный пункт всемирной истории. Отношение ребенка к родителям, мужа к жене, брата к брату, друга к другу, вообще человека к человеку, короче, моральные отношения сами по себе суть истинно религиозные отношения. Вообще жизнь в своих существенных отношениях --- всецело божественной природы. Но свое религиозное освящение она получает вовсе не от благословения священника. Религия стремится сделать предмет священным, исключительно своим, по существу ей свойственным, внешним привнесением; тем самым она провозглашает себя исключительной священной силой; кроме себя она знает лишь земные, небожественные отношения; поэтому она и привходит к ним, чтоб своим присутствием их освятить и благословить.

Но брак, конечно, как свободный союз любви\dag\let\svthefootnote\thefootnote\let\thefootnote\relax\footnotetext{\dagда, только как свободный союз любви; ибо такой брак, узы которого являются лишь внешним ограничением, а не добровольным, самоудовлетворённым самоограничением любви, короче, всякий брак невольный, нежеланный, несамоудовлетворенным не есть брак истинный и, следовательно, моральный.}\let\thefootnote\svthefootnote, --- священен сам по себе, по природе заключенного здесь союза. Только тот брак есть брак религиозный или истинный, который соответствует сущности брака, любви. Так обстоит дело и в отношении всех других нравственных связей. Они только там являются моральными, они только там имеют нравственный смысл, где они сами по себе почитаются религиозными. Подлинная дружба существует только там, где дружба соблюдается с религиозной добросовестностью, как человек верующий соблюдает и хранит достоинство своего бога. Да будет же для тебя священна дружба, священна собственность, священен брак, священно благо каждого человека, но да будут они для тебя священны сами по себе!



В христианстве моральные законы понимаются как заповеди божии; нравственность сделана критерием религиозности; но тем не менее мораль имеет значение лишь подчиненное; сама по себе она не имеет значения религии. Это последнее значение приписывается только вере. Над моралью витает бог, как отличное от человека существо, которому принадлежит все лучшее, тогда как на долю человека приходится одно только падение. Все помыслы, которые должны были бы быть посвящены жизни и человеку, все лучшие силы свои человек отдает этому лишенному потребностей существу. Действительная причина обращается в безразличное средство; лишь представленная, лишь воображённая причина становится истинной, действительной причиной. Человек благодарит бога за благодеяния, которые оказал ему его же ближний, даже с ущербом для себя. Благодарность, приносимая им своему благодетелю, есть благодарность только призрачная, она относится не к нему, а к богу. Человек благодарен богу, но не благодарен человеку\ddag\let\svthefootnote\thefootnote\let\thefootnote\relax\footnotetext{\ddag<<Бог творит благо через начальство, господ и всякую тварь, вследствие чего народ бывает привержен к твари, а не к творцу и не восходит через них к своему создателю. Отсюда произошло, что язычники обращали в богов своих царей\dots Ибо человек или не может, или не хочет понять, что всякое деяние, или благо приходит от бога, а не от твари, хотя она и является средством, которым бог действует, помогает и наделяет нас>> (Лютер, ч. \rom{4}, стр. 237).}\let\thefootnote\svthefootnote. Так гибнет нравственное настроение в религии! Так жертвует человек человеком богу! Кровавое человеческое жертвоприношение есть на самом деле только грубо чувственное выражение сокровенной тайны религии. Где приносились богу кровавые человеческие жертвы, там эти жертвы считались наивысшей жертвой, а чувственная жизнь --- наивысшим благом. Поэтому человек приносил жизнь в жертву богу лишь в необычайных случаях и верил, что он оказывает этим богу величайшую честь. Если христианство, по крайней мере в наше время, не приносит больше кровавых жертв своему богу, то это происходит, если откинуть другое основание, только потому, что человеческая жизнь не считается уже наивысшим благом. Вместо нее жертвуют богу душой, настроением, ибо они считаются благом более высоким. Но общее между этими жертвами заключается в том, что человек в религии приносит в жертву обязанности человеческие, --- как, например, обязанность уважать жизнь ближнего или быть благодарным, --- обязанностям религиозным, отношение к человеку --- отношению к богу. Христиане, благодаря понятию вседовольства божия, являющегося лишь предметом чистого поклонения, устранили во всяком случае многие нелепые и дикие представления. Но это вседовольство есть только абстрактное, метафизическое понятие, которое никоим образом не оправдывает особой сущности религии. Потребность поклонения только с одной стороны охлаждает, как всякая односторонность, религиозное чувство; поэтому, чтобы установить взаимность, необходимо было, если не прямо на словах, то на деле, допустить в боге определения, отвечающие субъективным потребностям. Все действительные определения религии покоятся на взаимности*\let\svthefootnote\thefootnote\let\thefootnote\relax\footnotetext{*<<Я прославлю прославляющих меня, а бесславящие меня будут посрамлены!>> (Первая книга Самуила, 2:30). <<О преблагий отец! Даже низкий червь, достойный вечного презрения, исполнен уверенности, что ты любишь его, так как он чувствует, что он любит, или, вернее, так как он предчувствует, что ты любишь его. Следовательно, всякий, кто уже любит, не сомневается в том, что и его любят>> (Бернард ad Thomam, Epist. 107). Прекрасное и очень важное изречение. Если я не заступаюсь за бога, то и бог не заступится за меня; если я не люблю, то и меня не любят. Страдательный залог является сознающим себя действительным залогом, объект является сознающим себя субъектом. Любить --- значит быть человеком, а быть любимым --- значит быть богом. Я любим, говорит бог; я люблю, говорит человек. Лишь позднее это отношение изменяется, и страдательный залог обращается в действительный.}\let\thefootnote\svthefootnote. Религиозный человек мыслит о боге, потому что бог мыслит о нем, он любит бога, потому что бог прежде возлюбил его и т.д. Бог ревнует к человеку --- религия ревнует к морали\dag\let\svthefootnote\thefootnote\let\thefootnote\relax\footnotetext{\dag<<И сказал господь Гедеону: народа с тобой слишком много, не могу я предать мидианитян в руки их, чтобы не возгордился Израиль предо мною и не сказал: <<моя рука спасла меня>>, то есть <<Израиль не должен присваивать себе то, что мне принадлежит>> (Richter, 7, 2). <<Так говорит господь: проклят человек, надеющийся на людей. Благословен человек, который надеется на господа и которого упование --- господь>> (Иеремия, 17: 5 и 7). <<Бог не требует наших денег, тела и имущества, все это он предает кесарю (т. е. представителю мира и государства) и через кесаря --- нам. Лишь сердце, которое есть высшее и лучшее в человеке, оставил он ему, и это сердце надо отдать богу, если мы верим в него>> (Лютер, ч. \rom{16}, стр. 505).}\let\thefootnote\svthefootnote; она высасывает из морали лучшие силы; она отдает человеку человеческое, а богу --- божеское. И богу принадлежит истинное, полное душевности настроение, ему принадлежит сердце.







Если в те времена, когда религия была священной, брак, собственность и государственные законы встречали уважение к себе, то основание к тому было не в религии, а в том первоначальном, естественно-моральном и правовом сознании, которому правовые и нравственные отношения казались священными как таковые. Если право не является священным само по себе, то оно тем более не станет священным чрез религию. Собственность стала священной не потому, что она представлялась божественным учреждением, а потому, что она считалась сама по себе священной, она и стала рассматриваться как учреждение божественное. Любовь не потому священна, что она есть предикат бога, но она является предикатом бога потому, что она божественна сама по себе. Язычники поклонялись свету и родникам не потому, что это --- дары божии, но потому, что они сами по себе представлялись человеку как нечто благодетельное, потому что они услаждали страждущего; за эти прекрасные их качества и воздавалась им божеская честь.

Где мораль утверждается на теологии, а право --- на божьих постановлениях, там можно оправдать и обосновать самые безнравственные, несправедливые и позорные вещи. Я могу обосновать мораль на богословии лишь в том случае, если я уже определяю божественную сущность помощью морали. В противном случае у меня не будет никакого критерия нравственного и безнравственного, а будет лишь безнравственное, произвольное основание, откуда я могу вывести все что угодно. Таким образом, если я хочу обосновать мораль на боге, я должен её уже предположить в боге, то есть я могу обосновать мораль и право, короче, все существенные отношения лишь ими самими, и я их обосновываю правильно, согласно истине, лишь тогда, когда обосновываю их ими самими. Полагать что-либо в боге или выводить из бога --- значит уклоняться от испытующего разума и, не отдавая себе отчета, устанавливать что-либо как нечто несомненное, неприкосновенное и священное. Поэтому, если и не прямо злая, коварная цель, то во всяком случае самоослепление лежит в основе всех определений морали и права с помощью богословия. Если смотреть на право серьезно, то ему не нужно никакого поощрения или помощи свыше. Нам не нужно христианского государственного права: мы нуждаемся лишь в разумном, справедливом, человеческом государственном праве. Все справедливое, истинное и доброе везде имеет в себе самом, в своем качестве, основание своей святости. Там, где относятся к морали серьезно, там она уже сама по себе почитается божественной силой. Если мораль не обоснована на себе самой, тогда не существует внутренней необходимости для морали, и она отдается на безграничный произвол религии.

Таким образом в вопросе об отношении самосознающего разума к религии речь идет только об уничтожении иллюзии --- иллюзии не безразличной, но, напротив, действующей очень вредно на человечество и отнимающей у человека не только силу действительной жизни, но и понимание истины и добродетели; ведь даже любовь, как самое искреннее, внутреннее настроение, становится благодаря религиозности лишь мнимой, иллюзорной любовью, ибо религиозная любовь любит человека только ради бога, то есть она лишь призрачно любит человека, а на самом деле любит только бога.

Если мы, как сказано, изменим в обратном порядке религиозные отношения и то, что религия считает средством, будем неизменно рассматривать как цель и то, в чем она видит лишь нечто подчиненное, побочное и обусловленное, возвысим до значения причины, тогда мы разрушим иллюзию и осветим вопрос непомрачённым светом истины. Таинства крещения и причастия, эти существенные, характерные символы христианской религии, помогут нам утвердить эту истину и сделать её очевидной.

Вода крещения является для религии только средством, с помощью которого святой дух сообщается человеку. Но это определение ставит религию в противоречие с разумом и истинной природой вещей. С одной стороны, естественное свойство воды как бы играет здесь некоторую роль, но с другой --- оно не имеет никакого значения, и вода представляется чисто произвольным орудием божией благодати и всемогущества. Но мы освободимся от этих и других невыносимых противоречий и уразумеем истинное значение крещения лишь при взгляде на него как на указание значения воды для человека. Крещение должно представлять нам чудесное и вместе с тем естественное действие воды на человека. И действительно, вода оказывает не только физическое, но также нравственное и интеллектуальное влияние на человека. Вода очищает человека не только от телесной грязи, но в воде спадает у него также и слепота с глаз: он видит, он мыслит яснее; он чувствует себя свободнее; вода угашает пыл страстей. Сколько святых прибегали к помощи естественных свойств воды, чтобы преодолеть наваждение дьявола! В чем отказывала им благодать, то давала им природа. Вода оказывает услуги не только диэтетике, но и педагогике. Опрятность, омовение --- это первая, хотя и низшая добродетель\ddag\let\svthefootnote\thefootnote\let\thefootnote\relax\footnotetext{\ddagочевидно, христианское крещение водою есть только пережиток древних естественных религий, где, как, например, у персов, вода была религиозным средством очищения (S. Rhode, Die нeilige Sage ets., p. 305, 425 u f.). Но здесь крещение водою имело более истинный и, следовательно, более глубокий смысл, чем у христиан, так как оно опиралось на естественное свойство и значение воды. Но, разумеется, наш умозрительный и богословский супранатурализм не понимает этих простых воззрений на природу древних религий. Поэтому, если персы, индусы, египтяне и евреи обратили физическую чистоплотность в религиозную обязанность, то в этом отношении они были гораздо более благоразумны, чем христианские святые, которые в физической нечистоплотности видели и осуществляли сверхъестественное начало своей религии. Сверхъестественность в теории становится противоестественностью на практике. Сверхъестественность есть только эвфемизм для противоестественности.}\let\thefootnote\svthefootnote. В холодной воде потухает пыл эгоизма. Вода есть самое естественное и первое средство для сближения с природой. Водяная баня есть как бы химический процесс, в котором наше я растворяется в объективной сущности природы. Вышедший из воды человек является новым, возрождённым человеком. Учение, что мораль ничего не значит без средств благодати, правильно в том случае, если мы на место воображаемых сверхъестественных средств благодати поставим естественные средства. Мораль бессильна без природы, она должна опираться на простейшие естественные средства. Глубочайшие тайны скрываются во всем обычном и повседневном, игнорируемом супранатуралистической религией и умозрением, причем действительные тайны приносятся в жертву тайнам иллюзорным: так здесь, например, действительное чудесное свойство воды приносится в жертву её воображаемой чудесной силе. Вода есть самое простое из средств благодати или врачевания от болезней как души, так и тела. Но вода действует только тогда, когда ею пользуются часто и правильно. Крещение, как единичный акт, есть или совершенно бесполезное, или, если с ним соединяются надежды на реальный эффект, вполне суеверное учреждение. Напротив, оно представляется разумным и почтенным учреждением, если в нем символизируется и прославляется моральная и физическая целебная сила воды и природы вообще.



Но таинство воды нуждается ещё в дополнении. Вода, как всеобщий элемент жизни, напоминает нам о нашем происхождении от природы, и это напоминание роднит нас с растениями и животными. При крещений водою мы преклоняемся перед могуществом чистых сил природы; вода есть материал для естественного равенства и свободы, зеркало золотого века. Мы, люди, отличаем себя от растительного и животного мира, который мы вместе с неорганическим царством обозначаем одним общим названием природы, --- отличаем себя от природы. Поэтому мы должны ознаменовать ещё и это наше существенное отличие. Символами этого нашего отличия служат вино и хлеб. Вино и хлеб по своему веществу суть продукты природы, а по своей форме --- продукты человека. Если мы заявляем при помощи воды, что человек бессилен без природы, то с помощью вина и хлеба мы утверждаем, что природа, по крайней море в области духовной, бессильна без человека; природа нуждается в человеке, как человек в природе. В воде прекращается человеческая, духовная деятельность; а в вине и хлебе она достигает наслаждения собою. Вино и хлеб суть сверхъестественные продукты --- в единственно возможном и истинном смысле, не противоречащем разуму и природе. Если мы в воде поклоняемся чистой силе природы, то в вине и хлебе мы поклоняемся сверхъестественной силе духа и сознания человеческого. Поэтому этот праздник вина и хлеба доступен только человеку зрелому, с развитым сознанием, а крещение доступно уже и детям. Но вместе с тем мы прославляем здесь и истинное соотношение духа и природы: природа дает материю, а дух --- форму. Праздник крещения водою вливает в нас благодарность к природе, а праздник хлеба и вина --- благодарность к человеку. Вино и хлеб принадлежат к древнейшим изобретениям. Вино и хлеб объективируют, символизируют ту истину, что человек есть бог и спаситель.

Еда и питье составляют мистерию причастия --- еда и питье на самом деле сами по себе являются актом религиозным, по крайней мере должны быть им*\let\svthefootnote\thefootnote\let\thefootnote\relax\footnotetext{*<<Еда и питье --- дело нетрудное и самое любезное для людей, даже самым радостным делом на свете являются еда и питье, и, например, существует поговорка ,,перед едой не пляшут`` или ,,на сытом брюхе сидит тяжелая башка``. В итоге еда и питье есть любезное и необходимое дело, которому люди легко и быстро научаются. Такое же приятное и необходимое дело предлагает нам и господь наш Христос, когда говорит: Я приготовил вам веселое и приятное пиршество, я не хочу возложить на вас тяжелого дела\dots Я учредил для вас тайную вечерю>> и т. д. (Лютер, ч. \rom{16}, стр. 222).}\let\thefootnote\svthefootnote. Поэтому вспоминайте при каждом куске хлеба, утоляющем муки голода, и при каждом глотке вина, веселящем сердце, о том боге, который расточает эти благодатные дары --- о человеке! Но из-за благодарности к человеку не забывайте о благодарности к природе! Не забывайте, что вино есть кровь, а хлеб есть плоть растений, которые ради блага вашего существования пожертвовали собою! Помните, что растение символизирует сущность природы, которая самоотверженно жертвует собою ради нашего наслаждения! Итак, не забывайте о благодарности, которой вы обязаны естественным свойствам хлеба и вина! И если покажется вам смешным, что я называю религиозным актом еду и питье, эти обычные, повседневные акты, которые совершаются людьми без чувства и мысли, то вспомните, что и принятие даров для многих представляется бессмысленным и бездушным актом, так как оно часто совершается; и чтоб уразуметь религиозный смысл вкушения хлеба и вина, представьте себе, что этот столь повседневный акт неестественно и насильственно прерывается. Голод и жажда разрушают не только физические, но также и духовные и нравственные силы человека; они лишают его человеческого образа, лишают рассудка и сознания. О, если б ты испытал подобную нищету и бедствие, то как бы ты стал благословлять и хвалить естественные свойства хлеба и вина, опять даровавших тебе твою человечность и твой ум! Поэтому стоит изменить обычный ход вещей, чтобы придать обычному значение необычного и озарить жизнь, как таковую, религиозным смыслом. Поэтому да будет нам священным хлеб, священно вино, а также священна вода! Аминь.

\bigskip
\hfill\emph{Людвиг Фейербах 1841.}












\part{Приложение}

\chapter{Объяснения, примечания, цитаты}

\emph{Сознание бесконечной сущности есть не что иное, как сознание человеком бесконечности своей собственной сущности, или в бесконечной сущности, предмете религии для человека объектом является лишь его собственная бесконечная сущность.}

<<Бог, --- говорит св. Фома Аквинский, --- не есть тело. Всякое тело конечно. Но мы можем в уме и воображении выйти за пределы всякого конечного тела. Если б Бог был телом, то наш ум и воображение могли бы мыслить что-нибудь выше Бога и это было бы противоречием>> (Summa contra gentiles, lib. \rom{1}, с. 20). <<Небо и ангелы обладают конечными силами, поэтому они не могут наполнить бесконечную способность нашего духа к постижению, поп potevunt mentis nos- trae captum imperlere, qui ad recipiendum immensus est>> (И.Л. Вивес, De Verit. Fidei Christ., lib. \rom{1} De fine bom). <<Блаженство есть наше конечное, наше единственное желание. И никакое земное благо не может утолить это желание, ибо все земное ниже человеческого духа. Только бог может удовлетворить желание человека, сделать его блаженным, ибо человеческий дух познает своим рассудком всеобщее (то есть бесконечное) благо и желает его своей волей; но только в Боге обретается всеобщее благо>> (Фома Акв., De Princ. Regim., lib. \rom{1}, с. 8). "Объект человеческого разума есть всеобщая истина (universale verum), то есть истина вообще, или ничем не ограниченная истина; объект же человеческой воли или желания есть всеобщее благо, которое обретается не в существе созданном (то есть конечном), а только в Боге. Таким образом, один только Бог может наполнить человеческую волю>> (он же, Summa theol. sac. Prima Secundae., Qu. \rom{2}, 8). Но если не телесное, земное, то есть не определенное, конечное, а только бесконечное существо есть соответствующий, приличествующий человеческому духу объект и если только оно может наполнить человеческую волю и разум, то очевидно, что для человека в этом бесконечном существе объектом является лишь бесконечность его собственной сущности и бесконечное существо есть для него не что иное, как выражение, явление, откровение или опредмечивание соб- ственной неограниченной сущности человека178.
<<Смертное существо ничего не ведает о бессмертном существе>> (Саллюстий у Г. Гроция, De verit. relig. christ., lib \rom{1}, \S 24, not. 1), т.е. существо, объектом которого служит бессмертное существо, само бессмертно. В бесконечном существе предметом для меня в качестве субъекта, в качестве сущности является лишь то, что есть мой предикат, мое собственное свойство. Бе- сконечное существо есть не что иное, как олицетворенная бесконечность человека; бог есть не что иное, как олицетворенная, представленная как существо божественность человека.

\bigskip

<<Познающие существа отличаются от непознающих тем, что последние имеют лишь свою собственную форму, а первые обладают также формой другого существа, ибо форма или образ познанного заключается в познающем. Отсюда ясно, что природа непознающего существа сравнительно ограниченна и узка, тогда как природа познающего существа гораздо более пространна и обширна. Поэтому философ (Аристотель) говорит: Душа есть некоторым образом все. Но ограниченные формы (какого-нибудь существа) исходят от природы или материи. Поэтому все нематериальное приближается к бесконечности>> (Фома Акв., Summa, P. \rom{1}, Qu. 14, Art. 2). <<Все обладающее бесконечной силой (virtutem infinitem), обладает и бесконечной сущностью; но дух или разум имеет бесконечную силу; ибо он объемлет всеобщее, которое может простираться на бесконечное. Что сила разума может некоторым образом простираться на бесконечное, происходит оттого, что сила разума не есть какая-либо материальная сила или форма, ибо она не есть деятельность какого-либо физического органа>> (там же, Qu. 7, Art. 2). <<Не наша плоть подобна образу божьему, но наша душа, которая свободна и всюду неограниченно парит, заносит нас в самые отдаленные места, видит отсутствующих и в одно мгновение обозревает вселенную>> (Амвросий 179, Hexaem., \rom{6}, с. 8). <<Беспредельность, неограниченность в истинном и собственном смысле слова приличествует только Богу-, но подобием этой неограниченности является человеческий дух, который в одно мгновение видит утро и вечер, север и юг, не- бо и землю, но не на самом деле, а лишь в представлении>> (Феодорит, Quaest. іn Genes., inter. 20). <<В действительности мы должны бесстрашно признать, что человек стоит выше самих богов или по крайней мере он совершенно равен им по силе. Когда небесный Бог спускается на землю, он оставляет за собой пределы неба. Но человек возносится на небо и измеряет его, видит его высоту и глубину и созерцает все мироздание, и, однако, --- и это самое важное - он не покидает землю, когда поднимается на высоты. До такой степени человек может быть всеобъемлющим. Поэтому мы смело можем сказать, что земной человек есть смертный Бог, а небесный Бог есть бессмертный человек>> (Hermes Trism., Poemander, с. 10, 24, 25). "Способность познавать и действовать у человека не ограничена, как у других животных, она неистощима и бесконечна и потому подобна богу" (Гуго Гр., там же). Бесконечность человеческого существа, которую мы сперва познали лишь косвенно, путем умозаключений, подтверждается в приведенных здесь отрывках, число которых можно было бы увеличить до бесконечности, непосредственно и неопровержимо. Психологическая бесконечность есть основание богословской или метафизической бесконечности. Неизмеримость, не ограниченное временем и пространством бытие, вездесущие бога есть наглядное объективированное вездесущие и неизмеримость силы человеческого представления и воображения.

\bigskip

\emph{Бесконечная или божественная сущность есть духовная сущность человека, которая, однако, обособляется от человека и представляется как самостоятельное существо. Бог есть дух - это значит в действительности: дух есть Бог. Каков субъект, таков и объект, какова мысль, таков и познаваемый объект. Бог как абстрактное, т.е. отвлеченное, нечувственное существо есть объект не чувств или чувственного воображения, а разума, следовательно, он есть только сущность разума, только разум, объективирующий себя как божественное существо.}

"Богу присуще всяческое совершенство, какое где-либо встречается в роде тех или иных существ. Но среди совершенств самым превосходным является разум, ибо благодаря ему существо есть некоторым образом все, так как оно объемлет в себе совершенства всех вещей; следовательно, Бог есть познающее, мыслящее существо" (Фома Акв., Summa с. gent., lib. I, с. 44, 5). "Так как древние называли богом всякую обособленную субстанцию, то отсюда следовало, что наша душа, т.е. разум, посредством которого мы познаем, имеет божественную природу; и некоторые христиане нашего времени, допускающие в нас обособленный (от человека), деятельный разум, буквально утверждают, что деятельный разум и есть Бог. Это мнение могло возникнуть из сходства нашей души с Богом, ибо познание есть прежде всего характерное свойство Бога и не приличествует никакому низшему существу, кроме человека, обладающего душой, поэтому и могло казаться, что душа обладает божественной природой" (он же, lib. II, с. 85). "Мудрец связан с Богом некоторым родством, ибо он занят вопросами разума, а последний и есть сущность самого Бога" (Синезий у Петавия, Teol. dog., Т. I, lib. IV, с. 1, § 1). "Разум составляет .связь между нами и Богом" (R. Moses, там же, Т. ГУ, lib. IV, с. 13, § 8)."Единство между Богом и человеком гораздо теснее, чем между душой и телом, ибо единение между двумя духовными существами бывает большее, чем между существом духовным и физическим. Но Бог духовной природы, так же как и человек духовной природы" (Гуго Виктор., там же, § 14). "Среди всех тварей один только разумный дух может возвыситься до вопроса о Боге, ибо он всего ближе подходит к нему по сходству своей сущности. Поэтому, чем больше разумный дух работает над самопознанием, тем более становится он способным к познанию Бога" (Ансельм, Monolog., с. 66). "Человеческий дух сравним только с самим Богом" (Цицерон, Quest, tusc.). "Сам Бог не может быть мыслим иным, как только свободным, ни с чем не связанным и обособленным от всех смертных сочетаний" (там же). Древние называли вообще душу, дух и разум Богом, тогда как христиане прямо или только на словах отрицают тождество Бога и человеческого духа, а косвенно или на деле утверждают его. Поэтому христиане упрекали язычников, что они обоготворили разум, но сами они делают то же. Разница только в том, что христиане одновременно со способностью человека к разуму или отвлечению боготворили также ее противоположность, чувственное существо, короче, все существо человека, и чрез это - не говоря уже о других указанных в этой книге основаниях - обращали отделенный от человека дух, имевший у древних несомненный характер субъективной, человеческой абстракции, в существо доступное ощущению и материальное, по крайней мере в их воображении. "Созерцательная жизнь, - справедливо говорит Фома Аквинский (в его Exposit. in Cantica Cantic., Parisiis, 1515, fol. 8), - имела у богословов (т.е. христиан) и у философов (т.е. язычников) не одно и то же значение. Философы полагают блаженство, цель созерцания, в мудрости, в упражнении силы мышления, а богословы полагают его больше во вкушении, чем в мышлении, больше в любви и сладости, чем в созерцании. Поэтому если мы захотим прибегнуть к чувствам для обозначения этого различия, то мы можем сказать: созерцательная жизнь философов услаждает зрение и слух, ибо среди всех чувств эти два чувства наиболее содействуют познанию и науке, а христианское созерцание услаждает вкус, обоняние и осязание". При этом следует еще заметить, что христиане потому глав- ным образом и отличают Бога от духа и души человека, что они отождествляют дух и душу с индивидуальным, т.е. действительным, чувственным, физическим существом, тогда как древние мыслили разум, дух в самом человеке как отдельную сущность, как сущность самое по себе. Поэтому когда христиане отрицают, что Бог есть дух или душа человека, то этот дух означает мыслящего человека, а душа - чувствующего, исполненного страстей человека или связанную с телом, чувственно определяемую душу. Так, например, Августин говорит ("Contra Fortun."): "Бог есть одно, а душа нечто другое. Бог неуязвим, непорочен, непроницаем и непогрешим. А душа грешит и т.д. Поэтому если душа есть субстанция Бога, то эта субстанция Бога заблуждается, подвергается порче и т.д." А Фома Аквинский (S. с. gent., lob. II, с. 25): "Бог вечен, а души не было раньше ее тела. Бог неизменяем, а в душе происходят изменения. Бог есть чистая деятельность, а душе свойственны страдания и деятельность, следовательно, душа не может быть частью божественной субстанции". Но разве душа есть нечто иное, чем сам человек? Сама душа точно так же непорочна, непроницаема, непогрешима, как и Бог, ибо, по учению самих же христиан, она неделима, проста, нематериальна, неистребима и бессмертна, короче все, что они в морали отрицают у души, они приписывают ей в метафизике. Метафизические определения, определения сущности божества и души, одни и те же. Это особенно видно в учении о бессмертии, в котором или буквально, или иносказательно душе присваиваются определения божества. Так, Гроций говорит (De verit. rel. chr., lib. I, c. 23): "Природа души ни от чего не зависит, противоположности для души не существует", т.е. она есть абсолютная, бесконечная сущность, ибо и Бог как "всеобщая причина и как универсальная природа" не имеет для себя противоположности, как буквально говорится, например, у Фомы Ак- винского в его комментарии к Дионисию. И поэтому, чтобы доказать существование Бога как отличной от души сущности, христиане отрицают существование души как сущности, отличной от тела и материи, и отождествляют душу с человеком; а чтобы доказать существование бессмертия, они отрицают различие между душой и божеством и обращают душу в независимую, отличную от тела божественную сущность180, следовательно, они признают, что душа Бога есть душа человека, ибо душа в богословии есть только бессмертная душа. "Только признание вечной жизни делает нас истинными христианами" (Августин, De civ. Dei, lib. VI, с. 9). 

Бог не есть тот или иной, твой или мой дух; он есть тот дух, который мыслится или представляется как единичный, единственный, самостоятельный дух. Бог есть вообще индивидуализированное или олицетворенное понятие рода, то есть род, мыслимый существующим как род в отличие от индивидуумов. Бог есть совокупность всех реальностей, это значит: Бог есть совокупность всех родов или понятий рода. В этом отношении различие между язычеством и христианством состоит в том, что у язычников род является лишь мыслимой сущностью, имеющей существование лишь в чувственных, действительных индивидуумах, а у христиан род, как таковой, как мыслимая сущность, имеет самостоятельное существование. Язычники различали мышление и бытие, а христиане отождествляли то и другое. Политеизм покоится на различии, монотеизм - на единстве мышления и бытия, рода и индивидуума. 

"Никто не говорит, что человек или животное есть белизна, ибо белизна существует не сама по себе, но обособляется, или индивидуализируется, в существующем субъекте. Точно так же никто не скажет, что Сократ или человек есть человечество; напротив того, божественная сущность существует отделено сама по себе и индивидуализируется в самой себе (Divina essentia est per se singulariter existens et in se ipsa individuata) (Фома Акв., S. с. gent., lib. I, c. 21). Поэтому божественная сущность делается предикатом бога, так что мы говорим: "Бог есть сущность", т.е., как верно объясняет это место комментатор Франциск из Феррары: "Бог есть божественность". "Абстрактное бытие может быть только едино, подобно тому как и белизна была бы едина, если бы она существовала абстрактно. Но Бог сам есть абстрактное бытие (Deus est ipsum esse abstractum, то есть бытие как бытие, как понятие бытия, которое одновременно мыслится как сущее), ибо он есть свое собственное бытие. Поэтому и может быть только один Бог" (он же, гл. 42). "У существ, составленных из формы и материи, необходимо различаются природа, или сущность, от субъекта, или индивидуума (suppositum), ибо сущность, или природа, объемлет в себе только то, что входит в определение вида или рода. Так, под человечеством разумеется то, что входит в определение человека, ибо этот человек есть человек, и человечество определяется именно тем, что человека делает человеком. Но индивидуальная материя со всеми индивидуализирующими акциденциями не входит в определение рода, ибо в определение человека не входят эта плоть и эти кости, белизна или чернота или вообще нечто подобное; эти акциденции исключаются из понятия человечества и включаются только в то, что составляет человека (т.е. человеческий индивидуум), поэтому отдельный человек заключает в себе нечто такое, чего не имеет человечество, и следовательно, человек и человечество не совсем одно и то же. Но там, где нет соединения формы и материи (т.е. духа и тела, рода и индивидуума), где индивидуализация совершается не через индивидуальную материю, то есть не через эту материю, а формы сами по себе индивидуализированы:, там неизбежно сами формы являются сущими субъектами или индивидуумами, так что в них нет разницы между индивидуумом и сущностью или родом (rion differt suppositium et natura). Но так как Бог не состоит из материи и формы, то поэтому Бог по необходимости есть своя собственная божественность (т.е. сама божественность), своя собственная жизнь (т.е. сама жизнь) и все прочие предикаты" (он же, Summa sacrae theol., P. I, Qu. 3, Art. 3). Это значит, что Бог есть только предикат, но, представляемый как субъект, он есть только понятие рода, которое, однако, мыслится как отдельное существо; поэтому он, будучи лишь существом абстрактным, мыслится нами как действительное, наличное существо. "Поэтому, чем меньше в имени определенности, чем больше носит оно всеобщий и безусловный характер, тем более оно подобает Богу. Следовательно, из всех имен наиболее приличествует Богу имя: "Я есмь сущий" (Исход., 3, 14), просто имя бытия как наиболее всеобщее" (он же, Qu. 13, Art. 11). " Чем спе- циальнее имена, тем больше определяют они соответствующий вид твари" (Он же, Qu. 33, Art. 1. См. об этом также Петавия. De trinit., lib. V, с. 6, § 10). "Когда мы переносим названия рождения и сына с вещей сотворенных на Бога, то мы отделяем мысленно от них все грубое (материальное) и преходящее, как, например, деление субстанции, порядок времени, и сохраняем только одно - общность или однородность сущности и природы. Точно так же, когда мы Бога называем Словом, то мы отбрасываем все, что лежит в понятии этого слова, насколько оно приличествует тварям, как, например, все преходящее и несамостоятельное" (Исидор Пелусиотский у Петавия. De trinit., lib. VII, с. 14, § 6). Это значит: сын, пребывающий в Боге, есть не что иное, как сын in abstractor понятие сына; слово, как является оно в Боге, божественное слово, не есть то или иное слово, высказанное человеком и про- звучащее в воздухе, не есть немецкое или греческое, латинское или еврейское слово, но слово само в себе, слово вообще, родовое понятие, слова, которому, естественно, приличествуют все определения божества: нечувственность и сверхчувственность, вечность, неизменяемость и простота. Поэтому вполне соответствует основному понятию или основной сущности божества и та особенность, что в троице внутренние свойства божества индивидуализированы как лица, как сущности. Бог есть не что иное, как совокупность, как множественность прилагательных, представляемых как существительные, множественность предикатов, абстракций, мыслимых как субъекты, как существа. По той же причине, почему дух, премудрость, промысл, благость, могущество, короче, все всеобщие, отвлеченные от человека и приро- () Л. Фейербах, т. 2 ды понятия превратились в боге в существа, по той же причине отвлеченные свойства отцовства и сыновства воплотились в лица. "Священное писание приписывает богу руки, глаза, сердце и другие органы, чтобы выразить этим известную активность или деятельность Бога, и хотя при этом отстраняется все грубое, несовершенное и плотское, тем не менее собственная, действительная деятельность этих органов ему приписывается. Ибо Бог действительно слышит и видит, желает и мыслит независимо от того, обладает ли он и пользуется ли теми частями тела, которым соответствуют эти понятия или эта деятельность. Точно так же писание говорит, что сын был рожден из утробы отца, ибо хотя у Бога и нет утробы, вообще в нем нет ничего телесного, тем не менее в нем произошло истинное зачатие, истинное рождение которое обозначается словом утроба матери"181 (Петавий, там же, lib. V, с. 7, § 4). В познании того, что Бог есть не что иное, как понятие рода, олицетворенное или индивидуализированное понятие рода, мы имеем ключ ко всем тайнам богословия, разъяснение всей ее необъяснимости и непостижимости, разрешение всех запутанных противоречий и трудностей, над которыми прежде тщетно ломали себе голову богословы и философы. Из этого мы видим, почему о Боге "можно говорить только en g6n6ral, вообще", а на все специальные вопросы, на все вопросы о путях и средствах, богословие дает отрицательные ответы; причина в том, что в понятии рода, как, впрочем, мы уже показали, обсуждая акт творения, исчезают как специальные и индивидуальные определения, те определения, которые вера или богословие приписывают Богу как существующие сами по себе, а не для нашего познания, ибо она представляет это понятие рода как действительное, предметное существо. Из этого мы узнаем, каков истинный смысл бесконечности, причинности, величия и совершенства, а также положительной и отрицательной природы бога. Бог в этом смысле есть существо, о котором все можно утверждать и отрицать, в котором все существует и вместе с тем не существует, в котором, например, цвет вообще представляет все цвета и ни одного из них; Бог бесконечен в том смысле, в каком бесконечен род, не ограниченный определенным местом, временем, индивидуумом и способом существования, ибо "общие понятия, роды (universalia) существуют всюду и все- гда" (Фома Акв., Summa theoL, P. I, Qu. 46, Art. 2). Бог в том же смысле выше человека, в каком цвет вообще выше отдельного цвета, ибо "человечество выше человека" (он же, в предисловии к Exposit. in Dionysii A., divina nomia182), он в том же смысле есть высшее существо, и, как таковое, есть основа и причина всех существ, в каком является им вообще род, когда последний мыслится в отличие от индивидуумов как самостоятельное существо; он в том же смысле есть совершенное существо, в каком является и род в противоположность индивидуумам, ибо цвет вообще вмещает в себя все цвета, тогда как действительный, отдельный цвет всегда лишь один цвет, исключающий все другие, и, следовательно, род есть совокупность всех совершенств, которыми наделены отдельные индивидуумы. "Бог есть само для себя пребывающее бытие (ipsum esse per se subsistens). Поэтому он объемлет в себе все совершенства бытия, так как очевидно, что если нечто теплое не содержит в себе все совершенство теплоты, то это только потому, что теплота распределена (т.е. осуществлена) несовершенным образом и что если бы теплота существовала для себя самой, то она обладала бы всем совершенством теплоты" (Фома Акв., там же, Qu. 4, Art. 2). Отсюда нам становится ясно, как нелепо представлять Бога в образе индивидуума; это столь же нелепо, как нелепо представление, что цвет вообще как отвлеченная от отдельных цветов сущность может быть осуществлен в каком-нибудь одном цвете; напротив, совершенно правильно и необходимо представление о том, что при разложении и низведении божественной сущности на действительные существа, от которых она абстрагирована, она осуществляется в целокупности всех индивидов; ибо Бог определяется и представляется как существо, которое в целом обладает всеми совершенствами и всеми добродетелями, которыми наделены частично все действительные существа. "Хотя бог, - говорит, например, Фома Аквинский в своем комментарии к Дионисию A. (cap. 11), - пребывает в себе самом нераздельно, все же его дары, т.е. его совершенства и силы, распределяются между отдельными созданиями и частично усвояются ими по степени их восприимчивости".

\bigskip

\emph{Бог есть не физиологическое или космическое, а психологическое существо.}

Кто не привносит Бога в природу, тот и не выносит его из нее. До- казательства бытия божия, заимствованные из природы, суть только доказательства невежества и высокомерия человека, который границы своего ума делает границами природы.
 Допуская цели в природе, надо иметь в виду, что цель природы не лежит вне природы или над ней, как и цель глаза, зрение, не лежит вне организации глаза, и потому она не приводит нас к существу вне природы или над ней. Цель в природе не отлична и независима от средства, от свойств органа; природа слышит только посредством уха, видит лишь посредством глаза, мыслит только посредством мозга, тогда как Бог слышит без ушей, видит без глаз, и мыслит без мозга. Откуда берется цель? - спрашивает теист, отделяющий мысленно цель от средств; но я спрашиваю: откуда же берутся средства? Как может от существа, которое мыслит без посредства головы, произойти другое существо, которое мыслит исключительно головой? К чему понадобились нематериальному, без средств действующему всемогущему существу материальные средства? Поэтому заключение от природы к богу, т.е. к отличному от природы, сверхъестественному, духовному существу, как первопричине природы уместно и правильно только там, где человек верит, что можно видеть без глаз и слышать без ушей, где всемогущее, божественное существо является единственной связью между причиной и действием, средством и целью, органом и функцией. "Естественные предметы, - говорит, например, Кальвин, - суть не что иное, как орудия, которые Бог приводит в действие лишь в меру своего произволения и сообразно намеченной им цели. Ни одно творение не имеет такой удивительной и необыкновенной силы, как солнце. Оно своим блеском освещает весь земной шар, оно поддерживает и услаждает своей теплотой всех животных, оно оплодотворяет землю своими лучами... Однако же господь, дабы прославляли только его как истинного творца, прежде чем создать солнце, велел быть свету и наполнить землю всякого рода травами и плодами (Бытие, 1, 3, 11).
Поэтому ни один набожный человек не сделает солнце главной или необходимой причиной вещей, которые существовали еще до сотворения солнца; напротив, он признает в нем только орудие, которым пользуется бог, ибо так он хочет, и он мог бы столь же легко создать все существующее и без помощи солнца, из самого себя" (Jnstit. crist. relig., lib. I, с. 16, sect. 2). Во всяком случае, если б не было природы, не было бы и Бога, но природа есть только условие, а человечество - причина божества. Природа дает только материю для божества, но душу вдохнул в него человек. Так сила рождается из природы, а всемогущество - из человека. Конечно, бытие божие основывается на природе, но сущность Бога - только на человеке. "Два образа, - говорит Гуго в прологе своего комметария к Дионисию Аре- опагиту, - были представлены человеку, чтобы в них он мог видеть невидимое: образ природы и образ благодати. Первый был образом это- го мира, а второй был человечностью Слова. Природа могла, конечно, продемонстрировать, но просветить она не могла; но человечность Спасителя сначала просветила, прежде чем продемонстрировать. Картины природы лишь намекают на творца, а картины благодати показывают нам Бога в действительности: первые он создал только, чтоб мы познали, что он существует, а в последних он показал нам, что он присутствует". Природа, прибавим мы к этим словам Гуго, дает только хлеб и вино, но религиозное или теологическое содержание влагают в них вера, чувство и фантазия. Приписывать природе теологическое или теистическое значение - значит придавать хлебу значение тела, а вину - значение крови. Делать из природы произведение и выражение Бога - значит отнимать у нее субстанцию и оставлять лишь акциденции. "В чувственном, - говорит Фома Акв., - нельзя познать божественное существо как таковое; ибо чувственные создания суть произведения Бога, которые не представляют действия причины соответственным ей (адекватным) образом. Но так как следствия зависят от причины, то через них мы можем познать существование Бога и то, что ему присуще как первопричине всех вещей" (Summa, P.
Внимание!
Если вам нужна помощь в написании работы, то рекомендуем обратиться к профессионалам. Более 70 000 авторов готовы помочь вам прямо сейчас. Бесплатные корректировки и доработки.Узнать стоимость своей работы
%I, Qu. 12, Art. 12). Но простая причинность, будь это первая и всеобщая, не составляет еще божества. Причина есть физическое понятие, хотя, являясь основанием (предпосылкой) божества, и образует понятие вполне отвлеченное и гиперфизическое, ибо понятие это есть не что иное, как олицетворение родового понятия причины. "Естественное познание (т.е. только опирающееся на природу) не приводит к Богу, поскольку он есть предмет блаженства" (там же, Sec. P., sec. Partis Qu. 4, 7). Но только тот Бог, который является предметом блаженства, есть Бог религии, истинный и соответствующий понятию или имени божества. "В природе, - говорит он же, - находятся лишь следы, но не образ божества. Следы лишь показывают, что кто-то приходил, но не дают понятия о качествах его. Образ Бога находится лишь в разумном создании, в человеке" (там же, P. I, Qu. 45, Art. 7). Поэтому вера в сверхъестественное происхождение природы опирается лишь на веру в сверхъестественность человека. Объяснение происхождения природы из сущности, отличной от природы, основывается на невозможности объяснить и вывести из природы человеческое существо, отличное от природы. Бог есть творец природы, ибо человек (с точки зрения религии и богословия) не есть создание природы. Человек (в его собственном понимании не произошел от природы, но тем не менее в нем живет сознание, что он не вечен, что он произошел, возник. Но откуда же он взялся? От Бога, т.е. от существа его же сущности, равного ему, которое* только тем и отличается от него, что оно не возникло. Бог только косвенно, посредственно является творцом природы и лишь потому, что он есть творец или, вернее, отец человека, что он не мог бы быть создателем человека, если б не был творцом природы, в которую, несмотря на всю свою супранатуралистическую сущность, вплетен человек. Следовательно, природа только потому произошла от Бога, что сам человек произошел от Бога, а человек потому божествен по происхождению, что он есть божественное существо, которое, однако, - не касаясь уже того, что вообще в боге он мыслит свое существо как род, а в себе - как индивидуум, в Боге - как неограниченную и бесплотную сущность, а в себе самом - как ограниченную плоть, - он представляет себе как другое существо, ибо сознание его происхождения стоит в противоречии с сознанием или представлением его божественности. Поэтому сознание божественности, сознание: я - создание бога, чадо божие, есть высшее самосознание человека. "Если бы тебя, - говорит Эпиктет, - усыновил царь, твое высокомерие не знало бы пределов. Почему же ты не гордишься сознанием, что ты сын божий?" (Arrian, Epict., lib. I, с. 3). Природа, мир не имеет никакой ценности, никакого интереса для христианина. Христианин думает только о себе, о спасении своей души, или, что то же, о Боге. "Твоей первой и последней мыслью должен быть ты сам, твоим единственным ІҐОМЬІСЛОМ - твое спасение" (De inter, domo, Псевдо-Бернард). "Если ты внимательно присмотришься к себе, то удивишься, что прежде мог думать о чем-нибудь другом" (.Бернард, Tract, de XII. grad. hum. et sup.) "Разве ты сам не являешься для себя лучшим сокровищем" (Боэций, De consol. philos, lib. II, Prosa IV). "Больше ли солнце, чем земля, или оно шириной в фут, светит ли луна своим или отраженным светом - знать все это бесполезно, а не знать - безвредно. В опасности ваше благо: спасение ваших душ" (Лрнобий, Adv. gentes, lib. II, с. 61). "Итак, я спрашиваю: в чем предмет науки? Причины естественных вещей? Какого блаженства могу я ожидать от того, что узнаю об источнике Нила или о бреднях физиков относительно неба?" (Лактанций, Inst, div., lib. Ill, с. 8). "Мы не должны быть любознательны и любопытны (curiosi). Многие считают чем-то важным не познание Бога, а старательное исследование общей физической массы, называемой миром. Душа должна подавлять эту суетную жажду знания, которая в большинстве случаев приводит человека к мысли, что существует только телесное" {Августин, De mor. eccl. cath., lib. I, c. 21). "Воскресшая и живущая без конца плоть составляет более достойный познания предмет, чем все то, что успели узнать врачи (исследуя человеческое тело)" (он же, De an. et ejus orig., lib. IV, c. 10). "Оставь естественные знания. Достаточно, если ты знаешь, что огонь горяч, вода холодна и мокра. Если ты умеешь обращаться со своим полем, скотом, домом и ребенком, этого довольно по части знаний естества. Старайся познать лишь Христа, который тебе укажет, кто ты таков и в чем твое достояние. Таким образом, ты познаешь Бога и самого себя, чего не ведает ни один естествоиспытатель, ни один из разделов естествознания" (Лютер, ч. XIII, стр. 264). Из этих цитат, которые, впрочем, можно продолжать до бесконечности, достаточно ясно видно, что истинное, религиозное христианство не содержит в себе никакого принципа, никакого мотива для научной и материальной культуры. Практической целью и объектом для христианина является исключительно небо, т.е. реализованное спасение души. Но теоретической целью и объектом для христианина служит только Бог как существо, тождественное со спасением души. Кто познал Бога, познал все. И насколько Бог бесконечно важнее мира, настолько богословие бесконечно важнее познания вселенной. Богословие доставляет человеку блаженство, ибо его объект есть не что иное, как олицетворенное блаженство. "Несчастен тот, кто все знает, но тебя не знает, и счастлив тот, кто познал тебя, хотя бы и не знал он ничего другого" (Августин, Confess., lib. V, с. 4). Следовательно, кто захотел бы, кто мог бы променять блаженное божественное существо на жалкие, ничтожные вещи этого мира? Разумеется, Бог открывает себя в природе, но только в своих самых общих, самых неопределенных свойствах; себя самого, свое истинное, свое личное существо, открывает он только в религии, в христианстве. Познание Бога в природе есть язычество, а познание Бога в нем самом, в Христе, в котором пребывает живая полнота божественности, есть христианство. Поэтому какой же интерес может иметь для христианина занятие материальными, естественными вещами? Изучение природы, вообще культуры, предполагает или по крайней мере порождает языческое, т.е. светское, анти- богословское, антисупранатуралистическое понимание и верование. Поэтому культура современных христианских народов не только не проистекает из христианства, а становится понятной лишь из отрицания христианства, носившего вначале лишь практический характер. Вообще надо различать между тем, что христиане говорили и делали как христиане, и тем, что они говорили и делали как язычники, как естественные люди, то есть в согласии или в противоречии со своей верой. Поэтому сколь "пошлыми" являются современные христиане, когда они хвалятся искусствами и науками современных народов как созданиями христианства! Насколько в этом отношении почтенны древние христиане но сравнению с современными рекламистами! Они знали лишь то христианство, которое содержится в христианской вере; поэтому они не относили к христианству ни сокровищ и богатств этого мира, ни его наук и искусств. Во всех этих вещах они отдавали преиму- щество древним язычникам, грекам и римлянам. "Почему ты не удивляешься, Эразм, что от начала мира среди язычников встречались во все времена более высокие, более ценные люди, люди более разумные, более прилежные и более опытные во всех искусствах, чем среди христиан или народа божиего? Или, как говорит сам Христос, чада этого мира умнее, чем чада света\ и это очень важные слова. Да кого из христиан (я уже не говорю о греках, Демосфене и др.) можно было бы сравнить по уму и прилежанию хотя бы с Цицероном?" {Лютер, ч. XIX, стр. 37). "Чем же мы отличаемся от них? Быть может, умом, ученостью, нравственной культурой? Нет, только истинным познанием, почитанием Бога и обращением к нему" (Melanchthonis et alior. declam. Т. Ill, de vera invocat. Dei). В религии человек видит цель в себе самом, или он объективирует себя как цель и объект Бога. Тайна воплощения есть тайна любви Бога к человеку, но тайна любви Бога есть тайна любви человека к самому себе. Бог страдает, страдает за меня - в этом содержится высшее самоуслаждение, высшая самодостоверность человеческой души. "Ибо так возлюбил Бог мир, что отдал сына своего единородного" (Еванг., от Иоанна, 3, 16). "Если Бог за нас, кто против нас? Тот, который сына своего не пощадил, но предал его за всех нас" (К Римл., 8, 31, 32). "Бог свою любовь к нам доказывает тем, что Христос умер за нас" (там же, 5, 8). "А что ныне живу во плоти, то живу верою в сына божия, возлюбившего меня и предавшего себя за меня" (К Галатам, 2, 20. См. также Поел, к Титу, 3, 4; к Евреям, I, II). "Для христиан существование промысла доказывается всем этим миром, но всего более вочеловечением Бога, совершившимся ради нас, этим самым божественным и по силе выраженной любви к человеку самым невероятным актом промысла" (Григория Нисского Phil., lib. Ill, de provid. е., 1512. В. Rhenanus, Jo. Cono, interp.) "Братья, посмотрите, как унизился Бог ради людей. Поэтому человек не должен презирать себя, ибо сам Бог принял на себя его позор" (Августин, Serm. ad pop., serm. 371, с. 3). "О человек, ради тебя вочело- вечился Бог, и ты должен высоко ценить себя" (Serm. 380, с. 2). "Как может отчаиваться тот, ради кого так унизился сын божий?" (он же, De agone chr., с. 11). "Кто может ненавидеть человека, естество и образ которого являет вочеловечившийся Бог? Поистине, кто ненавидит человека, ненавидит Бога" (Manuale, с. 26, Псевдо-Августин). "Мысль, что Бог так высоко ценил нас, что сын божий вступил в общение с нами и. своей смертью искупил грехи наши, должна возвышать наш дух и освобождать его от сомнения в бессмертии" (Петр Ломб., lib. Ill, dist. 20, с. 1). "Главным делом божественного промысла является вочеловечение. Ни небо, ни земля, ни море, ни воздух, ни солнце, ни луна, ни звезды не доказыва- ют так убедительно безмерной доброты Бога к нам, как вочеловечение единородного сына божия. Следовательно, Бог печется о нас, он печется о нас с любовью (Феодорит, De provid., orat. X, Opp. Paris, 1642. Т. IV, § 442). "Только в забвении достоинства своего существа человек может прилепляться к вещам, недостойным Бога (т.е. только Бог есть объект, достойный человека и соответствующий достоинству человека). И поэтому, чтобы доказать человеку достойнейшим образом, что только в Боге состоит полное блаженство человека, Бог непосредственно и воспринял человеческое естество" (Фома Акв., Summa с. gentil., lib. IV, с. 54). "Бог не противник людей. Если бы Бог был нашим противником и врагом, то он поистине не воспринял бы бедное, жалкое человеческое естество" "Как высоко почтил нас Господь Бог наш, если он повелел сыну своему стать человеком! Ужели он мог бы еще более приблизиться к нам!" (Лютер, ч. XVI, стр. 533, 574). О сердце дивное, тобою Всех смертных пленены сердца Не потому, что над землею Твоей державе нет конца, А потому, что в сей юдоли, В людскую плоть одет, средь зол Земной и горестной недоли Ты путь страдальческий прошел. И хоть мы все отлично знаем, Что ты на небесах - Господь, Тебя мы братом ощущаем, В тебе родную видим плоть. И упоительней сознанья Нет для души моей больной, Чем то, что для меня страданья И смерть принял владыка мой. К тебе все помыслы я шлю: Я за любовь тебя люблю, За то, что ты - наш князь и Бог - Для нас стать кротким агнцем мог. Пусть верят все, что наш Господь Для нас приял людскую плоть И лютой смерти не избег, Чтоб был искуплен человек; Пусть верят в то, что он воскрес И ныне с высоты небес Царит, воссев на божий трон, В плоть человечью облечен. (Gesangbuch der evangel. Briidergemeinde, Gnadau, 1824) (Книга песен евангелической братской общины, Гнадау, 1824)



























































\footnote{человек человеку бог (лат.).}
\footnote{времена меняются (лат.).}
\footnote{чрезвычайно трудное дело (лат.).}
\footnote{опасность в промедлении, промедление опасно (лат.).}
\footnote{по преимуществу (фр.).}
\footnote{см. примеч. 121.}
\footnote{супружеская связь в целях деторождения есть грех (лат.).}
\footnote{в данном случае Фейербах неточно цитирует сочинение Ф. Бэкона. <<О достоинстве и приумножении наук>>. См.: Бэкон Ф. Соч. в 2-х т. Т. 1. М., 1971 г., с. 144.}
\footnote{слово присоединяется к стихии и образует таинство крещения (лат.).}
\footnote{оскорбление величества Божьего (лат.).}
\footnote{апостолы\dots в своих посланиях прокляли еретиков (лат.).}
\footnote{теологическое заблуждение (лат.).}
\footnote{прегрешение против бога (лат.).}
\footnote{объект любви (лат.).}
\footnote{на самом деле (лат.).}
\footnote{на практике (лат.).}
\footnote{фактически, на деле (лат.).}
\footnote{по праву, на основании права, по закону (лат.).}
\footnote{см.: Иоан., \rom{2}, 41--42.}
\footnote{имя собственное (лат.).}
\footnote{любовь торжествует над богом (лат.).}
\footnote{непознаваемый в себе познается в другом (лат.).}
\footnote{см.: Вторая книга Моисеева. Исход (20,2).}

\end{document}